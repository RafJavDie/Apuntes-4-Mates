%\documentclass[12pt,a4paper,spanish]{amsart}
%\usepackage[margin=1.5cm]{geometry}
%%\usepackage[spanish]{babel}
%%\usepackage[all]{xy}
%%\CompileMatrices
%%\OnlyOutlines
%%\ShowOutlines
%\usepackage{amsmath,amssymb,varioref,enumerate,showkeys}
%%\usepackage{emlines2}
%\usepackage[dvipsone]{graphicx}
%\usepackage{epsfig}
%%\usepackage{psfrag}
%%\newdir{ >}{{}*!/-5pt/\dir{>}}
%%\externaldocument{}
%%begin numlast
%\newcommand{\lga}{\longrightarrow}
%\newcommand{\lgaf}{\longleftarrow}
%\newcommand {\sub}{\subset}
%\newtheorem{dummy}{realdumb}[section]
%\newtheorem{theorem}[dummy]{Theorem}
%\newtheorem{lemma}[dummy]{Lemma}
%\newtheorem{corollary}[dummy]{Corollary}
%\newtheorem{proposition}[dummy]{Proposition}
%\newtheorem*{theoremun}{Theorem}
%\newtheorem*{corollaryun}{Corollary}
%\theoremstyle{definition}               %%Change Theoremstyle
%\newtheorem{definition}[dummy]{Definition}
%\newtheorem{conjecture}[dummy]{Conjecture}
%\newtheorem{question}[dummy]{Question}
%\newtheorem{example}[dummy]{Example}
%%\newtheorem{nothing}[dummy]{Ejercicio}
%\newtheorem{nothing}{Ejercicio}
%%\theoremstyle{remark}
%\newtheorem{remark}[dummy]{Remark}
%\newenvironment{display}{\refstepcounter{dummy} $$}%
%{\leqno{\rm ({\thedummy})} $$} \numberwithin{equation}{dummy}
%\theoremstyle{plain}
%%end numlast
%\DeclareMathOperator{\co}{co} \DeclareMathOperator{\Diag}{Diag}
%\DeclareMathOperator{\Ar}{Ar} \DeclareMathOperator{\Hom}{Hom}
%\DeclareMathOperator{\coker}{coker} \DeclareMathOperator{\im}{Im}
%\DeclareMathOperator{\R}{\mathbb R}
%\DeclareMathOperator{\N}{\mathbb N}
%\newcommand{\loc}[2]{{\mathcal#1}^{-1}{\mathcal#2}}
%\newcommand{\cat}[1]{\mathcal#1}
%\newcommand{\scat}[1]{\overset{\sim}{\mathcal#1}}
%\newcommand{\ccat}[2]{{\mathcal #1}^{\wedge #2}}
%\newcommand{\complex}[1]{C({\mathcal#1})}
%\newcommand{\ccomp}[2]{C\left({\mathcal#1}^{\wedge#2}\right)}
%\newcommand{\quot}[2]{\left.{\mathcal#1}\!\right/\negmedspace{\mathcal#2}}
%\newcommand{\cquot}[2]{C\left(\left.{\mathcal#1}\!\right/\negmedspace{\mathcal#2}\right)}
%\newcommand{\ccquot}[3]{C\left(\left.
%{\mathcal#1}^{\wedge#3}\right/\negmedspace{\mathcal#2}^{\wedge#3}\right)}
%\newcommand{\qquot}[3]{{\mathcal#1}^{\wedge#3}\negmedspace
%\left.\right/\negmedspace{\mathcal#2}^{\wedge#3}}
%\title{Geometr\'{\i}a y Topolog\'{\i}a de Superficies 2016/17\\ Relaci\'on 6.}
\documentclass{article}
\usepackage{amsmath,accents}%
\usepackage{amsfonts}%
\usepackage{amssymb}%
\usepackage{comment}
\usepackage{graphicx}
\usepackage{mathrsfs}
\usepackage[utf8]{inputenc}
\usepackage{amsfonts}
\usepackage{amssymb}
\usepackage{graphicx}
\usepackage{mathrsfs}
\usepackage{setspace}
\usepackage{amsthm}
\usepackage{nccmath}
\usepackage[spanish]{babel}
\usepackage{multirow}
\usepackage{hyperref}
\usepackage{tikz-cd}
\usepackage{pgf,tikz}
\usetikzlibrary{arrows}
\usetikzlibrary{cd}
\usetikzlibrary{babel}
\theoremstyle{plain}
\hypersetup{colorlinks=true,citecolor=red, linkcolor=blue}

\renewcommand{\baselinestretch}{1,4}
\setlength{\oddsidemargin}{0.5in}
\setlength{\evensidemargin}{0.5in}
\setlength{\textwidth}{5.4in}
\setlength{\topmargin}{-0.25in}
\setlength{\headheight}{0.5in}
\setlength{\headsep}{0.6in}
\setlength{\textheight}{8in}
\setlength{\footskip}{0.75in}

\theoremstyle{definition}

\newtheorem{theorem}{Teorema}[section]
\newtheorem{acknowledgement}{Acknowledgement}
\newtheorem{algorithm}{Algorithm}
\newtheorem{axiom}{Axiom}
\newtheorem{case}{Case}
\newtheorem{claim}{Claim}
\newtheorem{propi}[theorem]{Propiedades}
\newtheorem{condition}{Condition}
\newtheorem{conjecture}{Conjecture}
\newtheorem{coro}[theorem]{Corolario}
\newtheorem{criterion}{Criterion}
\newtheorem{defi}[theorem]{Definición}
\newtheorem{example}[theorem]{Ejemplo}
\newtheorem{exercise}{Ejercicio}
\newtheorem{lemma}[theorem]{Lema}
\newtheorem{nota}[theorem]{Nota}
\newtheorem{sol}{Solución}
\newtheorem*{sol*}{Solución}
\newtheorem{prop}[theorem]{Proposición}
\newtheorem{remark}{Remark}

\newtheorem{dem}[theorem]{Demostración}

\newtheorem{summary}{Summary}

\providecommand{\abs}[1]{\lvert#1\rvert}
\providecommand{\norm}[1]{\lVert#1\rVert}
\providecommand{\ninf}[1]{\norm{#1}_\infty}
\providecommand{\numn}[1]{\norm{#1}_1}
\providecommand{\gabs}[1]{\left|{#1}\right|}
\newcommand{\bor}[1]{\mathcal{B}(#1)}
\providecommand{\func}[2]{\colon{#1}\longrightarrow{#2}}
\newcommand{\R}{\mathbb{R}}
\newcommand{\Q}{\mathbb{Q}}
\newcommand{\Z}{\mathbb{Z}}
\newcommand{\F}{\mathbb{F}}
\newcommand{\C}{\mathbb{C}}
\newcommand{\X}{\chi}
\providecommand{\Zn}[1]{\Z / \Z #1}
\newcommand{\resi}{\varepsilon_L}
\newcommand{\cee}{\mathbb{C}}
\providecommand{\conv}[1]{\overset{#1}{\longrightarrow}}
\providecommand{\gene}[1]{\langle{#1}\rangle}
\providecommand{\convcs}{\xrightarrow{CS}}
% xrightarrow{d}[d]
\setcounter{exercise}{0}
\newcommand{\cicl}{\mathcal{C}}

\begin{document}
\title{Relación 6 - Geometría y Topología de superficies }
\author{Javi, Rafa, Diego}
\maketitle

\begin{exercise}

Calcula el grupo fundamental del cilindro $\mathbb{R}\times S^1$.

\end{exercise}
\begin{sol*}
$\pi_1(\mathbb{R}\times S^1)\cong\pi_1(\R)\times\pi_1(S^1)\cong\{1\}\times\Z\cong\Z.$
\end{sol*}

\newpage


\begin{exercise}\label{2}

Calcula el grupo fundamental de la suma puntual de dos copias de $S^2$.

\end{exercise}
\begin{sol*}
Para diferenciar cada una de las copias las llamaremos $S^2_1$ y $S^2_2$. Sea $U$ el abierto formado por $S^2_1$ y la mitad de $S^2_2$. Sea $V$ el abierto formado por $S^2_2$ y la mitad de $S^2_1$. Se tiene que $U\cup V=S^2_1\vee S^2_2$. Tenemos que tanto $U$ como $V$ son simplemente conexos y $U\cap V$ es contráctil, por lo que usando el teorema de Seifert-Van Kampen obtenemos que $\pi_1(S^2_1\vee S^2_2)\cong\{1\}$.
\end{sol*}

\newpage

\begin{exercise}

Calcula el grupo fundamental de la suma puntual de dos copias de $S^1$, y generaliza el resultado al caso de $n$ copias, para un $n$ cualquiera.

\end{exercise}
\begin{sol*}\

\begin{tikzpicture}[line cap=round,line join=round,>=triangle 45,x=1.0cm,y=1.0cm]
\clip(-4.132863999999993,-4.614538666666662) rectangle (8.133802666666664,1.326794666666668);
\draw(-1.,0.) circle (1.cm);
\draw(1.,0.) circle (1.cm);
\draw(-3.,-3.) circle (1.cm);
\draw(3.,-3.) circle (1.cm);
\draw [shift={(-1.,-3.)}] plot[domain=1.5707963267948966:4.71238898038469,variable=\t]({1.*1.*cos(\t r)+0.*1.*sin(\t r)},{0.*1.*cos(\t r)+1.*1.*sin(\t r)});
\draw [shift={(1.,-3.)}] plot[domain=-1.5707963267948966:1.5707963267948966,variable=\t]({1.*1.*cos(\t r)+0.*1.*sin(\t r)},{0.*1.*cos(\t r)+1.*1.*sin(\t r)});
\draw [->] (-1.5546666666666642,-1.0653333333333301) -- (-2.,-2.);
\draw [->] (1.592,-0.9906666666666636) -- (2.,-2.);
\draw (-3.2,1) node[anchor=north west] {$S^1\vee S^1$};
\draw (-4.3,-1.8092053333333302) node[anchor=north west] {$X_1$};
\draw (3.8,-1.8518719999999969) node[anchor=north west] {$X_2$};
\draw [shift={(6.,-3.)}] plot[domain=-1.5707963267948966:1.5707963267948966,variable=\t]({1.*1.*cos(\t r)+0.*1.*sin(\t r)},{0.*1.*cos(\t r)+1.*1.*sin(\t r)});
\draw [shift={(8.,-3.)}] plot[domain=1.5707963267948966:4.71238898038469,variable=\t]({1.*1.*cos(\t r)+0.*1.*sin(\t r)},{0.*1.*cos(\t r)+1.*1.*sin(\t r)});
\draw (6.3,-1.5212053333333304) node[anchor=north west] {$X_1\cap X_2$};
\end{tikzpicture}

En este caso, $X_1\cap X_2$ es contráctil, por lo que $\pi_1(X_2\cap X_2,x_0)=\{1\}$. Por otro lado, tanto $X_1$ como $X_2$ tienen a $S^1$ como retracto de deformación fuerte, por lo que las inclusiones $k_1\func{S^1}{X_1}$ y $k_1\func{S^1}{X_2}$ inducen isomorfismos $k_{1*}\func{\pi_1(S^1,x_0)}{\pi_1(X_1,x_0)}$ y $k_{2*}\func{\pi_1(S^1,x_0)}{\pi_1(X_2,x_0)}$. Por tanto, $\pi_1(X_1,x_0)=\pi_1(X_2,x_0)\cong\Z$, siendo los generadores las clases $\varepsilon_i \in \pi_1(X_i,x_0)$ de las vueltas canónicas $t\mapsto e^{2\pi it}$ de las circunferencias completas de $X_1$ y $X_2$, respectivamente. El diagrama resultante del teorema de Seifert-Van Kampen es el siguiente
\[
\begin{tikzcd}
\pi_1(X_1\cap X_2,x_0)\cong\{1\} \ar[r, "i_1*"]\arrow[d,"i_2*"'] & \pi_1(X_1,x_0)=\langle\varepsilon_1|\ \rangle\arrow[d,dashed,"j_1*"]\\
\langle\varepsilon_2|\ \rangle =\pi_1(X_2,x_0)\arrow[r,dashrightarrow,"j_2*"'] & \pi_1(S^1\vee S^1,x_0)=\langle j_{1*}(\varepsilon_1), j_{2*}(\varepsilon_2)|\ \rangle
\end{tikzcd}
\]
El resultado es el grupo libre con dos generadores que son las clases en $\pi_1(S^1\vee S^1,x_0)$ de las vueltas canónicas de las dos circunferencias que componen $S^1\vee S^1$. Nótese que no es abeliano.

En general $\pi_1(S^1\vee\cdots\vee S^1,x_0)\cong\Z*\cdots *\Z$, donde $S^1\vee\cdots\vee S^1$ es el resultado de unir una cantidad finita de circunferencias por un solo punto común a todas.
\end{sol*}

\newpage

\begin{exercise}

Calcula el grupo fundamental de la banda de Möbius, y tambi\'en el de la botella de Klein.

\end{exercise}
\begin{sol*}
La banda de Möbius se retrae con deformación sobre una circunferencia, por lo que su grupo fundamental es isomorfo a $\Z$. Vayamos ahora  la Botella de Klein.\\
\definecolor{ffffff}{rgb}{1.,1.,1.}
\definecolor{zzttqq}{rgb}{0.6,0.2,0.}
\begin{tikzpicture}[line cap=round,line join=round,>=triangle 45,x=1.0cm,y=1.0cm]
\clip(-1.6541333333333335,-1.6) rectangle (9.140533333333332,2.6);
\fill[color=zzttqq,fill=zzttqq,fill opacity=0.2] (0.,0.) -- (1.,0.) -- (1.,1.) -- (0.,1.) -- cycle;
\fill[color=zzttqq,fill=zzttqq,fill opacity=0.2] (2.,-1.5) -- (3.,-1.5) -- (3.,-0.5) -- (2.,-0.5) -- cycle;
\draw [color=ffffff,fill=ffffff,fill opacity=1.0] (2.5,-1.) circle (0.20010805969664447cm);
\draw [line width=0.4pt,color=zzttqq,fill=zzttqq,fill opacity=0.2] (5.,0.5) circle (0.45cm);
\draw [line width=0.4pt] (5.,0.5) circle (0.35cm);
\draw [color=ffffff,fill=ffffff,fill opacity=1.0] (5.,0.5) circle (0.17782688210729017cm);
\draw [line width=0.4pt,color=zzttqq,fill=zzttqq,fill opacity=0.2] (2.5,1.5) circle (0.45cm);
\draw [color=zzttqq] (0.,0.)-- (1.,0.);
\draw [color=zzttqq] (1.,0.)-- (1.,1.);
\draw [color=zzttqq] (1.,1.)-- (0.,1.);
\draw [color=zzttqq] (0.,1.)-- (0.,0.);
\draw [->] (0.,1.) -- (0.6,1.);
\draw [->] (1.,0.) -- (0.4,0.);
\draw [->] (0.,0.) -- (0.,0.6);
\draw [->] (1.,0.) -- (1.,0.6);
\draw [->] (2.,-0.5) -- (2.6,-0.5);
\draw [->] (3.,-1.5) -- (2.4,-1.5);
\draw [->] (2.,-1.5) -- (2,-0.9);
\draw [->] (3.,-1.5) -- (3,-0.9);
\draw [->] (1.780533333333333,1.3104) -- (1.2429333333333332,0.9946666666666657);
\draw [->] (1.8146666666666664,-0.34506666666666636) -- (1.2856,0.);
\draw [color=zzttqq] (2.,-1.5)-- (3.,-1.5);
\draw [color=zzttqq] (3.,-1.5)-- (3.,-0.5);
\draw [color=zzttqq] (3.,-0.5)-- (2.,-0.5);
\draw [color=zzttqq] (2.,-0.5)-- (2.,-1.5);
\draw [->] (4.016266666666666,0.19253333333333317)--(3.3165333333333327,-0.2597333333333331);
\draw [->] (3.9992,0.9008)--(3.3165333333333327,1.3274666666666655);
\draw [->] (5.,-0.5) -- (5,0.15);
\draw (1.8,2.5) node[anchor=north west] {$X_1=\text{disco abierto}$};
\draw (2.2,0) node[anchor=north west] {$X_2$};
\draw (0.1,-0.5) node[anchor=north west] {$\text{ Hacemos}$};
\draw (0.1,-0.8) node[anchor=north west] {$\text{un agujero}$};
\draw (5.5,0.74464) node[anchor=north west] {$X_1\cap X_2$};
\draw (-0.5,0.6) node[anchor=north west] {$a$};
\draw (1.1,0.6) node[anchor=north west] {$a$};
\draw (0.4,1.4) node[anchor=north west] {$b$};
\draw (0.4,0) node[anchor=north west] {$b$};
\draw (1.2,1.8) node[anchor=north west] {$j_1$};
\draw (1.5,0.3) node[anchor=north west] {$j_2$};
\draw (3.3,0.4) node[anchor=north west] {$i_1$};
\draw (3.5,1.6) node[anchor=north west] {$i_2$};
\draw (4.75,-0.5) node[anchor=north west] {$S^1$};
\end{tikzpicture}

Se tiene que $X_1\cap X_2$ tiene por retracto de deformación fuerte a $S^1$, así que $\pi_1(X_1\cap X_2,x_0)=\langle\varepsilon_0|\ \rangle \cong\Z$, 
donde $\varepsilon^0$ es la clase de la vuelta canónica $\alpha^0$ de la circunferencia $S^1$. Además, $X_1$ es contráctil, con lo que $\pi_1(X_1,x_0)=\{1\}$. Para $X_2$ concluimos que $\pi_1(X_2,x_0)=\langle \varepsilon_2^a,\varepsilon_2^b|\ \rangle$ donde $\varepsilon_2^a$ y $\varepsilon_2^b$ son las clases en $\pi_1(X_2,x_0)$ de los lazos $\gamma*\alpha^a*\overline{\gamma}$ y $\gamma*\alpha^b*\overline{\gamma}$. Aquí, $\alpha^a$ y $\alpha^b$ son las vueltas canónicas de las circunferencias $a$ y $b$. Ahora, por el teorema de Seifert-Van Kampen y teniendo en cuenta que $i_{1*}(\varepsilon_0)=1$ se llega a que $\pi_1(X,x_0)=\langle j_{2*}(\varepsilon_2^a),j_{2*}(\varepsilon_2^b)|j_{2*}(i_{2*}(\varepsilon^0))^{-1}\rangle$ . Por último, calculamos $i_{2*}(\varepsilon_0)$.

\definecolor{qqffqq}{rgb}{0.,1.,0.}
\begin{tikzpicture}[line cap=round,line join=round,>=triangle 45,x=1.0cm,y=1.0cm]
\clip(-4,-0.5) rectangle (6.432288808308237,2.4209391043300577);

\draw (0.,0.)-- (2.,0.);
\draw (2.,0.)-- (2.,2.);
\draw (2.,2.)-- (0.,2.);
\draw (0.,2.)-- (0.,0.);
\draw (0.,0.)-- (0.7066666666666667,0.72);
\draw(1.,1.) circle (0.4055175020198813cm);
\draw [->] (2.,0.) -- (0.9,0.);
\draw [->] (2.,0.) -- (2.,1.);
\draw [->] (0.,0.) -- (0.,1.);
\draw [->] (0.,2.) -- (1.,2.);
\draw [->] (1.2185957084281869,1.341555794419043) -- (1.1392101437468405,1.3808739690795742);
\draw [line width=1.2pt,color=qqffqq] (0.7379455914298125,0.644459593784375)-- (0.21338395946064603,0.09709093433828794);
\draw [line width=1.2pt,color=qqffqq] (0.21338395946064603,0.09709093433828794)-- (1.910226803743515,0.08796812334751983);
\draw [line width=1.2pt,color=qqffqq] (1.910226803743515,0.08796812334751983)-- (1.9193496147342832,1.9125303215011433);
\draw [line width=1.2pt,color=qqffqq] (1.9193496147342832,1.9125303215011433)-- (0.0674189836083562,1.9216531324919113);
\draw [line width=1.2pt,color=qqffqq] (0.0674189836083562,1.9216531324919113)-- (0.08566460558989243,0.2020032607321213);
\draw [line width=1.2pt,color=qqffqq] (0.08566460558989243,0.2020032607321213)-- (0.6056648320636749,0.7311262981966721);
\draw [->,color=qqffqq] (0.7379455914298125,0.644459593784375) -- (0.4745729277044789,0.3696359446796789);
\draw [->,color=qqffqq] (0.21338395946064603,0.09709093433828794) -- (1.0389744903683416,0.092652275569967);
\draw [->,color=qqffqq] (1.910226803743515,0.08796812334751983) -- (1.9150849212485737,1.2);
\draw [->,color=qqffqq] (1.9193496147342832,1.9125303215011433) -- (0.98430664841938,1.9171364444879162);
\draw [->,color=qqffqq] (0.0674189836083562,1.9216531324919113) -- (0.0770015131620486,1.0184997220564025);
\draw [->,color=qqffqq] (0.08566460558989243,0.2020032607321213) -- (0.38419490345889523,0.5057709322479488);
\draw (1.3,1.1787653001727818) node[anchor=north west] {$\alpha^0$};
\draw (0.7757231517425831,1) node[anchor=north west] {$x_0$};
\draw (-0.5,1.138695177458031) node[anchor=north west] {$a$};
\draw (2.1381073240441104,1.1119817623148638) node[anchor=north west] {$a$};
\draw (0.6,0.6) node[anchor=north west] {$\gamma$};
\draw (0.8,0.) node[anchor=north west] {$b$};
\draw (0.8,2.5) node[anchor=north west] {$b$};
\draw (-0.5,0) node[anchor=north west] {$x_1$};
\draw (-0.5,2) node[anchor=north west] {$x_1$};
\draw (2,2) node[anchor=north west] {$x_1$};
\draw (2,0) node[anchor=north west] {$x_1$};
\begin{scriptsize}
\draw [fill=black] (0.7066666666666667,0.72) circle (2.5pt);
\draw [fill=black] (0.,0.) circle (2.pt);
\draw [fill=black] (2.,0.) circle (2.pt);
\draw [fill=black] (2.,2.) circle (2.pt);
\draw [fill=black] (0.,2.) circle (2.pt);
\end{scriptsize}
\end{tikzpicture}
\[
i_{2*}(\varepsilon_0)=[\gamma*\overline{\alpha^b}*\alpha^a*\overline{\alpha^b}*\overline{\alpha^a}*\overline{\gamma}]=(\varepsilon_2^{b})^{-1}\varepsilon_2^{a}(\varepsilon_2^{b})^{-1}(\varepsilon_2^a)^{-1}.
\]
Por ello, $\pi_1(X,x_0)=\langle\varepsilon^a,\varepsilon^b|\varepsilon^{a}\varepsilon^{b}(\varepsilon^{a})^{-1}\varepsilon^b\rangle$, donde $\varepsilon^{a}=j_{2*}(\varepsilon^{a}_2)$ y $\varepsilon^{b}=j_{2*}(\varepsilon^{b}_2)$ son las clases de $\gamma*\alpha^a*\overline{\gamma}$ y $\gamma*\alpha^b*\overline{\gamma}$ en $\pi_1(X,x_0)$. Si, como en el caso del toro, se deshace el cambio de punto base a $x_1$, tenemos $\pi_1(X,x_1)=\langle\eta^a,\eta^b|\eta^{a}\eta^{b}(\eta^{a})^{-1}\eta^b\rangle$, donde $\eta^a$ y $\eta^b$ son las clases de las vueltas canónicas de las circunferencias $a$ y $b$.
\end{sol*}

\newpage

\begin{exercise}

Calcula el grupo fundamental del \emph{sombrero bobo} (i.e. un tri\'angulo relleno con los lados identificados seg\'un la palabra $aa^{-1}a$), y del plano proyectivo real.

\end{exercise}

\begin{sol*}
Empezamos por el plano proyectivo que es más simple y el sombrero bobo será análogo.\
\definecolor{ffffff}{rgb}{1.,1.,1.}
\begin{tikzpicture}[line cap=round,line join=round,>=triangle 45,x=1.0cm,y=1.0cm]
\clip(-3.9736363636363694,-2) rectangle (6.480909090909101,2);
\draw [fill=black,fill opacity=0.13] (0.,0.) circle (1.5cm);
\draw [fill=black,fill opacity=1.0] (0.,0.) circle (0.2700550907983698cm);
\draw [color=black] (0.,0.) circle (0.5cm);
\draw [color=ffffff,fill=ffffff,fill opacity=1.0] (0.,0.) circle (0.2700550907983698cm);
\draw [dash pattern=on 2pt off 2pt] (0.,0.) circle (0.9cm);
\draw [->] (0.07096361834016676,1.498320447992375) -- (-0.07232341452011842,1.4982554267254136);
\draw [->] (0.,-1.5) -- (0.16255704268110133,-1.4911657211305438);
\draw (-0.1,1.9) node[anchor=north west] {$a$};
\draw (-0.1,-1.5) node[anchor=north west] {$a$};
\draw (0.35,0.5) node[anchor=north west] {$S^1$};
\begin{scriptsize}
\draw [fill=black] (-1.5,0.) circle (1.5pt);
\draw [fill=black] (1.5,0.) circle (1.5pt);


\end{scriptsize}
\end{tikzpicture}

Tomamos como $X_1$ un disco abierto, que al ser contráctil cumple que $\pi_1(X,x_0)=\{1\}$. El conjunto $X_2$ será el resultado de eliminar un disco cerrado de $X$. Por tanto, $X_1\cap X_2$ tendrá como retracto de deformación fuerte a $S^1$, es decir, $\pi_1(X_1\cap X_2,x_0)=\langle\varepsilon_0|\ \rangle$, donde $\varepsilon_0$ es la clase de la vuelta canónica de $S^1$. En $X_2$ observemos lo siguiente

\definecolor{ffffff}{rgb}{1.,1.,1.}
\begin{tikzpicture}[line cap=round,line join=round,>=triangle 45,x=1.0cm,y=1.0cm]
\clip(-2.,-2) rectangle (13.348732398112169,2.6);
\draw [fill=black,fill opacity=0.11] (0.,0.) circle (1.5cm);
\draw [color=ffffff,fill=ffffff,fill opacity=1.0] (0.,0.) circle (0.2700550907983698cm);
\draw [->] (0.07096361834016676,1.498320447992375) -- (-0.07232341452011842,1.4982554267254136);
\draw [->] (0.,-1.5) -- (0.16255704268110133,-1.4911657211305438);
\draw (-0.08609868635317827,1.9283962211701489) node[anchor=north west] {$a$};
\draw (-0.16768672937624718,-1.6) node[anchor=north west] {$a$};
\draw (-0.17039989202214118,-0.20950806396164895)-- (-0.8549582108626596,-1.232496027449387);
\draw(0.,0.) circle (0.2700550907983698cm);
\draw (1.6952402529838262,0.37822340373183405) node[anchor=north west] {$\text{Se retrae con }$};
\draw(6.,0.) circle (1.5cm);
\draw (6.,0.)-- (5.070163695513291,-1.177032050055775);
\draw (-0.16768672937624718,-0.17929489025913883) node[anchor=north west] {$x_0$};
\draw (0.15,0.5) node[anchor=north west] {$S^1$};
\draw (-0.9155771237543788,-1.3) node[anchor=north west] {$x_1$};
\draw (6.033004540376989,0.13345927466262644) node[anchor=north west] {$x_0$};
\draw (4.836379909371979,-1.3) node[anchor=north west] {$x_1$};
\draw (5.652260339602668,-0.42405901932834644) node[anchor=north west] {$\gamma$};
\draw (9.5,-0.42405901932834644) node[anchor=north west] {$\gamma$};
\draw (1.7360342744953605,-0.02971681138351197) node[anchor=north west] {$\text{deformación a}$};
\draw [->] (6.071929341419882,1.4982743973794994) -- (5.901274342189669,1.4967475553646037);
\draw [->] (6.,-1.5) -- (6.146625481272907,-1.4928164549741165);
\draw (5.910622475842386,-1.5118995929692691) node[anchor=north west] {$a$};
\draw (5.842632439989829,1.9827882498521952) node[anchor=north west] {$a$};
\draw (7.6,0.4326154324138802) node[anchor=north west] {$\cong Y\equiv$};
\draw (9.5,-1.5)-- (9.5,0.);
\draw(9.5,1.) circle (1.cm);
\draw [->] (9.58738738655073,1.9961744047463943) -- (9.5,2.);
\draw (9.473300354516395,1.9) node[anchor=north west] {$a$};
\draw (9.60928042622151,-1.3215274925821077) node[anchor=north west] {$x_0$};
\draw (9.622878433392023,0.09266525315109184) node[anchor=north west] {$x_1$};
\begin{scriptsize}
\draw [fill=black] (1.5,0.) circle (1.5pt);
\draw [fill=black] (-0.17039989202214118,-0.20950806396164895) circle (2.5pt);
\draw [fill=black] (-0.8549582108626596,-1.232496027449387) circle (2.5pt);
\draw [fill=black] (6.,0.) circle (2.5pt);
\draw [fill=black] (4.5,0.) circle (1.5pt);
\draw [fill=black] (5.070163695513291,-1.177032050055775) circle (2.5pt);
\draw [fill=black] (7.5,0.) circle (1.5pt);
\draw [fill=black] (9.5,-1.5) circle (2.5pt);
\draw [fill=black] (9.5,0.) circle (2.5pt);
\draw [fill=black] (-1.5,0.) circle (1.5pt);
\end{scriptsize}
\end{tikzpicture}

Como se puede observar, $\pi_1(Y,x_0)\cong\Z$ está generado por la clase del lazo $\gamma*\alpha^a*\overline{\gamma}$, donde $\alpha^a$ es la vuelta canónica de la circunferencia $a$. Por tanto, como la inclusión $k\func{Y}{X_2}$ induce un isomorfismo $\pi_1(Y,x_0)\cong\pi_1(X_2,x_0)$, tenemos que $\pi_1(X_2,x_0)\cong\Z$ generado por la clase $\varepsilon_2$ del lazo $\gamma*\alpha^a*\overline{\gamma}$ en $\pi_1(X_2,x_0)$. Del teorema \ref{SVK} de Seifert-Van Kampen obtenemos el siguiente diagrama

\[
\begin{tikzcd}
\pi_1(X_1\cap X_2,x_0)=\langle \varepsilon_0|\ \rangle \ar[r, "i_1*"]\arrow[d,"i_2*"'] & \pi_1(X_1,x_0)=\{1\}=\langle\ |\ \rangle\arrow[d,dashed,"j_1*"]\\
\pi_1(X_2,x_0)=\langle \varepsilon_2|\ \rangle\arrow[r,dashrightarrow,"j_2*"'] & \pi_1(X,x_0)
\end{tikzcd}
\]

Como $i_{1*}(\varepsilon_0)=1$, solo tenemos que calcular $i_{2*}(\varepsilon_0)$, que nos da
\[
i_{2*}(\varepsilon_0)=[i_2\circ\alpha^0]=[\gamma*\alpha^a*\alpha^a*\overline{\gamma}]=[\gamma*\alpha^a*\overline{\gamma}*\gamma*\alpha^a*\overline{\gamma}]=\varepsilon_2\varepsilon_2=\varepsilon_2^2
\]

Esto quiere decir que $\pi_1(X,x_0)=\langle\varepsilon|\varepsilon^{-2}\rangle =\langle\varepsilon|\varepsilon^{2}\rangle\cong\Z_2$, donde $\varepsilon=j_{2*}(\varepsilon_2)$ es la clase de $\gamma*\alpha^a*\overline{\gamma}$ en $\pi_1(X,x_0)$. Si deshacemos el cambio de punto base, 
tenemos  $\pi_1(X,x_1)=\langle\eta|\eta^{2}\rangle$, donde $\eta$ es la clase de $\alpha^a$, la vuelta canónica de la circunferencia $a$.

Usando el mismo razonamiento que para el plano proyectivo llegaríamos a que el grupo fundamental del sombrero bobo es de la forma $\langle a\mid aa^{-1}a\rangle\cong\langle a\mid a\rangle\cong\langle\mid\rangle$.
\end{sol*}

\newpage

\begin{exercise}

Calcula el grupo fundamental de de cualquier superficie compacta sin borde.

\end{exercise}
\begin{sol*}
Por el teorema de clasificación de superficies sabemos que solo existen dos tipos de superficies compactas sin borde. Dentro de las del tipo I (orientables) tratamos la esfera aparte, pues ya sabemos que $\pi_1(S^2)\cong\{1\}$. Calculemos pues el grupo fundamental de una superficie orientable arbitraria.\

Sea $S$ una superficie de tipo I. Recordemos que estas superficies vienen representadas por un modelo que tiene el código $a^{}_1b^{}_1a^{-1}_1b^{-1}_1\dots a^{}_nb^{}_na^{-1}_nb^{-1}_n$. Esto es, $S$ está representada por la identificación de los lados de un polígono regular de $4n$ lados de acuerdo con el siguiente gráfico:

\definecolor{zzttqq}{rgb}{0.6,0.2,0.}
\begin{tikzpicture}[line cap=round,line join=round,>=triangle 45,x=1.0cm,y=1.0cm]
\clip(-1.5,-0.85) rectangle (8,3.4);
\draw (4.,3.)-- (5.5,3.);
\draw (5.5,3.)-- (6.5,2.);
\draw (6.5,2.)-- (6.5,0.5);
\draw (6.5,0.5)-- (5.5,-0.5);
\draw (4.,3.)-- (3.,2.);
\draw (3.,2.)-- (3.,0.5);
\draw [dash pattern=on 2pt off 2pt] (3.,0.5)-- (5.5,-0.5);
\draw(4.8,1.4) circle (0.6027172390626455cm);
\draw (4.405738478652503,1.8558792856097492)-- (4.,3.);
\draw [->] (4.,3.) -- (4.8,3.);
\draw [->] (5.5,3.) -- (6.,2.5);
\draw [->] (6.5,0.5) -- (6.5,1.1957501053417074);
\draw [->] (5.5,-0.5) -- (6.,0.);
\draw [->] (4.,3.) -- (3.5,2.5);
\draw [->] (3.,2.) -- (3.,1.2);
\draw [dash pattern=on 2pt off 2pt] (4.8,1.4) circle (1cm);
\draw (4.55,1.5838433463488795) node[anchor=north west] {$A$};
\draw (3.414187200315089,0.8683484392203448) node[anchor=north west] {$B$};
\draw (1,1.7) node[anchor=north west] {\large{$S\ \equiv$}};
\draw (2.35,1.5) node[anchor=north west] {$a_n$};
\draw (3.,3) node[anchor=north west] {$b_n$};
\draw (4.7,3.35) node[anchor=north west] {$a_1$};
\draw (6.048151968315743,2.8025434628864936) node[anchor=north west] {$b_1$};
\draw (6.622120410297976,1.387278811423458) node[anchor=north west] {$a_2$};
\draw (6.142502945079946,-0.04371100283361169) node[anchor=north west] {$b_2$};
\draw (4.2,2.78681830009246) node[anchor=north west] {$\gamma$};
\draw (5.25,1.552393020760812) node[anchor=north west] {$S^1$};
\draw (4.4284602005302665,1.8904840208325373) node[anchor=north west] {$x_0$};
\begin{scriptsize}
\draw [fill=black] (4.405738478652503,1.8558792856097492) circle (2.5pt);
\fill[color=zzttqq,fill=zzttqq,fill opacity=0.1](3,0.5)--(3.,2.) -- (4,3)--(5.5,3)--(6.5,2)-- (6.5,0.5)--(5.5,-0.5)--cycle;


\end{scriptsize}
\end{tikzpicture}

Procedemos de manera similar a los ejemplo del capítulo anterior: descomponemos $S$ como la unión $S=A\cup B$ donde $A$ es un disco abierto, por lo que es contráctil, esto es, $\pi_1(A,x_0)=\{1\}$, y $B$ es $S$ menos un disco cerrado contenido en $A$. En particular, $B$ se retrae al siguiente grafo:

\begin{tikzpicture}[line cap=round,line join=round,>=triangle 45,x=1.0cm,y=1.0cm]
\clip(-3,0.5) rectangle (6.393300959292774,3.2);
\draw [rotate around={-12.485467490857165:(1.6364672736480006,2.086385207554635)}] (1.6364672736480006,2.086385207554635) ellipse (0.4955243419426075cm and 0.13041747704333986cm);
\draw [rotate around={86.157546872085:(2.173228013337387,2.4831214064554827)}] (2.173228013337387,2.4831214064554827) ellipse (0.46750069112046705cm and 0.1344452240202966cm);
\draw [shift={(2.2965829659368096,2.189736654327123)},dash pattern=on 2pt off 2pt]  plot[domain=-0.020405330686538825:1.4601391056210007,variable=\t]({1.*0.6535839211181232*cos(\t r)+0.*0.6535839211181232*sin(\t r)},{0.*0.6535839211181232*cos(\t r)+1.*0.6535839211181232*sin(\t r)});
\draw [rotate around={1.1708850280784644:(2.58494704784979,1.9915168803910728)}] (2.58494704784979,1.9915168803910728) ellipse (0.4223626584291288cm and 0.07777723849776252cm);
\draw [rotate around={-64.02560603756869:(2.3199203894015676,1.5529583855030706)}] (2.3199203894015676,1.5529583855030706) ellipse (0.4572702113179464cm and 0.14432623016288876cm);
\draw (2.1432227545969864,1.9963694313334333)-- (1.,1.);
\draw (1.2,2.6) node[anchor=north west] {$a_1$};
\draw (1.6,2.9298663699236602) node[anchor=north west] {$b_1$};
\draw (2.812673416271692,1.9430267491282782) node[anchor=north west] {$a_n$};
\draw (2.6259740285536464,1.4829461151088095) node[anchor=north west] {$b_n$};
\draw (1.0723684093284844,0.996194139986763) node[anchor=north west] {$x_0$};
\draw (1.645802243033909,1.4162677623523647) node[anchor=north west] {$\gamma$};
\draw (-0.3,2.2) node[anchor=north west] {\large{$L\equiv$}};
\draw [->] (1.35,2.2308648946743626) -- (1.45,2.2512894804092745);
\draw [->] (2.099185545523773,2.8155770684977686) -- (2.0664202368358655,2.6958193007182714);
\draw [->] (2.8,1.92) -- (2.7,1.925);
\draw [->] (2.539517043578322,1.3753231851333148) -- (2.5,1.5);
\begin{scriptsize}
\draw [fill=black] (2.1432227545969864,1.9963694313334333) circle (2.0pt);
\draw [fill=black] (1.,1.) circle (2.5pt);
\end{scriptsize}
\end{tikzpicture}

Por otro lado, $A\cap B$ se retrae con deformación fuerte a la circunferencia $S^1$. Sea $\varepsilon$  la clase de la vuelta canónica de $S^1$. Sea $k_*\func{\pi_1(L,x_0)}{\pi_1(B,x_0)}$ el isomorfismo inducido por la inclusión $k\func{L}{B}$. Sabemos que $\pi_1(L,x_0)$ es el grupo libre engendrado por las clases de los lazos $\gamma*a_i*\overline{\gamma}$ y $\gamma*b_i*\overline{\gamma}$, donde $a_i$ y $b_i$ son las vueltas canónicas indicadas de la misma manera.  Por tanto las clases, que denotamos $\alpha'_i$ y $\beta'_i$ respectivamente, de esos mismos lazos en $B$ generan $\pi_1(B,x_0)$. Tenemos el siguiente diagrama
\[
\begin{tikzcd}
\pi_1(A\cap B,x_0)=\langle\varepsilon|\ \rangle \ar[r, "i_{1*}"]\arrow[d,"i_{2*}"'] & \pi_1(A,x_0)=\{1\}\arrow[d, dashed, "j_{1*}"]\\
\pi_1(B,x_0)=\langle \alpha'_1,\beta'_1,\dots,\alpha'_n,\beta'_n| \rangle\arrow[r,dashed,"j_{2*}"'] & \pi_1(S,x_0)
\end{tikzcd}
\]
Como se ha hecho repetidas veces ya, en este diagrama se comprueba que $i_{2*}(\varepsilon)=[\alpha'_1,\beta'_1]\cdots[\alpha'_n,\beta'_n]$, donde $[\alpha'_i,\beta'_i]$ indica la relación de conmutación. Además, como $i_{1*}(\varepsilon)=1$, usando el teorema de Seifert-Van Kampen, se llega a 
\[
\pi_1(S,x)=\langle \alpha_1,\beta_1,\dots,\alpha_n,\beta_n|[\alpha_1,\beta_1]\cdots[\alpha_n,\beta_n]\rangle,
\]
donde $\alpha_i=j_{2*}(\alpha'_i)$ y $\beta_i=j_{2*}(\beta'_i)$ son las clases de los lazos  $\gamma*a_i*\overline{\gamma}$ y $\gamma*b_i*\overline{\gamma}$ en $X$. Ahora, al abelianizar la relación del grupo, que es un producto de conmutadores, sevuelve trivial y nos queda el grupo abeliano libre
\begin{gather*}
(\pi_1(S,x))^{ab}=\langle \alpha_i, \beta_i|[\alpha_i,\beta_i],[\alpha_i,\beta_j],\ 1\leq i, j\leq n\rangle=\\
\langle \alpha_i, \beta_i|[\alpha_i,\beta_j],\ 1\leq i, j\leq n\rangle\cong\Z\underbrace{\times\cdots\times}_{2n\ veces}\Z.
\end{gather*}

Vamos ahora con el otro tipo. Una superficie $S$ de tipo II viene representada por el código $a_1 a_1 a_2 a_2\dots a_n a_n$. Esto es, $S$ es el resultado de identificar los lados de un polígono de $2n$ lados de acuerdo con el siguiente gráfico:

\begin{tikzpicture}[line cap=round,line join=round,>=triangle 45,x=1.0cm,y=1.0cm]
\clip(-1.5,-0.85) rectangle (8,3.4);
\draw (4.,3.)-- (5.5,3.);
\draw (5.5,3.)-- (6.5,2.);
\draw (6.5,2.)-- (6.5,0.5);
\draw (6.5,0.5)-- (5.5,-0.5);
\draw (4.,3.)-- (3.,2.);
\draw (3.,2.)-- (3.,0.5);
\draw [dash pattern=on 2pt off 2pt] (3.,0.5)-- (5.5,-0.5);
\draw(4.8,1.4) circle (0.6027172390626455cm);
\draw (4.405738478652503,1.8558792856097492)-- (4.,3.);
\draw [->] (4.,3.) -- (4.8,3.);
\draw [->] (5.5,3.) -- (6.,2.5);
\draw [->] (6.5,1.5) -- (6.5,1.);
\draw [->] (6.5,0.5) -- (6.,0.);
\draw [->] (3.,2) -- (3.5,2.5);
\draw [->] (3.,1.) -- (3.,1.5);
\draw [dash pattern=on 2pt off 2pt] (4.8,1.4) circle (1cm);
\draw (4.55,1.5838433463488795) node[anchor=north west] {$A$};
\draw (3.414187200315089,0.8683484392203448) node[anchor=north west] {$B$};
\draw (1,1.7) node[anchor=north west] {\large{$S\ \equiv$}};
\draw (2.35,1.5) node[anchor=north west] {$a_n$};
\draw (3.,3) node[anchor=north west] {$a_n$};
\draw (4.7,3.35) node[anchor=north west] {$a_1$};
\draw (6.048151968315743,2.8025434628864936) node[anchor=north west] {$a_1$};
\draw (6.622120410297976,1.387278811423458) node[anchor=north west] {$a_2$};
\draw (6.142502945079946,-0.04371100283361169) node[anchor=north west] {$a_2$};
\draw (4.2,2.78681830009246) node[anchor=north west] {$\gamma$};
\draw (5.25,1.552393020760812) node[anchor=north west] {$S^1$};
\draw (4.4284602005302665,1.8904840208325373) node[anchor=north west] {$x_0$};
\begin{scriptsize}
\draw [fill=black] (4.405738478652503,1.8558792856097492) circle (2.5pt);
\fill[color=zzttqq,fill=zzttqq,fill opacity=0.1](3,0.5)--(3.,2.) -- (4,3)--(5.5,3)--(6.5,2)-- (6.5,0.5)--(5.5,-0.5)--cycle;
\end{scriptsize}
\end{tikzpicture}

Como en el caso anterior, escribimos $S=A\cup B$, donde $A$ es un disco abierto, por lo que es contráctil, esto es, $\pi_1(A,x_0)=\{1\}$, y $B$ es $S$ menos un disco cerrado contenido en $A$, por lo que $B$ se retrae al siguiente grafo:

\begin{tikzpicture}[line cap=round,line join=round,>=triangle 45,x=1.0cm,y=1.0cm]
\clip(-3,0.5) rectangle (6.393300959292774,3.2);
\draw [rotate around={-12.485467490857165:(1.6364672736480006,2.086385207554635)}] (1.6364672736480006,2.086385207554635) ellipse (0.4955243419426075cm and 0.13041747704333986cm);
\draw [rotate around={86.157546872085:(2.173228013337387,2.4831214064554827)}] (2.173228013337387,2.4831214064554827) ellipse (0.46750069112046705cm and 0.1344452240202966cm);
\draw [shift={(2.2965829659368096,2.189736654327123)},dash pattern=on 2pt off 2pt]  plot[domain=-0.020405330686538825:1.4601391056210007,variable=\t]({1.*0.6535839211181232*cos(\t r)+0.*0.6535839211181232*sin(\t r)},{0.*0.6535839211181232*cos(\t r)+1.*0.6535839211181232*sin(\t r)});
\draw [rotate around={1.1708850280784644:(2.58494704784979,1.9915168803910728)}] (2.58494704784979,1.9915168803910728) ellipse (0.4223626584291288cm and 0.07777723849776252cm);
\draw [rotate around={-64.02560603756869:(2.3199203894015676,1.5529583855030706)}] (2.3199203894015676,1.5529583855030706) ellipse (0.4572702113179464cm and 0.14432623016288876cm);
\draw (2.1432227545969864,1.9963694313334333)-- (1.,1.);
\draw (1.2,2.6) node[anchor=north west] {$a_1$};
\draw (1.6,2.9298663699236602) node[anchor=north west] {$a_2$};
\draw (2.812673416271692,1.9430267491282782) node[anchor=north west] {$a_{n-1}$};
\draw (2.6259740285536464,1.4829461151088095) node[anchor=north west] {$a_n$};
\draw (1.0723684093284844,0.996194139986763) node[anchor=north west] {$x_0$};
\draw (1.645802243033909,1.4162677623523647) node[anchor=north west] {$\gamma$};
\draw (-0.3,2.2) node[anchor=north west] {\large{$L\equiv$}};
%\draw [->] (1.35,2.2308648946743626) -- (1.45,2.2512894804092745);
%\draw [->] (2.,2.9) -- (2.1,3);
%\draw [->] (2.8,1.92) -- (2.7,1.925);
%\draw [->] (2.539517043578322,1.3753231851333148) -- (2.5,1.5);
\begin{scriptsize}
\draw [fill=black] (2.1432227545969864,1.9963694313334333) circle (2.0pt);
\draw [fill=black] (1.,1.) circle (2.5pt);
\end{scriptsize}
\end{tikzpicture}

Por otro lado, $A\cap B$ se retrae con deformación fuerte a la circunferencia $S^1$. Sean $\alpha'_i=\gamma_\sharp[a_i]$ las clases de los lazos $\gamma*a_i*\overline{\gamma}$ ya vistos en $B$, que al representar generadores de $\pi_1(L,x_0)$ también son representantes de generadores de $\pi_1(B,x_0)$. Tenemos el siguiente diagrama, donde $\varepsilon$ la clase de la vuelta canónica de $S^1$ en $A\cap B$,
\[
\begin{tikzcd}
\pi_1(A\cap B,x_0)=\langle\varepsilon|\ \rangle \ar[r, "i_{1*}"]\arrow[d,"i_{2*}"'] & \pi_1(A,x_0)=\{1\}\arrow[d, dashed, "j_{1*}"]\\
\pi_1(B,x_0)=\langle \alpha'_1,\dots,\alpha'_n|\ \rangle\arrow[r,dashed,"j_{2*}"'] & \pi_1(S,x_0)
\end{tikzcd}
\]
Se tiene que $i_{1*}(\varepsilon)=1$ y $i_{2*}(\varepsilon)=\alpha'_1\alpha'_1\cdots\ \alpha_n'\alpha_n'={\alpha'}_1^2\cdots\ {\alpha'}_n^2$, así que sin más que aplicar el teorema de Seifert-Van Kampen obtenemos

\[
\pi_1(S,x_0)=\langle\alpha_1,\dots,\alpha_n|\alpha_1^2\cdots\ \alpha_n^2\rangle,
\]
donde $\alpha_i=j_{2*}(\alpha'_i)$ son las clases de los lazos $\gamma*a_i*\overline{\gamma}$ en $X$. Ahora, abelianizando
\[
(\pi_1(S,x))^{ab}=\langle\alpha_i|\alpha_1^2\cdots\ \alpha_n^2,\ [\alpha^{}_i,\alpha^{}_j]\ 1\leq i, j\leq n \rangle.
\]
Si ahora llamamos $\beta_i=\alpha_i$ para $1\leq i\leq n-1$ y $\beta_n=\alpha_1\cdots\ \alpha_n$ nos queda
\[
(\pi_1(S,x))^{ab}=\langle\beta_i|[\beta^{}_i,\beta^{}_j],\ \beta^2_n,\ 1, i\leq j\leq n-1\rangle\cong\Z\underbrace{\times\cdots\times}_{n-1\ veces}\Z\times\Z_2
\]
\end{sol*}

\newpage

\begin{exercise}

Calcula el grupo fundamental del toro menos un disco.

\end{exercise}
\begin{sol*}
El toro menos un disco se retrae con deformación sobre $S^1\vee S^1$, por lo que su grupo fundamental es isomorfo a $\Z*\Z$.
\end{sol*}

\newpage

\begin{exercise}

Calcula el grupo fundamental del toro con una membrana.

\end{exercise}
\begin{sol*}
Sea $U$ el abierto formado por la membrana y una sección abierta del toro. Sea $V$ el abierto formado por el toro unido al exterior de un disco cerrado estrictamente contenido en la membrana. Se tiene que $U$ es contráctil, $V$ se retrae con deformación sobre el toro y $\pi_1(U\cap V)\cong\Z$. Para utilizar el teorema de Seifert-Van Kampen debemos calcular la imagen del generador $\varepsilon$ de $\pi_1(U\cap V)$ por la inclusión en $\pi_1(V)$ (ya que en $\pi_1(U)$ es trivial). Como se trata de la clase de un lazo en la membrana, se proyecta sobre el toro como un lazo horizontal (recordemos que el toro tiene un generador que se corresponde con un lazo y otro que se corresponde con un lazo vertical), por lo que el generador de lazos horizontales del toro se debe trivializar. En conclusión, el grupo fundamental del toro con membrana es simplemente $\Z$. Una forma de ver esto es que cualquier lazo horizontal se puede llevar de forma continua hasta la membrana y ahí contraerlo.
\end{sol*}

\newpage

\begin{exercise}

Calcula el grupo fundamental de la esfera $S^2$ con un di\'ametro.

\end{exercise}
\begin{sol*} Podemos representar el espacio de la siguiente forma:\

\begin{tikzpicture}[line cap=round,line join=round,>=triangle 45,x=1.0cm,y=1.0cm]
\clip(-2.357333333333334,-1.455111111111109) rectangle (5.820444444444447,2.5057777777777774);
\draw [fill=black,fill opacity=0.10000000149011612] (0.,0.) circle (1.cm);
\draw [shift={(0.,0.5)}] plot[domain=-0.46364760900080615:3.6052402625905993,variable=\t]({1.*1.118033988749895*cos(\t r)+0.*1.118033988749895*sin(\t r)},{0.*1.118033988749895*cos(\t r)+1.*1.118033988749895*sin(\t r)});
\draw [->] (0.,1.) -- (0.11052935837081118,0.9938728595439845);
\draw [->] (0.,-1.) -- (0.10879432554464447,-0.9940642809845276);
\draw (-0.024888888888888495,2) node[anchor=north west] {$b$};
\draw (-0.017777777777777382,1.3) node[anchor=north west] {$a$};
\draw (-0.017777777777777382,-1.0213333333333316) node[anchor=north west] {$a$};
\begin{scriptsize}
\draw [fill=black] (1.,0.) circle (2.0pt);
\draw [fill=black] (-1.,0.) circle (2.0pt);
\end{scriptsize}
\end{tikzpicture}

Como en otras ocasiones, tomamos un disco abierto en el interior que llamaremos $U$, de modo que $\pi_1(U)=\{1\}$. Por otro lado, $V$ se retrae con deformación sobre $S^1$, por lo que $\pi_1(V)=\Z$. Por último, $\pi_1(U\cap V)=\Z$. Utilizando Seifert-Van Kampen, vemos que el generador de $\pi_1(V)$ se neutraliza dado que dentro de la esfera menos el disco cualquier lazo se puede contraer a un punto, sin importar el diámetro que hemos añadido. Esto nos da la relación $1=1$ en el grupo fundamental. Así que finalmente el grupo fundamental quedaría isomorfo a $\Z$. 
\end{sol*}

\newpage

\begin{exercise}

Sean $A$ un conjunto finito de puntos del plano, y $B$ un conjunto finito de puntos del espacio. Calcula los grupos fundamentales de $\mathbb{R}^2\setminus A$ y $\mathbb{R}^3\setminus B$.

\end{exercise}
\begin{sol*}
Sean $n=|A|$ y $m=|B|$ los cardinales correspondientes. Tenemos que $\mathbb{R}^2\setminus A$ se retrae sobre la suma puntual de $n$ copias de $S^1$, por lo su grupo fundamental es el producto libre de $n$ copias de $\Z$. En el segundo caso. $\mathbb{R}^3\setminus B$ se retrae sobre la suma puntual de $m$ copias de $S^2$, por lo que, generalizando el ejercicio \ref{2}, se tiene que su grupo fundamental es trivial.
\end{sol*}

\newpage

\begin{exercise}

Calcula el grupo fundamental de los siguientes espacios:

\begin{itemize}

\item $\mathbb{R}^3$ menos una recta.

\item $\mathbb{R}^3$ menos una circunferencia.

\item $\mathbb{R}^3$ menos la uni\'on del eje OZ y la circunferencia unidad del plano $XY$.

\item $\mathbb{R}^3$ menos la uni\'on puntual de dos circunferencias.

\item $\mathbb{R}^3$ menos la uni\'on de dos circunferencias coplanarias disjuntas.

\item $\mathbb{R}^3$ menos la uni\'on de dos circunferencias disjuntas pero engarzadas como eslabones.



\end{itemize}

\end{exercise}

\begin{sol*}
Se va a usar reiteradamente que los retractos de deformación conservan el grupo fundamental.
\begin{itemize}
\item Se retrae con deformación sobre $S^1$, luego su grupo fundamental es $\Z$.
\item Se retrae con deformación sobre un toro con membrana, luego su grupo fundamental es $\Z$.
\item Se retrae con deformación sobre un toro, luego su grupo fundamental es $\Z\times\Z$.
\item Se retrae sobre un doble toro con dos membranas, por lo que su grupo fundamental es $\Z$. Similarmente al toro con una membrana, los únicos lazos que no se pueden llevar a una membrana para contraerlos son los verticales.
\item Se retrae con deformación sobre la suma puntual de dos toros con membranas. En este caso los generadores de lazos verticales también pueden contraerse gracias al punto de unión, por lo que el grupo fundamental es trivial.
\item Se retrae con deformación sobre $S^2\vee T^2$, como ya vimos en una relación anterior. Usaremos ahora Seifer-Van Kampen. Tomamos $U$ como la esfera con un abierto del toro, y $V$ como el toro con un abierto de la esfera. $U$ retrae con deformación sobre la esfera, por lo que $\pi_1(U)=\{1\}$. Por otro lado, $\pi_1(V)=\Z\times\Z$, ya que $V$ retrae con deformación sobre el toro. Finalmente, $\pi_1(U\cap V)=\{1\}$ puesto que esta intersección es contrácil. Por tanto, $\pi_1(S^2\vee T^2)\cong\pi_1(T^2)\cong\Z\times\Z$. Una forma alternativa se puede encontrar a partir de la página 22 del siguiente pdf \url{https://www.math.cornell.edu/~hatcher/AT/ATch1.pdf}
\end{itemize}
\end{sol*}

\newpage

\begin{exercise}

Calcular el grupo fundamental de la uni\'on puntual de un toro con la uni\'on puntual de dos circunferencias.



\end{exercise}

\begin{sol*}
Vamos a analizar los dos casos posibles que se dan. En primer lugar que el punto de unión del toro como $S^1\vee S^1$ sea el mismo que por el que las circunferencias están unidas y en segundo lugar que sea un punto distinto.
\begin{itemize}
\item Tomamos como $U$ el toro unido a una sección abierta de $S^1\vee S^1$ homeomorfa a cuatro segmentos abiertos por un extremo. Como $V$ tomamos la unión puntual de circunferencias unida a una bola abierta centrada en punto de unión sobre el toro. $U$ se retrae con deformación sobre el toro, por lo que $\pi_1(U)\cong\Z\times\Z$. $V$ se retrae sobre $S^1\vee S^1$, por lo que $\pi_1(V)\cong\Z*\Z$. En la intersección podemos retraer tanto la bola como los segmentos sobre el punto de unión, luego $\pi_1(U\cap V)\cong\{1\}$. Así pues, el grupo fundamental del espacio original queda $\Z\times\Z*\Z*\Z$. 
\item Tomamos $U$ igual al toro unido a una sección abierta de la circunferencia a la que está unida. Como $V$ la unión puntual de circunferencias y una bola abierta centrada en el punto de unión sobre el toro. A partir de aquí totalmente análogo al caso anterior, por lo que el grupo fundamental vuelve a ser $\Z\times\Z*\Z*\Z$.
\end{itemize}
\end{sol*}

\newpage

\begin{exercise}

Dados dos toros por las identificaciones $aba^{-1}b^{-1}$ y $cdc^{-1}d^{-1}$, calcular el grupo fundamental del cociente de la uni\'on disjunta de esos toros por la identificaci\'on $a=c$.

\end{exercise}
\begin{sol*}
El polígono resultante tendría las identificaciones $abda^{-1}d^{-1}b^{-1}$. Vamos a ir directamente al grupo fundamental ya que se haría de forma análoga a lo que se ha hecho ya repetidas veces. Tendríamos como grupo fundamental $\langle a,b,d\mid abda^{-1}d^{-1}b^{-1}\rangle$. Para ver más claro de qué grupo se trata hacemos el cambio de generador $\gamma=bd$, con lo que nos queda $\langle a,\gamma, d\mid a\gamma a^{-1}\gamma^{-1}\rangle\cong\Z\times\Z*\Z$.
\end{sol*}

\newpage

\begin{exercise}

Consideramos el subespacio $Y$ de $\mathbb{R}^2$ dado por la uni\'on de $X=\{(x,\textrm{sen }1/x);0<x\leq 1\}$ y un arco entre $(0,-1)$ y $(1,0)$ disjunto con $X$. Calcular el grupo fundamental de $Y$.

\end{exercise}
\begin{sol*}
En primer lugar podemos retraer el arco sobre el $(0,0)$ con deformación. Cualquier camino que empiece en el $(0,0)$ debe ser constante puesto que el espacio no es conexo por caminos. Por otra parte, $X$ es homeomorfo al intervalo $(0,+\infty)$ mediante el homeomorfismo $x\mapsto\textrm{sen }1/x$, por lo que es contráctil. Así que $\pi_1(Y)\cong\{1\}$.
\end{sol*}

\newpage

\begin{exercise}
Se sabe que las identificaciones del per\'{\i}metro de  un pol\'{\i}gono de diez lados dadas por los c\'odigos $abcb^{-1}daec^{-1}e^{-1}d^{-1}$ y $abcbdaece^{-1}d^{-1}$ son
superficies. Determinar sus grupos fundamentales. A partir de la abelianizaci\'on de los mismos determinar el modelo de cada superficie.
\end{exercise}
\begin{sol*}
A estas alturas estamos capacitados para ver que el grupo fundamental de la primera superficie es $G=\langle a,b,c,d,e\mid abcb^{-1}daec^{-1}e^{-1}d^{-1}\rangle$ y el de la segunda $H=\langle a,b,c,d,e\mid abcbdaece^{-1}d^{-1}\rangle$. Ahora abelianizamos:
\[
G^{ab}=\langle a,b,c,d,e\mid a^2,aba^{-1}b^{-1},aca^{-1}c^{-1},\dots,bcb^{-1}c^{-1},bdb^{-1}d^{-1},\dots\rangle\cong\Z\times\Z\times\Z\times\Z\times\Z_2.
\]
Esto quiere decir que la primera superficie es la suma conexa de cinco planos proyectivos. 
\[
H^{ab}=\langle a,b,c,d,e \mid (abc)^2,aba^{-1}b^{-1},aca^{-1}c^{-1},bcb^{-1}c^{-1},\dots\rangle,
\]
Haciendo el cambio de generador $\gamma=abc$, obtenemos el grupo 
\[
H^{ab}=\langle a,b,\gamma,d,e|\gamma^2,aba^{-1}b^{-1},\gamma a\gamma^{-1}a^{-1},\gamma b\gamma^{-1}b^{-1},\dots\rangle\cong\Z\times\Z\times\Z\times\Z\times Z_2,
\] por lo que se trata también de la suma conexa de cinco planos proyectivos.
\end{sol*}

\newpage

\begin{exercise}
Usar el abelianizado del grupo fundamental para determinar qu\'e superficies son las representadas por los c\'odigos
$$
a_1a_2\dots a_n a^{-1}_1a^{-1}_2\dots a^{-1}_{n-1}a_n \mbox{ y } a_1a_2\dots a_na^{-1}_1a^{-1}_2\dots a^{-1}_{n-1}a^{-1}_n.
$$
\end{exercise}
\begin{sol*}
En el primer caso, además de las relaciones de abelianización, obtendríamos $a_n^2$, ya que el resto se cancelaría con su inverso, por lo que simplemente nos quedaría un grupo isomorfo a $\Z\times\Z\times\cdots\times\Z\times\Z_2$. Esto quiere decir que la primera superficie es la suma conexa de $n$ planos proyectivos. En el segundo caso, tras abelianizar todos se cancelan, por lo que el grupo fundamental es el grupo libre abeliano de $n$ generadores y esto significa que la superficie es una suma conexa de $n$ toros.
\end{sol*}

\newpage

\begin{exercise}
Determinar con la ayuda del grupo fundamental el modelo de la superficie dada por  el c\'odigo $abcd^{-1}ad^{-1}b^{-1}c^{-1}$.
\end{exercise}
\begin{sol*}
Como se ha hecho ya varias veces, el grupo fundamental es $G=\langle a,b,c,d\mid abcd^{-1}ad^{-1}b^{-1}c^{-1}\rangle$. Vamos a abelianizarlo para reconocer la superficie. Obviaremos las relaciones de abelianización por comodidad.
\[
G^{ab}=\langle a,b,c,d\mid a^2(d^{-1})^2\rangle.
\]
La relación se puede escribir equialentemente como $a^2d^{-1}=d$, así que llamamos $\gamma= a^2d^{-1}$ y tenemos el grupo isomorfo $\langle \gamma,b,c,d\mid \gamma =d\rangle\cong\langle\gamma,b,c\mid\rangle\cong\Z\times\Z\times\Z$. Esto quiere decir que la superficie es la suma conexa de tres toros.
\end{sol*}

\end{document}
