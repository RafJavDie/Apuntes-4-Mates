	\documentclass[twoside]{article}
\usepackage{../../estilo-ejercicios}

%--------------------------------------------------------
\begin{document}

\title{Ejercicios de Variedades Diferenciables - Tema 3}
\author{Javi, Rafa, Diego}
\maketitle



\begin{ejercicio}{1}\label{1}
Sean $M$ una variedad diferenciable, $p \in M$ y $(U, \varphi = (x_1, \dots , x_m))$, $(V, \psi =
(y_1, \dots , y_m))$ dos s.l.c.entorno de $p$, tales que $x_1 = y_1$. ¿Es $\left(
\frac{\partial}{\partial x_1}\right)_p = \left(\frac{\partial}{\partial y_1}\right)_p$?
\end{ejercicio}
\begin{solucion}

\end{solucion}

\newpage

\begin{ejercicio}{2}
Sean $M$ una variedad diferenciable, $p \in M$, $W$ un entorno abierto de $p$ y
$f,g \in \mathcal{F}(W)$. Probar que:

\begin{itemize}
\item[(a)] Si $\forall q \in W$ y $\forall u \in  T_q(M)$ se tiene que $u(f) = 0$, entonces $f$ es constante
en un entorno de $p$.
\item[(b)] Si $\forall q \in W$ y $\forall u \in T_q(M)$ se verifica que $u(f) = u(g)$, entonces $f - g$ es
constante en un entorno de $p$.
\end{itemize}
\end{ejercicio}
\begin{solucion}

\end{solucion}

\newpage

\begin{ejercicio}{3}
Sean $M$ una variedad diferenciable, $p \in M$ y $\alpha : I \to M$ una curva diferenciable
pasando por $p$ (es decir, se puede suponer que $0 \in I$ y que $\alpha(0) = p$).
Si $\alpha'(0) = 0$, probar que la aplicación
$$L:\calF\to\R:f\mapsto L(f)=\left.\frac{d^2(f\circ\alpha)}{dt^2}\right|_{t=0}$$
define un vector tangente a $M$ en $p$.
\end{ejercicio}
\begin{solucion}

\end{solucion}

\newpage

\begin{ejercicio}{4}
Sean $M$ una variedad diferenciable, $p \in M$ y $(U, \varphi = (x_1, \dots , x_m))$ un s.l.c.
entorno de $p$. Dado cualquier covector $\omega \in T^*_p (M)$, calcular su expresión en
función de la base $\{(dx_i)_p \mid i = 1, \dots ,m\}$.
\end{ejercicio}
\begin{solucion}

\end{solucion}

\end{document}