	\documentclass[twoside]{article}
\usepackage{../../estilo-ejercicios}

%--------------------------------------------------------
\begin{document}

\title{Ejercicios de Variedades Diferenciables - Tema 3}
\author{Javi, Rafa, Diego}
\maketitle



\begin{ejercicio}{1}\label{1}
Sean $M$ una variedad diferenciable, $p \in M$ y $(U, \varphi = (x_1, \dots , x_m))$, $(V, \psi =
(y_1, \dots , y_m))$ dos s.l.c.entorno de $p$, tales que $x_1 = y_1$. ¿Es $\left(
\frac{\partial}{\partial x_1}\right)_p = \left(\frac{\partial}{\partial y_1}\right)_p$?
\end{ejercicio}
\begin{solucion}
No, basta tomar $M=U=V=\R^2$ y $φ(u,v)=(u,u+v)$, $ψ(u,v)=(u,v)$.
Por la fórmula del cambio de base:
\begin{gather*} \left(\frac{\partial}{\partial y_1}\right)_p = \left(\frac{\partial x_1}{\partial y_1}\right)_p \left(\frac{\partial}{\partial x_1}\right)_p +  \left(\frac{\partial x_2}{\partial y_1}\right)_p  \left(\frac{\partial}{\partial x_2}\right)_p =
 \left(\frac{\partial}{\partial x_1}\right)_p + \left(\frac{\partial}{\partial x_2}\right)_p
\end{gather*}
\end{solucion}
\newpage

\begin{ejercicio}{2}
Sean $M$ una variedad diferenciable, $p \in M$, $W$ un entorno abierto de $p$ y
$f,g \in \mathcal{F}(W)$. Probar que:

\begin{itemize}
\item[(a)] Si $\forall q \in W$ y $\forall u \in  T_q(M)$ se tiene que $u(f) = 0$, entonces $f$ es constante
en un entorno de $p$.
\item[(b)] Si $\forall q \in W$ y $\forall u \in T_q(M)$ se verifica que $u(f) = u(g)$, entonces $f - g$ es
constante en un entorno de $p$.
\end{itemize}
\end{ejercicio}
\begin{solucion}\
\begin{itemize}
\item[(a)] Sea $(W,\varphi=(x_1,\dots,x_n))$ un s.l.c. Dado $q\in W$ usamos la expresión $u = \sum u(x_i) \left(\dfrac{\partial}{\partial x_i}\right)_q$. Así pues, 
\[
0=u(f) = \left(\sum u(x_i) \left(\dfrac{\partial}{\partial x_i}\right)_q\right)f=\sum u(x_i) \left(\frac{\partial f}{\partial x_i}\right)_q = u(x)\cdot\nabla(f\circ\varphi^{-1})(\varphi(q))
\]
Como $u$ y $q$ son arbitrarios, esto implica que $\nabla f\circ\varphi^{-1}=0$, por lo que $f \circ \varphi^{-1}$ es constante en $\R^m$, en un entorno de $\varphi(p)$, pero como $\varphi$ es difeomorfismo, no puede ser constante (salvo el caso trivial), luego $f$ ha de ser constante. 
\item[(b)] $u(f) = u(g)\Rightarrow u(f)-u(g)=0$, Por linealidad, $u(f-g)=0$, que por el apartado anterior implica que $f-g$ es constante.
\end{itemize}
\end{solucion}

\newpage

\begin{ejercicio}{3}
Sean $M$ una variedad diferenciable, $p \in M$ y $\alpha : I \to M$ una curva diferenciable
pasando por $p$ (es decir, se puede suponer que $0 \in I$ y que $\alpha(0) = p$).
Si $\alpha'(0) = 0$, probar que la aplicación
$$L:\calF(M)\to\R:f\mapsto L(f)=\left.\frac{d^2(f\circ\alpha)}{dt^2}\right|_{t=0}$$
define un vector tangente a $M$ en $p$.
\end{ejercicio}
\begin{solucion}
Basta verificar que $L(f)$ cumple las propiedades de todo vector tangente. 
\begin{enumerate}
\item ($\R$-linealidad) Se tiene por ser la derivada un operador lineal. La composición de operadores lineales es también lineal, por lo que la segudna derivada lo es.
\item (Condición de Leibniz) Sean $f,g\in\calF(M)$. 
\begin{gather*}
L(fg)=\left.\frac{d^2[(f\circ\alpha)(g\circ\alpha)]}{dt^2}\right|_{t=0}=\left.\frac{d}{dt}\left[\frac{d(f\circ\alpha)}{dt}g(\alpha)+f(\alpha)\frac{d(g\circ\alpha)}{dt}\right]\right|_{t=0}=\\
\left.\frac{d^2(f\circ\alpha)}{dt^2}g(\alpha)\right|_{t=0}+\frac{d(f\circ\alpha)}{dt}g'(\alpha)\alpha(0)+\left.\frac{d^2(g\circ\alpha)}{dt^2}f(\alpha)\right|_{t=0}+\frac{d(g\circ\alpha)}{dt}f'(\alpha)\alpha(0)\overset{\alpha'(0)=0}{=}\\
\left.\frac{d^2(f\circ\alpha)}{dt^2}g(\alpha)\right|_{t=0}+\left.\frac{d^2(g\circ\alpha)}{dt^2}f(\alpha)\right|_{t=0}=L(f)g(p)+L(g)f(p)
\end{gather*}
\item Que $L(f)=L(g)$ si $f,g\in\calF(M)$ coinciden en un entorno de $p$ se tiene trivialmente.
\end{enumerate}
\end{solucion}

\newpage

\begin{ejercicio}{4}
Sean $M$ una variedad diferenciable, $p \in M$ y $(U, \varphi = (x_1, \dots , x_m))$ un s.l.c.
entorno de $p$. Dado cualquier covector $\omega \in T^*_p (M)$, calcular su expresión en
función de la base $\{(dx_i)_p \mid i = 1, \dots ,m\}$.
\end{ejercicio}
\begin{solucion}
Sea $v\in T_p(M)$, entonces
\[
\omega(v)=\omega\left(\sum_{i=1}^m v(x_i)\left(\frac{\partial}{\partial x_i}\right)_p\right)=\sum_{i=1}^nv(x_i)\omega\left(\left(\frac{\partial}{\partial x_i}\right)_p\right)=\sum_{i=1}^n(dx_i)_p(v)\omega\left(\left(\frac{\partial}{\partial x_i}\right)_p\right),
\]
por lo que la expresión de $w$ es
$$\omega=\sum_{i=1}^n\omega\left(\left(\frac{\partial}{\partial x_i}\right)_p\right)(dx_i)_p.$$
\end{solucion}

\end{document}