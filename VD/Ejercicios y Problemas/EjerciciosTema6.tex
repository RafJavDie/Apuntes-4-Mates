 	\documentclass[twoside]{article}
\usepackage{../../estilo-ejercicios}
\newcounter{ejercicio}
%--------------------------------------------------------
\begin{document}

\title{Ejercicios de Variedades Diferenciables - Tema 6}
\author{Javi, Rafa, Diego}
\maketitle

\begin{ejercicio}{1}
Probar que la diferencial de cualquier función constante es nula.
\end{ejercicio}
\begin{solucion}
Sea $df:\mathcal{X}(U)\to\calF(U)$ dada por $(df)X=Xf$, y sea $f=c\in\R$. Entonces, dado $p\in U$
$$((df)X)(p)=(Xf)(p)=X_pf=u(f)$$
para cualquier $u\in T_p(M)$ tal que $X_p=u$, y como $u(f)=0$ por ser $f$ constante, se tiene que $((df)X)(p)=0\ \forall p\in U$, de modo que $(df)X=0\ \forall X\in\mathcal{X}(U)$, es decir, que $df=0$ como campo de tensores. 
\end{solucion}

\newpage

\begin{ejercicio}{2}
Sea $(U, \varphi = (x_1, \dots, x_m))$ un s.l.c. y $f \in \calF(U)$. Probar que:
$$df =
\sum^m_{i=1}
\parcial{f}{x_i}dx_i$$
\end{ejercicio}
\begin{solucion}
De la propia definición del campo de tensores $df$ se obtiene que $df\left(\parcial{}{x_i}\right)=\parcial{f}{x_i}$, y como es un campo de tensores 1-covariante, aplicando el correspondiente ``lema 3'' se llega al resultado.
\end{solucion}

\end{document}