	\documentclass[twoside]{article}
\usepackage{../../estilo-ejercicios}

%--------------------------------------------------------
\begin{document}

\title{Problemas de Variedades Diferenciables - Tema 4}
\author{Javi, Rafa, Diego}
\maketitle

\begin{ejercicio}{6}
Sean $\R^3$, $X \in \mathcal{X}(\R^3)$ dado por $X = x\frac{\partial}{\partial x} + e^z\frac{\partial}{\partial y}$ y sea $f :
\R^3 \to \R^3 : (x, y, z) \mapsto (x + y, y, 7)$. ¿Está $X$ $f-$relacionado con $X$?
\end{ejercicio}
\begin{solucion}

\end{solucion}
\newpage
\begin{ejercicio}{7}
Sea $f\func{M}{N}$ una aplicación diferenciable. Probar:
\begin{enumerate}
\item Si $f$ es sobreyectiva, entonces, $\forall X \in \mathcal{X}(M)$ fijo, existe a lo más un campo de vectores $Y\in \mathcal{X}(N)$ tal que $X$ está $f$-relacionado con $Y$. 
\item Si $f$ es una inmersión, entonces, $\forall Y \in \mathcal{X}(N)$, existe a lo más un campo de vectores $X\in \mathcal{X}(M)$ tal que $X$ está $f$-relacionado con $Y$. 
\item Si $f$ es una inmersión, entonces, dado $\forall Y \in \mathcal{X}(N)$, existe $X\in \mathcal{X}(M)$, $f$-relacionado con $Y$ si y solo si $\forall p \in M$, $Y_{f(p)}\in f_{\ast p}(T_p(M))$.
\end{enumerate}
\end{ejercicio}
\begin{solucion}
\end{solucion}

\end{document}