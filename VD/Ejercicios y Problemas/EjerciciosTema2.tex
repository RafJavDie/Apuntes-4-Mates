	\documentclass[twoside]{article}
\usepackage{../../estilo-ejercicios}

%--------------------------------------------------------
\begin{document}

\title{Ejercicios de Variedades Diferenciables - Tema 2}
\author{Javi, Rafa, Diego}
\maketitle



\begin{ejercicio}{1}\label{1}
Sean $M$ y $N$ dos variedades diferenciables y $G \subseteq M$ un abierto.
\begin{itemize}
\item[(a)] Si $f \in \calF(G,N)$ y $H \subseteq G$ es un abierto, probar que $f|_H \in \calF(H,N)$.
\item[(b)] Si $f: G \to N$ es una aplicación y $\{G_i\}_{i\in I}$ es un recubrimiento por
abiertos de $G$ (es decir, $G = \cup_{i\in I} G_i$) tal que $f|_{G_i} \in \calF(G_i,N)$, para
todo $i \in I$, probar que $f \in \calF(G,N)$.
\end{itemize}
\end{ejercicio}
\begin{solucion}

\end{solucion}

\newpage

\begin{ejercicio}{2}
Sean $M$ una variedad diferenciable y $G \subseteq M$ un abierto. Probar que la
aplicación inclusión $i : G \to M$ es diferenciable.
\end{ejercicio}
\begin{solucion}

\end{solucion}

\newpage

\begin{ejercicio}{3}
Probar que la composición de aplicaciones diferenciables es una aplicación
diferenciable.
\end{ejercicio}
\begin{solucion}

\end{solucion}

\newpage

\begin{ejercicio}{4}
Probar que la aplicación identidad entre una variedad diferenciable y ella
misma es un difeomorfismo.
\end{ejercicio}
\begin{solucion}

\end{solucion}

\newpage

\begin{ejercicio}{5}
Sean $(M,\calA_1)$ y $(M,\calA_2)$ dos variedades diferenciables. ¿Bajo qué condiciones
es $Id : (M,\calA_1) \to (M,\calA_2)$ una aplicación diferenciable? ¿Cuándo es un
difeomorfismo?
\end{ejercicio}
\begin{solucion}

\end{solucion}

\newpage

\begin{ejercicio}{6}
Probar que cualquier aplicación constante es diferenciable.
\end{ejercicio}
\begin{solucion}

\end{solucion}



\end{document}