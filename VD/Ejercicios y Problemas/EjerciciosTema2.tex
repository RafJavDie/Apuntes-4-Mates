	\documentclass[twoside]{article}
\usepackage{../../estilo-ejercicios}

%--------------------------------------------------------
\begin{document}

\title{Ejercicios de Variedades Diferenciables - Tema 2}
\author{Javi, Rafa, Diego}
\maketitle



\begin{ejercicio}{1}\label{1}
Sean $M$ y $N$ dos variedades diferenciables y $G \subseteq M$ un abierto.
\begin{itemize}
\item[(a)] Si $f \in \calF(G,N)$ y $H \subseteq G$ es un abierto, probar que $f|_H \in \calF(H,N)$.
\item[(b)] Si $f: G \to N$ es una aplicación y $\{G_i\}_{i\in I}$ es un recubrimiento por
abiertos de $G$ (es decir, $G = \cup_{i\in I} G_i$) tal que $f|_{G_i} \in \calF(G_i,N)$, para
todo $i \in I$, probar que $f \in \calF(G,N)$.
\end{itemize}
\end{ejercicio}
\begin{solucion}\
\begin{itemize}
\item[(a)]Como $f\in\calF(G,N)$, tenemos que dadas dos cartas $(U,\varphi)$ de $G$ y $(V,\psi)$ de $N$ tales que $U\cap f^{-1}(V)\neq\emptyset$ entonces
$$\psi\circ f\circ\varphi^{-1}:\varphi(U\cap f^{-1}(V))\to\psi f(U\cap f^{-1}(V))\in\mathcal{C}^\infty$$ 
Como $H\subseteq G$, posee estructura de subvariedad abierta, por lo que podemos tomar una carta de la forma $(U\cap H,\varphi|_H)$. Supongamos entonces que $(U\cap H)\cap f^{-1}(V)\neq\emptyset$. Tenemos que comprobar que 
$$\psi\circ f|_H\circ\varphi|_H^{-1}:\varphi|_H((U\cap H)\cap f|_H^{-1}(V))\to\psi f|_H((U\cap H)\cap f^{-1}(V))\in\mathcal{C}^\infty$$ 
Pero esta aplicación no es más que la restricción de la aplicación anterior al abierto $H$, por lo que efectivamente es diferenciable.
\item[(b)]
\end{itemize}
\end{solucion}

\newpage

\begin{ejercicio}{2}
Sean $M$ una variedad diferenciable y $G \subseteq M$ un abierto. Probar que la
aplicación inclusión $i : G \to M$ es diferenciable.
\end{ejercicio}
\begin{solucion}

\end{solucion}

\newpage

\begin{ejercicio}{3}
Probar que la composición de aplicaciones diferenciables es una aplicación
diferenciable.
\end{ejercicio}
\begin{solucion}
Sea $f \in \calF(G,N)$ y $g \in \calF(N,P)$, veamos que $g \circ f \in \calF(M,P)$.
Como $f,g$ son continuas, $g\circ f$ es continua. Para todo $h \in \calF(P)$: $ h \circ (g \circ f) = (h \circ g) \circ f$. Como $h \in \calF(P)$ y $g \in \calF(N,P)$, por el teorema 2.1.1, $h \circ g \in \calF(N)$. Entonces, de nuevo usando el teorema 2.1.1, $h \circ g \circ f \in \calF(M)$. Como $h$ era cualquiera, aplicando una última vez el teorema 2.1.1, $g \circ f \in \calF(M,P)$.
\end{solucion}

\newpage

\begin{ejercicio}{4}
Probar que la aplicación identidad entre una variedad diferenciable y ella
misma es un difeomorfismo.
\end{ejercicio}
\begin{solucion}

\end{solucion}

\newpage

\begin{ejercicio}{5}
Sean $(M,\calA_1)$ y $(M,\calA_2)$ dos variedades diferenciables. ¿Bajo qué condiciones
es $Id : (M,\calA_1) \to (M,\calA_2)$ una aplicación diferenciable? ¿Cuándo es un
difeomorfismo?
\end{ejercicio}
\begin{solucion}

\end{solucion}

\newpage

\begin{ejercicio}{6}
Probar que cualquier aplicación constante es diferenciable.
\end{ejercicio}
\begin{solucion}

\end{solucion}



\end{document}