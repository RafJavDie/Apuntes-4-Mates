\documentclass[cursovd_portada.tex]{subfiles}

\begin{document}
\chapter{Variedades Diferenciables}
\section{Introducción.}
\hs La Geometría Diferencial consiste en el estudio de aquellos problemas geo\-mé\-tri\-cos que pueden ser
tratados usando el Cálculo Diferencial e Integral. Por tanto, los objetos básicos a trabajar deben ser espacios en
los cuales nociones como la diferenciación e integración tengan sentido. Dichos espacios serán denominados {\it
Variedades Diferenciables}. En pocas palabras, una variedad diferenciable es un espacio topológico localmente
homeomorfo (en realidad, difeomorfo, pero habrá que precisar esta noción) a un espacio euclídeo. Así, los primeros
ejemplos de variedad diferenciable lo constituyen los propios espacios euclídeos. Además, las curvas regulares y
las superficies regulares son también variedades diferenciables. Recuérdese que un subconjunto $M\subseteq\R^3$ se
dice que es una superficie regular si para cualquier punto $p\in M$ existe un entorno $V$ de $p$ en $\R^3$ y una
aplicación $\vec{x}:U\subseteq\R^2\fl V\subseteq M$ de un abierto $U$ de $\R^2$ sobre $M$, tales que (a) $\vec{x}$
es un homeomorfismo diferenciable y (b) la diferencial $(D\vec{x})_q:\R^2\fl\R^3$ es inyectiva en todo punto $q\in
U$.
\par
A la aplicación $\vec{x}$ se le denomina parametrización de $M$ en $p$. La consecuencia más importante de  la
definición de superficie regular es el hecho de que el cambio de parámetros es un difeomorfismo. Por consiguiente,
una superficie regular es, intuitivamente, una reunión de abiertos de $\R^3$ organizados de tal forma que cuando
dos de tales abiertos se intersectan la transición de uno a otro se realiza de manera diferenciable. Como
consecuencia, tiene sentido hablar, en una superficie regular, de funciones diferenciables y aplicar los métodos
del Cálculo Diferencial.
\par
El defecto más importante de la definición de superficie regular es su depencia respecto de $\R^3$. La idea
natural de superficie debiera ser la de un conjunto bidimensional (en un sentido a precisar) y al que se le pueda
aplicar el Cálculo Diferencial de $\R^2$; la presencia innecesaria de $\R^3$ es, simplemente, una imposición de
nuestra naturaleza física.
\par
Aunque la necesidad de una idea abstracta de superficie (esto es,
sin involucrar ningún espacio ambiente) fue intuida por Gauss
en 1827 en su trabajo titulado \emph{Disquisitions generales
circa superficies curvas}, se hizo necesario el paso de algo
más de un siglo para que tal idea tuviera una forma
definitiva, pues la formulación explícita del concepto de
variedad diferenciable tal y como se conoce actualmente y que
contiene como caso particular a la noción de superficie
regular abstracta, no apareció hasta 1932, en el libro de
Veblen y Whitehead titulado \emph{Foundations of Differential
Geometry.} Una de las razones de esta demora estuvo en el hecho
de que el papel fundamental del cambio de parámetros no fue
bien comprendido.
\section{Cartas Locales, Atlas, Estructuras Diferenciables.}
\hs En todo lo que sigue y salvo mención explícita en
contra, se supondrá que $M$ es un espacio topológíco
$T_2$ y $2^{\underline{o}}N$\footnote{En la literatura es posible
encontrar definiciones de variedades diferenciables no $T_2$ ni
$2^{\underline{o}}N$. Para los objetivos de este curso, dichas
variedades se considerarán casos patológicos.}.
\begin{defi}
Una {\bf Carta Local de dimensión $m$} en $M$ es un par $(U,\vp)$ tal que:
\begin{enumerate}
\item $U$ es un abierto de $M$, denominado {\bf Dominio} de la carta.
\item $\vp$ es un homeomorfismo de $U$ en un abierto $\vp(U)$ de $\R^m$, llamado {\bf Aplicación Coordenada} de la
carta.
\end{enumerate}
\end{defi}
\begin{defi}
Dada una carta local $(U,\vp)$ de $M$ de dimensión $m$ y dado $p\in U$, a las coordenadas $(\lambda_1,\dots
,\lambda_m)$ de $\vp(p)\in\vp(U)\subseteq\R^m$ se les llama {\bf Coordenadas Locales} de $p$ respecto de la carta
$(U,\vp)$. Por esta razón, también se denomina a las cartas {\bf Sistemas Locales de Coordenadas} (s.l.c.).
\end{defi}
Considerando las proyecciones canónicas $u_i:\R^m\fl\R$,
$i=1,\dots ,m$ (en particular, se pueden pensar con dominio en
$\vp(U)$) y denotando por $x_i$ a la función
$x_i=u_i\circ\vp:U\fl\R$, se tiene que $\vp(p)=(\lambda_1,\dots
,\lambda_m)=(x_1(p),\dots ,x_m(p))$, pues, para cualquier
$i=1,\dots ,m$, $\lambda_i=u_i(\lambda_1,\dots
,\lambda_m)=u_i(\vp(p))=x_i(p)$.
\begin{defi}
Dada una carta local $(U,\vp)$ de $M$, las funciones $x_i$,
$i=1,\dots m$, se llaman {\bf Funciones Componentes} (o {\bf
Funciones Coordenadas}) de la carta y se escribe $\vp=(x_1,\dots
,x_m)$.
\end{defi}
\begin{ej}
{\rm
\begin{enumerate}
\item Sea $C\subset\R^3$ una curva regular alabeada (podría
hacerse del mismo modo con una curva plana) ``sin cruces", dotada
de la topología inducida por la euclídea y sea
$\a:(a,b)\subseteq\R\fl C$ una parametrización local
inyectiva. Se sabe que $\a$ es un difeomorfismo (en particular, un
homeo\-mor\-fis\-mo) del intervalo $(a,b)$ en $\a(a,b)$, que es
abierto de $C$. Llamando $\vp$ a la aplicación $\a^{-1}$ de
$\a(a,b)$ en $(a,b)$, se tiene que $(\a(a,b),\vp)$ es una carta
local de dimensión 1 en $C$. \item Sea $M$ una superficie
regular de $\R^3$ y sea $\vec{x}:U\subseteq\R^2\fl M$ una
superficie simple de $M$. Se sabe que $\vec{x}(U)$ es un abierto
de $M$ y que $\vec{x}$ es un difeomorfismo de $U$ en $\vec{x}(U)$.
Llamando $\vp$ a la aplicación $\vec{x}^{-1}$ de $\vec{x}(U)$
en $U$, se tiene que $(\vec{x}(U),\vp)$ es una carta local de
dimensión 2 en $M$. \item Sea $\R^m$ y $\vp={\rm
id}:\R^m\fl\R^m$. Entonces, $(\R^m,\vp)$ es una carta local de
dimensión $m$ de $\R^m$, con funciones coordenadas las
proyecciones canónicas $u_1,\dots ,u_m$.
\end{enumerate}}
\end{ej}
\begin{defi} Se dice que dos cartas locales de dimensión $m$ sobre $M$, $(U,\vp)$ y $(V,\psi)$ están {\bf
Relacionadas} si se verifica una de las dos condiciones siguientes:
\begin{enumerate}
\item $U\cap V=\emptyset$, ó
\item Si $U\cap V\neq\emptyset$, entonces las aplicaciones (llamadas {\bf Aplicaciones de Transición} o {\bf Aplicaciones
de Cambio} de las cartas),
$$\psi\circ\vp^{-1}:\vp(U\cap V)\fl\psi(U\cap V)$$
y
$$\vp\circ\psi^{-1}:\psi(U\cap V)\fl\vp(U\cap V)$$
son de clase $\esp{C}^{\infty}$.
\end{enumerate}
\hs Obsérvese que tanto $\vp(U\cap V)$ como $\psi(U\cap V)$ son abiertos de $\R^m$.
\end{defi}
\begin{ej}
{\rm
\begin{enumerate}
\item Probar que dos parametrizaciones locales inyectivas de una
misma curva regular dan lugar a dos cartas locales de dicha curva
que están relacionadas. \item Probar que dos superficies
simples de una misma superficie regular dan lugar a dos cartas
locales de dicha superficie que están relacionadas.
\end{enumerate}}
\end{ej}
\begin{defi}
Un {\bf Atlas} (de dimensión $m$) en $M$ es una familia de cartas locales (de dimensión $m$) sobre $M$ tales que
sus dominios recubren a $M$ y que dos a dos están relacionadas. Un atlas se dice {\bf Maximal} si no está
propiamente contenido en otro atlas.
\end{defi}
\begin{ej}
{\rm
\begin{enumerate}
\item Las parametrizaciones locales inyectivas de una curva regular forman un atlas de dimensión 1 sobre la curva y las
superficies simples de una superficie regular forman un atlas de dimensión 2 sobre la superficie.
\item $\{(\R^m,{\rm id})\}$ es un atlas de dimensión $m$ sobre $\R^m$.
\end{enumerate}}
\end{ej}
\begin{defi}
Una carta local (de dimensión $m$) se dice {\bf Admisible} en un atlas (de dimensión $m$) si está relacionada con
todas las cartas de dicho atlas.
\end{defi}
En particular, todas las cartas de un atlas son admisibles en él. Además, si $(U,\vp)$ es una carta admisible en
un atlas $\esp{A}$, entonces $\esp{A}\cup\{(U,\vp)\}$ es otro atlas.
\begin{prop}
Todo atlas está contenido en un único atlas maximal.
\end{prop}

\begin{dem}
Sean $\At$ atlas y definimos:
\[ \At^+ = \At \cup \{\text{todas las cartas admisibles en }\At \} \]
A demostrar:
\begin{enumerate}
	\item $\At^+$ atlas.
	\begin{itemize}
		\item $\At^+$ es una familia de cartas.
		\item $\At^+$ recubren la variedad.
		\item Sean $(U,φ)$, $(V,ψ) \in \At^+$:
		\begin{itemize}
			\item Si $(U,φ), (V,ψ) \in \At$, entonces están relacionadas.
			\item Si $(U,φ) \in \At$ y $(V,ψ)$ es admisible (o viceversa), $(U,φ)$ y $(V,ψ)$ están relacionadas.
			\item Supongamos que $(U,φ)$ y $(V,ψ)$ son admisibles en $\At$. Veamos que:
			\[ ψ \circ φ^{-1} : φ(U\cap V) \to ψ(U \cap V) \in C^{\infty} \]
			\[ φ \circ ψ^{-1} : ψ(U\cap V) \to φ(U \cap V) \in C^{\infty} \]
			Como $U,V \subset M$, entonces $U \cap V \subset M$, luego:
			\[ U \cap V = U \cap V \cap M = U \cap V \cap \left(\bigcup_{i \in I}U_i\right) = \bigcup_{i \in I}(U \cap V \cap U_i) \]
			donde $\At = \{(U_i,φ_i)\}_{i \in I}$. Usando que $φ$ es biyectiva: $φ(U \cap V) = \bigcup_{i \in I} φ(U \cap V \cap U_i)$, que es un abierto de $\R^m$. Bata probar que $ψ \cap φ^{-1}$ es $C^\infty$ en cada $φ(U \cap V \cap U_i)$ $\forall i \in I$.

			En $φ(U \cap V \cap U_i)$: $ψ \circ φ^{-1} = ψ \circ φ_i^{-1} \circ φ_i \circ φ^{-1}$. Asociando los términos y observando que $ψ \circ φ_i^{-1}$ y $φ_i \circ φ^{-1}$ son cambios de cartas en un atlas $C^\infty$, deducimos que $ψ \circ φ^{-1}$ es $C^\infty$ en $U \cap V$.

			Por lo tanto todas las cartas están relacionadas dos a dos.
		\end{itemize}
	\end{itemize}
	\item $\At \subseteq \At^+$. Evidente.
	\item $\At^+$ maximal. Sea $\tilde{\At}$ un atlas tal que $\At \subseteq \At^+ \subseteq \tilde{\At}$. Como todas las cartas de $\tilde{\At}$ son admisibles en $\At$, entonces $\tilde{\At} \subseteq \At^+$ por definición, luego $\tilde{\At} = \At^+$.
	\item $\At^+$ único. Sea $\At^{++}$ un atlas maximal tal que $\At \subseteq \At^{++}$. Como todas las cartas de $\At^{++}$ son admisibles en $\At$, por definición $\At^{++} \subseteq \At^+$. Por la propiedad 2, $\At^{++} = \At^+$.
\end{enumerate}
\end{dem}

\begin{defi}
Dos atlas de la misma dimensión sobre $M$ se dicen {\bf Compatibles} o {\bf Equivalentes} si su unión es otro
atlas.
\end{defi}
Como consecuencia de esta definición se tiene que dos atlas
son compatibles si y sólo si están contenidos en un mismo
atlas maximal. Además, las cartas de dos atlas compatibles
están relacionadas todas entre sí.
\begin{prop}
La relación de compatibilidad entre atlas de la misma dimensión es una relación de equivalencia.
\end{prop}

\begin{dem}\mbox{}
\begin{itemize}
	\item (Reflexiva) $\At \sim \At$ pues $\At \cup \At = \At$ atlas.
	\item (Simétrica) $\At_1 \sim \At_2$ implica que $\At_2 \sim \At_1$, pues $\At_2 \cup \At_1 = \At_1 \cup \At_2$ es un atlas.
	\item (Transitiva) Veamos que $\At_1 \sim \At_2$ y $\At_2 \sim \At_3$ implica que $\At_1 \sim \At_3$. Dado que $\At_1,\At_2$ está en un atlas maximal $\At^+$ y $\At_2,\At_3$ está en un atlas maximal $\At^{++}$, se deduce que $\At^+=\At^{++}$ por unicidad de atlas maximal que contiene a $\At_2$. Entonces $\At_1 \sim \At_3$.
\end{itemize}
\end{dem}

\begin{defi} Una {\bf Variedad Diferenciable} de dimensión $m$ es un par $(M,\esp{A})$ donde $M$ es un espacio
topológico $T_2$ y $2^{\underline{o}}N$ y $\esp{A}$ un atlas de dimensión $m$ sobre $M$. A la clase de
equivalencia por la relación anterior del atlas $\esp{A}$ (o, por abuso del lenguaje, al propio atlas $\esp{A}$)
se le llama {\bf Estructura Diferenciable} de la variedad.
\end{defi}
Cuando no haya lugar a confusión, se dirá que $M$ es la variedad diferenciable, omitiendo nombrar explícitamente
el atlas.
\par
Por otra parte y en virtud de la definición de atlas, dado un punto $p$ de una variedad diferenciable $M$, siempre
existe una carta local $(U,\vp)$ de la estructura diferenciable tal que $p\in U$. Esta carta se denomina {\bf
Carta Entorno de $p$}.

\newpage

\section{Ejercicios.}
\begin{enumerate}
\item Sean $\esp{A}$ un atlas sobre $M$ y $(U,\vp)\in\esp{A}$ una
carta local. Si $V\subseteq U$ es un abierto, probar que
$(V,\vp|_V)$ es una carta local admisible en $\esp{A}$. \item Sea
$M$ una variedad diferenciable. Probar que, dado cualquier $p\in
M$, existe una carta local $(U,\vp)$ de la estructura
diferenciable centrada en $p$ (es decir, tal que $\vp(p)=0$).
\item Sea $M$ una variedad diferenciable y $\esp{A}$ el atlas
maximal de la estructura diferenciable. Probar que los dominios de
las cartas locales de $\esp{A}$ forman base de la topología
de $M$. \item Sea $(M,\esp{A})$ una variedad diferenciable y
$G\subseteq M$ un abierto. Probar que, dando a $G$ la
topología relativa de la de $M$,
$$\esp{B}=\{(G\cap U,\vp|_{G\cap U})\}_{(U,\vp)\in\esp{A}}$$
es un atlas sobre $G$ que lo dota de estructura de variedad
diferenciable de la misma dimensión que $M$, llamada {\bf
Estructura de Subvariedad Abierta} de $G$. \item Probar que todo
espacio vectorial real $m$-dimensional admite una estructura
diferenciable canónica de dimensión $m$. \item Probar que
$\C^m$ es una variedad diferenciable de dimensión $2m$. \item
Probar que el producto cartesiano de dos variedades diferenciables
$M$ y $N$ es una variedad diferenciable de dimensión la suma
de las dimensiones de las variedades factores. De hecho, si
$(U,\vp)$ es una carta de $M$ y $(V,\psi)$ una carta de $N$, se
verifica que $(U\times V,\vp\times\psi)$ es una carta sobre
$M\times N$.
\end{enumerate}
\end{document}