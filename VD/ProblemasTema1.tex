	\documentclass[twoside]{article}
\usepackage{../estilo-ejercicios}

%--------------------------------------------------------
\begin{document}

\title{Problemas de Variedades Diferenciables - Tema 1}
\author{Javi, Rafa, Diego}
\maketitle



\begin{ejercicio}{1}\label{1}
En $\R^{n+1}-\{0\}$ se considera la relación de equivalencia:
$$x\sim y\Leftrightarrow\exists\lambda\neq 0\text{ tal que } x=\lambda y.$$
Al espacio cociente $\R^{n+1}-\{0\}/\sim$ se le denota por $\mathbb{P}^n(\R)$ y se le llama
\emph{Espacio Proyectivo Real n-dimensional}. Dotarlo de estructura de variedad diferencial.
\end{ejercicio}
\begin{solucion}
Se dota a $\mathbb{P}^n(\R)$ de la topología cociente. La aplicación cociente $\pi$ es abierta. Sea $H\subseteq\R^{n+1}-\{0\}$ abierto, entonces $\pi^{-1}(\pi(H))$ es un cono abierto que contiene a $H$. Veamos que el espacio proyectivo es $T_2$. Sean $p\neq q$ en el proyectivo. Entonces existen $x,y\in\R^{n+1}-\{0\}$ tales que $p=[x]$ y $q=[y]$, de modo que $x\not\sim y$. Sean $r\neq s$ rectas por el origen tales que $x\in r, y\in s$. Llamamos $\theta\neq 0$ al ángulo que forman las dos rectas. Construimos el cono abierto $C_1$ de eje $r$ y ángulo $\frac{\theta}{4}$ y $C_2$ de eje $s$ y el mismo ángulo. La imagen de estos conos en el cociente son abiertos disjuntos que contienen respectivamente a $p$ y $q$, y cuya preimagen es de nuevo dichos conos. 

Vamos a probar ahora que es $2º$ numerable. Sea $\mathcal{B}$ una base numerable de abiertos de $\R^{n+1}-\{0\}$. Tomamos $\mathcal{B}_\pi=\{\pi(B)\mid B\in\mathcal{B}\}$ y vamos a probar que esta es una base numerable de abiertos del proyectivo. Es trivialmente numerable, ya que como mucho hay la misma cantidad de conjuntos que en $\mathcal{B}$. También es trivial que son abiertos porque $\pi$ es abierta. Sea $p\in G$ abierto proyectivo.  Entonces $p=[x]$ para algún $x\in\pi^{-1}(G)$, que es abierto. Por tanto, existe $B\in\mathcal{B}$ tal que $x\in B\subseteq\pi^{-1}(G)$. Aplicando $\pi$, $\pi(x)\in\pi(B)\subseteq\pi(\pi^{-1}(G))=G$ (por ser sobreyectiva). 

Sea $\mathcal{A}=\{(U_i,\varphi_i)\}_{i=1,\dots,n+1}$ con $U_i=\{[(x_1,\dots,x_n)]\mid x_i\neq 0\}\subseteq\mathbb{P}^n(\R)$ y 
\begin{gather*}
\varphi_i : U_i\longrightarrow\R^n\\
[(\xn{n})]\mapsto\frac{(\xn{n})}{x_i}\text{quitarle el i-esimo}
\end{gather*}
Vamos a probar que es un atlas, así que primero veamos que sus elementos son cartas. Sea $U_i$ y veamos que es abierto. $\pi^{-1}(U_i)=\{(\xn{n})\mid x_i\neq 0\}\subseteq\R^{n+1}-\{0\}$. Se tiene que $\varphi_i(U_i)=\R^n$, pues para cada todo punto real se puede dar $x_i$ tal que $(u_1,\dots,u_n)=(x_i\frac{u_1}{x_i},\dots, x_i\frac{u_n}{x_i})=\varphi([u_1,\dots, x_i,\dots,u_n])$.

Nos queda probar que $\varphi_i$ es homeomorfismo. Antes de nada debemos probar que está bien definida, lo cual es trivial porque si tomamos otro representante, habrá un escalar que se cancelará al dividir. Veamos ahora que es biyectiva. Para ello definimos su inversa $\varphi^{-1}(u_1,\dots,u_n)=[u_1,\dots,1,\dots,u_n]$. Las dos composiciones son claramente la identidad correspondiente. A continuación probamos que $\varphi_i$ es continua. Sea $H\subseteq\R^n$ abierto. Si $\varphi^{-1}(H)$ es abierto en el proyectivo, también lo es en la topología relativa a $U_i$. $\pi^{-1}\varphi_i^{-1}(H)=(\varphi_i\circ\pi)^{-1}(H)$. La aplicación $\varphi\circ\pi$ es continua por su definición, luego dicho conjunto es abierto en $\pi^{-1}(U_i)$, que es abierto en $\R^{n+1}-\{0\}$, así que $(\varphi_i\circ\pi)^{-1}(H)$ es abierto en $\R^{n+1}-\{0\}$. Por último debemos probar que $\varphi_i^{-1}$ es continua, lo cual es trivial por su definición.

Seguidamente, $\bigcup_{i=1}^{n+1} U_i=\mathbb{P}^n(\R)$, lo cual es bastante claro. Lo último que nos queda es probar que las cartas están relacionadas dos a dos. Sean $U_i,U_j$ no disjuntos. $\varphi_i(U_i\cap U_j)=\{(u_1,\dots,u_n)\in\R^n\mid u_{j-1}\neq 0\}$ y $\varphi_j(U_i\cap U_j)=\{(u_1,\dots,u_n)\in\R^n\mid u_{i}\neq 0\}$. Veamos las composiciones. $\varphi_j\circ\varphi^{-1}_i(u_1,\dots,u_n)=\varphi_j([u_1,\dots,u_{i-1},1,u_i,\dots,u_n)])$ que es diferenciable por su definición, ya que $u_{j-1}\neq 0$ en el dominio. De forma análoga se hace con la otra composición.
\end{solucion}

\begin{ejercicio}{2}
Sea $S^n = \{(x_1,\dots, x_{n+1})\in \R^{n+1}\mid x^2_1
+\cdots+x^2_{n+1} = 1\}$ la esfera
unidad de $\R^{n+1}$.\
\begin{enumerate}
\item Probar que las proyecciones estereográficas desde los puntos de $S^n$ a $\R^n$
dotan a $S^n$ de una estructura de variedad diferenciable de dimensión $n$.
¿Cuál es el número mínimo de proyecciones necesario para obtener un atlas?
Construir uno concreto.
\item Sean $U^+_i = \{(x_1, \dots, x_{n+1}) \in S^n\mid x_i > 0\}$ y $U^-_i = \{(x_1, \dots, x_{n+1}) \in S^n\mid x_i <
0\}$ los hemisferios de $S^n (i = 1,\dots, n + 1)$ y sea la bola unidad de $\R^n$,
$B_n = f\{u_1,\dots, u_n) \in \R^n\mid u^2_1
+ \cdots + u^2_n < 1\}$. Para cada $i = 1, \dots, n + 1$ se
%consideran las aplicaciones 'i : U+
%i 􀀀! Bn y i : U􀀀
%i 􀀀! Bn dadas por
%'i(x1; : : : ; xn+1) = (x1; : : : ; xi􀀀1; xi+1; : : : ; xn+1);
%i(x1; : : : ; xn+1) = (x1; : : : ; xi􀀀1; xi+1; : : : ; xn+1)
%(proyecciones ortogonales de los hemisferios sobre la bola). Probar que A =
%f(U+
%i ; 'i); (U􀀀
%i ; i)gi=1;:::;n+1 es un atlas de dimension n en Sn que la dota
de una estructura diferenciable que es la misma que la del apartado anterior.
\end{enumerate}
\end{ejercicio}
\begin{solucion}
Sea $(U,\psi)$ una carta cualquiera de la estructura diferenciable tal que $\psi(p)=q\neq 0$, basta tomar $(U,\varphi)$ con $\varphi=\psi-q$. Se tiene que $\varphi(p)=0$ y claramente sigue siendo homeomorfismo y manteniendo las propiedades de diferenciabilidad de $\psi$, por lo que está en la estructura diferenciable.
\end{solucion}

\begin{ejercicio}{3}
Sea $M$ una variedad diferenciable y $\mathcal{A}$ el atlas maximal de la estructura
diferenciable. Probar que los dominios de las cartas locales de $\mathcal{A}$ forman
base de la topología de $M$.
\end{ejercicio}
\begin{solucion}
Sea $U$ un abierto de $M$. Sea $x\in U$. Alrededor de $x$ existe un entorno abierto $C\subseteq U$ que admite un homeomorfismo $\psi_C$ con algún entorno abierto del origen en $\R^m$ (de hecho podemos suponer por el ejercicio anterior que $\psi_C(x)=0$). Sea ahora $S$ una bola abierta centrada en el origen, de modo que $0\in S\subseteq \psi_C(B)$. Así pues, $x\in\psi_C^{-1}(S)\subseteq U$. Por tanto, para cada abierto $U$ de $M$ y para todo punto $x\in U$ hemos encontrado una carta $(\psi_C^{-1}(S),\psi_C)$ tal que $x\in\psi_C^{-1}(S)\subseteq U$, lo cual significa que las cartas forman una base para la topología de $M$.
\end{solucion}

\begin{ejercicio}{4}
Sea $(M,\mathcal{A})$ una variedad diferenciable y $G\subseteq M$ un abierto. Probar que,
dando a $G$ la topología relativa de la de $M$,
$$\mathcal{B} = \{(G \cap U, \varphi|_{G\cap U})\}_{(U,\varphi)\in\mathcal{A}}$$
es un atlas sobre $G$ que lo dota de estructura de variedad diferenciable de la
misma dimensión que $M$, llamada \textbf{Estructura de Subvariedad Abierta}
de $G$.
\end{ejercicio}
\begin{solucion}
Primero veamos que los elementos de $\mathcal{B}$ son cartas locales. Por definición son abiertos de la topología relativa, y las restricciones de homeomorfismos siguen siendo homeomorfismos. De hecho, $\varphi|_{G\cap U}(G\cap U)=\varphi(G\cap U)=\varphi(G)\cap\varphi(U)$, que es abierto del mismo $\R^m$ (conservando así la dimensión) por ser intersección finita de abiertos. Para ver que las cartas están relacionadas entre sí basta hacer un razonamiento  análogo al del ejercicio \ref{1}. Por último, para ver que recubren $G$, 
$$G=M\cap G=(\bigcup_{(U,\varphi)\in\mathcal{A}}U)\cap G= \bigcup_{(U,\varphi)\in\mathcal{A}}(U\cap G).$$
\end{solucion}

\begin{ejercicio}{5}
Probar que todo espacio vectorial real $m$-dimensional admite una estructura
diferenciable canónica de dimensión $m$.
\end{ejercicio}
\begin{solucion}
Sea $V$ un espacio vectorial real $m$-dimensional. Entonces, por álgebra lineal existe una aplicación lineal biyectiva (y por tanto homeomorfismo con la topología inicial) $f:V\to\R^m$. Basta tomar el atlas unitario $\mathcal{A}=\{(V,f)\}$ para dotar a $V$ de estructura diferenciable.  
\end{solucion}

\begin{ejercicio}{6}
Probar que $\C^m$ es una variedad diferenciable de dimensión $2m$.
\end{ejercicio}
\begin{solucion}
Basta notar que $\C^m$ es un $\R$-espacio vectorial de dimensión $2m$ y aplicar el ejercicio anterior.
\end{solucion}

\begin{ejercicio}{7}
Probar que el producto cartesiano de dos variedades diferenciables $M$ y $N$
es una variedad diferenciable de dimensión la suma de las dimensiones de
las variedades factores. De hecho, si $(U,\varphi)$ es una carta de $M$ y $(V,\psi )$ una
carta de $N$, se verifica que $(U \times V,\varphi\times\psi )$ es una carta sobre $M \times N$.
\end{ejercicio}
\begin{solucion}
Sean $\mathcal{A}_1$ un atlas de dimensión $m$ para $M$ y $\mathcal{A}_2$ un atlas de dimensión $n$ para $N$. Para probar lo que se nos pide, construimos el atlas $\mathcal{A}=\{(U\times V,\varphi\times\psi): (U,\varphi)\in\mathcal{A}_1, (V,\psi)\in\mathcal{A}_2\}$. Veamos que sus elementos son cartas locales. En primer lugar, $U\times V$ es abierto de la topología producto y $\varphi\times\psi(U\times V)=\varphi(U)\times\psi(V)$ es un abierto de $\R^n\times\R^n=\R^{n+m}$. Además, el producto cartesiano de homeomorfismos es claramente homeomorfismo. Probemos ahora que las cartas están relacionadas dos a dos. 

Sean $(U_1\times V_1,\varphi_1\times\psi_1)\in\mathcal{A}_1$ y $(U_2\times V_2,\varphi_2\times\psi_2)\in\mathcal{A}_2$. Entonces 
$$(\varphi_1\times\psi_1)\circ(\varphi_2\times\psi_2)^{-1}=(\varphi_1\times\psi_1)\circ(\varphi_2^{-1}\times\psi_2^{-1})=(\varphi_1\circ\varphi_2^{-1},\psi_1\circ\psi_2^{-1}).$$
Como cada componente es infinitamente diferenciable por hipótesis, el proucto también lo es. Finalmente, veamos que las cartas recubren $M\times N$,
$$M\times N= \bigcup_{(U,\varphi)\in\mathcal{A}_1}U\times\bigcup_{(V,\psi)\in\mathcal{A}_1}V=\bigcup_{(U\times V,\varphi\times\psi)\in\mathcal{A}}U\times V.$$
\end{solucion}

\end{document}