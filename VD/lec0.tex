\documentclass[cursovd_portada.tex]{subfiles}

\begin{document}


%\hyphenation{equi-va-len-cia}\hyphenation{pro-pie-dad}\hyphenation{res-pec-ti-va-men-te}\hyphenation{sub-es-pa-cio}
\chapter{Complementos de Topología\\ General.}
\section{Espacios Topol\'{o}gicos.}
\begin{defi}
Un {\bf Espacio Topol\'{o}gico} es un par $(X,T)$, donde $X$ es un conjunto y $T$ una familia de subconjuntos de $X$,
llamada una {\bf Topolog\'{\i}a sobre $X$} y cuyos elementos son llamados {\bf Conjuntos Abiertos}, verific\'{a}ndose las
siguientes propiedades:
\begin{enumerate}
\item El conjunto vac\'{\i}o $\emptyset$ y el propio $X$ son (conjuntos) abiertos.
\item La intersecci\'{o}n de cualquier cantidad finita de abiertos es un abierto.
\item La uni\'{o}n de cualquier cantidad de abiertos es un abierto.
\end{enumerate}
\end{defi}
\begin{defi}
Un {\bf Espacio Topol\'{o}gico} es un conjunto $X$ tal que para cada punto $x\in X$ existe una familia $\esp{N}_x$ de
subconjuntos de $X$, llamados {\bf Entornos del punto $x$}, cumpli\'{e}ndose:
\begin{enumerate}
\item Cada punto est\'{a} en todos sus entornos.
\item La intersecci\'{o}n de dos entornos de un punto es tambi\'{e}n entorno de ese punto.
\item Un conjunto que contenga a un entorno de un punto es, a su vez, entorno de dicho punto.
\item Dado un entorno $N$ de un punto $x$, existe otro entorno $M$ de $x$ tal que $N$ es entorno de todos los
puntos de $M$.
\end{enumerate}
\end{defi}
\begin{teorema}
Las dos definiciones anteriores son equivalentes.
\end{teorema}
\begin{nota}
{\rm Seg\'{u}n se elija una u otra de las definiciones anteriores, las siguientes afirmaciones se tendr\'{a}n bien como
definici\'{o}n, bien como proposici\'{o}n:
\begin{enumerate}
\item Un conjunto es entorno de un punto si y s\'{o}lo si existe un abierto conteniendo al punto y contenido en el
conjunto.
\item Un conjunto es abierto si y s\'{o}lo si es entorno de todos sus puntos.
\end{enumerate}}
\end{nota}
\begin{prop}
Un conjunto es abierto si y s\'{o}lo si para cada punto suyo existe otro abierto que contiene al punto y est\'{a}
contenido en el conjunto.
\end{prop}
\begin{defi}
Sea $(X,T)$ un espacio topol\'{o}gico. Una familia de subconjuntos de $X$, $\esp{B}\subseteq T$, se dice que es una
{\bf Base de la Topolog\'{\i}a} si todo abierto es uni\'{o}n de elementos de $\esp{B}$.
\end{defi}
\begin{teorema}
Sea $\esp{B}\subseteq T$. Las condiciones siguientes son equivalentes:
\begin{enumerate}
\item $\esp{B}$ es base de $T$.
\item Para cada abierto y para cada punto suyo, existe un elemento de $\esp{B}$ que contiene al punto y est\'{a}
contenido en el abierto.
\end{enumerate}
\end{teorema}
\begin{defi}
Sea $(X,T)$ un espacio topol\'{o}gico y $x$ un punto de $X$. Una familia $\esp{B}_x$ de entornos de $x$ se dice {\bf
Base de Entornos} de $x$ si todo entorno de $x$ contiene alg\'{u}n elemento de la familia.
\end{defi}
\begin{defi}
En un espacio topol\'{o}gico $(X,T)$, los conjuntos complementarios de los conjuntos abiertos se llaman {\bf Conjuntos
Cerrados} y la familia de los conjuntos cerrados se denota por $\esp{F}$.
\end{defi}
\begin{prop}
La familia $\esp{F}$ de los conjuntos cerrados de un espacio to\-po\-l\'{o}\-gi\-co $(X,T)$ verifica las siguientes
propiedades:
\begin{enumerate}
\item El conjunto vac\'{\i}o $\emptyset$ y el propio conjunto $X$ son (conjuntos) cerrados.
\item La uni\'{o}n de cualquier cantidad finita de cerrados es un cerrado.
\item La intersecci\'{o}n de cualquier cantidad de cerrados es un cerrado.
\end{enumerate}
\end{prop}
\begin{teorema}
Sea $X$ un conjunto y $\esp{F}$ una familia de subconjuntos de $X$ que verifican las mismas propiedades que una
familia de conjuntos cerrados, recogidas en la proposici\'{o}n anterior. Entonces, existe una \'{u}nica topolog\'{\i}a sobre
$X$, definida por
$$T=\{G\subseteq X/X-G\in\esp{F}\},$$
para la que $\esp{F}$ es la familia de cerrados.
\end{teorema}
A continuaci\'{o}n, se van a presentar un resultado que permite construir una topolog\'{\i}a sobre un conjunto $X$ a partir
de cualquier familia de sus subconjuntos.
\begin{teorema}\label{teorema}
Sea $X$ un conjunto y sea $\esp{A}$ una familia de subconjuntos de $X$. Entonces la familia formada por el
conjunto vac\'{\i}o $\emptyset$, el propio conjunto $X$ y todas las uniones posibles que se puedan realizar con todas
las intersecciones finitas de elementos de $\esp{A}$ es una topolog\'{\i}a sobre $X$, que contiene a $\esp{A}$ y que,
adem\'{a}s, es la topolog\'{\i}a m\'{a}s peque\~{n}a de las que contienen a $\esp{A}$.
\end{teorema}
\begin{ejer}
\begin{enumerate}
{\rm
\item Escr\'{\i}banse con terminolog\'{\i}a matem\'{a}tica todas las definiciones y resultados anteriores. Ser\'{\i}a tambi\'{e}n
conveniente probar dichos resultados.
\item ?`Bajo qu\'{e} condiciones una familia $\esp{A}$ de subconjuntos de un conjunto $X$ es base de la topolog\'{\i}a que
se obtiene seg\'{u}n el proceso descrito en el Teorema \ref{teorema} (partiendo de la propia $\esp{A}$)?
\item Sea $X$ un conjunto y $d$ una m\'{e}trica sobre \'{e}l. Probar que existe una topolog\'{\i}a sobre $X$ para la que las
bolas abiertas de $d$ forman base. Dicha topolog\'{\i}a se llama {\bf Topolog\'{\i}a M\'{e}trica asociada a $d$}.}
\end{enumerate}
\end{ejer}
\begin{ejer}
\begin{enumerate}
{\rm
\item Sea $X$ un conjunto cualquiera y $T_{dis}=\esp{P}(X)$. $T_{dis}$ es una topolog\'{\i}a sobre $X$, llamada {\bf
Topolog\'{\i}a Discreta} y es la topolog\'{\i}a mayor, es decir, con m\'{a}s elementos, que puede construirse sobre $X$. Adem\'{a}s,
la familia formada por el conjunto vac\'{\i}o $\emptyset$ y todos los conjuntos unitarios es base de dicha topolog\'{\i}a.
\item En el conjunto de los n\'{u}meros reales $\R$, consid\'{e}rese la familia formada por todos los intervalos abiertos
y constr\'{u}yase la menor topolog\'{\i}a que contiene a tal familia, siguiendo el proceso descrito en el Teorema \ref{teorema}.
Dicha topolog\'{\i}a se llama {\bf Topolog\'{\i}a Eucl\'{\i}dea de $\R$}, es la topolog\'{\i}a m\'{e}trica asociada a la m\'{e}trica eucl\'{\i}dea
y tiene a la familia de todos los intervalos abiertos como base.
\item En $\R^m$ consid\'{e}rese la topolog\'{\i}a m\'{e}trica asociada a la m\'{e}trica eucl\'{\i}dea. Dicha topolog\'{\i}a se llama {\bf
Topolog\'{\i}a Eucl\'{\i}dea de $\R^m$} y las bolas abiertas (por ejemplo, en $\R^2$ son los discos abiertos) forman una
base de ella.}
\end{enumerate}
\end{ejer}
\begin{defi}
Sea $(X,T)$ un espacio topol\'{o}gico, $A\subseteq X$ y $x\in X$. Se dice que:
\begin{enumerate}
\item $x$ es un {\bf Punto Adherente} a $A$ si todo abierto que contenga a $x$ corta a $A$. Al conjunto de los
puntos adherentes a $A$ se le llama la {\bf Clausura} de $A$ y se denota por $\overline{A}$.
\item $x$ es un {\bf Punto de Acumulaci\'{o}n} de $A$ si todo abierto que contenga a $x$ corta a $A$ en alg\'{u}n punto distinto
de $x$. Al conjunto de los puntos de acumulaci\'{o}n de $A$ se le llama el {\bf Derivado} de $A$ y se denota por $A'$.
\item $x$ es un {\bf Punto Interior} de $A$ si hay alg\'{u}n abierto que contenga a $x$ contenido en $A$. Al conjunto de los
puntos interiores de $A$ se le llama el {\bf Interior} de $A$ y se denota por ${\rm int}(A)$.
\end{enumerate}
\end{defi}

\newpage

\begin{prop}
Sea $(X,T)$ un espacio topol\'{o}gico. Entonces:
\begin{enumerate}
\item La clausura de un conjunto es el menor cerrado que lo contiene. Por tanto, un conjunto es cerrado si y s\'{o}lo
si coincide con su clausura.
\item El interior de un conjunto es el mayo abierto contenido en \'{e}l. Por tanto, un conjunto es abierto si y s\'{o}lo
si coincide con su interior.
\end{enumerate}
\end{prop}
\begin{defi}
Sea $(X,T)$ un espacio topol\'{o}gico y $A\subseteq X$. Se llama {\bf To\-po\-lo\-g\'{\i}a Relativa} o {\bf Topolog\'{\i}a
Inducida} de $X$ a $A$ a:
$$T_A=\{G\cap A/G\in T\}.$$
\hs Al par $(A,T_A)$ se le llama {\bf Subespacio Topol\'{o}gico} de $(X,T)$. Una propiedad se dice que es {\bf
Hereditaria} para una clase de subconjuntos de $X$ si la verica el espacio topol\'{o}gico $(X,T)$ y cualquier
subespacio $(A,T_A)$, con $A$ perteneciendo a la clase especificada.
\end{defi}
\begin{ejer}
{\rm Probar que, efectivamente, $T_A$ es una topolog\'{\i}a sobre $A$.}
\end{ejer}
\section{Aplicaciones entre Espacios Topol\'{o}gicos. Ho\-meo\-mor\-fis\-mos.}
\begin{defi}
Una aplicaci\'{o}n entre dos espacios topol\'{o}gicos $f:(X,T_X)\fl (Y,T_Y)$ se dice {\bf Continua en un punto $x\in X$}
si para cualquier abierto $G\in T_Y$ entorno de $f(x)$, se tiene que $f^{-1}(G)$ es entorno de $x$ y se dice {\bf
Continua} si lo es en todo punto de $X$.
\end{defi}
\begin{teorema}
Sea $f:(X,T_X)\fl (Y,T_Y)$ una aplicaci\'{o}n. Las condiciones siguientes son equivalentes:
\begin{enumerate}
\item $f$ es continua.
\item La anti--imagen por $f$ de cualquier abierto de $Y$ es abierto de $X$.
\item La anti--imagen por $f$ de cualquier cerrado de $Y$ es cerrado de $X$.
\end{enumerate}
\end{teorema}
\begin{defi}
Una aplicaci\'{o}n $f:(X,T_X)\fl (Y,T_Y)$ se dice {\bf Abierta} (respectivamente, {\bf Cerrada})) si la imagen por $f$
de cualquier abierto de $X$ (respectivamente, de cualquier cerrado) es un abierto de $Y$ (respectivamente, un
cerrado).
\end{defi}
\begin{defi} Una aplicaci\'{o}n $f:(X,T_X)\fl (Y,T_Y)$ se dice que es un {\bf Homeomorfismo} si es biyectiva, continua
y su inversa es tambi\'{e}n continua.
\end{defi}
\begin{teorema}
Sea $f:(X,T_X)\fl (Y,T_Y)$ una aplicaci\'{o}n biyectiva. Las condiciones siguientes son equivalentes:
\begin{enumerate}
\item $f$ es un homeomorfismo.
\item $f$ es continua y abierta.
\item $f$ es continua y cerrada.
\end{enumerate}
\end{teorema}
\begin{defi} Una propiedad se dice {\bf Propiedad Topol\'{o}gica} o {\bf Invariante Topol\'{o}gico}, si se conserva por
homeomorfismos, es decir, si de verificarla un espacio topol\'{o}gico la verifican todos los espacios topol\'{o}gicos
homeomorfos a \'{e}l.
\end{defi}
\begin{defi}
Una aplicaci\'{o}n $f(X,T_X)\fl (Y,T_Y)$ continua se dice que es un {\bf Homeomorfismo Local} si para todo punto $x\in
X$, existe un entorno abierto $U_x\in T_X$ de $x$ tal que $f(U_x)$ es abierto en $Y$ y $f|_{U_x}:U_x\fl f(U_x)$ es
un homeomorfismo.
\end{defi}
\begin{prop}
Todo homeomorfismo local es una aplicaci\'{o}n abierta.
\end{prop}
\section{Construcci\'{o}n de Topolog\'{\i}as mediante aplicaciones. Topolog\'{\i}as Producto y Cociente.}
\begin{prop}
Sean $X$ un conjunto, $\{(Y_i,T_i)\}_{i\in I}$ una familia de espacios topol\'{o}gicos y $\{f_i:X\fl Y_i/i\in I\}$ una
familia de aplicaciones. Entonces, existe la menor topolog\'{\i}a $T$ sobre $X$ que hace continuas a todas las
aplicaciones $f_i$, llamada {\bf Topolog\'{\i}a Inicial} de las $f_i$ y que es la generada, siguiendo el proceso
descrito en el Teorema \ref{teorema}, por la familia $\esp{A}$ de subconjuntos de $X$:
$$\esp{A}=\bigcup_{i\in I}\{f_i^{-1}(G)/G\in T_i\}.$$
\end{prop}
\begin{coro}
Sean $X$ un conjunto, $(Y,T_Y)$ un espacio topol\'{o}gico y $f:X\fl (Y,T_Y)$ una aplicaci\'{o}n. Entonces,
$T=\{f^{-1}(G)/G\in T_Y\}$ es la menor topolog\'{\i}a sobre $X$ que hace continua a la aplicaci\'{o}n $f$ y se llama {\bf
Topolog\'{\i}a Inicial} de $f$. Adem\'{a}s, si $f$ es biyectiva, la topolog\'{\i}a inicial la convierte en un homeomorfismo.
\end{coro}
\begin{defi}
Dada una familia de espacio topol\'{o}gicos $\{(X_i,T_i)\}_{i\in I}$ y dado su producto cartesiano $X=\prod_{i\in
I}X_i$, a la topolog\'{\i}a inicial sobre $X$ de las proyecciones $\pi_i:X\fl X_i$ se le llama la {\bf Topolog\'{\i}a
Producto} de las $T_i$ y se denota por $T_{\prod}$ y al espacio topol\'{o}gico $(X,T_{\prod})$ se le llama {\bf
Espacio Producto} .
\end{defi}
\begin{nota}
{\rm En el caso de un n\'{u}mero finito de factores, una base de la topolog\'{\i}a producto est\'{a} formada por los productos
de abiertos de cada uno de los factores. Adem\'{a}s, tambi\'{e}n en este caso, la topolog\'{\i}a relativa de una producto es la
topolog\'{\i}a producto de las relativas de cada uno de los factores.}
\end{nota}
\begin{prop}
Sea $\{(X_i,T_i)\}_{i=1,\dots ,m}$ una familia finita de espacios to\-po\-l\'{o}\-gi\-cos y sea $X=\prod_{i=1}^mX_i$,
dotado de la topolog\'{\i}a producto. Dado otro espacio topol\'{o}gico $(Y,T_Y)$, una aplicaci\'{o}n $f:(Y,T_Y)\fl
(X,T_{\prod})$ es continua si y s\'{o}lo si $\pi_i\circ f:(Y,T_Y)\fl (X_i,T_i)$ es continua, para todo $i=1,\dots ,m$.
\end{prop}
\begin{prop}
Sean $X$ un conjunto, $\{(Y_i,T_i)\}_{i\in I}$ una familia de espacios topol\'{o}gicos y $\{f_i:Y_i\fl X/i\in I\}$ una
familia de aplicaciones. Entonces, existe la mayor topolog\'{\i}a $T$ sobre $X$ que hace continuas a todas las
aplicaciones $f_i$, llamada {\bf Topolog\'{\i}a Final} de las $f_i$.
\end{prop}
\begin{coro}
Sean $X$ un conjunto, $(Y,T_Y)$ un espacio topol\'{o}gico y $f:(Y,T_Y)\fl X$ una aplicaci\'{o}n. Entonces, $T=\{G\subseteq
X/ f^{-1}(G)\in T_Y\}$ es la mayor topolog\'{\i}a sobre $X$ que hace continua a la aplicaci\'{o}n $f$ y se llama {\bf
Topolog\'{\i}a Final} de $f$. Adem\'{a}s, si $f$ es biyectiva, la topolog\'{\i}a final la convierte en un homeomorfismo.
\end{coro}
\begin{defi}
Sean $(X,T)$ un espacio topol\'{o}gico y $\esp{R}$ una relaci\'{o}n de equivalencia sobre $X$. Si $\pi:X\fl X/\esp{R}$ es
la proyecci\'{o}n can\'{o}nica, entonces la topolog\'{\i}a final de $\pi$ sobre el espacio cociente $X/\esp{R}$, denotada por
$T_{\esp{R}}$ se llama {\bf Topolog\'{\i}a Cociente} y al espacio topol\'{o}gico $(X/\esp{R},T_{\esp{R}})$ se le llama {\bf
Espacio Cociente} sobre $X$ por la relaci\'{o}n $\esp{R}$.
\end{defi}
\begin{prop}
Sean (X,T) un espacio topol\'{o}gico, $(X/\esp{R},T_{\esp{R}})$ un espacio cociente sobre $X$, $(Y,T_Y)$ otro espacio
topol\'{o}gico y $f:(X/\esp{R},T_{\esp{R}})\fl (Y,T_Y)$ una aplicaci\'{o}n. Entonces, $f$ es continua si y s\'{o}lo si
$f\circ\pi:(X,T)\fl (Y,T_Y)$ es continua.
\end{prop}
\section{Axiomas de Separaci\'{o}n.}
\begin{defi}
Sea $(X,T)$ un espacio topol\'{o}gico. Se dice que es:
\begin{enumerate}
\item $T_1$ si todo par de puntos distintos de $X$ se puede separar por conjuntos abiertos, es decir, si para cada
uno de los puntos existe un abierto que lo contiene y no contiene al otro punto.
\item $T_2$ o de {\bf Haussdorf} si todo par de puntos distintos de $X$ se puede separar por abiertos dijuntos.
\item {\bf Regular} si todo conjunto cerrado y todo punto que no pertenezca a \'{e}l se pueden separar por abiertos
disjuntos.
\item $T_3$ si es regular y $T_1$.
\item {\bf Normal} si todo par de cerrados disjuntos se puede separar por abiertos disjuntos.
\item $T_4$ si es normal y $T_1$.
\end{enumerate}
\end{defi}
\begin{prop}
\begin{enumerate}
\item El axioma $T_i$ implica el axioma $T_j$, para todo $i>j$.
\item Los axiomas de separaci\'{o}n son propiedades topol\'{o}gicas.
\item Los axiomas $T_1$ y $T_2$ son propiedades hereditarias para todo subespacio.
\item Un espacio topol\'{o}gico es $T_1$ si y s\'{o}lo si todo conjunto unitario es cerrado.
\item Un espacio topol\'{o}gico es regular si y s\'{o}lo si para todo abierto $G$ y para todo punto $x\in G$,
existe otro abierto $U$ tal que $x\in U\subseteq\overline{U}\subseteq G$.
\item Un espacio topol\'{o}gico es normal si y s\'{o}lo si para todo abierto $G$ y para todo cerrado $F$ que lo contenga,
existe otro abierto $U$ tal que $G\subseteq U\subseteq\overline{U}\subseteq F$.
\end{enumerate}
\end{prop}
\begin{teorema}
{\bf (Lema de Uryshon).} Un espacio topol\'{o}gico $(X,T)$ es normal si y s\'{o}lo si para cada par de cerrados disjuntos
$F_1$ y $F_2$ de $X$, existe una aplicaci\'{o}n continua $f:(X,T)\fl [0,1]$ (con la topolog\'{\i}a eucl\'{\i}dea) tal que
$f(F_1)=0$ y $f(F_2)=1$.
\end{teorema}
\section{Axiomas de Numerabilidad.}
\begin{defi}
Sea $(X,T)$ un espacio topol\'{o}gico. Un subconjunto $D$ de $X$ se dice {\bf Denso} si $\overline{D}=X$.
\end{defi}
\begin{defi}
Se dice que un espacio topol\'{o}gico $(X,T)$ es:
\begin{enumerate}
\item {\bf Primero Numerable} ($1^{\underline{o}}N$) si todo punto de $X$ tiene una base de entornos numerable.
\item {\bf Segundo Numerable} ($2^{\underline{o}}N$) si $T$ tiene una base numerable.
\item De {\bf Lindeloff} si todo recubrimiento de $X$ por abiertos admite un subrecubrimiento numerable.
\item {\bf Separable} si existe un subconjunto de $X$ denso y numerable.
\end{enumerate}
\end{defi}
\begin{prop}
\begin{enumerate}
\item Los axiomas de numerabilidad son propiedades to\-po\-l\'{o}\-gi\-cas.
\item El axioma $2^{\underline{o}}N$ implica a todos los dem\'{a}s.
\item Los axiomas $1^{\underline{o}}N$ y $2^{\underline{o}}N$ son propiedades hereditarias para cualquier subespacio.
\item Todo espacio topol\'{o}gico regular y de Lindeloff es normal.
\end{enumerate}
\end{prop}
\section{Compacidad.}
\begin{defi}
Un espacio topol\'{o}gico $(X,T)$ se dice {\bf Compacto} si de todo recubrimiento de $X$ por abiertos se puede extraer
un subrecubrimiento finito.
\end{defi}
\begin{prop}
\begin{enumerate}
\item La compacidad se coserva por aplicaciones continuas. En consecuencia, es una propiedad topol\'{o}gica.
\item La compacidad es una propiedad hereditaria para subespacios cerrados.
\item En un espacio topol\'{o}gico $T_2$, todo subespacio compacto es cerrado.
\item Toda aplicaci\'{o}n continua y biyectiva de un espacio topol\'{o}gico compacto en un espacio topol\'{o}gico $T_2$ es un
homeomorfismo.
\item Todo espacio topol\'{o}gico $T_2$ y compacto es $T_4$.
\end{enumerate}
\end{prop}
\begin{defi}
Un espacio topol\'{o}gico $(X,T)$ se dice {\bf Localmente Compacto} si todo punto de $X$ tiene una base de entornos
formada por conjuntos compactos.
\end{defi}
\begin{teorema}
Un espacio topol\'{o}gico $T_2$ es localmente compacto si y s\'{o}lo si todo punto tiene un entorno compacto.
\end{teorema}
\begin{prop}
\begin{enumerate}
\item La compacidad local es una propiedad topol\'{o}gica.
\item La compacidad local es una propiedad hereditaria para subespacios abiertos y para subespacios cerrados.
\item Todo espacio $T_2$ y localmente compacto es $T_3$.
\end{enumerate}
\end{prop}
\begin{prop}
Sea $(X,T)$ un espacio topol\'{o}gico $T_2$ y localmente compacto. Entonces, para todo abierto $G$ y para todo punto
$p\in G$, existe otro abierto $H$ de clausura compacta y tal que $p\in H\subseteq\overline{H}\subseteq G$.
\end{prop}

\newpage

\section{Conexi\'{o}n.}
\begin{defi}
Un espacio topol\'{o}gico $(X,T)$ se dice {\bf Conexo} si no existen dos subconjuntos propios abiertos (o cerrados) de
$X$, disjuntos y que recubran a $X$.
\end{defi}
\begin{teorema}
Sea $(X,T)$ un espacio topol\'{o}gico. Las condiciones siguientes son equivalentes:
\begin{enumerate}
\item $(X,T)$ es conexo.
\item Los \'{u}nicos subconjuntos de $X$ que son a la vez abiertos y cerrados son el conjunto vac\'{\i}o $\emptyset$ y el
propio $X$.
\item Toda aplicaci\'{o}n continua de $(X,T)$ en $\R$ que tome dos valores, toma todos los valores intermedios.
\end{enumerate}
\end{teorema}
\begin{prop}
La conexi\'{o}n no es, en general, una propiedad hereditaria. Adem\'{a}s, se conserva por aplicaciones continuas, con lo
que, en consecuencia, es una propiedad topol\'{o}gica.
\end{prop}
\begin{defi}
En un espacio topol\'{o}gico $(X,T)$, se llama {\bf Componente Conexa} del punto $x\in X$ al mayor subconjunto conexo
de $X$ que contenga a $x$.
\end{defi}
\begin{prop}
Las componentes conexas de un espacio topol\'{o}gico $(X,T)$ son subconjuntos cerrados y forman una partici\'{o}n de $X$.
Adem\'{a}s, el n\'{u}mero de componentes conexas es una propiedad topol\'{o}gica.
\end{prop}
\begin{defi}
Sea $(X,T)$ un espacio topol\'{o}gico y $x\in X$. Se llama {\bf Orden de Conexi\'{o}n} de $x$ en $X$ al n\'{u}mero de
componentes conexas de $C_x-\{x\}$, donde $C_x$ denota la componente conexa de $x$ en $X$.
\end{defi}
\begin{prop}
Sea $f:(X,T_X)\fl (Y,T_Y)$ un homeomorfismo. Entonces, para todo $x\in X$, $x$ y $f(x)$ tiene  el mismo orden de
conexi\'{o}n. En consecuencia, el orden de conexi\'{o}n de los puntos es una propieda topol\'{o}gica.
\end{prop}
\begin{defi}
Un espacio topol\'{o}gico $(X,T)$ se dice {\bf Localmente Conexo} si todo punto de $X$ tiene una base de entornos
formada por conjuntos conexos.
\end{defi}
\begin{prop}
La conexi\'{o}n local es una propiedad topol\'{o}gica y hereditaria para abiertos. Adem\'{a}s, las componentes conexas de un
espacio topol\'{o}gico localmente conexo son tambi\'{e}n conjuntos abiertos.
\end{prop}
\begin{defi}
Un {\bf Arco} en un espacio topol\'{o}gico $(X,T)$ es la imagen por un homeomorfismo $f$ de $[0,1]$ (con la topolog\'{\i}a
eucl\'{\i}dea) en $(X,T)$. Dados dos puntos $x,y\in X$, se llama {\bf Arco desde $x$ hasta $y$} a un arco en $X$ tal
que $f(0)=x$ y $f(1)=y$.
\end{defi}
\begin{nota}
{\rm Obs\'{e}rvese que los arcos son, por definici\'{o}n, subconjuntos conexos de $X$. Por otra parte si la aplicaci\'{o}n $f$
es s\'{o}lo continua (y no un homeomorfismo), surge la noci\'{o}n de {\bf Camino} en $X$ y todas las definiciones y
resultados tienen su versi\'{o}n paralela.}
\end{nota}
\begin{defi}
Un espacio topol\'{o}gico $(X,T)$ se dice {\bf Conexo por Arcos} o {\bf Arcoconexo}, si dados dos puntos cualesquiera
de $X$, existe un arco en $X$ desde uno de los puntos hasta el otro.
\end{defi}
\begin{prop}
\begin{enumerate}
\item La conexi\'{o}n por arcos es una propiedad topol\'{o}gica.
\item Todo espacio topol\'{o}gico conexo por arcos es conexo.
\end{enumerate}
\end{prop}
\end{document}