\documentclass[twoside]{report}
\usepackage{../estilo-apuntes}
\addto\captionsspanish{\renewcommand{\chaptername}{Lección}}
\setcounter{chapter}{-1}
\def\a{\alpha}
\def\b{\beta}
\def\om{\omega}
\def\fl{\longrightarrow}
\def\vp{\varphi}
\def\hs{\hspace*{1.5 em}}
\def\dif{{\rm d}}
\newcommand {\At} {{\mathcal{A}}}

\newcommand{\de}[1]{{\rm d} #1}
\newcommand{\dep}[1]{\displaystyle{\frac{\partial}{\partial #1}}}
\newcommand{\deri}[1]{\displaystyle{\frac{{\rm d}}{{\rm d} #1}}}
\newcommand{\dderi}[2]{\displaystyle{\frac{{\rm d} #1}{{\rm d} #2}}}
\newcommand{\esp}[1]{{\cal #1}}
\newcommand{\ddep}[2]{\displaystyle{\frac{\partial #1}{\partial #2}}}
\newtheorem{teoap}{Teorema}
\newtheorem{propoap}{Proposición}
\newtheorem{lemaap}{Lema}
\newtheorem{coroap}{Corolario}
\newtheorem{defiap}{Definición}

\usepackage{pb-diagram}

\rhead{Variedades Diferenciables (Grado en Matemáticas)}
\lhead{Curso 2017/2018}


\begin{document}
\hyphenation{di-fe-ren-cia-ble}
\hyphenation{pro-pie-da-des}\hyphenation{re-gu-lar}
\hyphenation{Va-rie-dad}\hyphenation{si-guien-tes}\hyphenation{me-dian-te}\hyphenation{de-sa-rro-llo}
\hyphenation{ellas}
\leftmargini 0.2 in

\thispagestyle{empty}

\begin{center}
{\Large \bf RESUMEN DE LA ASIGNATURA}
\end{center}
\begin{center}
{\LARGE \bf VARIEDADES DIFERENCIABLES}
\end{center}
\begin{center}
{\Large \bf Curso 2017-2018}
\end{center}

\author{Alfonso Carriazo Rubio\\ Javier Aguilar Martín\\ Rafael González López\\ Diego Pedraza López}
	
\vspace{1.5cm}

\begin{center}
\includegraphics{ambigrama_big.jpg}
\end{center}

\noindent Ilustración: ambigrama de simetría central,
creado por Manuel Jesús Pérez García (noviembre de 2010).

\noindent Resumen realizado por: Alfonso Carriazo Rubio, Javier Aguilar Martín, Rafael González López y Diego Pedraza López.

\vspace{1.5cm}

\begin{center}
\includegraphics{sellous4.jpg}
\end{center}

\begin{center}
{\large \bf Departamento de Geometría y Topología}
\end{center}
\begin{center}
{\large \bf UNIVERSIDAD DE SEVILLA}
\end{center}

\setcounter{page}{0}

\tableofcontents


\subfile{lec0}
\subfile{lec1}
\subfile{lec2}
\subfile{lec3}
\subfile{lec4}
\subfile{lec5}
\subfile{lec6}
\subfile{apea}
\subfile{apeb}
\subfile{apec}

%\begin{thebibliography}{99}
%\addcontentsline{toc}{chapter}{Bibliografía}
%\bibitem{manual} {\it Manual for Authors of Mathematical Papers}, Bull. Am.
%Math. Soc, 68 (1962), 429--444.
%\end{thebibliography}
\end{document}
