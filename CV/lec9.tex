\documentclass[CV.tex]{subfiles}

\begin{document}


%\hyphenation{equi-va-len-cia}\hyphenation{pro-pie-dad}\hyphenation{res-pec-ti-va-men-te}\hyphenation{sub-es-pa-cio}

\chapter{Integración en variedades}






\section{Definición de la integral}

Sea $M$ una $n$-variedad diferenciable orientable. Se va a definir una integral
\[
\int_M:\Omega_c^n(M)\to\R
\]
el espacio vectorial de $n$-formas con soporte compacto. En el caso especial de que $M=\R^n$ (con la orientación estándar), podemos escribir $w\in\Omega_c^n(M)$ de forma única como $w=f(x)dx_1\land\dots\land dx_n$, donde $f\in C^{\infty}(\R^n;\R)$ tiene soporte compacto. Definimos entonces
\[
\int_{\R^n}f(x)dx_1\land\dots\land dx_n=\int_{\R^n} f(x)d\mu_n,
\]
donde $d\mu_n$ es la medida de Lebesgue usual de $\R^n$. La misma definición puede ser usada cuando $w\in\Omega_c^n(V)$ para un abierto $V\subseteq\R^n$, pues $w$ y $f$ son extensibles con diferenciabilidad a todo $\R^n$ haciendo que sean 0 en $\R^n-supp_V(w)$. 

\begin{lemma}
Sea $\phi:V\to W$ entre abiertos de $\R^n$ y supongamos que $\det(D_x\phi)$ tiene signo constante $\delta=\pm 1$ para todo $x\in V$. Para $w\in\Omega_c^n(W)$ tenemos que
\[
\int_V\phi^*(w)=\delta\int_Ww.
\]
\end{lemma}
\begin{proof}
Si $w=f(x)dx_1\land\dots\land dx_n$ se sigue que
\begin{align*}
\phi^*(w)=f(\phi(x))\det(D_x\phi)dx_1\land\dots\land dx_n=\delta f(\phi(x))|\det(D_x\phi)|dx_1\land\dots\land dx_n.
\end{align*}
El resultado se sigue entonces del teorema de reparametrización de integrales que afirma que
\[
\int_W f(x)d\mu_n=\int_V f(\phi(x))|\det(D_x\phi)|d\mu_n.
\]
\end{proof}

\begin{prop}
Para una $n$-variedad diferenciable orientable cualquiera $M$, existe un único operador lineal 
\[
\int_M:\Omega_c^n(M)\to\R
\]
con la propiedad de que si $w\in\Omega_c^n(\R)$ tiene soporte contenido en $U$, donde $(U,h)$ es una carta positivamente orientada, entonces
\[
\int_M w=\int_{h(U)}(h^{-1})^*w.
\]
\end{prop}
\begin{dem}
\QED
\end{dem}
\begin{lemma}
Sea $M$ una variedad diferenciable orientable de dimensión $n$.
\begin{enumerate}
\item  $\int_M w$ cambia de signo cuando se recierte la orientación de $M$.
\item Si $w\in \Omega_c^n(M)$ tiene soporte contenido en un abierto $W\subseteq M$, entonces
\[
\int_M w=\int_W w.
\]
\item Si $\phi:M\to N$ es un difeomorfismo que preserva la orientación, entonces se tiene que
\[
\int_M w=\int_N\phi^*(w)
\]
para $w\in\Omega_c^n(M)$. 
\end{enumerate}
\end{lemma}
\begin{proof}

\end{proof}
\end{document}
