\documentclass[twoside, 11pt]{report}
\usepackage{../estilo-apuntes}
\addto\captionsspanish{\renewcommand{\chaptername}{Lección}}
\setcounter{chapter}{0}
\def\a{\alpha}
\def\b{\beta}
\def\om{\omega}
\def\fl{\longrightarrow}
\def\vp{\varphi}
\def\hs{\hspace*{1.5 em}}
\def\dif{{\rm d}}
\newcommand {\At} {{\mathcal{A}}}

\newcommand{\de}[1]{{\rm d} #1}
\newcommand{\dep}[1]{\displaystyle{\frac{\partial}{\partial #1}}}
\newcommand{\deri}[1]{\displaystyle{\frac{{\rm d}}{{\rm d} #1}}}
\newcommand{\dderi}[2]{\displaystyle{\frac{{\rm d} #1}{{\rm d} #2}}}
\newcommand{\esp}[1]{{\cal #1}}
\newcommand{\ddep}[2]{\displaystyle{\frac{\partial #1}{\partial #2}}}


%\usepackage{pb-diagram}

\rhead{Cálculo en Variedades (Grado en Matemáticas)}
\lhead{Curso 2017/2018}


\usepackage{graphicx}
\begin{document}
\hyphenation{di-fe-ren-cia-ble}
\hyphenation{pro-pie-da-des}\hyphenation{re-gu-lar}
\hyphenation{Va-rie-dad}\hyphenation{si-guien-tes}\hyphenation{me-dian-te}\hyphenation{de-sa-rro-llo}
\hyphenation{ellas}
\leftmargini 0.2 in

\thispagestyle{empty}

\begin{titlepage}
	\centering
	{\huge\bfseries Cálculo en Variedades \par}
	\vspace{2cm}
	{\Large Javier Aguilar Martín\par}
	\vspace{2.5cm}
	\vfill
	Esta obra está licenciada bajo la Licencia Creative Commons Atribución 3.0 España. Para ver una copia de esta licencia, visite \url{http://creativecommons.org/licenses/by/3.0/es/} o envíe una carta a Creative Commons, PO Box 1866, Mountain View, CA 94042, USA.

Gran parte de esta obra es una traducción de los 10 primeros capítulos del libro \emph{From Calculus To Cohomology} de Madsen y Tornehave, que puede encontrarse en el siguiente enlace: \url{https://archive.org/details/MadsenI.H.TornehaveJ.FromCalculusToCohomologyDeRhamCohomologyAndCharacteristicClasses1996}

	{\large \today\par}
\end{titlepage}
	


\begin{center}
%\includegraphics{sellous4.jpg}
\end{center}



\setcounter{page}{0}

\tableofcontents



\subfile{lec1}
\subfile{lec2}
\subfile{lec3}
\subfile{lec4}
\subfile{lec5}
\subfile{lec6}
\subfile{lec7}
\subfile{lec8}
\subfile{lec9}

%\begin{thebibliography}{99}
%\addcontentsline{toc}{chapter}{Bibliografía}
%\bibitem{manual} {\it Manual for Authors of Mathematical Papers}, Bull. Am.
%Math. Soc, 68 (1962), 429--444.
%\end{thebibliography}
\end{document}
