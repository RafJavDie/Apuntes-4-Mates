\documentclass[CV.tex]{subfiles}

\begin{document}


%\hyphenation{equi-va-len-cia}\hyphenation{pro-pie-dad}\hyphenation{res-pec-ti-va-men-te}\hyphenation{sub-es-pa-cio}

\chapter{La sucesión de Mayer-Vietoris}

\section{Definición de la sucesión de Mayer-Vietoris}
Aquí desarrollaremos la llamada sucesión exacta larga (de cohomología) de Mayer-Vietoris que nos permitirá calcular la cohomología de de Rham de abiertos euclídeos.

\begin{teorema}
Sean $U_1,U_2\subseteq\R^n$ abiertos y $U=U_1\cup U_2$. Sean $i_{\nu}:U_{\nu}\to U$ y $j_{\nu}:U_1\cap U_2\to U_{\nu}$ ($\nu=1,2$) las correspondientes inclusiones. Entonces, la sucesión
\[
0\to \Omega^p(U)\overset{I^p}{\to}\Omega^p(U_1)\oplus\Omega^p(U_2)\overset{J^p}{\to}\Omega^p(U_1\cap U_2)\to 0
\]
es exacta, donde $I^p(w)=(i_1^*(w), i_2^*(w))$ y $J^p(w_1,w_2)=j_1^*(w_1)-j_2^*(w_2)$.
\end{teorema}
\begin{dem}
\QED
\end{dem}

Claramente, $I^*$ y $J^*$ son aplicaciones de complejos de cadenas, por lo que obtenemos una sucesión exacta
\[
0\to \Omega^*(U)\overset{I^*}{\to}\Omega^*(U_1)\oplus\Omega^*(U_2)\overset{J^*}{\to}\Omega^*(U_1\cap U_2)\to 0
\]
de complejos de cadenas, a la cual hay una sucesión exacta larga de cohomología asociada
\[
\cdots\to H^p(U)\overset{I^p}{\to}H^p(\Omega^*(U_1)\oplus \Omega^*(U_2))\overset{J^p}{\to}H^p(U_1\cap U_2)\overset{\partial^*}{\to}H^{p+1}(U)\to \cdots
\]
Además, sabemos que $H^p(\Omega^*(U_1)\oplus \Omega^p(U_2))=H^p(\Omega^*(U_1))\oplus H^p(\Omega^*(U_2))=H^p(U_1)\oplus H^p(U_2)$. Así que se tiene el siguiente teorema.

\begin{teorema}[sucesión de Mayer-Vietoris] Sean $U_1,U_2\in\R^n$ abiertos y $U=U_1\cup U_2$. Entonces existe una sucesión exacta larga en cohomología
\[
\cdots\to H^p(U)\overset{I^p}{\to}H^p(U_1)\oplus H^p(U_2)\overset{J^p}{\to}H^p(U_1\cap U_2)\overset{\partial^*}{\to}H^{p+1}(U)\to \cdots
\]
donde $I^p([w])=(i_1^*([w]), i_2^*([w]))$ y $J^p([w_1],[w_2])=j_1^*([w_1])-j_2^*([w_2])$.
\end{teorema}

\begin{coro}\label{coro}
Si $U_1$ y $U_2$ son abiertos disjuntos de $\R^n$ entonces
\[
I^*: H^p(U_1\cup U_2)\to H^p(U_1)\oplus H^p(U_2)
\]
es isomorfismo.
\end{coro}

La demostración es muy sencilla, así que se deja como ejercicio.

\begin{ej}
Vamos a calcular $H^*(\R^2-\{0\})$. Sean
\begin{align*}
&U_1=\R^2-\{(x_1,x_2)\mid x_1\geq 0, x_2=0\}\\
&U_2=\R^2-\{(x_1,x_2)\mid x_1\leq 0, x_2=0\}.
\end{align*}
Estos son conjuntos abiertos estrellados, así que $H^p(U_1)=H^p(U_2)=0$ para todo $p>0$ y $H^0(U_1)=H^0(U_2)=\R$. De hecho $H^0(U_{\nu})=\langle [\chi_{U_{\nu}}]\rangle$. Su intersección es $\R^2_+\sqcup\R^2_-$, así que por el corolario \ref{coro} aplicado a los semiplanos, $H^p(U_1\cap U_2)=0$ para todo $p>0$ y $H^0(U_1\cap U_2)=\R\oplus\R$, con generadores $[\chi_{\R^2_+}],[\chi_{\R^2_-}]$. En la sucesión de Mayer-Vietoris, para $p>0$ tenemos
\[
0\to H^p(U_1\cap U_2)\overset{\partial^*}{\to} H^{p+1}(\R^2-\{0\})\to 0,
\]
lo cual indica que $\partial^*$ es un isomorfismo, así que $H^{q}(\R^2-\{0\})=0$ para $q\geq 2$. Para $p=0$ obtenemos
\[
\begin{tikzcd}
0\arrow[r] & H^0(\R^2-\{0\})\arrow[r,"I^0"] & H^0(U_1)\oplus H^0(U_2)\arrow[r, "J^0"]\arrow[d,equal] & H^0(U_1\cap U_2)\arrow[r,"\partial^*"]\arrow[d, equals] & H^1(\R^2-\{0\})\arrow[r,"I^1"] & 0\\
 & & \R\oplus\R &\R\oplus\R & & 
\end{tikzcd}
\]
Como $\R^2-\{0\}$ es conexo, $H^0(\R^2-\{0\})\cong\R$ generador por $[\chi_{\R^2-\{0\}}]$. Así que como $I^0$ es inyectiva por exactitud, $\Ima{I^0}\cong\R$. Esto nos da a su vez $\ker{J^0}\cong\R$. Por el teorema del rango-nulidad, esto implica que $\Ima{J^0}\cong\R$. Utilizando el primer teorema de isomorfía teniendo en cuenta que $\partial^*$ es sobreyectiva y la exactitud, tenemos
\[
H^0(U_1\cap U_2)/\Ima{J^0}\cong H^1(\R^2-\{0\}).
\]
Pero $H^0(U_1\cap U_2)/\Ima{J^0}\cong\R$, por lo que concluimos
\[
H^p(\R^2-\{0\})=\begin{cases}
0 & p\geq 2\\
\R & p=1\\
\R & p=0.
\end{cases}
\]

Además, que $\partial^*$ sea sobreyectiva implica que $H^1(\R^2-\{0\})$ está generado por la imagen del algún elemento de $H^0(U_1\cap U_2)$. Además, por exactitud debe ser un elemento que no esté en la imagen de $J^0$. Siconsideramos $[\chi_{U_1}]$ y $[\chi_{U_2}]$, entonces $J^0([\chi_{U_1}])=j_2^*[\chi_{U_1}]=[\chi_{U_1}\circ j_1]=[\chi_{U_1\cap U_2}]=[\chi_{\R^2_+}]+[\chi_{\R^2_-}]$. Análogamente, $J^0[\chi_{U_2}]=-[\chi_{U_1\cap U_2}]=-[\chi_{\R^2_+}]-[\chi_{\R^2_-}]$. Tomando coordenadas, podemos decir que $\Ima{J^0}=\langle (1,1)\rangle$. Por tanto, dando un generador del cociente $H^0(U_1\cap U_2)/\Ima{J^0}=\langle (1,0),(0,1)\rangle/\langle (1,1)\rangle$ tenemos un elemento cuya imagen mediante $\partial^*$ genera $H^1(\R^2-\{0\})$. Esto es sencillo, pues podemos tomar, por ejemplo $(1,0)+\langle (1,1)\rangle$, que se corresponde con $[\chi_{\R^2_+}]+\Ima{J^0}$. También podríamos haber tomado de forma análoga como generador $[\chi_{\R^2_-}]+\Ima{J^0}$, o cualquier combinación de ambos.
\end{ej}

\end{document}

