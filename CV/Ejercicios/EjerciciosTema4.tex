	\documentclass[twoside]{article}
\usepackage{../../estilo-ejercicios}

%--------------------------------------------------------
\begin{document}

\title{Ejercicios de From Calculus to Cohomology, Capítulo 5}
\author{Javier Aguilar Martín}
\maketitle


\begin{ejercicio}{5.1}
Usando la notación del ejemplo 5.4, un punto $(x,y)\in U_1$ está unívocamente determinado por sus coordenadas polares $(r,\theta)\in (0,\infty)\times (0,2\pi)$. Sea $arg_1\in\Omega^0(U_1)$ la función $(x,y)\mapsto \theta\in(0,2\pi)$ (¿por qué es $arg_1$ diferenciable?). Definir similarmente $arg_2\in\Omega^0(U_2)$ usando coordenadas polares con $\theta\in (-\pi,\pi)$ y probar la existencia de una 1-forma cerrada $\tau\in\Omega^1(\R^2-\{0\})$ tal que
\[
\eta_{U_{\nu}}=i^*_{\nu}(\tau)=d(arg_{\nu})\quad (\nu=1,2).
\]
Probar que el homomorfismo 
\[
\partial^0:H^0(U_1\cap U_2)\to H^1(\R^2-\{0\})
\]
lleva funciones localmente constantes con valores $\{0,2\pi\}$ en los semiplanos superior e inferior respectivamente en $[\tau]$. 
\end{ejercicio}
\begin{solucion}
Sea $\varphi:(0,+\infty)\times (0,2\pi)\to U_1=\R^2-[0,+\infty)$ dada por $(r,\theta)\mapsto (r\cos\theta, r\sin\theta)$, que es diferenciable y biyectiva,   con inversa $(\sqrt{x^2+y^2}, arg_1(x,y))$. Como $|D_{r,\theta}\varphi|=r\neq 0$, se tiene que $\varphi$ es localmente difeomorfismo, esto es, dado $(r,\theta)\in U$, existe un abierto $V$ tal que $\varepsilon|_V: V\to\varphi(V)$ es difeomorfismo con jacobiano 
\[
D_{(x,y)}\varphi^{-1}=(D_{\varphi^{-1}(x,y)}\varphi)^{-1}=\frac{1}{r}\begin{pmatrix}
r\cos\theta & -\sin\theta\\
r\sin\theta & \cos\theta
\end{pmatrix}=\begin{pmatrix}
\frac{x}{\sqrt{x^2+y^2}} & \frac{-y}{x^2+y^2}\\
\frac{y}{\sqrt{x^2+y^2}} & \frac{x}{x^2+y^2}
\end{pmatrix}=\begin{pmatrix}
\parcial{\varphi}{x} & \parcial{arg_1}{x}\\
\parcial{\varphi}{y} & \parcial{arg_1}{y}
\end{pmatrix}
\]
Análogamente podemos definir $arg_2$. Para la siguiente parte, basta ver que $d(J^*(arg_1,arg_2))=J^*(d(arg_1),d(arg_2))=0$, ya que por exactitud esto implicará que $(d(arg_1),d(arg_2))=I^*(\tau)$ para algún $\tau\in\Omega^1(\R^2-\{0\})$. 
$J^*(arg_1,arg_2)=j^*_1(arg_1)-j_2^*(arg_2)=arg_1\circ j_1-arg_2\circ j_2\in\Omega^0(U_1\cap U_2)$. Sea $x\in U_1\cap U_2=\R^2_+\sqcup \R^2_-$. Si $x\in\R^2_+$, entonces $arg_1\circ j_1(x)=arg_1(x)\in (0,\pi)$ y $arg_2\circ j_2(x)=arg_2(x)\in (0,\pi)$, por lo que miden el mismo ángulo y su diferencia se anula. Si $x\in\R^2_-$, entonces $arg_i\circ j_1(x)\in (\pi,2\pi)$ y $arg_2\circ j_2(x)\in (-\pi, 0)$, luego su diferencia es $2\pi$. Esto significa que 
$J^*(arg_1,arg_2)=2\pi\chi_{\R^2_-}$. Así que su diferencial es nula, como queríamos demostrar.

Este resultado además prueba la última parte del ejercicio por la definición de $\partial^0$. 
\end{solucion}

\newpage

\begin{ejercicio}{5.2}
Probar que la 1-forma $\tau\in\Omega^1(\R^2-\{0\})$ del ejercicio \ref{ejer:5.1} y $\Ima(z^{-1}dz)$ del ejercicio 3.6 son la misma. %\ref{3.6}
\end{ejercicio}
\begin{solucion}

Con la expresión explícita que hemos dado al calcular la jacobiana es claro que son la misma. Recuérdese que $arg_1$ y $arg_2$ difieren en una constante, por lo que sus diferenciales son iguales. 

\end{solucion}
\newpage

\begin{ejercicio}{5.3}
¿Puede ser $\R^2$ cubierto por dos abiertos conexos $U,V$ tales que $U\cap V$ es disconexo? 
\end{ejercicio}
\begin{solucion}
Aplicamos Mayer-Vietoris a la situación descrita. Obtendríamos que $\R\cong H^0(U)\oplus H^0(V)/H^0(\R^2)\cong H^0(U\cap V)$, por lo que la intersección es conexa.
\end{solucion}
\newpage

\begin{ejercicio}{5.4}[Propiedad Phragmen-Brouwer de $\R^n$]  Supongamos que $p\neq q$ en $\R^n$. Un subconjunto cerrado $A\subseteq\R^n$ se dice que \emph{separa} $p$ y $q$ si $p$ y $q$ pertenecen a distintas componentes conexas de $\R^n-A$. 

Sean $A$ y $B$ dos subcojuntos cerrados disjuntos de $\R^n$. Dados dos puntos distintos $p$ y $q$ en $\R^n-(A\cup B)$. Probar que si ni $A$ ni $B$ separan $p$ y $q$, entonces $A\cup B$ no separa $p$ y $q$. (Aplicar Mayer-Vietoris a $U_1=\R^n-A$, $U_2=\R^n-B$).

\end{ejercicio}
\begin{solucion}
%No vamos a seguir exactamente la indicación aunque se va a fundamentar en lo mismo.
%
%Recordemos que en $\R^n$, abierto y conexo implica conexo por caminos, por lo que las componentes conexas lo serán por caminos. Sea $V$ un entorno del segmento $[p,q]$ en $\R^n$ (podemos suponer que es contráctil). Como ni $A$ ni $B$ separan a $p$ y $q$, sabemos que podemos escoger $V$ de forma que $V-A$ y $V-B$ son conexos por caminos. 
%
%Así que, en Mayer-Vietoris para $p=0$
%\[
%\begin{tikzcd}
%0\arrow[r] & H^0(V)\arrow[r,"I^0"] & H^0(V-A)\oplus H^0(V-B)\arrow[r, "J^0"] & H^0(V-(A\cup B))\arrow[r,"\partial^*"] & H^1(V).
%\end{tikzcd}
%\]
%Como $H^1(V)=0$, $H^0(V)=H^0(V-A)=H^0(V-B)=\R$,
%\[
%\begin{tikzcd}
%0\arrow[r] & \R\arrow[r,"I^0"] & \R\oplus \R\arrow[r, "J^0"] & H^0(V-(A\cup B))\arrow[r]& 0
%\end{tikzcd}
%\]
%de donde $\R\oplus\R\cong \R\oplus  H^0(V-(A\cup B))$, por lo que $H^0(V-(A\cup B))\cong\R$. Eso significa que $V-(A\cup B)$ es conexo por caminos, por lo que $A\cup B$ no separa $p$ y $q$. 

%{\bf\large ALTERNATIVA USANDO TAL CUAL LA INDICACIÓN}
Recordemos que todo abierto de $\R^n$ tiene como mucho una cantidad numerable de componentes conexas. Entonces $\R^n-A=A_1\sqcup A_2\sqcup\cdots$, $\R^n-B=B_1\sqcup B_2\sqcup\cdots$  y $\R^n-(A\cup B)=C_1\sqcup C_2\sqcup\cdots$. Supongamos que $p,q\in A_1$, $p,q\in B_1$ y $p\in C_1$. Tenemos que probar que $q\in C_1$. Se tiene que $\R^n$ Por Mayer-Vietoris tenemos
\[
0\to H^0(\R^n)\to H^0(\R^n-A)\oplus H^0(\R^n-B)\overset{J^*}{\to} H^0(\R^n-(A\cup B))\to 0
\]

$H^0(\R^n-A)=\langle \chi_{A_i}\rangle\cong\R$, $H^0(\R^n-B)=\langle \chi_{B_j}\rangle\cong\R$ y $H^0(\R^n-(A\cup B))=\langle \chi_{C_R}\rangle\cong\R$. Consideremos $\chi_{C_1}$. Como $J^*$ es sobreyectivo, existe $(\sum a_i\chi_{A_i},\sum b_j\chi_{B_j})$ con $J^*(\sum a_i\chi_{A_i},\sum b_j\chi_{B_j})=\chi_{C_1}$. Es decir, $\sum a_i\chi_{A_i}-\sum b_j\chi_{B_j}=\chi_{C_1}$. Tenemos que $\chi_{C_1}(p)=1$ y que  $(\sum a_i\chi_{A_i}-\sum b_j\chi_{B_j})(p)=a_1-b_1$. También sabemos que $\chi_{C_1}(q)=(\sum a_i\chi_{A_i}-\sum b_j\chi_{B_j})(q)=a_1-b_1=(\sum a_i\chi_{A_i}-\sum b_j\chi_{B_j})(p)=\chi_{C_1}(p)=1$, por lo que $q\in C_1$. 
\end{solucion}



\end{document}