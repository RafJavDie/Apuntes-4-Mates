	\documentclass[twoside]{article}
\usepackage{../../estilo-ejercicios}

%--------------------------------------------------------
\begin{document}

\title{Ejercicios de From Calculus to Cohomology, Capítulo 8}
\author{Javier Aguilar Martín}
\maketitle


\begin{ejercicio}{8.1}
Completar los detalles de la nota 8.2 del libro.
\end{ejercicio}
\begin{solucion}



\end{solucion}

\newpage

\begin{ejercicio}{8.2}
Sea $\varphi:N\to M$ una aplicación continua entre variedades diferenciables donde $M$ es subvariedad de $\R^k$. Sea $i:M\to \R^k$ la inclusión. Probar que $\varphi$ es diferenciable si y solo si $i\circ\varphi$ lo es.
\end{ejercicio}
\begin{solucion}
Probamos primero que la inclusión es diferenciable. Para ello, tomamos una carta $(U_m, h_m)$ en $M$ (suponemos que $M$ es de dimensión $m\leq k$). Entonces
\[
 i\circ h_m^{-1}: h_m(U_m)\to \R^k
\]
es diferenciable, ya que es la inclusión de un abierto de $\R^m$ en $\R^k$. Por tanto, si $\varphi$ es diferenciable, entonces $i\circ\varphi$ es diferenciable por composición. 

Recíprocamente, si $i\circ\varphi:N\to \R^k$ es diferenciable, significa que existe una carta $(U_n,h_n)$ de $N$ tal que
\[
i\circ\varphi\circ h_n^{-1}:h_n(U_n)\to\R^k
\]
es diferenciable. Como $i\circ\varphi=\varphi$ en cualquier abierto de $M$, entonces $\varphi$ es diferenciable.
\end{solucion}
\newpage

\begin{ejercicio}{8.3}
Supongamos que $M\subseteq\R^k$ (con la topología inducida por $\R^k$) es una variedad topológica $n$-dimensional. Incluyamos $M$ en $\R^{k+n}$. Probar que $M$ es localmente plana en $\R^{n+k}$.
\end{ejercicio}
\begin{solucion}
Recordemos hay que comprobar que se verifica la definición 8.8 pero sustituyendo las cartas por homeomorfismos. Entonces basta tomar la identidad $Id:\R^{n+k}\to\R^{n+k}$, de modo que para todo $x\in M\subseteq\R^n\subseteq\R^{n+k}$ se verifica que $x\in\R^{n+k}$ y además $Id(\R^{n+k}\cap M)=M=\R^n\cap M$.
\end{solucion}
\newpage

\begin{ejercicio}{8.4}
Sea $T^n=\R^n/\Z^n$ entendido como espacio de órbitas de la acción del grupo $\Z^n$ sobre $\R^n$. Sea $\pi:\R^n\to T^n$ la proyección canónica y equipemos a $T^n$ con la topología cociente.

Probar que $T^n$ es una variedad topológica compacta de dimensión $n$. Construir una estructura diferenciable en $T^n$ tal que $\pi$ se vuelva diferenciable y todo $p\in \R^n$ tiene un entorno abierto difeomorfo a un abierto de $T^n$ mediante $\pi$. Probar que $T^1$ es difeomorfo a $S^1$.  
\end{ejercicio}
\begin{solucion}

\end{solucion}

\newpage

\begin{ejercicio}{8.5}
Definimos $\tilde{A}:\R^2\to\R^2$ mediante $\tilde{A}(x,y)=(x+\frac{1}{2}, -y)$. Probar que existe una aplicación diferenciable $A:T^2\to T^2$ satisfaciendo $A\circ\pi =\pi\circ \tilde{A}$. Probar que $A$ es difeomorfismo con ella misma como inversa y que no tiene puntos fijos en $T^2$. 

Sea $K^2$ el conjunto de pares $\{q,A(q)\}$ con $q\in T^2$. Probar que $K^2$ con la topología cociente de $T^2$ es una 2-variedad topológica. Construir una estructura diferenciable sobre $K^2$.
\end{ejercicio}
\begin{solucion}

\end{solucion}

\newpage

\begin{ejercicio}{8.6}
Sea $p_0\in S^n$  el punto $p_0=(0,\dots, 0,1)$. Probar que $S^n-\{p_0\}$ es difeomorfo a $\R^n$ bajo la proyección estereográfica. 
\end{ejercicio}
\begin{solucion}
\end{solucion}


\end{document}