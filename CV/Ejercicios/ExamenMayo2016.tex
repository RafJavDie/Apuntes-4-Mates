	\documentclass[twoside]{article}
\usepackage{../../estilo-ejercicios}

%--------------------------------------------------------
\begin{document}

\title{Cálculo en Variedades, Mayo de 2016}
\author{Javier Aguilar Martín}
\maketitle


\begin{ejercicio}{1}
Recordemos del Ejercicio 2.4 que para un $\R$-espacio vectorial $V$ existe un isomorfismo $i:V\to Alt^1(V)=V^*$ dado por $i(v)\in Alt^1(V)$ que verifica $i(v)(w)=\langle w, v\rangle$ (producto escalar sobre $\R$ en $V$). 

Dados $\{v_1,\dots, v_{n-1},w\}\subset\R^n$, tenemos la existencia de $\varphi\in Alt^1(\R^n)$ definida por $\varphi(w)=\det\begin{pmatrix}
v_1\\
\vdots\\
v_{n-1}\\
w
\end{pmatrix}$. Entonces, por el isomorfismo anterior, existe un único $z=i^{-1}(\varphi)\in\R^n$ y, por tanto, $\varphi(w)=\langle w,z\rangle$. El vector $z$ es el llamado producto $v_1\times\dots\times v_{n-1}$.
\begin{enumerate}[a)]
\item Si hacemos lo anterior para $n=2$, ¿qué vector de $\R^2$ es $i^{-1}(\varphi)$?
\item Supongamos que $\{v_1,\dots, v_{n-1}\}\subset\R^n$ son linealmente independientes. Comprueba que $\{v_1,\dots, v_{n-1}, v_1\times\dots\times v_{n-1}\}$ describe la orientación estándar de $\R^n$, es decir, la base está positivamente orientada respecto a $\varepsilon_1\land\dots\land\varepsilon_n\in Alt^n(\R^n)$ interpretada como la forma estándar de orientación $dx_1\land\dots\land dx_n\in\Omega^n(\R^n)$ evaluada en $0\in\R^n$.
\end{enumerate}
\end{ejercicio}
\begin{solucion}\
\begin{enumerate}[a)]
\item  Tenemos que, dado $v\in\R^2$,  $\varphi(w)=\det\begin{pmatrix}
v\\
w
\end{pmatrix}=v_1w_1-v_2w_1$. Por otro lado, $z=i^{-1}(\varphi)\Leftrightarrow i(z)=\varphi$, así que $\varphi(w)=i(z)(w)=\langle w,z\rangle =w_1z_1+w_2z_2$. Esto implica que $z=(w_2,-w_1)$. 
\item
\end{enumerate}

\end{solucion}

\newpage

\begin{ejercicio}{2}
Demuestra que el homomorfismo $\land: Alt^1(\R^4)\times Alt^1(\R^4)\to Alt^2(\R^4)$ no es sobreyectivo. 

Ayuda: encuentra $w\in Alt^2(\R^4)$ tal que $w\land w\neq 0$.
\end{ejercicio}
\begin{solucion}
Sea $w=\varepsilon_1\land \varepsilon_2+\varepsilon_3\land\varepsilon_4$. Entonces $w\land w\neq 0$, tal como tengo en el Ejercicio 2.1. Si existieran $\alpha,\beta\in Alt^1(\R^4)$ con $w=\alpha\land\beta$, entonces 
\[
w\land w=(\alpha\land\beta)\land(\alpha\land\beta)=-(\alpha\land\alpha)\land(\beta\land\beta)=0,
\]
lo cual es una contradicción.
\end{solucion}
\newpage

\begin{ejercicio}{3}

\end{ejercicio}
\begin{solucion}

\end{solucion}
\newpage

\begin{ejercicio}{4}

\end{ejercicio}
\begin{solucion}

\end{solucion}

\newpage

\begin{ejercicio}{5}

\end{ejercicio}
\begin{solucion}

\end{solucion}


\end{document}