	\documentclass[twoside]{article}
\usepackage{../../estilo-ejercicios}

%--------------------------------------------------------
\begin{document}

\title{Ejercicios de From Calculus to Cohomology, Capítulo 9}
\author{Javier Aguilar Martín}
\maketitle

\begin{ejercicio}{9.1}
Sea $M\subseteq\R^l$ una subvariedad diferenciable y supongamos que los puntos $p\in\R^l$ y $p_0\in M$ son tales que $||p-p_0||\leq ||p-q||$ para todo $q\in M$. Probar que $p-p_0\in T_{p_0}M^{\perp}$.
\end{ejercicio}
\begin{solucion}
Podemos considerar $T_{p_0}M$ como subespacio de $\R^l$, por lo que el resultado se deduce del teorema de la proyección. 
%https://math.stackexchange.com/questions/1492497/projection-theorem-understanding-two-parts-of-the-proof


\end{solucion}
\newpage


\begin{ejercicio}{9.9}
En el espacio vectorial $M=M_n(\R)$ de matrices reales $n\times n$ tenemos el subespacio de matrices simétricas $S_n$. Definimos la aplicación diferenciable $\varphi:M\to S_n$ como 
\[
\varphi(A)=A^tA.
\]
Nótese que $\varphi^{-1}(I)$ es el espacio de matrices ortogonales $O_n$. Probar que para $A,B\in M$ tenemos
\[
D_A\varphi(B)=B^tA+A^tB.
\]
(Pista: usar la curva $A+sB$)
Aplicar el ejercicio 9.6 para probar que $O_n$ es una subvariedad diferenciable de $M$.
\end{ejercicio}
\begin{solucion}
Tomamos la curva $\gamma(s)=A+sB$ tal que $\gamma'(0)=B$. Entonces, 
\[
D_A\varphi(B)=D_A\varphi(\gamma'(0))=(\varphi\circ\gamma)'(0)=A^tB+B^tA.
\]

Para la segunda parte tenemos que probar que $D_A\varphi$ es sobreyectiva para toda matriz ortogonal $A$. Es decir, dada $\gamma:[0,1]\to S_n$ con $\gamma'(0)\in S_n$, tenemos que encontrar $B\in M$ tal que $\gamma'(0)=B^tA+A^tB$. Para ello hacemos la siguiente descomposición:
\[
\gamma'(0)=\frac{(\gamma'(0)A^t)}{2}A+A^t\frac{(A\gamma'(0))}{2}
\]
con lo que $B=\frac{A\gamma'(0)}{2}$.
\end{solucion}
\newpage

\begin{ejercicio}{9.10}
Un \emph{grupo de Lie} $G$ es una variedad diferenciable, que es también un grupo, tal que el producto y tomar inverso son diferenciables. Demostrar que $O_n$ es un grupo de Lie. (Aplicar \ref{ejer:9.9})
\end{ejercicio}
\begin{solucion}
$O_n$ hereda la estructura diferenciable de $M_n(\R)$, que es equivalente a la de $\R^{n\times n}$. Es claro que el producto de matrices es diferenciable, luego en $O_n$ lo es. En el caso de la inversa, como en el caso de $O_n$ es igual a trasponer, que no es más que una reordenación de las coordenadas es también claro que se trata de una aplicación diferenciable.
\end{solucion}
\newpage

\begin{ejercicio}{9.11}
Sea $\varphi:M\to N$ una aplicación diferenciable entre variedades diferenciables. Probar que $\varphi^*:\Omega^*(N)\to\Omega^*(M)$ es un morfismo de complejos de cadenas.
\end{ejercicio}
\begin{solucion}
Tenemos que probar que el siguiente diagrama es conmutativo 
\[
\begin{tikzcd}
\Omega^k(N)\arrow[r,"d^k"]\arrow[d,"\varphi^*"] & \Omega^{k+1}(N)\arrow[d,"\varphi^*"]\\
\Omega^k(M)\arrow[r, "d^k"] & \Omega^{k+1}(M)
\end{tikzcd}
\]
para todo $k$. Sea $w\in\Omega^k(N)$. Elegimos $p\in M,q=\varphi(p)\in N$ y parametrizaciones locales $g:W\to M$ y $h:V\to N$. Tenemos por definición
\[
d_qw=Alt^{k+1}((D_yh)^{-1})d_y(h^*(w))=Alt^{k+1}(D_qh^{-1})d_y(h^*(w))
\]
donde $y$ cumple $h(y)=q=\varphi(p)$, así que
\[
\varphi^*(d_qw)_p=Alt^{k+1}(D_p\varphi)\circ Alt^{k+1}(D_qh^{-1})(d_yh^*(w))=
\]
\[
Alt^{k+1}(D_qh^{-1}D_p\varphi)(d_yh^*(w))=Alt^{k+1}(D_ph^{-1}\varphi)(d_y(h^*(w)))=
\]
%\[
%Alt^{k+1}(D_ph^{-1}\varphi)(d_yAlt^k(D_yh)(w_{h(y)})).
%\]
%Ahora, por la conmutatividad de $d$ con $\Omega^k(h)$ a nivel de complejo de cocadenas de abiertos euclídeos, esto es igual a 
%\[
%Alt^{k+1}(D_ph^{-1}\varphi)Alt^{k+1}(D_yh)(d_q w_{\varphi(p)})=Alt^{k+1}(D_p\varphi)(d_yw_{\varphi(p)}).
%\]
% 
Por otro lado, 
%\[
%\varphi^*(w)_p=Alt^k(D_p\varphi)(w_{\varphi(p)}),
%\]
%así que 
eligiendo $x$ con  $g(x)=p$ tenemos
\[
d_p\varphi^*(w)=Alt^{k+1}((D_xg)^{-1})d_x(g^*(\varphi^*(w)))=
\]
%\[
%Alt^{k+1}(D_xg^{-1})d_x(Alt^k(D_xg)\circ Alt^k(D_p\varphi)(w_{\varphi(p)}))=
%\]
%\[
%Alt^{k+1}(D_xg^{-1})d_x(Alt^k(D_x\varphi g)(w_{\varphi(p)})=Alt^{k+1}(D_p\varphi)(d_x w_{\varphi(p)})
%\]

SUPONGO QUE NO ME QUEDA MÁS REMEDIO QUE EVALUAR EN VECTORES
\end{solucion}

\newpage

\begin{ejercicio}{9.12}
El producto escalar usual en $\R^n$ induce un producto escalar en $Alt^n(\R^n)$ (ver ejercicio 2.5). Probar que $w\in Alt^n(\R^n)$ es un vector unitario si y solo si $w(v_1,\dots, v_n)=\pm 1$ para toda base ortonormal $\{v_1,\dots, v_n\}$ de $\R^n$.
\end{ejercicio}
\begin{solucion}
Dada $w\in Alt^n(\R^n)$ tenemos que $w=\lambda \varepsilon_1\land\dots\varepsilon_n=\lambda\varepsilon_I$ con $\lambda\in\R$. Entonces $\gene{w,w}=\lambda^2\det(\gene{\varepsilon_i,\varepsilon_j})=\lambda^2\det(\gene{i^{-1}(\varepsilon_i),i^{-1}(\varepsilon_j)})=\lambda^2$. La última igualdad se tiene porque $i$ lleva bases ortonormales en bases ortonormales y estamos tomando la dual de la base estándar. Por otra parte, $w(v_1,\dots, v_n)=\lambda\det(\varepsilon_i(v_j))=\pm\lambda$. Así que $\gene{w,w}=1\Leftrightarrow \lambda=1\Leftrightarrow w(v_1,\dots, v_n)=\pm 1$.
\end{solucion}

\newpage

\begin{ejercicio}{9.13}
Probar que la botella de Klein no es orientable.
\end{ejercicio}
\begin{solucion}
\end{solucion}
\newpage

\begin{ejercicio}{9.15}
Sea $M$ una variedad diferenciable $n$-dimensional y denotemos por $\widetilde{M}$ el conjunto de los pares $(p, o_p)$, donde $p\in M$ y $o_p$ es una de las dos orientaciones de $T_pM$. La proyección $\pi:\widetilde{M}\to M$ envía $(p,o_p)$ a $p$. 

Para un abierto orientado $W\subseteq M$ con forma de orientación $w\in\Omega^n(W)$ denotamos $\widetilde{W}\subseteq\widetilde{M}$ al conjunto de pares $(p,o_p)$, donde $p\in W$ y $o_p$ es la orientación de $T_pM$ determinada por $w_p\in Alt^n T_pM$.

Probar que $\widetilde{M}$ tiene una topología tal que $\widetilde{W}$ es abierto y $\pi$ es un homeomorfismo entre $\widetilde{W}$ y $W$ para todo abierto orientado $W\subseteq M$. Probar que $\widetilde{M}$ tiene una orientación canónica. El par consistente en $\widetilde{M}$ y $\pi$ se llama \emph{oriented double covering} (doble recubridor orientado) de $M$.
\end{ejercicio}
\begin{solucion}
Para cada abierto orientado $W\subseteq M$ con forma de orientación $w\in\Omega^n(W)$ definimos los conjuntos $W_+=\{(p,o_p)\in\widetilde{M}\mid p\in W, o_p\in [w]\}$ y $W_-\{(p,o_p)\in\widetilde{M}\mid p\in W, o_p\notin[w]\}$. Dotamos a $\widetilde{M}$ de la topología que tiene como base el de abiertos la familia $\{W_+,W_-\}_{W\subseteq M}$. Con la notación del enunciado, $\widetilde{W}=W_+$ y es evidente que $\pi:W_+\to W$ es homeomorfismo con esta topología.

Construimos un atlas sobre $\widetilde{M}$ a partir del atlas $\{(U,\varphi)\}$ de $M$. Para ello consideramos la familia $\{(U_\pm, \varphi_\pm)\}$ con $\varphi_\pm:=\varphi\circ\pi$. Es claro que los abiertos recubren $\widetilde{M}$ y por la observación anterior, $\varphi_\pm$ es un homeomorfismo entre $U_\pm$ y $\varphi(U)$. Además, dadas dos cartas $(U_\pm,\varphi_\pm)$ y $(V_\pm,\psi_\pm)$, se tiene que la aplicación
\[
\psi_\pm\circ \varphi_\pm^{-1}=\psi\circ\pi\circ\pi^{-1}\circ\varphi^{-1}=\psi\circ\varphi\in C^{\infty}
\]
pues $\pi$ se está restringiendo a un dominio donde es homeomorfismo.

Se puede orientar $\widetilde{M}$ con la forma de orientación $w\in\Omega^n(\widetilde{M})$ que en cada punto $(p,o_p)$ vale $o_p$. 


\end{solucion}
\newpage

\begin{ejercicio}{9.16}
Sea $M$ una variedad diferenciable conexa. Probar que $\widetilde{M}$ consiste en a lo sumo dos componentes conexas, y que $M$ es orientable si y solo si $\widetilde{M}$ es conexa.
\end{ejercicio}
\begin{solucion}
Si $M$ es orientable, entonces podemos considerar $\widetilde{M}_+$ y $\widetilde{M}_-$, que son abiertos disjuntos que recubren $\widetilde{M}$, de modo que $\widetilde{M}$ tiene dos componentes conexas.


Supongamos que $M$ es no orientable y sea $W\subseteq\widetilde{M}$ una componente conexa. La restricción $\pi|_W: W\to M$ sigue siendo un recubridor diferenciable. Efectivamente, para cada abierto $U\subseteq M$, cada componente conexa de $\pi^{-1}(U)$ está en una única componente conexa de $\widetilde{M}$, por lo que podemos quedarnos con las preimágenes que estén en $W$. Así que todas las fibras tienen el mismo cardinal. Esto quiere decir que  las fibras en $W$ tienen cardinalidad 1 o 2. Si tuvieran cardinalidad 1, entonces $\pi|_W$ sería un difeomorfismo y su inversa induciría una orientación en $M$. De esta forma, la cardinalidad de las fibras es 2 y eso implica que necesariamente $W=\widetilde{M}$, con lo que se tienen la conexión.
\end{solucion}
\newpage

\begin{ejercicio}{9.17}
Sea $V\subseteq\R^{n+k}$ un entorno tubular de una subvariedad diferenciable $M\subseteq\R^{n+k}$ de dimensión $n$ con la proyección asociada $r:V\to M$. Se define la aplicación diferenciable $f:V-M\to\R$ mediante
\[
f(x)=||x-r(x)||=\min_{y\in M}||x-y||.
\]
Probar que $f$ es una solución a la ecuación diferencial 
\[
\sum_{j=1}^{n+k}\left(\parcial{f}{x_j}\right)^2=1.
\]
Supongamos que $k=1$ y que $M$ está orientada por una aplicación de Gauss $Y$. Podemos definir la distancia signada de $M$, $\varphi:V\to R$ requiriendo
\[
\varphi(x)Y(r(x))=x-r(x)\quad (x\in V).
\]
Probar que $\varphi$ es una solución de la ecuación 
\[
\sum_{j=1}^{n+1}\left(\parcial{\varphi}{x_j}\right)^2=1.
\]
\end{ejercicio}
\begin{solucion}
Las derivadas salen unos chorizos enormes pero al final sale lo que tiene que salir. Para la segunda parte observar $||\varphi(x)||=f(x)$. 
\end{solucion}
\newpage

\begin{ejercicio}{9.18}
Sea $\pi:\widetilde{M}\to M$ el doble recubridor orientado del ejercicio \ref{ejer:9.15}. Sea $A:\widetilde{M}\to \widetilde{M}$ la aplicación que para cada $p\in M$ intercambia los dos puntos en $\pi^{-1}(p)$. Probar que $A$ es un difeomorfismo de orden 2 y que
\[
\Omega^r(\widetilde{M})=\Omega^r(\widetilde{M})_+\oplus\Omega^r(\widetilde{M})_-,
\]
donde $\Omega^r(\widetilde{M})_\pm$ es el subespacio propio asociado a $\pm 1$ para el isomorfismo
\[
A^*:\Omega^r(\widetilde{M})\to \Omega^r(\widetilde{M}).
\]
Probar que el complejo de deRham $(\Omega^*(\widetilde{M}),d)$ se descompone en la suma directa de dos subcomplejos
\[
(\Omega^*(\widetilde{M})_+,d)\text{ y } (\Omega^*(\widetilde{M})_-,d).
\]
Probar que $\pi^*$ es un isomorfismo entre el complejo $(\Omega^*(M),d)$ y $(\Omega^*(\widetilde{M})_+,d)$. Probar que para todo $k\in\Z$ se tiene que
\[
H^k(\pi):H^k(M)\to H^k(\widetilde{M})
\]
envía $H^k(M)$ mediante isomorfismo a al subespacio propio asociado a 1 en $H^k(\widetilde{M})$ de $H^k(A)$. 
\end{ejercicio}
\begin{solucion}
Es evidente que $A$ tiene orden 2 porque solo hay dos fibras para cada punto. Por tanto es su propia inversa y basta probar que es diferenciable para comprobar que es difeomorfismo, pero esto es claro puesto que $$\psi_{\pm}\circ A\circ\varphi_{\pm}=\psi\circ\pi\circ A\circ \pi^{-1}\circ\varphi^{-1}=\psi\circ\varphi$$
pues las proyecciones actúan igual sobre las dos fibras, es decir, $\pi\circ A=\pi$.

Como $A$ es un difeomorfismo de orden 2, $A^*:\Omega^r(\widetilde{M})\to \Omega^r(\widetilde{M})$ es un isomorfismo de orden 2, con lo que podemos descomponer $\Omega^r(\widetilde{M})=\Omega^r(\widetilde{M})_+\oplus\Omega^r(\widetilde{M})_-$ para todo $r$, donde $\Omega^r(\widetilde{M})_+=\{w\in \Omega^r(\widetilde{M})\mid A^*w=w\}$ y $\Omega^r(\widetilde{M})_-$ es su complemento ortogonal, es decir el subespacio formado por las formas diferenciales tales que $(A^*w)_{(p,o_p)}=w_{(p,-o_p)}=-w_{(p,o_p)}$.

Definimos el isomorfismo $\phi: \Omega^r(\widetilde{M})\to \Omega^r(\widetilde{M})$ como $\phi(w)=\frac{w+A^*(w)}{2}\oplus \frac{w-A^*(w)}{2}$. Con esto es claro que $d$ respeta los subespacios de forma análoga a como se hizo para el espacio proyectivo. Por tanto, se tiene la descomposición $(\Omega^*(\widetilde{M}),d)=(\Omega^*(\widetilde{M})_+,d)\oplus(\Omega^*(\widetilde{M})_-,d).$

Esto induce una descomposición similar en cohomología, $H^k(\widetilde{M})=H^k(\widetilde{M})_+\oplus H^k(\widetilde{M})_-$. 

Para probar que $\pi^*$ induce tal isomorfismo, observemos primero que como $\pi\circ A=\pi$, para cada $r$, $\Ima\pi^*\cong \{w\in \Omega^r(\widetilde{M})\mid w=A^*(w)\}=\Omega^r(\widetilde{M})_+$. Por tanto, tenemos que ver que $\pi^*$ es inyectiva. Supongamos que $\pi^*(w)=0$, entonces para todo $q=(p,o_p)\in\widetilde{M}$, $\pi^*(w)_q=0$. Equivalentemente, $\pi^*(w)_q(w_1,\dots, w_r)=w_p(D_p\pi(w_r),\dots, D_p\pi(w_r))$ para todo $w_i\in T_q\widetilde{M}$. Pero $D_p\pi$ es isomorfismo por ser $\pi$ difeomorfismo local, así que $w_p(v_1,\dots, v_p)=w_p(D_p\pi(w_r),\dots, D_p\pi(w_r))$ para todo $v_i\in T_pM$.

A partir de esto y la descomposición del complejo de cadenas se tiene directamente que $H^k(\pi)$ es un isomorfismo entre $H^k(M)$ y $H^k(\widetilde{M})_+$.
\end{solucion}
\end{document}