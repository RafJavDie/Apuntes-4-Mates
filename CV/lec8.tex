\documentclass[CV.tex]{subfiles}

\begin{document}


%\hyphenation{equi-va-len-cia}\hyphenation{pro-pie-dad}\hyphenation{res-pec-ti-va-men-te}\hyphenation{sub-es-pa-cio}

\chapter{Formas diferenciables sobre variedades}

En el primer capítulo habíamos definido las formas diferenciables como aplicaciones $w\in\Omega^p(U)$, $w:U\subseteq\R^n\to Alt^p(\R^n)$. En este capítulo veremos cómo construirlas sobre variedades diferenciables.

\section{Construcción de las formas diferenciables}

Fijado $p\in U\subseteq\R^n$ abierto, definimos el conjunto $\{\gamma:I\to U$ diferenciable con $I$ intervalo abierto conteniendo el 0 y $\gamma(0)=p\}=C_p(U)$. Definimos sobre él la relación de equivalencia $\gamma_1\sim\gamma_2$ si y solo si $D_0\gamma_1=\gamma_1'(0)=\gamma_2'(0)=D_0\gamma_2$. Entonces $C_p(U)/\sim\cong\R^n$. Si $F:U\to V$ es diferenciable con $p\in U\subseteq\R^n$ y $V\subseteq\R^m$ abiertos, entonces dada $\gamma:I\to U$ curva diferenciable con $\gamma(0)=p$, la aplicación $D_pF:\R^n\to \R^m$ se puede definir como $\gamma(0)'\mapsto D_p(F(\gamma'(0))=D_0(F\circ\gamma)=(F\circ \gamma)'(0)$.  Así que definimos $D_pF:C_p(U)/\sim\to C_{F(p)}(V)/\sim$ como $[\gamma]\mapsto[F\circ\gamma]$.

Dado $p\in M$ variedad $n$-dimensional, tomamos una carta $(U,h)$ en $M$ contieniendo a $p$ y consideramos $C_p(M)=\{\alpha:I\to M$ diferenciable con $I$ intervalo abierto conteniendo el 0 y $\gamma(0)=p\}$, donde definimos $\alpha_1\sim\alpha_2$ si y solo si $(h\circ\alpha_1)'(0)=(h\circ\alpha_2)'(0)$. Supongamos que tomáramos otra carta $(\widetilde{U},\tilde{h})$. Vamos a ver que la relación anterior no depende de la carta. Consideramos el intercambio de cartas $F=h\circ\tilde{h}^{-1}$ (que es difeomorfismo en un entorno de $\tilde{h}(p)$ en un entorno de $h(p)$). Esto implica que $D_{\tilde{h}(p)}F$ es isomorfismo. Así que $D_{\tilde{h}(p)}([\tilde{h}\circ\alpha_i])=[h\circ \alpha_i]$ para $i=1,2$, por lo que si las derivadas tomando $\tilde{h}$ son iguales, también lo son tomando $h$.

\begin{defi}
Definimos el \textbf{espacio tangente} $T_pM=C_p(M)/\sim$.
\end{defi}

\begin{nota}
\item $T_pM$ es un $\R$-espacio vectorial de dimensión $n$. Dada una carta $(U,h)$ alrededor de $p\in M$,
\[
\phi_h:T_pM\to \R^n=C_{h(p)}(h(U))/\sim 
\]
\[
[\alpha]\mapsto \phi_h([\alpha])=(h\circ\alpha)'(0)=D_0(h\circ\alpha)
\]
$\phi_h$ es una biyección e induce una estructura de $\R$-espacio vectorial de dimensión $n$. Podemos considerar $\phi_h$ como un $\R$-isomorfismo de espacios vectoriales. Si $(\widetilde{U},\tilde{h})$ es otra carta para $p\in M$, entonces como $D_{h(p)}F:C_{h(p)}(h(U))/\sim\to C_{\tilde{h}(p)}(\tilde{h}(\widetilde{U})/\sim$ es isomorfismo, tenemos que $\phi_h=D_{h(p)}F^{-1}\circ \phi_{\tilde{h}}$, ya que $\phi_h([\alpha])=[h\circ\alpha]$ y $\phi_{\tilde{h}}([\alpha])=[\tilde{h}\circ\alpha]=[\tilde{h} \circ h^{-1}\circ h\circ \alpha]=[F^{-1} \circ h \circ \alpha]$. 
\end{nota}

\begin{lemma}
Sea $f:M\to N$ un diferenciable entre variedades de dimensión $m$ y $n$ respectivamente. Sea $p\in M$, entonces
\begin{enumerate}
\item Existe $D_pf:T_pM\to T_{f(p)}N$ $\R$-homomorfismo de espacios vectoriales dado por $D_pf[\alpha]=[f\circ\alpha]$.
\item Si $(U,h)$ es una carta para $p\in M$ y $(V,g)$ alrededor de $f(p)\in N$, se tiene que conmuta el diagrama
\[
\begin{tikzcd}
T_pM\arrow[r, "D_pf"]\arrow[d, "\phi_h"'] & T_{f(p)}N\arrow[d, "\phi_{\tilde{h}}"]\\
\R^m\arrow[r, "D_{h(p)}(g\circ f\circ h^{-1})"'] & \R^n
\end{tikzcd}
\]
\end{enumerate}
\end{lemma}

\begin{proof}
 Tomando cartas $(U,h),(V,g)$ de un atlas maximal, tendremos que
\[
F=g\circ f\circ h^{-1}:h(U\cap f^{-1}(V))\to g(f(U\cap f^{-1}(V)))
\]
es diferenciable, luego induce $D_{h(p)}F:\R^m\to\R^n$  dada por $[\gamma]\mapsto[F\circ\gamma]$. Luego dado $[\alpha]\in T_pM$ se tiene que 
\[
D_{h(p)}F(\phi_h[\alpha])=[F\circ h\circ\alpha]=[g\circ f\circ \alpha]=\phi_g[f\circ\alpha]
\]
Basta definir $D_pf[\alpha]=[f\circ \alpha]$, por lo que el diagrama conmuta. Como $\phi_h,\phi_g$ son isomorfimos se tiene entonces por el diagrama que $D_pf$ es homomorfismo de $\R$-espacios vectoriales.
\end{proof}

\begin{nota}\
\begin{enumerate}
\item $(U,h)$ , $D_ph=\phi_h:T_pM\to\R^m$ es isomorfismo si $M$ es una $m$-variedad. De hecho $\R^m\cong C_{h(p)}(h(U))/\sim =T_{h(p)}\R^m$. Además $\phi^{-1}_h=D_{h(p)}\phi=(D_{h^{-1}(p)}\phi)^{-1}$.
\item Sea $i:M\hookrightarrow \R^n$ diferenciable y $p\in M$. $D_pi:T_pM\to T_{i(p)}\R^n\cong\R^n$ es una inyección y podemos identificar $T_pM=\Ima D_pi\subseteq\R^n$. 
\item Si $M\overset{f}{\to}N\overset{\varphi}{\to}P$ son diferenciables, entonces $D_p(\varphi\circ f)=D_{f(p)}\varphi\circ D_pf$ (regla de la cadena).
\item Dado $p\in M$, y una carta $(U,h)$ alrededor de $p$, entonces la base del $\R$-e.v $T_pM$ será $\{(\parcial{}{x_i})_p\}_{i=1
}^n$ donde $(\parcial{}{x_i})_p=\phi^{-1}_h(e_i)$ siendo $\{e_1,\dots, e_n\}$ la base estándar.
\item Un vector tangente $X_p\in T_pM$ se podrá escribir de forma única como
\begin{equation}\label{coord}
X_p=\sum_{i=1}^m a_i\left(\parcial{}{x_i}\right)_p
\end{equation}
con $a=(a_1,\dots, a_m)\in\R^m$. Si $X_p=[\alpha]$, donde $\alpha:I\to U$ con $\alpha(0)=0$ se tiene $a=(h\circ \alpha)'(0)$. 

\item Dada $f\in\CC^{\infty}(M;\R)$ tenemos la aplicación tantente
\[
D_pf:T_pM\to T_{f(p)}\R\cong\R.
\]
La \textbf{derivada direccional} $X_pf\in\R$ está definida como la imagen en $\R$ de $X_p$ mediante $D_pf$, es decir, $X_pf=(f\circ\alpha)'(0)$. En términos de $f\circ h^{-1}$ tenemos por la regla de la cadena
\[
X_pf=\frac{d}{dt}(f\circ h^{-1}\circ h\circ \alpha(t))|_{t=0}=\sum_{i=1}^m\parcial{f\circ h^{-1}}{x_i}(h(p))a_i.
\]
En particular 
\[
\left(\parcial{}{x_j}\right)_pf=\parcial{f\circ h^{-1}}{x_i}(h(p)).
\]
\item Tenemos una base similar $(\parcial{}{y_j})_{f(p)}$ ($1\leq j\leq n$) para $T_{f(p)}N$ y la matriz de $D_pf$ con respecto a las bases de $T_pM$ y $T_{f(p)}N$ es la matriz jacobiana en $h(p)$ de $g\circ f\circ h^{-1}$. En particular, cuando $N=M$ y $f=Id_M$ se tiene que
\begin{equation}\label{cambio}
\left(\parcial{}{x_i}\right)_pf=\parcial{\varphi}{x_i}(h(p))\left(\parcial{}{y_j}\right)_p
\end{equation}
donde $\varphi=g\circ h^{-1}$, siendo $(U,h)$ y $(V,g)$ cartas alrededor de $p$. 
\end{enumerate}
\end{nota}

Supongamos que $X$ es una función que a cada $p\in M$ le asigna un vector tangente $X_p\in T_pM$. Dada una carta $(U,h)$, la fórmula \ref{coord} se cumple para unas ciertas funciones $a_i:U\to\R$. Si estas funciones son diferenciables en un entorno de $p$, entonces se dice que $X$ es diferenciable en $p$. Además esta condición es independiente de las cartas por \ref{cambio}. Si $X$ es diferenciable en todo punto $p\in M$ se dice que $X$ es un \textbf{campo de vectores diferenciable} en $M$. 

Consideremos familias $w=\{w_p\}_{p\in M}$ de $k$-formas alternadas en $T_pM$, donde $w_p\in Alt^k(T_pM)$. Necesitamos la noción de deferenciabilidad en $p$ de $w$. Sea $g:W\to M$ una parametrización local (la inversa de una carta) donde $W\subseteq\R^m$ es abierto. Para $x\in W$,
\[
D_x g:\R^m\to T_{g(x)}M
\]
es un isomorfismo e induce un isomorfismo
\[
Alt^k(D_xg):Alt^k(T_{g(x)}M)\to Alt^k(\R^m).
\]

Definimos $g^*(w):W\to Alt^k(\R^m)$ como la función cuyo valor en $x$ es
\[
g^*(w)_x=Alt^k(D_xg)(w_{g(x)}).
\]
Recordemos que para $k=0$, $g^*(w)=w_{g(x)}$. 

\begin{defi}
Una familia $w=\{w_p\}_{p\in M}$ de $k$-formas alternadas en $T_pM$ se dice \textbf{diferenciable} si $g^*(w)$ es una función diferenciable para toda parametrización loca. El conjunto de tales familias diferenciables es un espacio vectorial denotado $\Omega^k(M)$. En particular, $\Omega^0(M)=\CC^{\infty}(M;\R)$. 
\end{defi}

\begin{lemma}
Sea $g_i:W_i\to N$ una familia de parametrizaciones locales con $N=\bigcup g_i(W_i)$. Si $g_i^*(w)$ es diferenciable para todo $i$, entonces $w$ es diferenciable.
\end{lemma}
\begin{proof}
Sea $g:W\to N$ una parametrización local cualquiera y $z\in W$. Probamos que $g^*(w)$ es diferenciable entorno a $z$. Elegimos $i$ con $g(z)\in g_i(W_i)$. En un entorno de $z$ podemos escribir $g=g_i\circ g_i^{-1}\circ g\circ h$, donde $h=g_i^{-1}\circ g:g^{-1}(g_i(W_i))\to W_i$ es una aplicación diferenciable entre abiertos euclídeos. Entonces
\[
g^*(w)=(g_i\circ h)^*(w)=h^*(g_i^*(w))
\]
en un entorno de $z$ y el lado derecho es diferenciable por hipótesis. 
\end{proof}

La diferencial exterior
\[
d:\Omega^k(M)\to\Omega^{k+1}(M)
\]
se puede definir a través de las parametrizaciones locales $g:W\to M$ como sigue. Si $w=\{w_p\}_{p\in M}$ es una $k$-forma diferenciable en $M$ entonces
\[
d_pw=Alt^{k+1}((D_xg)^{-1})\circ d_x(g^*w),\quad p=g(x).
\]
No es inmediato que $d_pw$ no dependa de la elección de $g$. Dada parametrización local $g$, entonces cualquier otra tiene localmente la forma $g\circ\phi$ con $\phi:U\to W$ difeomorfismo. Sean $z_1,\dots, z_{k+1}\in T_pM$. Elegimos $v_1,\dots, v_{k+1}\in\R^n$ de modo que $D_x(g\circ \phi)(v_i)=z_i$. Debemos probar que
\[
d_yg^*(w)(w)(w_1,\dots, w_{k+1})=d_x(g\circ\phi)^*(w)(v_1,\dots,v_{k+1})
\]
donde $\phi(x)=y$ y $D_x\phi(v_i)=w_i$ ESTO NO LO ENTIENDO . Esto se sigue de las ecuaciones 
\begin{gather*}
(g\circ\phi)^*(w)=\phi^*(g^*(w))\\
d\phi^*(\tau)=\phi^*d(\tau),
\end{gather*}
donde $\tau=g^*(w)$. Es evidente que $d\circ d=0$, con lo que tenemos definido un complejo de cadenas. Tenemos que $\Omega^k(M)=0$ si $k>\dim(M)$ pues $Alt^k(T_pM)=0$ cuando $k>\dim(T_pM)$. Una aplicación diferenciable $\phi:M\to N$ induce un morfismo de complejo de cadenas $\phi^*:\Omega^*(N)\to\Omega^*(M)$
\[
\phi^*(\tau)_p=Alt^k(D_p\phi)(\tau_{\phi(p)}),\quad \tau\in \Omega^k(M),
\]
siendo $\phi^*(\tau)_p=\tau_{\phi(p)}$ para $k=0$. Se define el producto bilineal para $w\in\Omega^k(M)$ y $\tau\in\Omega^l(M)$, $w\land\tau$ como $(w\land\tau)_p)=w_p\land\tau_p\in\Omega^{k+l}(M)$.

Se prueba usando parametrizaciones locales que $\phi^*w$ y $w\land\tau$ son diferenciables PROBARLO. Es también fácil de ver que
\[
d(w\land\tau)=dw\land\tau+(-1)^kw\land d\tau
\]
\[
w\land\tau=(-1)^{kl}\tau\land w.
\]

\begin{defi}
La $p$-ésima \textbf{cohomología de deRham} de una variedad $M$, denotada $H^p(M)$, es la $p$-ésima cohomología de $\Omega^*(M)$. 
\end{defi}

La diferencial exterior induce un producto $H^p(M)\times H^q(M)\to H^{p+q}(M)$ que convierte a $H^*(M)$ en un álgebra graduada. Nótese que $H^p(M)=0$ para $p>n=\dim(M)$ y para $p<0$. 

El morfismo de complejos de cadenas $\phi^*$ inducido por una aplicación diferenciable $\phi:M\to N$ induce aplicaciones lineales
\[
\phi^*=H^p(\phi):H^p(N)\to H^p(M),
\]
lo cual convierte a la cohomología de deRham en un functor contravariante entre la categoría de variedades diferenciables y la categoría de $\R$-álgebras graduadas anticonmutativas.

\begin{defi}\
\begin{enumerate}
\item Una variedad diferenciable $M$ de dimensión $n$ se dice \textbf{orientable} si existe $w\in\Omega^n(M)$ con $w_p\neq 0$ para todo $p\in M$. Una tal $w$ se denomina \textbf{forma de orientación} de $M$.
\item Dos formas de orientación $w$ y $\tau$ de $M$ son equivalentes si $\tau=fw$ para alguna $f\in\Omega^0(M)$ con $f(p)>0$ para todo $p\in M$. Una \textbf{orientación} en $M$ es una clase de equivalencia de formas de orientación de $M$. 
\end{enumerate}
\end{defi}

En el espacio euclídeo $\R^n$ tenemos la forma de orientación $dx_1\land\dots\land dx_n$, que representa la \textbf{orientación estándar} de $\R^n$. 

Sea $M$ orientada con una forma de orientación $w$. Una base $b_1,\dots, b_n$ de $T_pM$ se dice \textbf{positivamente} o \textbf{negativamente  orientada} con respecto a $w$ dependiendo del signo de $w_p(b_1,\dots, b_n)\in\R$ (no puede ser 0 porque $w_p\neq 0$). El signo depende solo de la orientación determinada por $w$. Si $\tau$ es otra orientación, $\tau=fw$ para una unívocamente determinada $f\in\Omega^0(M)$ con $f(p)\neq 0$ para todo $p\in M$. Decimos que $w$ y $\tau$ determinan la misma orientación en $p$ si $f(p)>0$. Equivalentemente, $w$ y $\tau$ inducen las mismas bases positivamente orientadas de $T_pM$. Si $M$ es conexa, entonces $f$ tiene signo constante en $M$, así que se tiene:

\begin{lemma}
En una variedad diferenciable orientable conexa hay exactamente dos orientaciones.
\end{lemma}

Si $U$ es un abierto de una variedad diferenciable $M$, entonces una orientación de $U$ estará inducida por la restricción de la forma de orientación de $M$ a $U$. Recíprocamente se tiene:

\begin{lemma}
Sea $V=(V_i)_{i\in I}$ un recubrimiento por abiertos de la variedad diferenciable $M$. Supongamos que todos los $V_i$ tienen orientaciones y que en las intersecciones estas orientaciones coinciden. Entonces $M$ tiene una orientación cuya restricción a cada $V_i$ coincide con la orientación original de $V_i$. 
\end{lemma}


\end{document}
