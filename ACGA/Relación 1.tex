\documentclass[twoside]{article}
\usepackage{../estilo-ejercicios}

%--------------------------------------------------------
\begin{document}

\title{Algebra Conmutativa y Geometría Aplicada}
\author{Rafael González López, Javier Aguilar Martín}
\maketitle

\begin{ejercicio}{1}
Describir todos los subconjuntos algebraicos de $\mathbb{A}_k^1$.
\begin{solucion}
Quitando el caso $X=\mathbb{A}_k^1$, veamos que el resto de conjuntos algebraicos son precisamente los conjuntos finitos. Sabemos que los conjuntos finitos son siempre algebraicos. Si existiese un conjunto algebraico infinito $X$, entonces habría de existir $f \in k[x]$ tal que $V(f)=X$. Además $f$ habría de anularse en cada punto de $X$, pero como $X$ es infinito, esto implica que $f\equiv 0$ y $X=\mathbb{A}_k^1$.
\end{solucion}
\end{ejercicio}

\newpage 
\begin{ejercicio}{2}
Probar que en $\mathbb{A}^1_k$ con la topología de Zariski, dos abiertos no vacíos tienen intersección no vacía (todo abierto es denso).
\begin{solucion}
Por reducción al absurdo, supongamos que $\exists A,B$ abiertos no vacíos de $\mathbb{A}_k^1$ con la topología de Zariski, tales que $A\cap B =\emptyset$. Esto es equivalente a que $A^c \cup B^c = \mathbb{A}_k^n$. Podemos suponer que $A\neq B$, pues el otro caso es trivial. Sabemos además que existen $f,g\in k[x]$, $f\neq g$, polinomios tales que $\V(f) = A^c$ y $\V(g)=B^c$. En tal caso $\V(f)\cup \V(g)=\V(fg) = \mathbb{A}^1_k$, pero esto solo puede ser si $fg = 0$ $\forall x\in \mathbb{A}^1_k$, pero al estar en un cuerpo algebraicamente cerrado (y por tanto infinito), $fg\equiv 0$. Dado que $k[x]$ es dominio, se tiene que, o bien $f\equiv 0$, o bien $g \equiv 0$. Sin pérdida de generalidad, supongamos que es $f$. Entones $A^c = \V(f) = \mathbb{A}_k^n$, luego $A=\emptyset$, lo cual contradice las hipótesis.
\end{solucion}
\end{ejercicio}

\newpage 
\begin{ejercicio}{3}
Probar que $\I(X)$ es un ideal radical de $k[\xn{n}]$.
\begin{solucion}
Que $\I(X)$ es una ideal es claro, pues sean $f,g \in \I(X)$ entonces $f(x)+g(x)=0$ $\forall x\in X$. Además, si $f\in \I(X)$ y $g \in k[\xn{x}]$, entonces $f(x)g(x)=0$ $\forall x\in X$. El hecho de ser radical se desprende de que si $f^n\in \I(X)$ entonces $f^n(x) =0$ $\forall x \in X$, pero como estamos en un dominio, se tiene que $f(x)=0$ $\forall x\in X$, luego $f\in \I(X)$.
\end{solucion}
\end{ejercicio}	


\newpage 
\begin{ejercicio}{4}
Sea $S\subset k[\xn{n}]$. Probar que $\sqrt{\langle{S}\rangle}\subset \I(\V(S))$.
\begin{solucion}
Sea $f\in \sqrt{\langle{S}\rangle}$. Entonces $\exists n \in \N$ tal que $f^n \in \gene{S}$. En particular, sabemos que $\exists f_1,\dotsc,f_k\in S$, $g_1,\dotsc,g_k \in k[\xn{n}]$ tales que
$$
f^n = f_1 g_1 + \dotsc + f_k g_k 
$$
Dado que $f_i\in S$, $f_i(x)=0$ $\forall x \in \V(S)$. Por tanto, $f^n(x) = 0$ $\forall x \in \V(s)$. Por definición
$$
\I(\V(S)) =  \{ h \in k[\xn{n}] \mid h(x)=0 \; \forall x\in \V(S)\}
$$
Por tanto, $f^n \in \I(\V(S))$. Pero por el Ejercicio 3, sabemos que $\I(\V(S))$ es un ideal radical, luego $f\in \I(\V(S))$, como queríamos probar.
\end{solucion}
\end{ejercicio}

\newpage 
\begin{ejercicio}{5}
Recíprocamente, probar que toda $k$-álgebra finitamente generada
y reducida es isomorfa al anillo de coordenadas de algún conjunto algebraico $X \subset \mathbb{A}^n_k$ ara algún $n$.\begin{solucion}
\end{solucion}
\end{ejercicio}

\newpage 
\begin{ejercicio}{6}
Probar que $\mathfrak{m}_a$ es un ideal maximal de $k[\xn{n}]$.
\end{ejercicio}
\begin{solucion}
Sea $a\in \mathbb{A}^n_k$, $\V(\mathfrak{m}_a)=\{a\}$. Al ser un conjunto unitario, es el conjunto algebraico no vacío más pequeño posible, pues solo contiene al vacío. Como el operador $\I$ invierte las inclusiones, se tiene que $I(\{a\})=\I(\V(\mathfrak{m}_a))=\mathfrak{m}_a$ es maximal. En efecto, si existiera $I\supsetneq\mathfrak{m}_a$, entonces $\V(I)\subsetneq \V(\mathfrak{m}_a)$, por lo que $\V(I)=\emptyset$, lo cual ocurre si y solo si $I=\mathbb{A}^n_k$. Esto prueba además que todos los ideales máximales son de esa forma, pues los ideales maximales se corresponden biunívocamente con los conjuntos unitarios.
\end{solucion}

\newpage 
\begin{ejercicio}{7}
Sea $Z ⊆ \mathbb{A}^n_k$ una variedad, y $X, Y ⊆ \mathbb{A}^n_k$ conjuntos algebraicos.
Si $Z ⊆ X ∪ Y$, probar que $Z ⊆ X$ o $Z ⊆ Y$.
\end{ejercicio}
\begin{solucion}
Si $Z$ es variedad, entonces es irreducible. $Z=Z\cap( X\cup Y)=(Z\cap X)\cup(Z\cap Y)$, luego por ser $Z$ irreducible, o bien $Z\cap X=Z$ o bien $Z\cap Y=Z$, de donde se deduce el resultado. Por inducción se prueba para cualquier unión finita.
\end{solucion}

\newpage
\begin{ejercicio}{12} Sean $S_1 ⊆ S_2 ⊆ k[\xn{n}]$. Probar que $\V(S_2) ⊆ \V(S_1)$. ¿Es cierto el recíproco?
\end{ejercicio}
\begin{solucion}
\begin{gather*}
\V(S_2) = \{x\in \mathbb{A}_k^n \mid f(x) = 0 \; \forall f \in S_2\} =  \\
\{x\in \mathbb{A}_k^n \mid f(x) = 0 \; \forall f \in S_1\}\cap  \{x\in \mathbb{A}_k^n \mid f(x) = 0 \; \forall f \in S_2\setminus S_1\} = V(S_1)\cap V(S_2\setminus S_1)
\end{gather*}
El recíproco es falso en general. Para ver un contraejemplo consideremos en $\mathbb{A}_k^1$ los conjuntos $S_1 = \{0\}$ y $S_2 = \{x\}$. Se tiene que $V(S_2) = \{0\} \subset \V(S_1)=\mathbb{A}_k^1$, pero $S_1 \not\subset S_2$.
\end{solucion}
\newpage


\begin{ejercicio}{23}
 Sea $X\subseteq \mathbb{A}^2_\C$ la curva definida por la ecuación $x^a=y^b$, donde $a,b$ son enteros positivos. Probar que $X$ es irreducible si y sólo si $a$ y $b$ son primos entre sí (\emph{Ayuda: } Probar que el homomorfismo $\C[x,y]/\langle x^a-y^b\rangle\to\C[t]$ dado por $x\mapsto t^b$, $y\mapsto t^a$ es inyectivo si $a$ y $b$ son primos entre sí).
\end{ejercicio}
\begin{solucion}
Si $(a,b)\neq 1$ entonces $\exists n,k,k'\in\N$ tales que $a=nk$, $b=nk'$ y $(k,k')=1$. Entonces claramente $X$ no es irreducible puesto que, en ese caso, $\langle x^a-y^b\rangle$ debería ser primo y 

Si $(a,b)=1$, veamos que el homomorfismo del enunciado es inyectivo. Si $P(x,y)$ tal que $\phi(P(x,y)) = P(t^b,t^a) = 0$. Multiplicando por $t$
\end{solucion}

\newpage

\begin{ejercicio}{27}
 Si $V\subseteq W\subseteq\mathbb{A}^n_k$ son dos variedades de la misma dimensión, probar que $V=W$.
\end{ejercicio}
\begin{solucion}
Supongamos que la contención es estricta y sea una cadena creciente de variedades algebraicas de longitud $\dim(V)=\dim(W)=r$
$$
Z_0 \subset Z_1 \subset \dotsc \subset Z_r = V
$$
Como $V$ es variedad algebraica contenida estrictamente en $W$ podemos considerar
$$
Z_0 \subset Z_1 \subset \dotsc \subset V \subset W
$$
Pero entonces $\dim(W)>r$ lo cual es una contradicción con las hipótesis.
\end{solucion}
\newpage

\begin{ejercicio}{30}
 Sea $Z\subseteq\mathbb{A}^n_k$ una variedad, y $f,g:Z\to k$ dos funciones regulares. Si existe un abierto no vacío $U\subseteq Z$ tal que $f_{|U}=g_{|U}$, probar que $f=g$. ¿Es cierto si $Z$ es un conjunto algebraico arbitrario?
\end{ejercicio}
\begin{sol}
Si $V$ no es una variedad entonces el resultado no es necesariamente cierto. Si $X=V(x^2-1)$ y $U=\{1\}$, $f(x)=(x-1)$ y $g(x)=x(x-1)$. 
\end{sol}
\end{document}