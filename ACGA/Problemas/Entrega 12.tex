\documentclass[twoside]{article}
\usepackage{../../estilo-ejercicios}
\DeclareMathOperator{\Ima}{Im}
\DeclareMathOperator{\Char}{char}
%--------------------------------------------------------
\begin{document}

\title{Algebra Conmutativa y Geometría Algebraica}
\author{Rafael González López,Javier Aguilar Martín,  Diego Pedraza López}
\maketitle

\begin{ejercicio}{1}
Sea $A = \bigoplus_{n≥0} A_n$ un anillo graduado.
\begin{enumerate}
	\item Probar que $A^{+k} := \bigoplus_{n≥k} A_n$ es un ideal de $A$ para todo $k ≥ 1$.
	\item Probar que $A$ es noetheriano si y sólo si $A_0$ es noetheriano y $A$ es una $A_0$-álgebra finitamente generada.
\end{enumerate}
\end{ejercicio}
\begin{solucion}\mbox{}
\begin{enumerate}
	\item Sean $a,b \in A^{+k}$. Existen $a_i \in A_i$ con $i=k,\dots,m$ tal que $a = \sum_{i=k}^{m} a_i$. También existen $b_j \in A_j$ con $j=k,\dots,n$ tal que $b = \sum_{j=k}^n b_j$. Sin pérdida de generalidad, podemos suponer que $m≥n$. Para todo $j$ tal que $n≥j>m$, podemos tomar $b_j = 0 \in A_j$. Entonces podemos escribir $b = \sum_{i=k}^m b_i$. Finalmente, como $A_i$ es subgrupo aditivo, $a_i+b_i \in A_i$ para todo $i$, luego:
	\[ a + b = \sum_{i=1}^m (a_i + b_i) \in A^{+k}\]
	Sea $a \in A^{+k}$ y $b \in A$. Existen $a_i \in A_i$ con $i=k,\dots,m$ tal que $a = \sum_{i=k}^{m} a_i$. Además existen $b_j \in A_j$ con $j=0,\dots,n$ tal que $b = \sum_{j=0}^{n} b_j$. Como $A_i \cdot A_j \subseteq A_{i+j}$, tenemos que $a_i \cdot b_j \in A_{i+j}$. Tenemos que $i+j ≥ k$ para todo $i≥k$ y $j≥0$, luego $a_i \cdot b_j \in A^{+k}$. Entonces:
	\[ ab = \sum_{i=k}^{m} \sum_{j=0}^{n} a_i b_j \in A^{+k} \]

	Luego $A^{+k}$ es ideal de $A$ para todo $k≥0$.

	\item Supongamos primero que $A$ es noetheriano. Tenemos entonces que $A_0 \cong A/A^{+1}$. Como $A$ es noetheriano y $A^{+1}$ es ideal, $A/A^{+1}$ es noetheriano. Cogemos un sistema generador finito de elementos homogéneos de $A^{+1}=\langle a_1,\dots,a_n \rangle$ de grados $d_1,\dots,d_n$ respectivamente. Sea $\tilde{A}=A_0[a_1,\dots,a_n]$ un $A_0$-álgebra. Veamos que $\tilde{A}=A$. Claramente $\tilde{A} \subseteq A$.

	Demostraremos por inducción que $A_n \subseteq \tilde{A}$ para todo $n ≥ 0$, lo que implica que $A = \bigoplus_{i≥0} A_n \subseteq \tilde{A}$. 
	Para $n=0$ está clara la inclusión. Para $n≥1$, supuesta la hipótesis como cierta para $0,\dots,n-1$, tomamos un $x \in A_n$. Como $x \in A^{+1}$, existen $x_i \in A$ tal que $x = \sum x_i a_i$.
	Como $x$ es elemento homogéneo de grado $n$, tenemos que todas las componentes homogéneas de grado distinto de $n$ se anulan, luego podemos quedarnos con la componente homogénea $\tilde{x}_i$ de $x_i$ de grado $\max(n-d_i,0)$. Entonces $\tilde{x}_i \in M_{\max(n-d_i,0)}$. Como $d_i > 0$ podemos usar la hipótesis de inducción de manera que $x_i \in \tilde{A}$, luego $x=\sum x_ia_i \in \tilde{A}$.

	Luego $A=\tilde{A}$ y, en consecuencia, $A$ es un $A_0$-álgebra finitamente generada.

	Supongamos ahora que $A_0$ es noetheriano y $A$ es una $A_0$-álgebra finitamente generada. Entonces $A=A_0[a_1,\dots,a_n]$ para $a_1,\dots,a_n \in A$. Como $A_0$ es noetheriano, por el teorema de la base de Hilbert, $A$ es noetheriano.
\end{enumerate}
\end{solucion}

\newpage

\begin{ejercicio}{2}
 
Sean $X, Y ⊆ \mathbb{P}^n_k$
dos conjuntos algebraicos de dimensión $d$ tales
que $\dim(X ∩Y ) < d$. Probar que el grado de $X ∪Y$ es la suma de los grados de
$X$ e $Y$ . Deducir que el grado de un conjunto finito formado por $d$ puntos es $d$.
\end{ejercicio}

\begin{solucion}\
\begin{itemize}
\item Vamos a tratar primero el caso en el que $X$ e $Y$ sean irreducibles. Sea $I_1=\I(X),  I_2=\I(Y), I=\I(X\cup Y)$. Entonces, como $I_1\cap I_2=I$ y la dimensión de la intersección es menor que la de los conjuntos por separado, tenemos las siguiente sucesión exacta
\[
0\to R/I\to R/I_1\oplus R/I_2\to R/(I_1+I_2)\to 0
\]

Por el primer apartado del ejercicio 4 de esta relación tenemos que $\varphi_{R/I}+\varphi_{R/(I_1+I_2)}=\varphi_{R/I_1\oplus R/I_2}$ y por el segundo apartado de dicho ejercicio $\varphi_{R/I_1\otimes R/I_2}=\varphi_{R/I_1} + \varphi_{R/I_2}$, o lo que es lo mismo,
\[
\varphi_{X\cup Y}+\varphi_{X\cap Y}=\varphi_{X} + \varphi_{Y}
\]
Como $\dim(X)=\dim(Y)$, los dos polinomios del lado derecho tienen el mismo grado, por lo que el término líder de la suma es $\deg(X)/d!+\deg(Y)/d!$. Por otro lado, como la intersección tiene dimensión menor, entonces el coeficiente líder del polinomio de la izquierda es del de $\varphi_{X\cup Y}$, que es simplemente $\deg(X\cup Y)/d!$, por lo que tenemos la igualdad $\deg(X\cup Y)=\deg(X)+\deg(Y)$.


Supongamos ahora que $X$ se descompone en componentes irreducibles como $X_1\cup\dots\cup X_r$, y supongamos que $X_1,\dots, X_i$ con $i\leq r$ son de dimensión máxima $d$. Análogamente $Y=Y_1\cup\dots\cup Y_s$ con $Y_1,\dots, Y_k$, $k\leq s$, de dimensión $d$. Sean $I=\I(X\cup Y)$, $I_j=\I(X_j)$ y $J_j=\I(Y_j)$. Aplicamos un razonamiento análogo anterior para tener la sucesión exacta 

\[
0\to R/I\to \bigoplus_{j=1}^r R/I_j \oplus \bigoplus_{j=1}^s R/J_j\to R/(\sum_{j=1}^r I_j+\sum_{j=1}^s J_j)\to 0
\]

Tendríamos entonces que $\varphi_{R/I}+\varphi_{R/(\sum I_j+\sum J_j)}=\varphi_{ \bigoplus R/I_j \oplus \bigoplus R/J_j}=\sum_{j=1}^r\varphi_{R/I_j}+\sum_{j=1}^s \varphi_{R/J_j}$, por lo que 
\[
\varphi_{X\cup Y}+\varphi_{\bigcap X_j\cap \bigcap Y_j}=\sum_{j=1}^r\varphi_{X_j}+\sum_{j=1}^s \varphi_{Y_j}
\]
En el lado derecho, el coeficiente líder vendrá dado por las componentes de máxima dimension, es decir, $\frac{1}{d!}(\sum_{j=1}^i\deg(X_j)+\sum_{j=1}^k\deg(Y_j))$. En en lazo izquierdo, $\bigcap_{j=1}^r X_j\cap \bigcap_{j=1}^s Y_j=X\cap Y$, que tiene dimensión estrictamente menor que $d$, luego no afecta al grado del polinomio, así que obtenemos que $\frac{1}{d!}\deg(X\cup Y)=\frac{1}{d!}(\sum_{j=1}^i\deg(X_j)+\sum_{j=1}^k\deg(Y_j))$, es decir, que $\deg(X\cup Y)=\sum_{j=1}^i\deg(X_j)+\sum_{j=1}^k\deg(Y_j)$. Ahora, bien, como los $X_j$ y los $Y_j$ son irreducibles, $\sum_{j=1}^i\deg(X_j)=\deg (\bigcup_{j=1}^i X_j)$ y $\sum_{j=1}^k\deg(Y_j)=\deg (\bigcup_{j=1}^k Y_j)$. 

Como hemos visto, las componentes de menor dimensión no alteran el grado, luego $\deg (\bigcup_{j=1}^i X_j)=\deg (\bigcup_{j=1}^r X_j)=\deg(X)$ y de igual modo $\deg (\bigcup_{j=1}^k Y_j)=\deg (\bigcup_{j=1}^s Y_j)=\deg(Y)$. En efecto, si tenemos una variedad $H$ y otra variedad $L$ de dimensión estrictamente menor que $H$, realizando el mismo argumento de la sucesión exacta obtendríamos que

\[
\varphi_{H\cup L}+\varphi_{H\cap L}=\varphi_{H}+ \varphi_{L}
\]
Como $\dim(H\cap L)\leq\dim(L)<\dim(H)$, tendríamos que $\deg(H\cup L)=\deg(H)$, con lo que finalmente hemos obtenido que $\deg(X\cup Y)=\deg(X)+\deg(Y)$, como queríamos demostrar.


\item Dado $X=\{x_1,\dots, x_d\}$ un conjunto finito de $d$ puntos, podemos descomponerlo en conjuntos de la forma $X_i=\{x_i\}\ \forall i=1,\dots, d$. Entonces, $\dim((X\setminus X_d)\cap X_d)=\dim(\emptyset)<0$, luego el grado de $X$ es el grado de $X\setminus X_d$ más el grado de $X_d$. Sucesivamente podemos repetir este proceso hasta encontrar que el grado de $X$ es igual a la suma de los grados de los $X_i$. Como $\dim(X_i)=0$, el grado de $X_i$ será el número de puntos de corte de $X_i$ con una variedad lineal genérica de dimensión $n$, que como estamos en $\mathbb{P}^n_k$ será necesariamente 1 punto. Así que como cada conjunto unitario tiene grado 1, en virtud del resultado anterior, $X$ tiene grado $d$. Lo cual además está respaldado por un argumento de intersección similar al anterior, puesto que $\dim(X)=0$ e intersecará con $\PP^n_k$ en precisamente $d$ puntos. 


\end{itemize}
\end{solucion}

\end{document}