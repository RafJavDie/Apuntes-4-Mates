\documentclass[twoside]{article}
\usepackage{../../estilo-ejercicios}
%\DeclareMathOperator{\Ima}{Im}

%--------------------------------------------------------
\begin{document}

\title{Algebra Conmutativa y Geometría Algebraica}
\author{Javier Aguilar Martín, Rafael González López, Diego Pedraza López}
\maketitle

\begin{ejercicio}{1}\
Sean $X, Y,Z$ variedades cuasi-proyectivas, y $\phi : X \dashrightarrow Y$ y $\psi : Y \dashrightarrow Z$ aplicaciones racionales.
\begin{enumerate}
\item Probar que si $(U, \tilde{\phi})$ y $(V, \tilde{\phi}')$ (donde $U, V \subseteq X$ son abiertos densos) son
dos representaciones de $\phi$, entonces $\tilde{\phi}$ es dominante si y solo si $\tilde{\phi}'$ lo es, y
que por tanto la definición de aplicación racional dominante tiene sentido.
\item Probar que si $\phi$ y $\psi$ son dominantes, la composición  $\psi\circ\phi$ está bien definida
(en el sentido de que existen representaciones de $\phi$ y $\psi$ que se pueden
componer, y la composición no depende de las representaciones elegidas)
y que la composición es también dominante.
\end{enumerate}
\end{ejercicio}
\begin{solucion}\
\begin{enumerate}
\item Para probar este apartado se usará el siguiente resultado de topología general.
\begin{lemma} \label{lema1}
Dada una aplicación continua $f:X\to Y$, para cualquier subconjunto $A\subseteq X$, $\overline{f(\overline{A})}=\overline{f(A)}$. 
\end{lemma}
\begin{proof}
Por monotonía de la clausura se tiene $\overline{f(A)}\subseteq\overline{f(\overline{A})}$. Para la otra inclusión, recordemos que dado $A\subseteq X$, $f$ es continua si y solo si $f(\overline{A})\subseteq\overline{f(A)}$, por lo que basta volver a tomar clausura para obtener $\overline{f(\overline{A})}\subseteq\overline{f(A)}$.
\end{proof}
 Por la definición de función racional tenemos que $\tilde{\phi}|_{U\cap V}=\tilde{\phi}'|_{U\cap V}$. Supongamos que $\tilde{\phi}$ es dominante. Entonces $\overline{\tilde{\phi}'(U\cap V)}=\overline{\tilde{\phi}(U\cap V)}=\overline{\tilde{\phi}(\overline{U\cap V})}$, ya que los morfismos son continuos en la topología de Zariski y por tanto podemos aplicar el lema \ref{lema1}. Usando ahora que los abiertos no vacíos de Zariski son densos y la intersección de dos cualesquiera de ellos también es denso, obtenemos $\overline{\tilde{\phi}(\overline{U\cap V})}=\overline{\tilde{\phi}(X)}=Y,$ por ser $\tilde{\phi}$ dominante. En definitiva, $\overline{\tilde{\phi}'(U\cap V)}=Y$, por lo que claramente $\tilde{\phi}'$ es dominante, ya que $\overline{\tilde{\phi}'(V)}\supseteq\overline{\tilde{\phi}'(U\cap V)}=Y$. El recíproco es análogo.
 \newpage
\item Sean $(U,\tilde{\phi}),(V,\tilde{\psi})$ representaciones de $\phi$ y $\psi$ respectivamente. Claramente la composición $\tilde{\psi}\circ\tilde{\phi}$ está bien definida en el abierto denso $W=U\cap\tilde{\phi}^{-1}(V)$ (lo es por ser intersección de abiertos densos, gracias a la continuidad de $\tilde{\phi}$), luego basta tomar las representaciones $(W,\tilde{\phi}|_W),(V,\tilde{\psi})$. Nótese que la composición está bien definida pues $\tilde{\phi}(W)\subseteq \tilde{\phi}(\tilde{\phi}^{-1}( V))\subseteq V$.

Veamos que la composición no depende de las representaciones elegidas. Sean $(U,\tilde{\psi}\circ\tilde{\phi}),(V,\hat{\psi}\circ\hat{\phi})$ dos representaciones de la composición. Tenemos, por ser $\phi$ racional, que $\tilde{\phi}|_{U\cap V}=\hat{\phi}|_{U\cap V}$. En particular esto implica que $\tilde{\phi}(U\cap V)= \hat{\phi}(U\cap V)$. Sean $\tilde{W}$ y $\hat{W}$ dominios abiertos de $\tilde{\psi}$ y $\hat{\psi}$ respectivamente tales que $\tilde{\phi}(U\cap V)\subseteq\tilde{W},\hat{\phi}(U\cap V)\subseteq\hat{W}$. Dichos abiertos existen puesto que ambos morfismos deben estar definidos en abiertos no vacíos, los cuales deben contener a $\tilde{\phi}(U\cap V)=\hat{\phi}(U\cap V)$ puesto que la composición está bien definida.  Entonces, por ser $\psi$ racional, $\tilde{\psi}|_{\tilde{W}\cap\hat{W}}=\hat{\psi}|_{\tilde{W}\cap\hat{W}}$, luego en particular $\tilde{\psi}|_{\tilde{\phi}(U\cap V)}=\hat{\psi}|_{\hat{\phi}(U\cap V)}$, es decir, $$\tilde{\psi}\circ\tilde{\phi}|_{U\cap V}=\hat{\psi}\circ\hat{\phi}|_{U\cap V}$$ como queríamos demostrar.

Para ver que la composición es dominante, sea $(W,\tilde{\psi}\circ\tilde{\phi})$ un representación de $\psi\circ\phi$. Veamos primero el siguiente resultado.

\begin{lemma}\label{lema2}
Un morfismo $g:X\dashrightarrow Y$ es dominante si y solo si $g^{-1}(U)\neq\emptyset\ \forall U\subseteq Y$ abierto denso.
\end{lemma}
\begin{proof}
Supongamos que $g$ es dominante y sea $U$ un abierto denso. Por reducción al absurdo, supongamos que $g^{-1}(U)=\emptyset$. Entonces $\Im{g}\subseteq Y\setminus U$, que es cerrado por ser complementario de un abierto, luego es igual a su clausura, que no es el total.

Recíprocamente, supongamos que $g^{-1}(U)\neq\emptyset\ \forall U\subseteq Y$. Si, $g$ no fuera dominante, entonces existiría un abierto no vacío $V\subseteq Y$ tal que $\Im{g}\subseteq Y\setminus V$, pero entonces $g^{-1}(V)=\emptyset$, lo cual contradice la hipótesis.

\end{proof}

 Sea $U\subseteq Z$ un abierto denso, entonces aplicando el lema \ref{lema2}, $V=\tilde{\psi}^{-1}(U)\neq\emptyset$ por ser $\tilde{\psi}$ dominante y abierto por ser continua.  Por el mismo motivo, $\tilde{\phi}^{-1}(V)$ es un abierto no vacío. En definitiva, $(\tilde{\psi}\circ\tilde{\phi})^{-1}(U)\neq\emptyset$, por lo que la aplicación es dominante utilizando el lema \ref{lema2}.

\end{enumerate}
\end{solucion}

\newpage

\begin{ejercicio}{2} Sea $X\subset \A^3_\C$ la cuádrica definida por la ecuación $x^2 + yz =1$. Construir una aplicación birracional entre $X$ y $\A^2_\C$, probando que $X$ es racional. Determinar dos abiertos $U\subset X$ y $V\subset \A^2_\C$ tales que la aplicación birracional definida anteriormente induzca un isomorfismo entre ellos.
\end{ejercicio}
\begin{solucion}
Vamos a empezar la casa por el tejado. Vamos a ver que existen dos abiertos isomorfos dentro y veremos que este isomorfismo induce una aplicación birracional sobre $X$ y $\A^2_\C$. Vamos a considerar los abiertos $U=X\setminus\V(x(2-y-z))$ y $V=\A^2_\C\setminus \V((yz+1)(y+z))$. Estos abiertos son claramente no vacíos $(1,0,0)\in U$. Ahora vamos a proyectar $U$ sobre $V$ (identificándolo con el hiperplano $H\equiv x=1$ través del punto $P=(0,1,1)\in X\setminus U$. Sea $(x,y,z)$ otro punto de $U$. Consideremos la recta que pasar por este punto y por $P$.
$$
r(\lambda)= (0,1,1)+\lambda (x,y-1,z-1) = (\lambda x,\lambda (y-1),\lambda (z-1))
$$
Si lo intersecamos con nuestro hiperplano $x=1$ tenemos que $\lambda x= 1$. Como $x\neq 0$ $\forall (x,y,z)\in U$ podemos definir la aplicación que a cada $(x,y,z)\in X$ le asigna $(1,\frac{y-1}{x},\frac{z-1}{x})$. Naturalmente esta aplicación es un morfismo entre $U$ y $H$. Componiendo con el isomorfismo $\pi\func{H}{\A^2_\C}$, $\pi (1,y,z) = (y,z)$ tenemos que
$$
\tilde{f}\colon U\rightarrow \A_\C^2 \qquad \tilde{f}(x,y,z) =\left(\frac{y-1}{x},\frac{z-1}{x}\right)
$$
Vamos a probar que $Im(\tilde{f})\subset V$. Sea $(x,y,z)\in U$ entonces tenemos que comprobar que $f(x,y,z)\notin U^c$. Pero $f(x,y,z)\in U^c$ si y solo sí $(x,y,z)$ verifica
$$
\frac{y-1}{x}\frac{z-1}{x} + 1 = 0 \equiv x^2 + (y-1)(z-1) = 0 \equiv 2-y-z = 0
$$
$$
\frac{y-1}{x}+\frac{z-1}{x} = \frac{y+x-2}{x} = 0 \equiv 2-y-x = 0
$$
Lo cual sabemos que no es posible por definición de $U$. Por tanto, podemos considerar
$$
{f}\colon U\rightarrow V \qquad {f}(x,y,z) =\left(\frac{y-1}{x},\frac{z-1}{x}\right)
$$
Solo nos queda probar, pues, que $f$ tiene una inversa. Consideremos ahora la recta que pasa por un punto cualquiera de $\pi^{-1}(V)$ ($(1,y,z)$ con ciertas restricciones sobre $y$ y $z$), y por $P$. 
$$
s(\lambda) = (0,1,1)+\lambda(1,y,z)=(\lambda, 1+\lambda y, 1+ \lambda z)
$$
Tenemos que ver la intersección de $s$ con $X$. Imponiendo la ecuación de la cuádrica tenemos que
$$\lambda^2 + (\lambda y+1)(\lambda z + 1) = \lambda^2 (1+yz)+\lambda(y+z)+1 = 1 \Leftrightarrow \lambda(\lambda(1+yz)+(y+z))$$
Naturalmente el caso $\lambda =0$ nos da el propio punto $P$, luego consideramos $\lambda = -\dfrac{y+z}{1+yz}$, el cuál está bien definido por estar $(y,z)\in V$. Podemos considerar por tanto
$$
\tilde{g}\colon  V \rightarrow X \qquad \tilde{g}(y,z) =\left(-\dfrac{y+z}{1+yz},\dfrac{1-y^2}{1+yz},\dfrac{1-z^2}{1+yz}\right) 
$$
Vamos a ver que $Im(\tilde{g})\subset U$. Para ello, análogamente al caso anterior, vamos a ver que $g(y,z)\notin U^c$. 
$$
-\frac{y+z}{1+yz} = 0 \equiv y+z = 0
$$
$$
2-\frac{1-y^2}{1+yz}-\frac{1-z^2}{1+yz} = 2 + \frac{y^2+z^2-2}{1+yz} =0 \equiv 2(1+yz) +y^2 +z^2 -2 = (y+z)^2 = 0 
$$
Lo cuál no puede ocurrir por definición de $V$. Definiendo ahora
$$
{g}\colon  V \rightarrow U \qquad {g}(y,z) =\left(-\dfrac{y+z}{1+yz},\dfrac{1-y^2}{1+yz},\dfrac{1-z^2}{1+yz}\right) 
$$
Tenemos por tanto dos morfismos bien definidos $f,g$ tales que la imagen del uno es el dominio del otro y recíprocamente. Vamos a comprobar que $f$ y $g$ son inversas

\begin{align*}
(g\circ f)(x,y,z)& = g\left(\frac{y-1}{x},\frac{z-1}{x}\right) =\left(-\dfrac{\frac{z-1}{x}+\frac{y-1}{x}}{1+\frac{y-1}{x}\frac{z-1}{x}},\dfrac{1-\left(\frac{y-1}{x}\right)^2}{1+\frac{y-1}{x}\frac{z-1}{x}},\dfrac{1-\left(\frac{z-1}{x}\right)^2}{1+\frac{y-1}{x}\frac{z-1}{x}}\right) \\
&= \left(-x\frac{y+z-2}{x^2+(y-1)(z-1)},\frac{x^2-(y-1)^2}{x^2+(y-1)(z-1)},\frac{x^2-(z-1)^2}{x^2+(y-1)(z-1)}\right)\\
&=\left(-x\frac{y+z-2}{2 -y-z},\frac{x^2-(y-1)^2}{2 -y-z},\frac{x^2-(z-1)^2}{2 -y-z}\right)\\
&= \left(x,\frac{(1-yz)-(y^2+1-2y)}{2 -y-z},\frac{(1-yz)-(z^2+1-2z)}{2 -y-z}\right)\\
&= \left(x,\frac{-yz-y^2+2y}{2 -y-z},\frac{-yz-z^2+2z}{2 -y-z}\right)=(x,y,z)\\
(f\circ g)(y,z) &= f\left(-\dfrac{y+z}{1+yz},\dfrac{1-y^2}{1+yz},\dfrac{1-z^2}{1+yz}\right) = \left( \frac{\dfrac{1-y^2}{1+yz}-1}{-\dfrac{y+z}{1+yz}},\frac{\dfrac{1-z^2}{1+yz}-1}{-\dfrac{y+z}{1+yz}}\right) \\
&=  \left( \frac{{y^2}+yz}{y+z},\frac{{z^2}+yz}{y+z}\right) =   \left( y\frac{{y}+z}{y+z},z\frac{{z}+y}{y+z}\right)= (y,z)
\end{align*}
Por tanto, hemos probad que $U$ y $V$ son isomorfos. Como $u$ y $V$ son abiertos, $\tilde{f}$ y $\tilde{g}$ son dominantes. Podemos definir
$$
\phi \colon X \dashrightarrow \A_\C^2 \qquad \psi \colon \A_\C^2 \dashrightarrow X
$$
Como las clases de equivalencia de $(U,\tilde{f})$ y $(V,\tilde{g})$ respectivamente. Por definición, $\phi$ y $\psi$ son dominantes. Naturalmente $\phi$ y $\psi$ son aplicaciones racional una inversa de la otra, lo que prueba que $X$ y $\A_\C^2$ son birracionalmente equivalentes y, por tanto, $X$ es racional. Obviamente, una vez construidas estas aplicaciones $\phi$ induce un isomorfismo en su restricción a $U$ sobre su imagen, es decir, la propia función $f$.
\end{solucion}

\newpage

\begin{ejercicio}{4}\emph{(La curva de Fermat)} Probar que la curva $X \subset A^2_\C$ definida
por la ecuación $x^n + y^n = 1$ no es racional si $n \geq 3$ (Ayuda: Sean $P, Q,R \in
\C[t]$ primos entre sí dos a dos tales que $P(t)^n + Q(t)^n = R(t)^n$. Probar que
$R(t)Q'(t) - Q(t)R'(t)$ es un múltiplo de $P(t)^{n-1}$ y $R(t)P'(t) - P(t)R'(t)$ es un
múltiplo de $Q(t)^{n-1}$. Obtener así una contradicción.
\end{ejercicio}
\begin{solucion}
Denotamos $X: x^n+y^n=1$, del cual queremos probar que no es racional. Supongamos por reducción al absurdo que existe $\varphi:\A^1\to X$ birracional, tal que $t\mapsto (r(t)^n+s(t)^n)$ con $r(t)^n+s(t)^n=1$.  Quitando denominadores, obtendríamos $P(t)^n+Q(t)^n=R(t)^n$. Por ser DFU podemos eliminar factores comunes, así que suponemos que $P,Q,R$ son primos dos a dos. Tomamos derivadas
$$P(t)^{n-1}P'(t)+Q(t)^{n-1}Q'(t)=R(t)^{n-1}R'(t)$$
Multiplicando por $R$, 
$$P^{n-1}P'R+Q^{n-1}Q'R=P^nR'+Q^nR'$$
$$P^{n-1}(P'R-PR')=Q^{n-1}(QR'-Q'R)$$
Por lo que $P^{n-1}|QR'-Q'R$ y $Q^{n-1}|P'R-PR'$, lo cual no es posible observando los grados.
\end{solucion}

\newpage

\begin{ejercicio}{5}\
Sean $a$ y $b$ enteros positivos, y $X = \V(x^b-y^a) \subseteq \A_\C^2$. Probar que la aplicación $φ : \A_\C^1 \to X$ dada por $φ(t) = (t^a,t^b)$ es birracional si y sólo si $a$ y $b$ son primos entre sí, y es un isomorfismo si y sólo si $a=1$ ó $b=1$.
\end{ejercicio}

\begin{sol}
Supongamos primero que $φ$ es birracional. Esto sólo tiene sentido si $X$ es irreducible. Entonces $\langle x^b-y^a \rangle$ es un ideal primo, es decir, $x^b-y^a$ es un polinomio irreducible. Sea $\text{mcd}(a,b)=c$. Tomamos $z=x^{b/c}$ y $w=y^{a/c}$. Entonces el polinomio $x^b-y^a$ es equivalente a $z^c-w^c$. Como $z^c-w^c=(z-w)\cdot(z^{c-1}+z^{c-2}w^1+\cdots+z^1w^{c-2}+w^{c-1})$, tenemos que $x^{b/c}-y^{a/c}$ divide a $x^b-y^a$. Como $x^b-y^a$ es irreducible, debe cumplirse que $c=1$. Es decir, $a$ y $b$ deben ser primos entre sí.

Supongamos ahora que $a$ y $b$ son primos, entonces por la identidad de Bezout existen $c$ y $d$ tal que $ac+bd=1$. Sea $ψ : X \dashrightarrow \A_C^1$ dada por:
\[ ψ(x,y) = x^cy^d \]
Ya que si $c<0$ ó $d<0$, $ψ$ no está definida en los puntos $x=0$ y $y=0$. Como en $X$ se tiene que $x=0$ si y sólo si $y=0$, el único punto de $X$ donde $ψ$ no está definida es $(0,0)$. Como $ψ$ está definida al menos en el abierto $X\setminus(0,0)$ y además:
\[ (ψ \circ φ)(t) = ψ(t^a,t^b) = t^{ac}t^{bd}=t^{ac+bd}=t \]
Por otro lado, usando que en $X$, $x^b=y^a$:
\[ (φ \circ ψ)(x,y) = φ(x^cy^d) = ((x^cy^d)^a,(x^cy^d)^b) = (x^{ac}y^{ad},x^{bc}y^{bd}) = (x^{ac}x^{bd},y^{ac}y^{bd}) = (x,y) \]
Luego $φ$ es birracional.

Para que $φ$ sea isomorfismo, basta asegurar que $ψ$ esté definida en todo punto de $X$.
\begin{itemize}
	\item Si $a=1$ ó $b=1$: Supongamos sin pérdida de generalidad que $a=1$, entonces $c=1$ y $d=0$. En este caso definimos $ψ(x,y)=x$, que está definida en todo punto. Entonces $φ$ es isomorfismo.
	\item Si $a>1$ y $b>1$: En este caso, para que se pueda cumplir la fórmula de Bezout, $c<0$ ó $d<0$. En este caso $ψ$ no está definida en $(0,0)$ y $φ$ no es isomorfismo.
\end{itemize}
Por lo tanto, $φ$ es isomorfismo si y sólo si $a=1$ ó $b=1$.
\end{sol}
\end{document}
