\documentclass[twoside]{article}
\usepackage{../../estilo-ejercicios}

%--------------------------------------------------------
\begin{document}

\title{Algebra Conmutativa y Geometría Aplicada}
\author{Javier Aguilar Martín, Rafael González López}
\maketitle

\begin{ejercicio}{1}\
\begin{itemize}

\item[1)] Probar el Teorema de la Base de Hilbert: si $A$ es un anillo
noetheriano, entonces el anillo de polinomios $A[x]$ también lo es.

\item[2)] Probar que el anillo $A$ formado por las funciones $f : \R \to \R$ continuas
no es noetheriano.
\end{itemize}
\end{ejercicio}
\begin{solucion}\
\begin{itemize}


\item[1)]Sea $I\subseteq A[x]$ un ideal. Vamos a probar que está finitamente generado. 

Sea $f_1\in I$ un elemento de grado mínimo. Para $i\geq 1$, si tenemos el ideal $\langle f_1,\dots, f_i\rangle$, elegimos $f_{i+1}$ de grado mínimo en $I\setminus\langle f_1,\dots, f_i\rangle$. Si para algún $i$, $I=\langle f_1,\dots, f_i\rangle$, ya hemos terminado. Si no, sea $a_j$ el coeficiente líder de $f_j$. El conjunto de coeficientes líderes forma un ideal en $A$, y como $A$ es noetheriano por hipótesis, el ideal $\langle a_1,a_2,\dots\rangle$ está finitamente generado por $a_1,\dots,a_m$ para algún $m\in\N$. Se va a probar que $f_1,\dots,f_m$ generan $I$.

Supongamos que no. Entonces elegimos como explicamos anteriormente $f_{m+1}$, cuyo coeficiente líder es $a_{m+1}=\sum_{j=1}^m u_j a_j$ para ciertos $u_j\in A$. Dado que el grado de $f_{m+1}$ es mayor o igual que el grado de $f_j$ $\forall j=1,\dots, m$, el polinomio
$$g=\sum_{j=1}^m u_j f_j x^{\deg{f_{m+1}}-\deg{f_j}}\in\langle f_1,\dots,f_m\rangle$$
tiene el mismo término líder (grado y coeficiente) que $f_{m+1}$. Por tanto, la diferencia $f_{m+1}-g$ tiene grado estrictamente menor que $f_{m+1}$ a pesar de no estar en $\langle f_1,\dots,f_m\rangle$, lo cual contradice la elección de $f_{m+1}$. Con lo cual, $I=\langle f_1,\dots,f_m\rangle$ está finitamente generado.

\item[2)] Denotemos por $\mathcal{C}$ al anillo de funciones continuas. Sea $I_n=\{f\in\mathcal{C}\mid f(x)=0\ \forall x\geq n\}$. Para cada $n\in\N$, $I_n$ es un ideal, pues el producto y la suma de funciones continuas que se anulan en un conjunto fijo sigue anulándose en el mismo conjunto. Sin embargo, la cadena $I_0\subset I_1\subset\cdots\subset I_n\subset\cdots$ no termina, pues $I_{n}\subsetneq I_{n+1}\ \forall n\in\N$. Por la caracterización de anillo noetheriano esto significa que el anillo de las funciones continuas no es noetheriano. 
\end{itemize}
\end{solucion}

\newpage

\begin{ejercicio}{2} Demostrar el siguiente resultado usado en la prueba del Nullstellensatz: si $k$ es un cuerpo y $k \subset K$ una extensión de cuerpos que es finitamente generada como $k$-álgebra, entonces $K$ es una extensión finita (es decir, finitamente generada como $k$-espacio vectorial) de $k$.
\begin{solucion}
Sea $K$ en las condiciones del enunciado, entonces $\exists a_1,\dotsc,a_n \in K$ de forma que $K=k[a_1,\dotsc,a_n]$, es decir, es finitamente generado como $k$-álgebra. 
\begin{lema} Si $K=k[a_1,\dotsc,a_n]$ es una extensión algebraica de $k$, entonces es una extensión finita de $k$.
\end{lema}
\begin{dem}
Podemos verlo fácilmente por inducción. El caso $K=k[a_1]$ es trivial, pues si $K$ es algebraico, $a_1$ lo es en particular, por lo que existe $f\in k[X]$ tal que $f(a_1)=0$. Si $f$ tiene grado $d$, es sencillo ver que $\{1,a_1,\dotsc,a_1^{d-1}\}$ genera $k[a_1]$. Ahora basta aplicar la hipótesis de inducción teniendo en cuenta que $k[a_1,\dotsc,a_n]=k[a_1,\dotsc,a_{n-1}][a_n]$ y que si $a_i$ es algebraico entonces $k[a_i]$ es cuerpo.
\end{dem}

Vamos a probar que $K$ es necesariamente una extensión algebraica. Por reducción al absurdo, supongamos que $K$ no es una extensión algebraica de $k$, es decir, existen elementos trascentendes sobre $k$. Obviamente, al menos uno de los generadores de $K$ ha de ser trascentente sobre $k$ (supongamos que es $a_1$), entonces el anillo $k[a_1]$ puede verse como un anillo de polinomios sobre un cuerpo, por lo que es además DFU.

\begin{lema}El anillo $k[a_1]$ contiene infinitos elementos primos.
\end{lema}
\begin{dem}
Basta usar el argumetno de Euclides. Sabemos que el conjunto de los primos de $k[a_1]$ es no vacío, pues $t$ siempre primo. Si este conjunto fuese finito $\{p_1,\dotsc,p_r\}$, entonces consideramos $p=p_1\cdots p_r +1$. Entonces, o bien $p$ es primo o bien ha de existir algún primo que lo divida, pero ninguno de los primos $p_i$ lo hace. En cualquier caso, nuestro conjunto no está completo, lo que es un absurdo.

Notemos que hemos usado que $k[a_1]$ es infinito, lo cual siempre es cierto aunque $k$ no lo sea.
\end{dem}
Podemos considerar ahora $k(a_1)$ el cuerpo de fracciones como extensión de $k$. Claramente $k\subset k(a_1)\subset K$. Además, sabemos que $k(a_1)$ no puede ser finitamente generado como $k$-álgebra, pues basta usar el Lema 2 y un argumento similar al de la prueba de que $\Q$ no es finitamente generado como $\Z$-álgebra. La contradicción buscada se tiene a través del siguiente lema.
\begin{lema} Sean $k \subset L \subset K$ cuerpos. Si $K$ es finitamente generado como $k$-álgebra, entonces $L$ también lo es.
\end{lema}

%Veámoslo por inducción sobre la dimensión de $K$ como $k$-álgebra. Para $m=1$ tenemos que $K=k[a_1]$ con $a_1 \in K\setminus k$ (otro caso es trivial). En tal caso, $\forall p \in K$, $\exists s\in \N\cup\{0\}$, $\exists b_0,\dotsc,b_s \in k$ tales que $p = \sum_{k=0}^{s} b_k a_1^k$. Como estamos en un cuerpo, podemos considerar $p=a_1^{-1}$. Multiplicando $a_1$ obtenemos
%$$\sum_{k=0}^s b_k a_1^{k+1}  = 1$$
%Por tanto, $a_1$ es raíz del polinomio $-1+\sum_{k=0}^s b_k X^{k+1}$, que tiene coeficientes en $k$, luego $a_1$ es algebraico sobre $k$ y, por tanto, la extensión es finita. 

%Supongamos que el resultado es cierto para $m=n-1$, veámoslo para $m=n$. Dado que $K = k[a_1,\dotsc,a_{n-1}][a_n]$, sabemos que $K'=k[a_1,\dotsc,a_{n-1}]$ es una extensión algebraica y finita de $k$ por hipótesis de inducción. Con un razonamiento análogo al caso $n=1$, tenemos que $a_n$ es algebraico sobre $K'$, luego es una extensión finita de $K'$, pero entonces:
%Sabemos que $\exists f\func{k[a_1,\dotsc,a_n]}{K}$ sobreyectiva, de forma que $K\cong k[a_1,\dotsc,a_n]/\ker(f)$. Podemos suponer que $a_i \in K\setminus k$. Como $K$ es un cuerpo, deducimos que $%\ker(f)$ es maximal.
%$$
%[K\colon k] = [K \colon K'] \cdot [K' \colon k] < \infty
%$$
\end{solucion}
\end{ejercicio}
\end{document}