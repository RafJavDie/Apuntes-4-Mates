\documentclass[twoside]{article}
\usepackage{../../estilo-ejercicios}
\DeclareMathOperator{\Ima}{Im}
\DeclareMathOperator{\Char}{char}
%--------------------------------------------------------
\begin{document}

\title{Algebra Conmutativa y Geometría Algebraica}
\author{Javier Aguilar Martín, Rafael González López, Diego Pedraza López}
\maketitle

\begin{ejercicio}{4}\
 
\begin{enumerate}
\item Sea $X = \V(f)\subset \PP^n_k$ una hipersuperficie proyectiva, donde $f\in k[x_0,\dotsc,x_n]$ es un polinomio homogéneo irreducible. Probar que en el punto $a\in X$ es singular si y solo si $\partial_i f(a) = 0$ para todo $i=0,\dotsc,n$. 
\item Sea $V\subset \PP^n_k$ una variedad proyectiva irreducible, y sea $I(V )= \gene{F_1,\dotsc,F_n}$ con $F_1,\dotsc,F_n \in k[x_0,\dotsc,x_n]$ homogéneos. Probar que los puntos singulares de $V$ son aquellos en los que el rango de la matriz 
$$
\left(
\frac{\partial F_i}{\partial_j}\right)_{1\leq i \leq r, 0\leq j \leq n}
$$
es menor $n-\dim(V)$.
\end{enumerate}
\end{ejercicio}
\begin{solucion}
Notemos que si $f$ es homogéneo de grado $d$ entonces es claro que $\partial_i f$ sigue siendo un polinomio homogéneo, o bien de grado $d-1$, o bien de grado $0$. Por tanto, si $\partial_i f$ se anula en $a\in \A^n_k$ entonces también se anula en cualquier representante de $[a]\in \PP^n_k$. 
\begin{enumerate}
\item Supongamos que $\partial_i f(a)=0$ para todo $i=0,\dotsc,n$. Si $a=(a_0:\dotsc:a_n)$, sabemos que existe $a_i \neq 0$. Por simplicidad consideramos $i=0$. Si tomamos la carta afín $U_0$ podemos considerar $U_0 \cap X$ abierto de afín de $X$ y calcular ahí el espacio tangente, considerando el representante de $a=(1,a_1/a_0 \dotsc, a_n/a_0)$ (al aplicarlo sobre $f$ tendríamos $f$ deshomogeneizado). Por teoría, sabemos que un punto es singular si 
$$
rg
\begin{pmatrix}
\partial_1 f(a) & \dotsc & \partial_n f(a)
\end{pmatrix} < n-\dim(X) \Leftrightarrow 
rg
\begin{pmatrix}
\partial_1 f(a) & \dotsc & \partial_n f(a)
\end{pmatrix} = 0
$$
Esto es si y solo si $\partial_i f(a) = 0$ para todo $i=1,\dotsc,n$, lo cual se tiene por hipótesis.

Supongamos que $a$ es un punto singular de $X$. Supongamos que $a=[e_i]$, siendo $e_i = (0,\dotsc,1,\dotsc,0)$. Como para cada $i$ es análogo, supongamos que $i=0$. Es decir, supongamos que $a=(1:0:\dotsc:0)$ es singular en $X$. Tomando la carta afín $U_0$ y procediendo como en el apartado anterior, junto con la consideración inicial, obtenemos que $\partial_i f(a) = 0$ para todo $i=1,\dotsc,n$. Veamos qué ocurre con $\partial_0 f(a)$. Como es homogéneo, para que no se anulase en $a$ tendría que existir un monomio úicamente $x_0$ (o ser constante no nulo), pero entonces $f$ también tendría un monomio en $x_0$ y no se anularía en $a$, lo cuál sabemos que ocurre. Por tanto, el resultado está probado para los elementos de esta forma.

Quitando estos casos, podemos suponer que en $a=(a_0:\dotsc,a_n)$ existen 2 índices distintos $i,j$ tales que $a_i,a_j \neq 0$. Aplicando el procedimiento anterior para las cartas $U_i$ y $U_j$ cubrimos todos los índices, obteniendo que $\partial_i f(a)=0$ para todo $i=0,\dotsc,n$.
\item 
\end{enumerate}
\end{solucion}


\newpage
\begin{ejercicio}{6}\
 
\begin{enumerate}
\item\label{1} Sea $X \subseteq \PP^2_k$ una cónica (es decir, $\I(X$) está generado por
un polinomio de grado 2). Probar que, si $X$ tiene algún punto singular,
es la unión de dos rectas.

\item\label{2} Sea $X \subseteq \PP^2_k$ una cúbica (es decir, $\I(X)$ está generado por un polinomio
de grado 3). Probar que, si $X$ tiene dos puntos singulares distintos, es la
unión de una recta y una cónica (Ayuda: Reducir al caso en el que la recta
que une los dos puntos singulares es $x_0 = 0$).

\end{enumerate}
\end{ejercicio}

\begin{solucion}\
\begin{enumerate}
\item Sea $\langle c_{0,0}x_0^2+c_{1,1}x_1^2+c_{2,2}x_2^2+c_{0,1}x_0x_1+c_{0,2}x_0x_2+c_{1,2}x_1x_2\rangle=\I(X)$ y supongamos que $\Char(k)\neq 2$. A este polinomio le podemos asociar la matriz  
$$C=\begin{pmatrix}
c_{0,0} & \frac{c_{0,1}}{2} &\frac{c_{0,2}}{2}\\
\frac{c_{0,1}}{2}       & c_{1,1} & \frac{c_{1,2}}{2} \\
\frac{c_{0,2}}{2}        &  \frac{c_{1,2}}{2}       & c_{2,2}
\end{pmatrix}.$$

Por la asignatura \emph{Álgebra Lineal y Geometría II}, $X$ es unión de dos rectas distintas si y solo si $rg(C)=2$. Por tanto, vamos a probar que si tiene algún punto singular, entonces $rg(C)= 2$. Si $X$ tiene algún punto singular, entonces $\exists a=(a_0:a_1:a_2)\in X$ tal que la matriz de sus derivadas parciales
$$\begin{pmatrix}
2c_{0,0}a_0 +c_{0,1}a_1+c_{0,2}a_2 & 2c_{1,1}a_1 +c_{0,1}a_0+c_{1,2}a_2 & 2c_{0,0}a_2+c_{0,2}a_0+c_{1,2}a_1
\end{pmatrix}=(0\ 0\ 0).$$
Esto equivale a 
$$rg\begin{pmatrix}
2c_{0,0} & c_{0,1} &c_{0,2}\\
c_{0,1}       & 2c_{1,1} & c_{1,2} \\
c_{0,2}        &  c_{1,2}       & 2c_{2,2}
\end{pmatrix}\leq 2\Leftrightarrow rg\begin{pmatrix}
c_{0,0} & \frac{c_{0,1}}{2} &\frac{c_{0,2}}{2}\\
\frac{c_{0,1}}{2}       & c_{1,1} & \frac{c_{1,2}}{2} \\
\frac{c_{0,2}}{2}        &  \frac{c_{1,2}}{2}       & c_{2,2}
\end{pmatrix}\leq 2.$$

Además, el rango no puede ser 1, pues por la misma asignatura, tendríamos que la cónica es una recta doble. Pero $\I(X)$ no puede estar generado por un polinomio de grado 1 elevado al cuadrado, ya que entonces no sería radical. Por lo tanto es exactamente igual a 2, como queríamos probar.

En el caso $\Char(k)=2$ tendremos al derivar que 
$$\begin{pmatrix}
c_{0,1}a_1+c_{0,2}a_2 & c_{0,1}a_0+c_{1,2}a_2 & c_{0,2}a_0+c_{1,2}a_1
\end{pmatrix}=(0\ 0\ 0),$$
o lo que es lo mismo 
$$rg\begin{pmatrix}
0 & c_{0,1} &c_{0,2}\\
c_{0,1}       & 0 & c_{1,2} \\
c_{0,2}        &  c_{1,2}       & 0
\end{pmatrix}\leq 2.$$
Pero en $\Char(k)=2$, esto ocurre siempre puesto que su determinante es $2c_{0,1}c_{1,2}c_{0,2}=0$, por lo que todo punto es singular. Esto implica que $X$ no es variedad, por lo que se descompone como unión de variedades. Como el polinomio que lo genera es de grado 2, esto quiere decir que es reducible en 2 polinomios de grado 1, por lo que $X$ es unión de dos rectas.

\item Aplicando un cambio proyectivo de coordenadas podemos suponer que los puntos singulares son $P=(0:1:0)$ y $Q=(0:0:1)$. Vamos a probar que la recta que los une, $x_0=0$, está contenida en la cúbica, y por tanto es reducible a una recta y una cónica (ya que el polinomio factoriza al menos en un término de grado 1 y otro de grado 2).

Sea $\langle f\rangle= \I(X)$ de modo que $f(x_0,x_1,x_2)$ es el siguiente polinomio 
\begin{gather*}
c_{0,0,0}x_0^3+c_{1,1,1}x_1^3+c_{2,2,2}x_2^3+c_{0,0,1}x_0^2x_1+c_{0,0,2}x_0^2x_2+c_{1,1,2}x_1^2x_2+c_{0,1,1}x_1^2x_0\\
+c_{0,2,2}x_2^2x_0+c_{1,2,2}x_2^2x_1+c_{0,1,2}x_0x_1x_2
\end{gather*}
En primer lugar, como los puntos pertenecen a la cúbica, podemos sustituir en el polinomio y obtenemos $c_{1,1,1}=c_{2,2,2}=0$. Por lo tanto nos queda que
$$f(x_0,x_1,x_2)=c_{0,0,0}x_0^3+c_{0,0,1}x_0^2x_1+c_{0,0,2}x_0^2x_2+c_{1,1,2}x_1^2x_2+c_{0,1,1}x_1^2x_0
+c_{0,2,2}x_2^2x_0+c_{1,2,2}x_2^2x_1+c_{0,1,2}x_0x_1x_2$$
Calculamos sus derivadas evaluadas en los puntos singulares y deducimos
$$\begin{pmatrix}
c_{0,1,1} & 0 & c_{1,1,2}
\end{pmatrix}=\begin{pmatrix}
0 & 0 & 0
\end{pmatrix}$$
$$\begin{pmatrix}
c_{0,2,2} & c_{1,2,2} & 0
\end{pmatrix}=\begin{pmatrix}
0 & 0 & 0
\end{pmatrix}$$
Por lo que
$$f(x_0,x_1,x_2)=c_{0,0,0}x_0^3+c_{0,0,2}x_0^2x_2+c_{0,1,1}x_1^2x_0
+c_{0,1,2}x_0x_1x_2=x_0(c_{0,0,0}x_0^2+c_{0,0,2}x_0x_2+c_{0,1,1}x_1^2
+c_{0,1,2}x_1x_2)$$
Así que hemos descompuesto la cúbica como una recta ($x_0=0$) y una cónica ($c_{0,0,0}x_0^2+c_{0,0,2}x_0x_2+c_{0,1,1}x_1^2
+c_{0,1,2}x_1x_2=0$).
\end{enumerate}
\end{solucion}

\end{document}