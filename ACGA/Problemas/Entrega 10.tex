\documentclass[twoside]{article}
\usepackage{../../estilo-ejercicios}
\DeclareMathOperator{\Ima}{Im}

%--------------------------------------------------------
\begin{document}

\title{Algebra Conmutativa y Geometría Aplicada}
\author{Javier Aguilar Martín, Rafael González López}
\maketitle

\begin{ejercicio}{3}\
Sea $A \subseteq B$ una extensión entera de dominios de integridad.
 
\begin{enumerate}
\item\label{1} Si
$$\mathfrak{p}_0 \subsetneq \mathfrak{p}_1 \subsetneq \dots \subsetneq \mathfrak{p}_r$$
es una cadena creciente de ideales primos en $A$, probar que existe una
cadena creciente
$$\mathfrak{q}_0 \subsetneq \mathfrak{q}_1 \subsetneq \dots \subsetneq \mathfrak{q}_r$$
de ideales primos en $B$ tales que $\mathfrak{p}_i = A \cap \mathfrak{q}_i$.
\item\label{2} Si $\mathfrak{p} \subseteq A$ es un ideal primo y $\mathfrak{q}_1 \subseteq \mathfrak{q}_2 \subseteq B$ dos ideales primos de $B$ tales
que $\mathfrak{q}_1 \cap A = \mathfrak{q}_2 \cap A = \mathfrak{p}$, probar que $\mathfrak{q}_1 = \mathfrak{q}_2$.
\item Deducir de los apartados anteriores que si $f : X \to Y$ es un morfismo
finito y dominante entre variedades afines, entonces $\dim(Y ) = \dim(X)$.

\end{enumerate}
\end{ejercicio}
\begin{solucion}
Empezamos con algunos resultados que serán útiles en la prueba del ejercicio.
\begin{lemma}\label{lema}
Sean $A\subseteq B$ anillos, con $B$ entero sobre $A$. Si $\mathfrak{b}$ es un ideal maximal de $B$ y $\mathfrak{a}=\mathfrak{b}\cap A$, entonces $B/\mathfrak{b}$ es entero sobre $A/\mathfrak{a}$.
\end{lemma}
\begin{proof}
Si $x\in B$, entonces existe un polinomio mónico $x^n+a_{n-1}x^{n-1}+\dots+a_0=0$ con $a_i\in A,\ i=0,\dots,n-1$. Basta reducir esta ecuación módulo $\mathfrak{b}$, pues la clase de $x$ está en $B/\mathfrak{b}$. Además, los coeficientes pasarían al cociente como elementos del subanillo $(A+\mathfrak{b})/\mathfrak{b}\subseteq B/\mathfrak{b}$. Pero usando el segundo teorema de isomorfía para anillos tenemos que, como $\mathfrak{a}=\mathfrak{b}\cap A$, entonces $(A+\mathfrak{b})/\mathfrak{b}\cong A/\mathfrak{a}$, luego podemos ver los coeficientes en dicho anillo.
\end{proof}

\newpage

\begin{coro}\label{coro}
Sean $A\subseteq B$ anillos, $B$ entero sobre $A$. Sea $\qq$ un ideal primo de $B$ y sea $\pp=A\cap \qq$. Entonces $\qq$ es maximal si y solo si $\pp$ es maximal.
\end{coro}
\begin{proof}
Por el lema \ref{lema}, $B/\qq$ es entero sobre $A/\pp$, y ambos son dominios de integridad. Ahora, como $B/\qq$ es cuerpo si y solo si $A/\pp$ es cuerpo, se tiene el resultado.
%asegurarme de que hemos dado la equivalencia de ser cuerpo
\end{proof}

Pasamos a probar los apartados del ejercicio.

\begin{enumerate}
\item Veamos que para todo ideal primo $\pp\subseteq A$ se tiene que existe un ideal primo $\qq\subseteq B$ tal que $\qq=\pp\cap A$. Esto será suficiente pues la cadena resultante es claramente creciente. 

Por el apartado 3 del ejercicio 1 de esta misma relación, $B_\pp$ es entero sobre $A_\pp$ y el diagrama
\[
\begin{tikzcd}
A\arrow[r, hookrightarrow]\arrow[d,"\alpha"']& B\arrow[d,"\beta"]\\
A_\pp\arrow[r, hookrightarrow]& B_\pp
\end{tikzcd}
\]
donde $\alpha$ y $\beta$ son los homomorfismos inducidos, es conmutativo. Sea $\mathfrak{n}$ el ideal maximal de $B_\pp$, entonces $\mm=\mathfrak{n}\cap A_\pp$ es el ideal maximal de $A_\pp$ por el corolario \ref{coro}. Si tomamos $\qq=\beta^{-1}(\mathfrak{n})$, entonces $\qq$ es primo por ser preimagen de un primo, y tenemos que $\qq\cap A=\alpha^{-1}(\mm)=\pp$ por la correspondecia entre ideales de $A$ que no cortan a $A\setminus\pp$ y los de $A_\pp$.

\item  Usando el apartado 3 del ejercicio 1 de esta relación, $B_\pp$ es entero sobre $A_\pp$. Sea $\mm$ el ideal extendido de $\pp$ en $A_\pp$ y sean $\mathfrak{n},\mathfrak{n}'$ los ideales extendidos de $\qq_1$ y $\qq_2$ en $B_\pp$. Entonces $\mm$ es el ideal maximal de $A_\pp$ por la correspondecia entre ideales. Por otro lado, $\mathfrak{n}\subseteq\mathfrak{n}'$ y por hipótesis $\mathfrak{n}\cap A_\pp=\mathfrak{n}'\cap A_\pp=\mm$. Por el corolario \ref{coro} se sigue que $\mathfrak{n},\mathfrak{n}'$ son maximales, por lo que $\mathfrak{n}=\mathfrak{n}'$, ya que solo hay un ideal maximal. De aquí se deduce que $\qq_1=\qq_2$, ya que al estar trabajando en dominios de integridad, el homomorfismo inducido hacia el anillo localizado es inyectivo.


\item Por ser $f$ finito tenemos que $\calA(X)$ es entero sobre $f^*(\calA(Y))$, que son ambos dominios de integridad al ser $\calA(X)$ isomorfo a un cociente por un ideal primo y porque $f^*(\calA(Y))\subseteq\calA(X)$. También tenemos que por ser $f$ dominante, $f^*$ es inyectiva. Sea 
$$\mathfrak{p}_0 \subsetneq \mathfrak{p}_1 \subsetneq \dots \subsetneq \mathfrak{p}_r$$ 
una cadena creciente de ideales primos de $f^*(\calA(Y))$ de longitud igual a la dimensión de dicho anillo. Entonces, por el apartado \ref{1} existe una cadena creciente
$$\mathfrak{q}_0 \subsetneq \mathfrak{q}_1 \subsetneq \dots \subsetneq \mathfrak{q}_r$$ 
de ideales primos en $\calA(X)$ tales que $\mathfrak{p}_i = f^*(\calA(Y)) \cap \mathfrak{q}_i$. Además, esta es la dimensión de Krull de $\calA(X)$, pues si tuviéramos una cadena $\qq'_0\subseteq\dots\subseteq \qq'_d$ con $d>r$, entonces 
$$\qq'_0\cap f^*(\calA(Y))\subseteq \qq'_1\cap f^*(\calA(Y))\subseteq\dots\subseteq \qq'_d\cap f^*(\calA(Y))$$ 
es una cadena creciente de ideales primos en $f^*(\calA(Y))$, luego por ser $r$ su dimensión, deben existir algunos $\qq'_i\cap f^*(\calA(Y))=\qq'_{i+1}\cap f^*(\calA(Y))$, lo cual implica por el apartado \ref{2} que $\qq'_i=\qq'_{i+1}$, luego la cadena solo podrá tener como mucho longitud $r$ para ser estrictamente creciente. 

Falta probar que también $\dim(Y)=r$. Tomando la misma cadena que antes en $f^*(\calA(Y))$, obtenemos por la inyectividad de $f^*$ la siguiente cadena creciente de ideales primos
$$(f^*)^{-1}(\pp_0)\subsetneq (f^*)^{-1}(\pp_1)\subsetneq\dots\subsetneq (f^*)^{-1}(\pp_r),$$
por lo que $\dim(Y)=\dim_{Krull}(\calA(Y))\geq\dim_{Krull}(f^*(\calA(Y))=r$. Recíprocamente, si tenemos una cadena creciente de ideales en $\calA(Y)$
$$\mathfrak{p}'_0 \subsetneq \mathfrak{p}'_1 \subsetneq \dots \subsetneq \mathfrak{p}'_d.$$
Entonces, como $f^*$ es isomorfismo sobre su imagen, tenemos en $f^*(\calA(Y))$ la cadena creciente de ideales primos
$$f^*(\mathfrak{p}'_0) \subsetneq f^*(\mathfrak{p}'_1) \subsetneq \dots \subsetneq f^*(\mathfrak{p}'_d).$$
Como $\dim_{Krull}(f^*(\calA(Y)))=r$, esto implica que $d\leq r$, por lo que $\dim_{Krull}(\calA(Y))\leq r$, con lo que finalmente tenemos la igualdad, como queríamos demostrar.

\end{enumerate}
\end{solucion}

\newpage

\end{document}