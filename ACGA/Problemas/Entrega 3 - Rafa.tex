\documentclass[twoside]{article}
\usepackage{../../estilo-ejercicios}

%--------------------------------------------------------
\begin{document}

\title{Algebra Conmutativa y Geometría Aplicada}
\author{Rafael González López}
\maketitle

\begin{ejercicio}{1}
Sea $A$ un anillo y $S\subset A$ un conjunto multiplicativa. Si $f\func{A}{B}$ es un homomorfismo de anillos tal que $f(S)\subset B^\times$, probar que existe un único homomorfismo de anillos $\tilde{f}\func{S^{-1}A}{B}$ que existe a $f$, en el sentido de que $\tilde{f}(a/1) = f(a)$.
\newline
Deducir que si $A$ es un dominio de integridad y $f\func{A}{L}$ un homomorfismo de anillos inyectivo, donde $L$ es un cuerpo, entonces $f$ se extiende de manera única a $K(A)$, el cuerpo de fracciones de $A$.
\begin{solucion}
Vamos a tratar de dar una definición de $\tilde{f}$. Sea $p/q$, sabemosq que $\tilde{f}(p/1)=f(p)$. Definimos $\tilde{f}(p/q)=k$ donde $k\in B$ es tal que $kf(q)=f(p)$. Tenemos que comprobar que está bien definido y que es homomorfismo.
\begin{itemize}
\item Supongamos que $\exists k,k'\in B$ tales que $kf(q)=k'f(q)=f(p)$. Como $f(q)\in B^\times$, podemos multiplicar por su inversa, obteniendo que $k=k'$. Por tanto, la aplicación está bien definida. De hecho, $\tilde{f}(p/q)=f(p)f(q)^{-1}$.
\item Para ver que es homomorfismo de anillos con unidad, comprobamos las tres propiedades de la definición. Notaremos indistamente $1$ al elemento neutro de la segunda operación tanto en $A$ como en $B$.
\begin{itemize}
\item Sea $1/1$ el neutro del producto en $S^{-1}A$, 
$$\tilde{f}(1/1)=f(1)f(1)^{-1}=1$$
\item Sean $p/q,p'/q' \in S^{-1}A$, entonces
\begin{align*}
\tilde{f}(p/q+p'/q') &= 	\tilde{f}\left(\frac{pq'+p'q}{qq'}\right) = f(pq'+p'q)f(qq')^{-1} = \\
 &= f(pq'+p'q)f(q)^{-1}f(q')^{-1} =
(f(pq')+f(p'q))f(q)^{-1}f(q')^{-1} = \\
&= (f(p)f(q')+f(p')f(q))f(q)^{-1}f(q')^{-1}  =\\
 &= f(p)f(q)^{-1}+f(p')f(q')^{-1} = \tilde{f}(p/q)+\tilde{f}(p'/q')
\end{align*}

\item Sean $p/q,p'/q' \in S^{-1}A$, entonces
\begin{align*}
\tilde{f}(p/q\cdot p'/q') &= 	\tilde{f}\left(\frac{pp'}{qq'}\right) = f(pp')f(qq')^{-1} = \\
 &= f(p)f(p')f(q)^{-1}f(q')^{-1} = \\
 &= f(p)f(q)^{-1}f(p')f(q')^{-1} =\\
 &= \tilde{f}(p/q)\tilde{f}(p'/q')
\end{align*}
\end{itemize}
Por tanto, hemos probado que nuestra definición de $\tilde{f}$ es un homomorfismo que extiende de anillo que extiende a $f$.
\end{itemize}
Vista la existencia, resta ver la unicidad. Supongamos que existe otro homomrfismo $\tilde{g}$ que extiende a $f$ además de $\tilde{f}$. En tal caso, como $\tilde{f}$ y $\tilde{g}$ extienden $f$, $\tilde{f}(a/1)=\tilde{g}(a/1)$. Por propiedades básicas de homomorfismo de anillos, basta ver que son iguales para los elementos de la forma $1/q$, con $q\in S$.
$$
f(q)f(q)^{-1} = \tilde{f}(q/q)= \tilde{f}(1/1) = 1 =\tilde{g}(1/1) =\tilde{g}(q/q) =\tilde{g}(q/1)\tilde{g}(1/q)=f(q)\tilde{g}(1/q)
$$
Usando que $f(q)\in B^\times$, obtenemos que $\tilde{g}(1/q)=f(q)^{-1}=\tilde{f}(1/q)$. Por tanto, si ${p/q\in S^{-1}A}$, entonces 
$$
\tilde{g}(p/q) = \tilde{g}(p/1\cdot 1/q) = \tilde{g}(p/1)\tilde{g}(1/q) = f(p)f(q)^{-1} = \tilde{f}(p/q)
$$
Para ver la segunda parte del ejercicio no tenemos que más que considerar que, al ser $f$ inyectiva, necesariamente $f(A^*)\subset L^*$, pues $L$ es cuerpo y su único elemento no unidad es el $0$. La hipótesis de que $A$ sea dominio es necesaria para definir el cuerpo de fracciones. No tenemos más que aplicar la primera parte a $S=A^{*}$, pero precisamente ${A^*}^{-1}A = K(A)$. 
\end{solucion}
\end{ejercicio}
\end{document}