\documentclass[twoside]{article}
\usepackage{../../estilo-ejercicios}
\DeclareMathOperator{\Ima}{Im}

%--------------------------------------------------------
\begin{document}

\title{Algebra Conmutativa y Geometría Aplicada}
\author{Javier Aguilar Martín, Rafael González López}
\maketitle
\begin{ejercicio}{2} Sea $X \subset \mathbb{P}^n_k$ una variedad proyectiva, $U\subset \mathbb{P}^1_k$ un abierto no vacío y $\phi\colon U\rightarrow X$ un morfismo. Probar que $\phi$ se extiende de manera única a un morfismo $\overline{\phi}\colon \mathbb{P}^1_k\rightarrow X$. Probar que no es cierto el caso en el que $X$ es afín.
\end{ejercicio}
\begin{solucion}
Como $U\subset \PP^1_k$ es un abierto contenido en una variedad (la propia recta poryectiva), es un variedad cuasiproyectiva, por lo que tiene sentido considerar el morfismo entre $U$ y $X$. Además, sabemos que existen $f_1,\dotsc,f_n$ polinomios homogéneos del mismo grado tales que $\phi = (f_1 : \dotsc : f_n)$. Si los $f_i$ no se anulan simultáneamente en ningún punto de $\PP^n_\C$, entonces podemos tomar como $\overline{\phi}=\phi$. En caso contrario, sabemos que se puede anular simultáneamente en, a lo sumo, una cantidad finita $s_1,\dotsc,s_k$ de puntos proyectivos. Supongamos que $s_i = (a_i:b_i)$. En tal caso, podemos escribir $f_i = \left(\prod_{i=1}^k (b_ix_0-a_ix_1)\right)\cdot f_i'$ para cualesquiera representantes de $s_i$. En tal caso podemos tomar como $\overline{\phi}=(f_1',\dotsc,f_n')$, pues no se anula en $\PP^1_\C$ y coincide con $\phi$ en $U$.
\end{solucion}
\newpage 
\begin{ejercicio}{5}\
\emph{La curva normal racional}. La imagen $X_d$ de la inmersión de
Veronese $\PP^1_k \to \PP^d_k$ para $n = 1$ se llama la \textbf{curva normal racional} en $\PP^d_k$.
\begin{enumerate}
\item Probar que $X_d$ es el conjunto de puntos $(x_0 : x_1 : \dots : x_d) \in \PP^d_k$ tales que
la matriz 
\[
\begin{pmatrix}
x_0 & x_1 &\cdots& x_{d-1}\\
x_1 & x_2 &\cdots& x_d
\end{pmatrix}
\]
tiene rango 1.
\item Probar que $X_d$ es la clausura proyectiva de la imagen del morfismo $\phi :\A^1_k \to \A^d_k$ dado por $\phi(t) = (t, t^2, \dots , t^d)$.
\item Probar que tres puntos distintos de $X_d$ nunca están en la misma recta.
\end{enumerate}
\end{ejercicio}
\begin{solucion}\

\end{solucion}
\newpage 

\end{document}