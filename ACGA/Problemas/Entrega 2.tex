\documentclass[twoside]{article}
\usepackage{../../estilo-ejercicios}
\DeclareMathOperator{\GL}{GL}
\DeclareMathOperator{\SL}{SL}
%--------------------------------------------------------
\begin{document}

\title{Algebra Conmutativa y Geometría Algebraica}
\author{Javier Aguilar Martín, Rafael González López, Diego Pedraza López}
\maketitle

\begin{ejercicio}{2}\
Sea $X \subseteq \mathbb{A}^n_k$
un conjunto algebraico, y $f \in \mathcal{A}(X)$ una función
regular que no se anule en ningún punto de $X$. Probar que $1/f : X \to k$ es
también una función regular.
\end{ejercicio}
\begin{solucion}\
Por ser $X$ algebraico se tiene que $X=\mathcal{V}(g_1,\dots,g_m)$, para algunos polinomios $g_i\in k[x_1,\dots,x_n]\ i=1,\dots,m$. Por definición de función regular, podemos escribir $f(x)=P(x)\in k[x_1,\dots, x_n]\ \forall x\in X$. Como $P(x)\neq 0\ \forall x\in X$, tenemos que 
$$\emptyset=\mathcal{V}(g_1,\dots,g_m)\cap\mathcal{V}(P)=\mathcal{V}(g_1,\dots,g_m,P).$$
Esto implica que $\langle g_1,\dots,g_m,P\rangle=k[x_1,\dots, x_n]$, por lo que $1\in \langle g_1,\dots,g_m,P\rangle$. Podemos suponer que $1\notin \langle g_1,\dots,g_m\rangle$, pues de lo contrario $k[x_1,\dots, x_n]=\langle g_1,\dots,g_m\rangle$, con lo que $P(x)$ sería un polinomio constante al ser $k$ algebraicamente cerrado y por tanto $1/P=1/f:X\to k$ sería trivialmente regular. Así pues, $1= h_1g_1+\cdots h_mg_m+hP$, por lo que, evaluando este polinomio en un punto $x\in X$, $1=hP$, para algún $h\in k[x_1,\dots, x_n]$. Por tanto, $1/f(x)=1/P(x)=h(x)\ \forall x\in X$, con lo que $1/f:X\to k$ es regular. 


\end{solucion}

\newpage

\begin{ejercicio}{6}
\begin{enumerate}
\item[]
\item Sea $X=\{(t,t^2,t^3)\mid t\in\C\} \subset \mathbb{A}^3_{\C}$. Probar que $X$ es una variedad algebraica, y hallar un conjunto de generadores de $\I(X)$.
\item Sea $X\subset \mathbb{A}^3_{\C}$ el conjunto algebraico definido por las ecuaciones $x^4-y^3=x^5-z^3= y^5 -z^4 = 0$. Probar que $X$ es irreducible y que no puede definirse usando solo dos de las ecuaciones dadas. (Ayuda: Usar la aplicación $t\mapsto (t^3,t^4,t^5)$).
\end{enumerate} 
\begin{solucion}
\item[]
\begin{enumerate}
\item Consideremos los polinomios $f(x,y,z) = x^2-y$ y $g(x,y,z)=x^3-z$. Es claro que $X=\V(f,g)$, por lo que $X$ es algebraico. Para probar que es variedad sabemos que basta que ver que $\gene{f,g}$ es un ideal primo.

Sean $p,q\in\C[x,y,z]$ tales que $pq\in \gene{g,f}$. Usando que $\C[x,y,z] \cong \C][x,y][z]$, podemos dividir $p$ y $q$ entre $f$, tomando $f$ como un monomio en la variable $z$, obteniendo un resto que no depende de $z$. 
\begin{gather*}
p(x,y,z)=(x^3-z)p_1(x,y,z) + r_1(x,y)\\
q(x,y,z)=(x^3-z)q_1(x,y,z) + r_2(x,y)
\end{gather*}
Además, podemos hacer un razonamiento análogo con los restos en $\C[x,y]\cong \C[x][y]$ y $g$, obteniendo restos que no dependen de $y$. 
\begin{gather*}
p(x,y,z)=(x^3-z)p_1(x,y,z) + (x^2-y)p_2(x,y) + p_3(x)\\
q(x,y,z)=(x^3-z)q_1(x,y,z) + (x^2-y)q_2(x,y) + q_3(x)
\end{gather*}
Si multiplicamos estas expresiones, tenemos que
\[
(p\cdot q)(x,y,z) = (x^3-z)h_1(x,y,z)+(x^2-y)h_2(x,y,z) + p_3(x)q_3(x)
\]
Se pueden hacer distintas agrupaciones ($h_1,h_2$ no son únicos), pero se obtienen resultados análogos. En cualquier caso, deducimos que $p_3(x)q_3(x)\in \gene{f,g}$. Sustituyendo $(x,y,z)\mapsto(t,t^2,t^3)$, deducimos que $p_3(t)q_3(t)=0$ $\forall t\in \C$. Como $\C$ es un cuerpo infinito, se tiene que $p_3(t)q_3(t)\equiv 0$. Al ser $\C[x]$ dominio, o bien $p_3 \equiv 0$, o $q_3 \equiv 0$. En cualquier caso, o bien $p$, o bien $q$ pertenecen a $\gene{f,g}$, como queríamos probar.
\item Notemos por $f(x,y,z)=x^4-y^3$, $g(x,y,z)=x^5-z^3$ y $h(x,y,z)=y^5-z^4$. Tenemos que demostrar que $X=\V(f,g,h)$ es irreducible, pero primeramente veamos qué forma tiene $X$.

Es claro que si $xyz = 0$ entonces, de las ecuaciones $f=g=h=0$ solamente dan lugar al origen. Podemos suponer, por tanto, que $xyz\neq 0$ y sea $t=\frac{y}{x}$. De $f(x,y,z)=0$, $x^4-y^3=0$, se tiene que $x=\dfrac{y^3}{x^3}=t^3$. Despejando $t=\frac{y}{x}$, tenemos que $y= tx = t^4$. De $g=0$ y $h=0$ deducimos que $z^3 = t^{15}$ y $z^4 = t^{20}$, dividiendo (ninguno de los términos es nulo), tenemos que $z=t^5$. 

Por tanto, si $a\in X$ (incluido el $(0,0,0)$) entonces $a\in Y=\{(t^3,t^4,t^5)\in\mathbb{A}^3_{\C}\mid t\in \C\}$. Sin embargo, es trivial ver que el recíproco también es cierto, por lo que $X=Y$. Para ver que $X$ es irreducible, vamos a probar dos lemas.

\begin{lemma}
El conjunto $\mathbb{A}^1_{\C}\subset \mathbb{A}^1_{\C}$ es irreducible.
\end{lemma}
\begin{dem}
Es trivial pues $\mathbb{A}^1_{\C}$ es infinito y los conjuntos algebraicos de $\mathbb{A}^1_{\C}$ son finitos (salvo $\mathbb{A}^1_{\C}$).
\end{dem}

\begin{lemma}
La aplicación $f\func{\mathbb{A}^1_{\C}}{\mathbb{A}^3_{\C}}$, $f(t)=(t^3,t^4,t^5)$ es continua (con la topología de Zariski) y lleva conjuntos irreducibles en conjuntos irreducibles.
\end{lemma}
\begin{dem}
Para ver que es continua, basta ver que la preimagen de un cerrado es cerrado. Sea $W \subset \mathbb{A}^3_{\C}$ un conjunto cerrado (algebraico), entonces existen ciertos polinomios $f_1,\dotsc,f_n\func{\mathbb{A}^3_{\C}}{\C}$ tales que $W=\V(f_1,\dotsc,f_n)$. Entonces los elementos de $f^{-1}(W)$ se anulan sobre $f_i \circ f$ $\forall i=1,\dotsc,n$. Efectivamente si $x\in f^{-1}(W)$, $f(x)\in W$ y, por definifición, $f_i(f(x)) = 0$. Dado que $f$ es una función polinómica, la composición también es un polinomio (trivial). Por tanto, $f^{-1}(W)$ también es cerrado.

Para la segunda parte operamos por reducción al absurdo. Supongamos que ${\exists W \subset \mathbb{A}^1_{\C}}$ irreducible tal que $f(W)=Z_1 \cup Z_2$, $Z_i$ algebraico y con $Z_i\subsetneq f(W)$. Como $f^{-1}(Z_i)$ también es algebraico (por ser $f$ continua), los conjuntos ${W_i = f^{-1}(Z_i)\cap W}$ son algebraicos (la intersección de cerrados es cerrada). Tenemos además que:
$$ W \subset f^{-1}(f(W)) = f^{-1}(Z_1\cup Z_2) = f^{-1}(Z_1)\cup f^{-1}(Z_2)$$
$$
W = W\cap (f^{-1}(Z_1)\cup f^{-1}(Z_2)) = (W\cap f^{-1}(Z_1))\cup(W \cap f^{-1}(Z_2)) = W_1 \cup W_2
$$
Como $W$ es irreducible, $W=W_1$ o $W=W_2$. Supongamos sin pérdida de generalidad que $W=W_1= f^{-1}(Z_1)\cap W$, luego $W \subset f^{-1}(Z_1)$. Se tiene pues que $f(W)\subset Z_1$, pero $Z_1 \subsetneq f(W)$, lo cual es absurdo.
\end{dem}
Volviendo a nuestro ejercicio, dado que $f(\mathbb{A}^1_{\C}) = X$, por los lemas anteriores deducimos que $X$ es irreducible.

Ahora vamos a ver la segunda parte del ejercicio, es decir, que usando solo dos de las ecuaciones que tenemos no podemos definir el conjunto $X$. Distingamos los tres casos.
\begin{itemize}
\item Si tenemos el sistema $x^4-y^3=x^5-z^3= 0$. Despejando en función de $x$ tomando raíces de manera adecuada, y haciendo $x=i$, tenemos que una solución del sistema es $(i,e^{\frac{2\pi i}{3}},e^{\frac{\pi i}{6}})$. Sin embargo, $\left(e^{\frac{2\pi i}{3}}\right)^5-\left(e^{\frac{\pi i}{6}}\right)^4 = -i\sqrt{3}$, luego no verifica $y^5-z^4 =0$.
\item Si tenemos el sistema $x^4-y^3 = y^5 - z^4=0$. De manera más sencilla que en el caso anterior, es obvio que la tripleta $(1,1,-1)$ es solución del sistema, pero que no verifica la ecuación restante $x^5-z^3=0$.
\item Si tenemos el sistema $x^5-z^3 = y^5 - z^4=0$. Análogo al caso primero, podemos observar que la tripleta $(i,-e^{\frac{13i\pi}{15}},e^{\frac{5\pi i}{6}})$ es solución de nuestras dos ecuaciones, pero si sustituimos en $f(i,-e^{\frac{13i\pi}{15}},e^{\frac{5\pi i}{6}}) = 1+(-1)^{\frac{3}{5}}$, luego no verifica la primera ecuación.
\end{itemize}
Por tanto, no podemos definir $X$ excluyendo alguna de las ecuaciones, como queríamos probar.
\end{enumerate}
\end{solucion}
\end{ejercicio}

\newpage

\begin{ejercicio}{8}
Sea $A$ un DFU, e $I = \langle p \rangle \subseteq A$ un ideal principal primo. Probar que no existe ningún ideal primo no nulo estrictamente contenido en $I$. Deducir que, si $X \subset \mathbb{A}_k^n$ es una hipersuperficie (es decir, un conjunto algebraico definido por una sola ecuación), entonces no existe ningún otro conjunto algebraico irreducible $Y$ distinto del total que contenga a $X$. ¿Es cierto si no suponemos que $Y$ sea irreducible?
\end{ejercicio}

\begin{solucion}
Supongamos que existe un ideal $J$ primo no nulo contenido $\langle p \rangle$. Sea $a \in J$, como $A$ es un DFU, entonces hay elementos primos $q_1,\dots, q_r$ en $A$ con $a=q_1\cdot \cdots \cdot q_r$. Como $a \in J$ y $J$ es primo, al menos un factor $q \in \{q_1,\dots,q_r\}$ debe estar en $J$. Como $q \in J \subseteq \langle p\rangle$, entonces $q=pr$ para algún $r \in A$.
\begin{itemize}
	\item Si $r$ es unidad, entonces $q\cdot r^{-1} = p$, luego $\langle p \rangle \subseteq J$ y, en consecuencia, $J = \langle p \rangle$.
	\item Si $r$ no fuera unidad, entonces $q$ sería reducible. Sin embargo, $A$ es un dominio de integridad, luego sus elementos primos son irreducibles. Por lo tanto, este caso no se puede dar.
\end{itemize}
Entonces si $J$ es un ideal primo contenido en $I$, necesariamente $I = J$.

Sea $X \subset \mathbb{A}_k^n$ una hipersuperficie. Como $k$ es un cuerpo, en particular es dominio de factorización única, luego $k[x_1,\dots,x_n]$ es un dominio de factorización única. Como $X$ es una hipersuperficie, $\I(X)$ está generado por un polinomio irreducible (proposición 1.26). Como estamos en un DFU, $\I(X)$ es un ideal principal primo. Sea $Y$ un conjunto algebraico irreducible $Y$ distinto del total y supongamos que $Y \supseteq X$. Entonces $\I(Y) \subseteq \I(X)$. Como $Y$ es irreducible, $\I(Y)$ es primo. Por lo demostrado anteriormente, entonces $\I(Y) = \I(X)$. Por la correspondencía biunívoca entre ideales primos y conjuntos algebraicos irreducibles (corolario 1.18), entonces $X=Y$.

Si $Y$ no es irreducible, basta tomar $Y = X \cup \{P\}$, donde $P \in \mathbb{A}_k^n$ es un punto que no está en $X$. Aquí estamos suponiendo que $X$ es distinto del total.
\end{solucion}

\end{document}