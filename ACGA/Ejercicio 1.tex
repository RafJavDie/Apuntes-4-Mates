\documentclass[twoside]{article}
\usepackage{amsmath,accents}%
\usepackage{amsfonts}%
\usepackage{amssymb}%
\usepackage{comment}
\usepackage{graphicx}
\usepackage{mathrsfs}
\usepackage[utf8]{inputenc}
\usepackage{amsfonts}
\usepackage{amssymb}
\usepackage{graphicx}
\usepackage{mathrsfs}
\usepackage{setspace}  
\usepackage{amsmath}
\usepackage{nccmath}
\usepackage[spanish]{babel}
\usepackage{multirow}

\newenvironment{ejercicio}[2][Estado]{\begin{trivlist}
\item[\hskip \labelsep {\bfseries Ejercicio}\hskip \labelsep {\bfseries #2.}]}{\end{trivlist}}
 \newenvironment{solucion}{\begin{trivlist}
\item[\hskip \labelsep {\textit{Solución}.}\hskip \labelsep]}{\end{trivlist}}
\setlength{\parindent}{0pt}
\setlength{\parskip}{5pt}

\renewcommand{\baselinestretch}{1,4}
\setlength{\oddsidemargin}{0.5in}
\setlength{\evensidemargin}{0.5in}
\setlength{\textwidth}{5.4in}
\setlength{\topmargin}{-0.25in}
\setlength{\headheight}{0.5in}
\setlength{\headsep}{0.6in}
\setlength{\textheight}{8in}
\setlength{\footskip}{0.75in}

\newtheorem{theorem}{Teorema}[section]
\newtheorem{acknowledgement}{Acknowledgement}
\newtheorem{algorithm}{Algorithm}
\newtheorem{axiom}{Axiom}
\newtheorem{case}{Case}
\newtheorem{claim}{Claim}
\newtheorem{propi}[theorem]{Propiedades}
\newtheorem{condition}{Condition}
\newtheorem{conjecture}{Conjecture}
\newtheorem{coro}[theorem]{Corolario}
\newtheorem{criterion}{Criterion}
\newtheorem{defi}[theorem]{Definición}
\newtheorem{example}[theorem]{Ejemplo}
\newtheorem{eje}{Ejercicio}
\newtheorem{lemma}[theorem]{Lema}
\newtheorem{nota}[theorem]{Nota}
\newtheorem{problem}{Problem}
\newtheorem{prop}[theorem]{Proposición}
\newtheorem{remark}{Remark}

\newtheorem{dem}[theorem]{Demostración}

\newtheorem{summary}{Summary}
\numberwithin{equation}{section}

\providecommand{\abs}[1]{\lvert#1\rvert}
\providecommand{\norm}[1]{\lVert#1\rVert}
\providecommand{\ninf}[1]{\norm{#1}_\infty}
\providecommand{\numn}[1]{\norm{#1}_1}
\providecommand{\gabs}[1]{\left|{#1}\right|}
\newcommand{\bor}[1]{\mathcal{B}(#1)}
\newcommand{\erre}{\mathbb{R}}
\newcommand{\resi}{\varepsilon_L}
\newcommand{\cee}{\mathbb{C}}
\providecommand{\conv}[1]{\overset{#1}{\longrightarrow}}
\providecommand{\convcs}{\xrightarrow{CS}}
% xrightarrow{d}[d]

%--------------------------------------------------------
\begin{document}

\title{Algebra Conmutativa y Geometría Aplicada}
\author{Rafael González López}
\maketitle

\begin{ejercicio}{1}
Calcular las soluciones $(x,y,z)$ enteras de $x^2+y^2=z^2$.
\begin{solucion}
Si dividimos por $z^2$ obtenemos $\left(\dfrac{x}{z}\right)^2+\left(\dfrac{y}{z}\right)^2=1$. Claramente existe una correspondencia biyectiva entre las soluciones racionales de nuestra ecuación y las de $a^2+b^2=1$. Consideremos el punto $(0,1)$ y la recta $y\equiv 0$. Si escogemos un punto de coordenadas racionales sobre la recta y lo unimos mediante otra recta con el $(0,1)$ obtenemos otro punto de corte sobre la circunferencia. Es decir, consideramos la recta que pasa por el $(0,1)$ y $(r,0)$ y resolvemos:
\[\begin{cases}
b=-\dfrac{a}{r}+1\\
a^2+b^2=1
\end{cases} \quad (a,b) = \left(\dfrac{2r}{r^2+1},\frac{r^2-1}{r^2+1}\right)
\]
Notamos que para cada punto de coordenadas racionales de la recta existe uno en la circunferencia menos un punto y recíprocamente. Además del caso especial del $(0,1)$. Como $r=\dfrac{p}{q}$, $(a,b)=\left(\dfrac{2pq}{p^2+q^2},\dfrac{p^2-q^2}{p^2+q^2}\right)$. Para pasar al problema inicial basta multiplicar por el denominador, es decir, la solución general del problema es:
\[(x,y,z)= \left(2pq,{p^2-q^2},{p^2+q^2}\right)\]
Donde $p,q\in\mathbb{Z}$ y $(x,y)$ pueden intercambiar papeles.
\end{solucion}
\end{ejercicio}
\newpage
\begin{ejercicio}{2}
Calcular las soluciones $(x,y,z)$ enteras de $x^2+y^2=2z^2$
\begin{solucion}
Podríamos usar un razonamiento análogo tomando el punto $(1,1)$ en la circunferencia $a^2+b^2=2$, pero lo cierto es que resulta demasiado tedioso. En su lugar observamos que las soluciones de esta ecuación verifican que:
\[
\left(\frac{x+y}{2}\right)^2+\left(\frac{x-y}{2}\right)^2=z^2
\]
Si hacemos un cambio de variable $2u=x+y$, $2v = x-y$. Resolver en $u,v$ es precisamente el ejercicio anterior, luego resta despejar $x$ e $y$:
\begin{align*}
x &= u + v = p^2+2pq -q^2\\
y &= u - v = - p^2+ 2pq +q^2\\
z &= \frac{\sqrt{x^2+y^2}}{2} =  p^2+q^2
\end{align*}
O, intercambiando las papeles de $u,v$: 
\begin{align*}
x &= u + v = p^2+2pq -q^2\\
y &= u - v =  p^2-2pq -q^2\\
z &= \frac{\sqrt{x^2+y^2}}{2} =  p^2+q^2
\end{align*}
\end{solucion}
Además, tener en cuenta que pueden cambiar los papeles de $x$ e $y$.
\end{ejercicio}
\newpage
\begin{ejercicio}{3}
Calcular las soluciones $(x,y,z)$ enteras de $x^2+y^2=3z^2$
\begin{solucion}
Vamos a demostrar que no existe ninguna solución entera. Supongamos que existe una tripleta $(x,y,z)$ de enteros que verifica la ecuación. En tal caso, dividimos por el $\gcd(x,y)$, la tripleta $\left(\dfrac{x}{d},\dfrac{y}{d},\dfrac{z}{d}\right)$ también es solución, que notaremos $(a,b,c)$. Nótese que $a$ y $b$ no pueden ser simultáneamente pares. Tengamos en cuenta que cualquier cuadrado ha de ser congruente con $0$ o $1$ módulo 4, por tanto:
\begin{align*}
a^2+b^2 \equiv 0,1,2 \mod 4 \quad 3c^2 \equiv 0, 3 \mod 4
\end{align*}
Por tato $a^2+b^2 \equiv 0 \mod 4$, de donde deducimos que $a^2 \equiv b^2 \equiv 0 \mod 4$. Esto es una contradicción, pues ambos no podían ser simultáneamente pares.
\end{solucion}
\end{ejercicio}
\end{document}