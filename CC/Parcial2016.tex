\documentclass[twoside]{article}
\usepackage{../estilo-ejercicios}
\usepackage{wasysym}
\usetikzlibrary{automata,positioning}
\usepackage{mathdots}
\newcommand{\sii}{{\Leftrightarrow}}
\usepackage{listings}
%--------------------------------------------------------
\begin{document}

\title{Ciencias de la Computación}

\author{Javier Aguilar Martín, Diego Pedraza López}
\maketitle

\begin{ejercicio}{1}
Sea $f : \N \to \N$ la función definida por
$$f(0) = 1,\quad f(n) = f(\lfloor\frac{n}{2}\rfloor) + n,\ \text{si }n \geq 1.$$
\begin{enumerate}
\item Diseñar un programa \texttt{GOTO} que calcule $f$.

(Solo podrán utilizarse las macros: \texttt{GOTO L}, $\texttt{W}\leftarrow \texttt{V}$ y $\texttt{W}\leftarrow   \lfloor\frac{\texttt{V}}{2}\rfloor$, siendo $\texttt{V}$ y $\texttt{W}$ variables
distintas. El programa tendrá 12 líneas como máximo).
\item Pruébese (utilizando la función recorrido de $f$) que $f$ es primitiva recursiva.
\end{enumerate}
\end{ejercicio}
\begin{solucion}\
\begin{enumerate}
\item Observemos que $f(\lfloor n/2\rfloor)=f(\lfloor\lfloor n/2\rfloor/2\rfloor)+\lfloor n/2\rfloor$. Con esto en mente podemos hacer el programa que se pide.

\begin{align*}
& Y\leftarrow Y+1\\
&Z_1\leftarrow X\\
& IF\ Z_1\neq 0\ GOTO\ A\\
& GOTO\ E\\
[C]&Z_2\leftarrow Z_1\\
[A]& Y\leftarrow Y+1\\
&Z_2\leftarrow Z_2-1\\
&IF\ Z_2\neq 0\ GOTO\ A\\
&GOTO\ B\\
[B]&Z_1\leftarrow \left\lfloor\frac{Z_1}{2}\right\rfloor\\
& IF\ Z_1\neq 0\ GOTO\ C\\
\end{align*}


\item Denotamos la función recorrido por $\hat{f}$. Sabemos que si $\hat{f}\in\mathcal{PR}$, entonces $f\in\mathcal{PR}$. $\hat{f}(0)=[f(0)]=2^1=2$, $$\hat{f}(n)=[1,f(1),f(2),\dots, f(n)]=\prod_{i=1}^{n} p_i^{f(i-1)}=\hat{f}(n-1)p_{n+1}^{f(n)}$$
Ahora, como $f(n)=f(\lfloor\frac{n}{2}\rfloor) + n$, y $f(\lfloor\frac{n}{2}\rfloor)$ está guardado en la función recorrido, 
$$p_{n+1}^{f(n)}=p_{n+1}^{f(\lfloor\frac{n}{2}\rfloor) + n}=p_{n+1}^{(\hat{f}(n-1))_{\lfloor\frac{n}{2}\rfloor+1}+n}$$
Luego la definición recursiva de $\hat{f}$ es la siguiente
\[
\begin{cases}
\hat{f}(0)=0 \\
\hat{f}(n)=h(n-1,\hat{f}(n-1)) & n\geq 1
\end{cases}
\]
Donde $h(x,y)=p_{x+2}^{((y)_{\lfloor\frac{x+1}{2}\rfloor+1}+x+1)}$.
\end{enumerate}
\end{solucion}

\newpage

\begin{ejercicio}{2}
Se pide:
\begin{enumerate}
\item Probar que la siguiente función $F : \N^2\dashrightarrow \N$ es \texttt{GOTO}-computable, siendo
\[
F(x,e)=\begin{cases}
1 & \text{si la computación del programa de código }e\text{ sobre }x\text{ tiene longitud impar}.\\
2 & \text{si la computación del programa de código }e\text{ sobre }x\text{ tiene longitud par}.\\
\uparrow & \text{en otro caso}.
\end{cases}
\]
(\textbf{Indicación}: Utilícese el predicado \texttt{STEP}).
\item Probar que existe una función primitiva recursiva $g : \N^2 \to \N$ tal que para todo
$e_1, e_2 \in \N$ y todo $x \in \N$ se tiene:
$$\varphi_{g(e_1,e_2)}^{(1)}=\varphi_{e_1}^{(1)}+\varphi_{e_2}^{(1)}.$$
\end{enumerate}
\end{ejercicio}
\begin{solucion}\
\begin{enumerate}
\item La computación del programa de código $e$ sobre $x$ tiene longitud impar$\Leftrightarrow (\exists t)_{\leq r}(\texttt{STEP}(x,e,t)\land 2\not| t\land\neg \texttt{STEP}(x,e,t-1))$, donde $r=(\mu s)(\texttt{STEP}(x,e,s))$, que es recursivo. Análogamente, la segunda condición equivale a $(\exists t)_{\leq r}(\texttt{STEP}(x,e,t)\land 2| t\land\neg \texttt{STEP}(x,e,t-1))$. Obsérvese que en caso de que el programa no pare nunca, ambos predicados no paran, por lo que $f$ no para, que es lo que deseamos.
\item Sea $\texttt{p}_1$ el programa de código $e_1$ y $\texttt{p}_2$ el programa de código $e_2.$ Entonces la suma vendrá dada por el siguiente programa
\begin{lstlisting}[mathescape=true]
                     $\texttt{p}_1$
                     $Z_1\leftarrow$ Y
                     Y$\leftarrow$ 0
                     $\texttt{p}_2$
                     $Z_2\leftarrow$ Y
                     Y$\leftarrow Z_1+Z_2$
\end{lstlisting}
Con lo que basata calcular el código de este programa. Al calcularlo estaremos usando las mismas funciones que se usan para calcular cualquier código, que son todas primitivas recursivas.
\end{enumerate}
\end{solucion}

\newpage

\begin{ejercicio}{3}
\end{ejercicio}\
\begin{enumerate}
\item Sea $\texttt{GOTO}^*$ el conjunto de los programas \texttt{GOTO} que solo contienen instrucciones condicionales
del tipo \texttt{IF V} $\neq 0$ \texttt{GOTO L} (etiquetadas o no) cuando \texttt{V} no es una variable de
entrada. Pruébese que el conjunto $S = \{\#(\texttt{P}) : \texttt{P} \in \texttt{GOTO}^*\}$ es primitivo recursivo.

\item Sea $\texttt{GOTO}_A$ el modelo de computación definido como \texttt{GOTO}, pero sin instrucciones de
decremento y añadiendo instrucciones de asignación del tipo $\texttt{V}\leftarrow \texttt{W}$ (con o sin etiquetas)
y condicionales del tipo \texttt{IF V} $=$ \texttt{W GOTO L}, para cada par de variables distintas \texttt{V} y \texttt{W}.
Probar que \texttt{GOTO} y $\texttt{GOTO}_A$ son equivalentes.
\end{enumerate}
\begin{solucion}\
\begin{enumerate}
\item $e\in S\Leftrightarrow (\forall j)_{\leq long(e+1)}(l(r((e+1)_j))\leq 2 impar(\lor r(r((e+1)_j)+1))$. El predicado $impar(x)$ es primitivo recursivo ya que $impar(x)\Leftrightarrow 2\not| x\Leftrightarrow \neg (\exists t)_{\leq x}(2t=x)\Leftrightarrow (\forall t)_{\leq x}(2t\neq x)$. 
\item Sabemos que las intrucciones de $\texttt{GOTO}_A$ son \texttt{GOTO}-computables ya que son macros definidas en teoría. Por tanto, tenemos que ver que podemos generar un decremento con las instrucciones de $\texttt{GOTO}_A$. El siguiente programa calcula la función decremento.
\begin{lstlisting}[mathescape=true]
                     $Z_1\leftarrow$Y
                     IF $Z_1\neq 0$ GOTO A
                     GOTO E
                 [A] $Z_3\leftarrow Z_2+1$
                     IF $Z_3=Z_1$ GOTO B
                     $Z_2\leftarrow Z_2+1$
                     GOTO A
                 [B] Y$\leftarrow Z_2$  
\end{lstlisting}

\end{enumerate}
\end{solucion}

\newpage

\begin{ejercicio}{4}
Sea $g : \N^2 \to \N$ definida por $g(n, x) = p^x_n - 1$. Se pide:
\begin{enumerate}
\item Calcular $\mathcal{U}_1(2, g(1000, 2))$.
\item Dado $n \in \N$. ¿se verifica $\texttt{STEP}^{(1)}(2, g(1000, n), 1000)$? Razónese la respuesta.
\end{enumerate}
\end{ejercicio}
\begin{solucion}\
\begin{enumerate}
\item En primer lugar, $g(1000,2)=p_{1000}^2-1$, luego buscamos el programa de código $\#(\texttt{p})=p_{1000}^2-1$. Esto significa que el programa tiene 999 instrucciones del tipo $Y\leftarrow Y$ al principio y acaba con la instrucción codificada por el número $2=\langle a,\langle b,c\rangle\rangle=2^a(2y+1)-1$, de donde deducimos que $a=0$, lo que quiere decir que la instrucción no está etiquetada. También deducimos que $y=1=\langle b,c\rangle=2^b(2c+1)-1$, de donde $b=1, c=0$, por lo que la instrucción es $Y\leftarrow Y+1$. Esto quiere decir que la función que se calcula es la constante 1, de modo que $\mathcal{U}_1(2, g(1000, 2))=1$.
\item El programa de código $g(1000,n)$ tiene exactamente 1000 instrucciones, y solo la última es distinta de $Y\leftarrow Y$, por lo que independientemente del tipo de instrucción que sea (se prueba por inspección) parará tras esa instrucción, así que sí se verifica el predicado. 
\end{enumerate}
\end{solucion}

\newpage

\begin{ejercicio}{5}
Consideremos el alfabeto $\Sigma = \{a, b\}$. Describir una máquina de
Turing que calcule la función $f : \Sigma^* \times \Sigma^* \to \Sigma^*$ definida por $f(x, y) = x$
\end{ejercicio}
\begin{solucion}
\url{http://turingmachinesimulator.com/shared/xsuwvlyyse}
\end{solucion}
\end{document}