\documentclass[twoside]{article}
\usepackage{../estilo-ejercicios}
\usepackage{wasysym}
\usetikzlibrary{automata,positioning}
\usepackage{mathdots}
%--------------------------------------------------------
\begin{document}

\title{Ciencias de la Computación}

\author{Javier Aguilar Martín}
\maketitle

\begin{ejercicio}{5}
Calcúlese:
\begin{enumerate}
\item $\mathcal{U}_1(2,575)$.
\item $\mathcal{U}_1(x,q^2-1)$, donde $q$ es un número primo.
\item $\mathcal{U}_2(1,1,3^m-1)$ donde $m$ tal que $m+1$ es potencia de 2.
\item $\mathcal{U}_2(1,1,\#(U_1))$ 
\end{enumerate}
\end{ejercicio}
\begin{solucion}\
\begin{enumerate}
\item  $\mathcal{U}_1(2,575)=[[U_1]]^{(2)}(2,575)=[[p]]^{(1)}(2)$, siendo $\#(P)=575$. $\#(P)+1=576=2^6 3^2=[6,2]$ (el programa tiene 2 instrucciones). Sabemos que $\#(I_1)=6,\#(I_2)=2$.$\#(I_1)=\langle a,\langle b,c\rangle\rangle=\langle a,d\rangle=2^a(2d+1)-1$, de donde deducimos que $a=0,d=3=\langle b,c\rangle=2^b(2c+1)-1$, lo que significa que $b=2,c=2$. Análogamente calculamos $\#(I_2)=\langle 0, \langle 1,0\rangle\rangle$. Por tanto $I_1=Y\leftarrow Y-1, I_2=Y\leftarrow Y+1$. Así pues, $[[p]]^{(1)}(2)=1$. 
\item  Buscamos $\#(Q)=q^-1$. Sea $k$ tal que $q=p_k$. $\#(Q)+1=q^2=p_k^2=[0,\dots,0,2]$, de longitud $k$. Las $k-1$ primeras instrucciones son $Y\leftarrow Y$, y la última es $Y\leftarrow Y+1$. 
\item  Buscamos $[[p]]^{(2)}(1,1)$ siendo $\#(P)=3^m-1$. Tenemos que $\#(P)+1=3^m=[0,m]$, luego la primera instrucción es $Y\leftarrow Y$, y la segunda cumple $\#(I)=m=\langle a,\langle b,c\rangle\rangle$, de donde deducimos que $m+1=2^a(2d+1)$. Como $m+1$ es potencia de 2, $d=0\Rightarrow b=c=0$. Por lo que la instrucción es $\langle a\langle 0,0\rangle\rangle$, que es de la forma $[L] Y\leftarrow Y$, con $\#(L)=a$. El valor del programa es siempre 0.
\item $\mathcal{U}_2(1,1,\#(U_1))=[[U_1]]^{(2)}(1,1)=[[p]]^{(1)}(1)$, siendo $\#(P)=1$, con lo que $\#(P)+1=2=[1]$. Es decir, que el programa consiste en la instrucción $[A_1] Y\leftarrow Y$, que vale 0.
\end{enumerate}
\end{solucion}

\newpage
\begin{ejercicio}{6}
Probar que son recursivos:
\begin{enumerate}
\item 
\item
\item
\item
\item
\end{enumerate}
\end{ejercicio}
\begin{solucion}\
\begin{enumerate}
\item Podemos escribir este conjunto como $\{e\in\N: e$ es el número de Gödel de un programa sin variables auxiliares$\}$. Sea $long(e+1)$ la longitud del programa que tiene ese código y $(e+1)_i$ la $i$-ésima instrucción de dicho programa. $e\in A\Leftrightarrow(\forall j)_{\leq long(e+1)}[j\geq 1\rightarrow (r(r((e+1)_j))=1\lor 2r(r((e+1)_j)))]$ que es primitivo recursivo.
\item $e\in B\Leftrightarrow long(e+1)\leq 7 \land (\exists i)_{\leq long(e+1)}(l(r((e+1)_i))=2$. Primitivo recursivo. Si el existencial no estuviera acotado seguiría sirviendo porque sería recursivo.
\item
\item
\item
\end{enumerate}
\end{solucion}
\newpage

\begin{ejercicio}{8}
Encontrar $f : \N \to \N$ primitiva recursiva e inyectiva tal que para todo $n$, $φ_{f(n)} = \mathcal{O}$.
\end{ejercicio}
\begin{solucion}
La estrategia que usaremos será hacer corresponde $f(n)$ con el programa que tiene $n$ ordenes de la forma $Y \leftarrow Y-1$. Como $\#(Y \leftarrow Y-1) = \langle 0, \langle 2,0\rangle \rangle = 4$, podemos definir
\[\begin{cases}
	f(0) = 2^4-1 \\
	f(n+1) = (f(n)+1)\cdot p_{n+2}-1
\end{cases}\]
\end{solucion}
que es claramente primitiva recursiva, inyectiva y $φ_{f(n)} = \mathcal{O}$.
\newpage
\begin{ejercicio}{9}
Probar que la función total $f : \N \to \N$ definida por
\[ f(x) = \begin{cases}
	φ_x(x)+1 &\text{ si }φ_x(x) \downarrow\\
	0 & \text{ en otro caso}
\end{cases}\]
no es recursiva (Indiciación: Si fuese recursiva, tomar $e$ tal que $φ_e = f$).
\end{ejercicio}
\begin{solucion}
Usaremos reducción al absurdo. Supongamos que $f \in \mathcal{P}$. En este caso, existiría $e \in \N$ tal que $f = φ_e$. Como $f$ es total, está definida en $f(e)$, luego $φ_e(e)\downarrow$. Pero entonces $φ_e(e) = f(e) = φ_e(e)+1$. Esto es absurdo.
\end{solucion}

\newpage
\begin{ejercicio}{11}
Sea $p \in \textsc{goto}_p$. Probar que es recursiva la función.
\[ g(x) = \begin{cases}
	\text{núm. de Gödel de la sucesión de valores que toma }Y\text{ en la ejecución de }p\text{ sobre }x &\text{ si }[\![p]\!](x)\downarrow\\
	\uparrow & \text{c.c}
\end{cases} \]
\end{ejercicio}
\begin{solucion}
Sea $f(x,t)=\left(r(di^{(1)}(x,\#(p),t)\right)_1$. Tenemos que $f : \N^2 \to \N$ es primitiva recursiva. Entonces la función recorrido $\hat{f}(x,t) = [f(x,0),\dots,f(x,t)]$ es primitiva recursiva. Sea $h(x) = (μt) (\textsc{step}^{(1)} (x,\#(p),t))$, que es recursiva (en particular es $μ$-recursión). Como $g(x) = \hat{f}(x,h(x))$, $g$ es recursiva.
\end{solucion}

\newpage
\begin{ejercicio}{12}
Probar que la siguiente función es recursiva: $h(x,y) = \max\{φ_0(y),\dots,φ_x(y)\}$.
\end{ejercicio}
\begin{solucion}
Se tiene que:
\[ \begin{cases}
	h(0,y) = U_1(y,0)\\
	h(x+1,y) = \max(h(x,y),U_1(y,x+1))
\end{cases}\]
Es decir, $h = R(g',h')$ donde $g'(y) = U_1(y,0)$ y $h'(x,y,z)=\max(z,U_1(y,x+1))$. Como $g'$ y $h'$ son recursiva, $h$ es recursiva.
\end{solucion}
HACER 3 y 18

\end{document}