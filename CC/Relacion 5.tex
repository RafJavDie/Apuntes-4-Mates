\documentclass[twoside]{article}
\usepackage{../estilo-ejercicios}
\usepackage{wasysym}
\usetikzlibrary{automata,positioning}
\usepackage{mathdots}
\newcommand{\sii}{{\Leftrightarrow}}
%--------------------------------------------------------
\begin{document}

\title{Ciencias de la Computación}

\author{Javier Aguilar Martín, Diego Pedraza López}
\maketitle

\begin{ejercicio}{1}
Sea $P(\vec{x},y_1,\dots,y_n)$ un predicado recursivo. Probar que el conjunto
\[ A = \{\vec{x} : (\exists y_1)\cdots(\exists y_n) P(\vec{x},y_1,\dots,y_n)\} \]
es recursivamente enumerable.
\end{ejercicio}
\begin{solucion}
Tenemos que:
\begin{align*}
	\vec{x} \in A & \sii (\exists y_1)\dots(\exists y_n) P(\vec{x},y_1,\dots,y_n)\\
	& \sii (\exists z) P(\vec{x},(z)_1,\dots,(z)_n)
\end{align*}
Como $P(\vec{x},(z)_1,\dots,(z)_n)$ es recursivo, $A$ es recursivo enumerable.
\end{solucion}

\newpage

\begin{ejercicio}{2}
Sea $P(x,y,u)$ un predicado recursivo. Probar que el conjunto
\[ A = \{(x,z) : (\forall y)_{≤z} (\exists u) P(x,y,u)\} \]
es recursivamente enumerable.
\end{ejercicio}
\begin{solucion}
Como hay una cantidad finita de $y≤z$, también hay una cantidad finita de $u$. Es decir podemos acotar el existencical.
\begin{align*}
	(x,z) \in A & \sii (\forall y)_{≤z} (\exists u) P(x,y,u)\\
	& \sii (\exists v)(\forall y)_{≤z} (\exists u)_{≤v} P(x,y,u)
\end{align*}
 que es recursivamente enumerable, luego $A$ es recursivamente enumerable.
\end{solucion}

\newpage

\begin{ejercicio}{3}
Sea $P(x,y,u)$ un predicado r.e. Probar que el conjunto
\[ A = \{(x,z) : (\forall y)_{≤z} (\exists u) P(x,y,u)\} \]
es recursivamente enumerable.
\end{ejercicio}
\begin{solucion}
Sea $R(x,y,u,v)$ recursivo tal que $P(x,y,u) \sii (\exists v) R(x,y,u,v)$ (existe por el teorema de la proyección).
\begin{align*}
	(x,z) \in A & \sii (\forall y)_{≤z} (\exists u) P(x,y,u)\\
	& \sii (\forall y)_{≤z} (\exists u) (\exists v) R(x,y,u,v)\\
	& \sii (\forall y_{≤z}) (\exists w) R(x,y,l(w),r(w))
\end{align*}
Por el ejercicio \ref{ejer:2}, este último predicado es recursivo enumerable. 
\end{solucion}

\newpage

\begin{ejercicio}{4}
Sea $f \in \mathcal{P}^{(1)}$. Probar que:
\begin{enumerate}
	\item $A \subseteq \N \land A$ r.e. $\Rightarrow f(A)$ r.e. y $f^{-1}(A)$ r.e.
	\item $f$ total $\land\ A \subseteq \N \land A$ recursivo $\Rightarrow f^{-1}(A)$ recursivo.
\end{enumerate}
\end{ejercicio}
\begin{solucion}\
\begin{enumerate}
\item $x\in f(A)\sii (\exists u) (u\in A\land f(u)=x)$. Como $f$ es recursiva $f(u)=x$ es recursivamente enumerable. Como también $u\in A$ es r.e, y añadir existenciales también, el resultado lo es. 

$x\in f^{-1}(A)\sii f(x)\in A\sii (\exists y) (f(x)=y\land y\in A)$. Por lo mismo de antes se tiene el resultado. 

\item Si $f\in\mathcal{R}$ y $A$ es recursivo, $x\in f^{-1}(A)\sii f(x)\in A\sii \mathcal{C}_A(f(x))=1$, que es composición de recursivas totales y ecuación entre funciones totales, por lo que es recursivo.
\end{enumerate}
\end{solucion}

\newpage

\begin{ejercicio}{5}
Sea $A \subseteq \N$. Probar que son equivalentes:
\begin{enumerate}
	\item $A$ es r.e.
	\item Existe $f \in \mathcal{P}$ (posiblemente parcial) tal que $\text{rang}(f) = A$.
	\item Existe $f \in \mathcal{P}\mathcal{R}$ tal que $\text{rang}(f)=A$.
\end{enumerate}
\end{ejercicio}
\begin{solucion}\mbox{}
\begin{itemize}
	\item[$(1\Rightarrow 3)$] Teorema de enumeración.
	\item[$(3\Rightarrow 2)$] Trivial.
	\item[$(2\Rightarrow 1)$] Teorema del rango.
	
	Sea $f \in \mathcal{P}$: Por el teorema de la forma normal. $f(x)\!\downarrow \Leftrightarrow \exists z \mathcal{T}_n(x,e,z)$ y $f(x) = l((μz)\mathcal{T}_n(x,e,z))$.
	Entonces:
	\begin{align*}
	y \in \text{rang}(f) & \Leftrightarrow (\exists x) (f(x)=y)\\
	& \Leftrightarrow (\exists x)(\exists z) (\mathcal{T}_n(x,e,z) \land l(z) = y)\\
	& \Leftrightarrow (\exists x) (\exists t) (\texttt{STEP}(x,e,t)) \land ((r(\texttt{di}(x,e,t)))_1 = y)\\
	& \Leftrightarrow (\exists z) (\texttt{STEP}(l(z),e,r(z))) \land ((r(\texttt{di}(l(z),e,r(z)))_1 = y)
\end{align*}
\end{itemize}
\end{solucion}

\newpage

\begin{ejercicio}{6}
Sea $A \subseteq \N$ tal que $A \neq \emptyset$. probar que son equivalentes:
\begin{enumerate}
	\item $A$ es recursivo.
	\item Existe $f \in \mathcal{R}^{(1)}$ tal que $\forall x \in \N$ $(f(x) ≤ f(x+1)) \land A = \text{rang}(f)$.
\end{enumerate}
\end{ejercicio}
\begin{solucion}\
\begin{itemize}
\item[$(2\Rightarrow 1)$]
$x\in A\sii x\in\text{rang}(f)\Leftrightarrow (\exists y)_{ \leq f(x+1)}(f(x)=y)$, que es recursivo. 
\item[$(1\Rightarrow 2)$]Sea $A$ recursivo no vacío. Entonces existe $g\in\mathcal{R}$ tal que $A=rang(g)$. Sea $r=(\mu y)(y\in A)$ y sea $s=(\mu y)(f(y)=r)$. Definimos
\begin{align*}
&f'(0)=[g(s)]\\
&f'(x+1)=f'(x)*[g((\mu u)((f'(x))_{Lt(f'(x))}< g(u))]
\end{align*}
Está claro que $f'\in\mathcal{R}$. Definimos $f(x)=(f'(x))_x$. Entonces $f\in\mathcal{R}$ y $rang(f)=A$. Es obvio que $rang(f)\subseteq rang(g)$. Y si $x\in rang(g)\Rightarrow(\exists u)(g(u)=x)\Rightarrow (\exists u)(g(u)\in f'(u))\Rightarrow (\exists u)(\exists i)_{\leq u} (x=f(i))$. 
\end{itemize}
\end{solucion}

\newpage

\begin{ejercicio}{7}
Sea $A\subseteq\N$ infinito. Probar que son equivalentes:
\begin{enumerate}
\item $A$ es r.e.,
\item Existe $f\in\mathcal{R}^{(1)}$ tal que $f$ es inyectiva y $A=\text{rang}(f)$.
\end{enumerate}
\end{ejercicio}
\begin{solucion}\
\begin{enumerate}
\item[$(2\Rightarrow 1)$]
$x\in A\sii x\in\text{rang}(f)\Leftrightarrow (\exists y)(f(x)=y)$, que es recursivamente enumerable. 
\item[$(1\Rightarrow 2)$] Sea $A$ r.e. infinito. Entonces existe $g\in\mathcal{R}$ tal que $A=rang(g)$. Definimos
\begin{align*}
&f'(0)=[g(0)]\\
&f'(x+1)=f'(x)*[g((\mu u)((\forall i)_{\leq x} ((f'(x))_i)\neq g(u))]
\end{align*}
Está claro que $f'\in\mathcal{R}$. Definimos $f(x)=(f'(x))_x$. Entonces $f\in\mathcal{R}$ y $rang(f)=A$. Es obvio que $rang(f)\subseteq rang(g)$. Y si $x\in rang(g)\Rightarrow(\exists u)(g(u)=x)\Rightarrow (\exists u)(g(u)\in f'(u))\Rightarrow (\exists u)(\exists i)_{\leq u} (x=f(i))$. 

\end{enumerate}
\end{solucion}

\newpage

\begin{ejercicio}{8}
Sea $A \subseteq \N$ infinito y r.e. Probar que existe $B \subseteq A$ recursivo e infinito.
\end{ejercicio}
\begin{solucion}
Por el teorema de la enumeración existe $f\in\mathcal{P}$ tal que $A=\{f(0),f(1),\dots\}=rang(f)$. Sea $i\in\mathcal{R}$ definida como sigue
\begin{align*}
&i(0)=0\\
&i(n)=(\mu m)(f(m)> f(i(n-1)))
\end{align*}
Efectivamente $i\in\mathcal{R}$. Si definimos ahora $g(x)=f(i(x))$, entonces $g\in\mathcal{R}$ por composición y es creciente, por lo que aplicando el ejercicio \ref{ejer:6}, $rang(g)$ es recursivo. Claramente $rang(g)$ es un conjunto infinito y además $rang(g)\subseteq A$, por lo que hemos probado el resultado.
\end{solucion}

\newpage

\begin{ejercicio}{11}
¿Es cierto que toda función total $f : \N^n \to \N$ cuyo rango sea un conjunto finito es
recursiva?
\end{ejercicio}
\begin{solucion}
Basta considerar la función característica de cualquier conjunto recursivamente enumerable no recursivo. 
\end{solucion}

\end{document}