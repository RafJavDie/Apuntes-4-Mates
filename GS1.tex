\documentclass[twoside]{report}
\usepackage[utf8x]{inputenc}
\usepackage[spanish]{babel}
\usepackage{amssymb}
\usepackage{amsmath}
\usepackage{amsthm}
\usepackage{hyperref}
\usepackage{subfiles}

\SetUnicodeOption{mathletters}
\SetUnicodeOption{autogenerated}

\renewcommand{\baselinestretch}{1,4}
\usepackage[papersize={210mm,297mm},
            twoside,
            includehead,
            top=1in,
            bottom=1in,
            inner=0.75in,
            outer=1.0in,
			bindingoffset=0.35in]{geometry}

\theoremstyle{definition}
\newtheorem{theorem}{Teorema}[section]
\newtheorem{propi}[theorem]{Propiedades}
\newtheorem{condition}{Condition}
\newtheorem{consecuencia}[theorem]{Consecuencia}
\newtheorem{observacion}[theorem]{Observación}
\newtheorem{coro}[theorem]{Corolario}
\newtheorem{defi}[theorem]{Definición}
\newtheorem{example}[theorem]{Ejemplo}
\newtheorem{lemma}[theorem]{Lema}
\newtheorem{nota}[theorem]{Nota}
\newtheorem{prop}[theorem]{Proposición}
\newtheorem*{dem}{Demostración}
\newcommand*{\QED}{\hfill\ensuremath{\blacksquare}}

\newtheorem{summary}{Summary}
\numberwithin{equation}{section}

\newcommand{\R}{\mathbb{R}}
\newcommand{\X}{\mathbb{X}}
\newcommand{\resi}{\varepsilon_L}
\providecommand{\conv}[1]{\overset{#1}{\longrightarrow}}
\providecommand{\convcs}{\xrightarrow{CS}}
\providecommand{\supf}{\mathbb{X}}
\providecommand{\lrg}{\longrightarrow}
%--------------------------------------------------------
\begin{document}
\chapter{Geometría}
\section{Paralelismo en superficie}
Sean $\X:U\to\R^3$ s.s., $\alpha=\X(u(t),v(t))$ c.p.r. en $\X$.

\begin{defi}
Diremos que $X(t)$ es un campo vectorial tangente a lo largo de $\alpha$ si $X(t) \in T_{\alpha(t)}(\X)$. Por ejemplo, $\dot{\alpha}(t)$ o $S(t)$.
\end{defi}

\begin{defi}
Sea X(t) un c.v.t. a lo largo de $\alpha$. Se dirá que $X(t)$ es paralelo a lo largo de $\alpha$ si $X'(t) \parallel N(t)$. El parámetro t no es esencial para la definición de campo paralelo.
\end{defi}

\begin{theorem}
El c.v.t. $X(t)=\sum_{k=1}^n X^k \X_k$ a lo largo de α es campo paralelo a lo largo de α si y sólo si:
\[ \frac{dX^k(t)}{dt} + \sum_{i,j=1}^2 Γ_{ij}^k X^i(t) \frac{du^j}{dt} = 0 \quad \forall k=1,2 \]
\end{theorem}
\begin{dem}
Derivando $X(t)$:
\begin{align*} X'(t) & = \sum_{k=1}^2 \left(\frac{dX^k(t)}{dt}\X_k + \sum_{j=1}^2 X^k(t)\X_{kj}\frac{du^j}{dt}\right) \\
 & \overset{\text{Ec. Gauss}}{=} \sum_{k=1}^2 \left(\frac{dX^k(t)}{dt}\X_k + \sum_{j=1}^2 X^k(t)\left[\sum_{i=1}^2 Γ_{kj}^i \X_i + L_{kj}N\right]\frac{du^j}{dt}\right) \\
 & = \sum_{k=1}^2 \left[ \frac{dX^k}{dt}(t) + \sum_{i,j=1}^2 Γ_{ij}^k X^i(t) \frac{du^j}{dt}\right]\X_k + \left[\sum_{i,j=1}^2 L_{ij} X^i(t) \frac{du^j}{dt}\right]N(t)
\end{align*}
Llamamos al primer sumando \textbf{parte  tangente} y al segundo sumando, \textbf{parte normal}. De aquí:
\[ X'(t) \parallel N(t) \Leftrightarrow \frac{dX^k}{dt} + \sum Γ_{ij}^k X^i(t) \frac{du^j}{dt} = 0\quad \forall k=1,2\]
$\QED$
\end{dem}

\begin{defi}
Llamamos \textbf{derivada covariante} de $X(t) = \sum_{k=1}^2 X^k \X_k$ a:
\[ \frac{DX(t)}{dt} = \sum_{k=1}^2 \left[ \frac{dX^k(t)}{dt} + \sum_{i,j=1}^2 Γ_{ij}^k X^i(t) \frac{du^j}{dt}\right]\X_k\]
\end{defi}

\begin{consecuencia}
$X(t)$ c.v.t. es paralelo a lo largo de $α$ si y sólo si $\frac{DX(t)}{dt} = 0$.
\end{consecuencia}

\begin{consecuencia}
El concepto de \emph{paralelismo} (campo paralelo) es intrínseco de la superficie. Recuérdese que lo símbolos de Christoffel son intrínsecos a la superficie, ya que:
\[ Γ_{ij}^k = \frac{1}{2} \sum_{h=1}^2 g^{hk} (g_{ih,j}+g_{jh,i}-g_{ij,h}) \]
donde $g_{xy,z} = \dfrac{\partial g_{xy}}{\partial u^z}$.
\end{consecuencia}

\begin{theorem}[Existencia y unicidad de campos paralelos]
Sea $α(t) = \X(u¹(t), u²(t))$ c.p.r. en $\X : U \to \R³$. Entonces existe un único c.v.t. $X(t)$ paralelo a lo largo de $α$ con $X(t_0)=v_0$.
\end{theorem}

\begin{dem}
Si existe $X(t) = \sum_{k=1}^2 X^k \X_k$ c.v.t. paralelo a lo largo de $α$, $X^k(t)$ cumple que:
\[ \frac{dX^k(t)}{dt} + \sum_{i,j=1}^2 Γ_{ij}^k(t) X^i(t) \frac{du^j}{dt} = 0 \quad k=1,2 \]
\[ X^1(t_0) = v_0^1  \qquad X^2(t_0) = v_0^2 \]
Estamos ante 2 problemas de Cauchy. Por el teorema de Picard, existe solución y es única.
$\QED$
\end{dem}

\begin{defi}
El campo $X(t)$ obtenido integrando el sistema anterior y que cumple $X(t_0) = v_0$, se llama el \textbf{transporte paralelo} de $v_0$ a lo largo de $α$.
\end{defi}

%\begin{observacion}
%Dada $α(t)$ y $v_0$, el campo $X(t)$ paralelo es único, pero si el camino $α$ va de $p$ a $q$ de $X^k$ de $X(t)$ satisface las ecuaciones diferenciales de $(1.13)$
%\end{observacion}

\begin{observacion}
El transporte paralelo depende del camino.
\end{observacion}

\begin{example}
En $S^2(1)$, consideramos la curva $α(θ) = (\cos θ \cos φ_0, \sen θ \cos φ_0, \sen φ_0)$ con ${θ \in [0,2\pi]}$. Se tiene:
\[ \X_1 = (- \sen θ \cos φ, \cos u \cos φ, 0) \]
\[ \X_2 = (- \cos θ \sen φ, - \sen θ \sen φ, \cos φ) \]
\[ g_{11} = \cos²φ \qquad g_{12} = 0 \qquad g_{22} = 1 \]
\[ Γ_{12}^1 = - \tan φ \qquad Γ_{11}^2 = \cos φ \sen φ \]
\[ u¹ = θ, u² = φ \]

Para que $X(θ) = X^1(θ) \X_1(θ,φ_0) + X^2 \X_2(θ, φ_0)$ sea c.v.t. paralelo, se debe cumplir que:
\[ \frac{dX^1}{dθ} - \tan φ_0 X^2 = 0 \]
\[ \frac{dX^2}{dθ} + \cos φ_0 \sen φ_0 X^1 = 0 \]
De ahí se obtiene integrando que:
\[ X^1(θ) = \frac{1}{\cos φ_0} \sen ( θ \sen φ0) \]
\[ X^2(θ) = \cos(θ \sen φ_0) \]
Luego:
\[ X(θ) = \frac{1}{\cos φ_0} \sen (θ \sen φ_0) \X_1 - \cos (θ \sen φ_0) \X_2 \]
Sin embargo, $X(0) = \X_2(0,φ_0) \neq X(2\pi) = \frac{1}{\cos φ}\sen (2\pi \sen φ_0) \X_1 - \cos (2 \pi \sen φ_0) \X_2$. El transporte paralelo del vector $X(0)$ no va a mantener siempre constante el vector al pasar la curva por el mismo punto.
\end{example}
\end{document}