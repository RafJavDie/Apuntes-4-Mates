\documentclass[twoside]{article}
\usepackage{../../estilo-ejercicios}
\newcommand{\colapso}{{\searrow\!\!\!\!\searrow}}
%--------------------------------------------------------
\begin{document}

\title{Ejercicios de Homología Simplicial}
\author{Diego Pedraza López, Javier Aguilar Martín}
\maketitle

\begin{ejercicio}{1.1}
Sea $K$ un complejo simplicial y $L \subseteq K$ un subcomplejo.
Se dice que $L$ es un \emph{subcomplejo lleno} de $K$ si para todo símplice $σ \in K$ la intersección $σ \cap |L|$ es una cara de $σ$ (posiblemente vacía).
Demostrar que son equivalentes:
\begin{enumerate}
\item $L$ es un subcomplejo lleno de $K$.
\item Ningún símplice de $K-L$ corta a $L$ en todo su borde.
\item Si los vértices de $σ$ están en $L$, $σ \in L$.
\end{enumerate}
\end{ejercicio}
\begin{solucion}
$(1\Rightarrow 2)$ Si $L$ es un conjunto de vértices, entonces $\partial L=\emptyset$, por lo que el resultado es trivial. En cualquier otro caso, el borde de $L$ está formado por más de una cara propia, por lo que si algún símplice $\sigma\in K-L$ corta a $L$ en todo su borde, $\sigma\cap |L|$ no puede ser una (única) cara de $\sigma$, lo que contradice la definición de subcomplejo lleno.

$(2\Rightarrow 3)$ Juraría haber encontrado un contraejemplo. Esta implicación es equivalente a $(\neg 3\Rightarrow\neg 2)$, es decir, si existe algún $\sigma$ cuyos vértices estén todos en $L$ pero $\sigma\notin L$, entonces existe algún símplice de $K-L$ que corta a $L$ todo su borde. ¿Pero qué pasa con esto?
\definecolor{ududff}{rgb}{0.30196078431372547,0.30196078431372547,1.}
\definecolor{xdxdff}{rgb}{0.49019607843137253,0.49019607843137253,1.}
\definecolor{uuuuuu}{rgb}{0.26666666666666666,0.26666666666666666,0.26666666666666666}
\begin{tikzpicture}[line cap=round,line join=round,>=triangle 45,x=1.0cm,y=1.0cm]
\clip(-2.689333333333333,-0.5) rectangle (9.577333333333332,3);
\fill[line width=2.pt,fill=black,fill opacity=0.10000000149011612] (0.,0.) -- (3.,0.) -- (1.5,2.5980762113533165) -- cycle;
\draw [line width=2.pt] (0.,0.)-- (3.,0.);
\draw [line width=2.pt] (3.,0.)-- (1.5,2.5980762113533165);
\draw [line width=2.pt] (1.5,2.5980762113533165)-- (0.,0.);
\draw [line width=2.pt] (3.,0.)-- (5.,2.);
\begin{scriptsize}
\draw [fill=uuuuuu] (0.,0.) circle (2.0pt);
\draw[color=uuuuuu] (-0.18266666666666662,0.17733333333333348) node {$A$};
\draw [fill=xdxdff] (3.,0.) circle (2.5pt);
\draw[color=xdxdff] (3.0173333333333328,0.252) node {$B$};
\draw [fill=uuuuuu] (1.5,2.5980762113533165) circle (2.0pt);
\draw[color=uuuuuu] (1.577333333333333,2.7693333333333356) node {$C$};
\draw [fill=ududff] (5.,2.) circle (2.5pt);
\draw[color=ududff] (5.076,2.193333333333335) node {$D$};
\draw[color=black] (4.148,0.9773333333333342) node {$\sigma$};
\end{scriptsize}
\end{tikzpicture}

Definimos $K$ como el complejo anterior, y como $L$, todos los símplices menos la arista $(BD)$ (como complejo simplicial los vértices no están aquí). Cogemos la arista $\sigma$, cuyos vértices son $B$ y $D$. Entonces ambos vértices están en $L$, y el único símplice de $K-L$ es $\sigma$, que claramente no corta a $L$ en todo su borde.
\end{solucion}

\newpage

\begin{ejercicio}{1.2}
Sean $σ = (v_0,\dots,v_p)$ y $τ = (w_0,\dots,w_q)$ símplices en $\R^n$.
Se dirá que $σ$ y $τ$ son \emph{unibles} si los $p+q+2$ puntos $v_0,\dots,v_p,w_0,\dots,w_q$ son afínmente independientes.
En ese caso, se donta por $στ$ el $(p+q+1)$-símplice generado por dichos puntos.
Dos complejos simpliciales $K$ y $L$ en $\R^n$ se llaman \emph{unibles} si:
\begin{itemize}
\item Para cada $σ \in K$ y $τ \in L$, $σ$ y $τ$ son unibles.
\item $στ$ y $σ'τ'$ se cortan en una cara común
\end{itemize}
Se pide:
\begin{enumerate}
\item Probar que si $K$ y $L$ son dos complejos unibles, entonces la familia de símplices dada por:
\[ KL = K \bigcup L \bigcup \{στ; σ \in K \text{ y }τ \in L\} \]
es un complejo simplicial en $\R^n$ llamado la \emph{unión simplicial} de $K$ y $L$.
\item Si $K = \{c\}$, entonces $KL$ es llamado \emph{complejo cono sobre} $L$ y es denotado por $cL$.
Si $K = \{c_1,c_2\}$ la unión simplicial $KL$ es llamada el \emph{complejo suspensión de} $L$ y denotada $ΣL$.
Más aún, si $L$ es finito $|cL|$ es homeomorfo al cono topológico $c|L|$ y $|ΣL|$ lo es a la suspenión topológica $Σ|L|$.
\item Si $K, L \neq \emptyset$, el poliedro $|KL|$ es siempre conexo por caminos.
\end{enumerate}
\end{ejercicio}
\begin{solucion}
\end{solucion}

\newpage

\begin{ejercicio}{1.3}
Sea $K$ complejo simplicial.
Para cada símplica $σ \in K$ probar la igualdad $st(σ;K) = σlk(σ;K)$ donde el término de la derecha indica la corespondiente unión simplicial.
\end{ejercicio}
\begin{solucion}
\end{solucion}

\newpage

\begin{ejercicio}{1.4}
Se dice que dos aplicaciones simpliciales $φ$, $ψ : K \to L$ son contiguas si para todo $σ \in K$ se verifica que $φ(σ) \cup ψ(σ) \in L$.
Probar que dos aproximaciones simpliciales $φ$, $ψ : K \to L$ de una misma aplicación continua $f : |K| \to |L|$ son contiguas.
\end{ejercicio}
\begin{solucion}
\end{solucion}

\newpage

\begin{ejercicio}{1.5}
Probar que si $|K|$ es conexo, entonces es conexo por caminos.
Más aún, se tiene que son equivalentes
\begin{enumerate}
\item $|K|$ es conexo.
\item El $1$-esqueleto de $K$ es conexo.
\item Dos vértices cualesquiera pueden unirse por un camino formado por aristas.
\end{enumerate}
\end{ejercicio}
\begin{solucion}
\end{solucion}

\newpage

\begin{ejercicio}{1.6}
Sea $K$ un complejo simplicial.
Un \emph{símplice principal} de $K$ es un símplice que no es cara propia de ningún símplice de $K$.
Si $τ < σ$ es una cara propia que no es cara propia de otro símplice, se dice que $τ$ es \emph{cara libre} de $σ$.
Obsérvese que necesariamente $\dim τ = \dim σ - 1$.

Supongamos que $K$ tiene un símplice principal $σ$ con cara libre $τ$.
Entonces, el subcomplejo $K_1 = K - \{σ,τ\}$ se llama \emph{colapso elemental} de $K$ através de $τ$, y se escribirá
\[ K \searrow K_1 \]
Una sucesión $K \searrow K_1 \searrow K_2 \searrow \dots \searrow K_n$ de colapsos elementales se llamará \emph{colapso} de $K$ a $K_n$, y se indicará mediante la notación $K \colapso K_n$.
Si $K_n$ es un punto, se dice que $K$ es \emph{colapsable}, y se usa la notación $K \colapso 0$.
\begin{enumerate}
\item Probar que el cono $cK$ sobre un complejo $K$ es colapsable.
\item Si $L$ es el complejo formado por un símplice y sus caras, entonces $L$ es colapsable.
Probar además, que en este caso, para todo complejo $K$ el complejo $LK$ es colapsable.
\item Dar un ejemplo de un subcomplejo $L$ de un complejo $K$ que sea colapsable, pero tal que $L$ no sea colapsable.
\item Probar que si $K$ es colapsable, entonces $|K|$ es contráctil.
\end{enumerate}
\end{ejercicio}
\begin{solucion}
\end{solucion}

\newpage

\begin{ejercicio}{1.7}
Un complejo simplicial de dimensión $1$ es llamado un \emph{grafo}.
Dado un grafo $G$, se dice que un subgrafo $T \subseteq G$ es un \emph{árbol} si $T$ es conexo y no contiene ninguna poligonal simple errada (que serán llamadas \emph{lazos}).
Probar que todo árbol colapsa a un punto.
Probar que $T \subseteq G$ es un árbol maximal si y sólo si $T$ contiene todos los vértices de $G$.
Probar que todo grafo conexo contiene un árbol maximal con respecto a la inclusión.
\end{ejercicio}
\begin{solucion}
\end{solucion}

\newpage

\begin{ejercicio}{1.8}
Dado un complejo simplicial $K$, probar que la relación binaria ``ser cara de'' induce un orden parcial entre los símplices de $K$.
El grafo que representa dicho orden se denomina \emph{diagrama de Hasse} de $K$ y se denota por $\mathcal{H}(K)$.
Probar que ambas estructuras son equivalentes, y caracterizar los colapsos en el diagrama de Hasse.
\end{ejercicio}
\begin{solucion}
\end{solucion}

\newpage

\begin{ejercicio}{1.9}
Dado un  espacio topológico $(X,\mathcal{T})$, consideremos un recubrimiento por abiertos de $X$ dado por $\mathcal{U} = \{\mathcal{U}_i\}_{i \in I}$.
Se define el \emph{nervio} de $\mathcal{U}$ como la familia de subconjuntos de índices $\{i_0,\dots,i_q\}$ tales que la intersección $\mathcal{U}_{i_0} \cap \dots \cap \mathcal{U}_{i_q}$ es no vacía.
Probar que el nervio de un recubrimiento es un complejo simplicial.
Estudiar qué propiedad interesante tienen los nervios en espacios topológicos compactos.
Decidir si todos los nervios sobre un mismo espacio son del mismo tipo de homotopía
(Considerar $X = S^1$ con la topología euclídea inducida).
\end{ejercicio}
\begin{solucion}
\end{solucion}

\newpage

\begin{ejercicio}{1.10}
Dado un grafo $G$, se define el complejo \emph{clique} de $G$ como el complejo simplicial $K(G)$ tal que:
\begin{itemize}
\item $K(G)^1 = G$.
\item $σ = (v_0,\dots,v_n)$ es un $n$-símplice de $K(G)$ si y sólo si el grafo complejo de vertices $v_0,\dots,v_n$ es un subgrafo de $G$.
\end{itemize}
Probar que $K(G)$ es efectivamente un complejo simplicial.
Comparar los tipos de homotopía de $G$ y $K(G)$.
Estudiar qué efecto tiene en $K(G)$ subdividir baricéntricamente $G$.
\end{ejercicio}
\begin{solucion}
\end{solucion}

\newpage

\begin{ejercicio}{1.11}
Sea $K$ una triangulación de una superficie.
Estudiar topológicamente los poliedros $|st(x;K)|$ para todo $x \in |K|$.
\end{ejercicio}
\begin{solucion}
\end{solucion}

\newpage

\begin{ejercicio}{1.12}
Sean $I = [0,1]$ y $K$ un complejo simplicial, dotar de estructura simplicial al cilindro de $|K|$, denotado por $|K \times I|$, de modo que $|K \times I| = |K| \times I$.
\end{ejercicio}
\begin{solucion}
\end{solucion}

\newpage

\begin{ejercicio}{1.13}
Indicar si alguna de las siguientes aplicaciones entre los complejos indicados más abajo es simplicial:
\begin{enumerate}[(a)]
\item
\[
\begin{tikzpicture}[line cap=round,line join=round,>=triangle 45,x=1.0cm,y=1.0cm]
\clip(0.068,-1.) rectangle (12.334666666666664,3);
\draw [line width=1.pt] (3.,0.)-- (6.,0.);
\draw [line width=1.pt] (6.,0.)-- (4.5,2.5980762113533165);
\draw [line width=1.pt] (4.5,2.5980762113533165)-- (3.,0.);
\draw [line width=1.pt] (7.,0.)-- (10.,0.);
\draw [line width=1.pt] (10.,0.)-- (8.5,2.5980762113533165);
\draw [line width=1.pt] (8.5,2.5980762113533165)-- (7.,0.);
\draw [line width=1.pt] (5.25,1.2990381056766587)-- (3.75,1.2990381056766582);
\draw [line width=1.pt] (3.75,1.2990381056766582)-- (4.5,0.);
\draw [line width=1.pt] (4.5,0.)-- (5.25,1.2990381056766587);
\draw (4.3,0.06) node[anchor=north west] {$v_1$};
\draw (2.8,0.06) node[anchor=north west] {$v_0$};
\draw (5.8,0.06) node[anchor=north west] {$v_2$};
\draw (3.1,1.506666666666661) node[anchor=north west] {$v_4$};
\draw (5.3,1.5173333333333276) node[anchor=north west] {$v_3$};
\draw (4.3,2.9573333333333234) node[anchor=north west] {$v_5$};
\draw (6.8,0.06) node[anchor=north west] {$w_0$};
\draw (9.6,0.06) node[anchor=north west] {$w_1$};
\draw (8.2,2.96) node[anchor=north west] {$w_2$};
\draw (1,3.0) node[anchor=north west] {$v_0\to w_0$};
\draw (1.,2.6) node[anchor=north west] {$v_1\to w_1$};
\draw (1.,2.2) node[anchor=north west] {$v_2\to w_1$};
\draw (1.,1.8) node[anchor=north west] {$v_3\to w_1$};
\draw (1.,1.4) node[anchor=north west] {$v_4\to w_2$};
\draw (1.,1.) node[anchor=north west] {$v_5\to w_1$};
\end{tikzpicture}
\]
\item
\[
\begin{tikzpicture}[line cap=round,line join=round,>=triangle 45,x=1.0cm,y=1.0cm]
\clip(0.068,-1.) rectangle (12.334666666666664,3);
\draw [line width=1.pt] (3.,0.)-- (6.,0.);
\draw [line width=1.pt] (6.,0.)-- (4.5,2.5980762113533165);
\draw [line width=1.pt] (4.5,2.5980762113533165)-- (3.,0.);
\draw [line width=1.pt] (7.,0.)-- (10.,0.);
\draw [line width=1.pt] (10.,0.)-- (8.5,2.5980762113533165);
\draw [line width=1.pt] (8.5,2.5980762113533165)-- (7.,0.);
\draw [line width=1.pt] (9.25,1.2990381056766587)-- (7.75,1.2990381056766582);
\draw [line width=1.pt] (7.75,1.2990381056766582)-- (8.5,0.);
\draw [line width=1.pt] (8.5,0.)-- (9.25,1.2990381056766587);
\draw [line width=1.pt] (4.5,0.)-- (4.5,2.6);
\draw (4.3,0.06) node[anchor=north west] {$v_2$};
\draw (2.8,0.06) node[anchor=north west] {$v_0$};
\draw (5.8,0.06) node[anchor=north west] {$v_3$};
\draw (4.3,2.9573333333333234) node[anchor=north west] {$v_1$};
\draw (6.8,0.06) node[anchor=north west] {$w_0$};
\draw (8.3,0.06) node[anchor=north west] {$w_5$};
\draw (9.6,0.06) node[anchor=north west] {$w_4$};
\draw (8.2,2.96) node[anchor=north west] {$w_2$};
\draw (7.1,1.506666666666661) node[anchor=north west] {$w_1$};
\draw (9.3,1.5173333333333276) node[anchor=north west] {$w_3$};
\draw (1,3.0) node[anchor=north west] {$v_0\to w_0$};
\draw (1.,2.6) node[anchor=north west] {$v_1\to w_4$};
\draw (1.,2.2) node[anchor=north west] {$v_2\to w_2$};
\draw (1.,1.8) node[anchor=north west] {$v_3\to w_1$};
\end{tikzpicture}
\]

\end{enumerate}
\end{ejercicio}
\begin{solucion}
\end{solucion}

\newpage

\begin{ejercicio}{1.14}
Sea $f : Δ^2 \to Δ^2$ la aplicacion simplicial defina por

\[
\begin{tikzpicture}[line cap=round,line join=round,>=triangle 45,x=1.0cm,y=1.0cm]
\clip(0.068,-1.) rectangle (12.334666666666664,3);
\draw [line width=1.pt] (3.,0.)-- (6.,0.);
\draw [line width=1.pt] (6.,0.)-- (4.5,2.5980762113533165);
\draw [line width=1.pt] (4.5,2.5980762113533165)-- (3.,0.);
\draw [line width=1.pt] (7.,0.)-- (10.,0.);
\draw [line width=1.pt] (10.,0.)-- (8.5,2.5980762113533165);
\draw [line width=1.pt] (8.5,2.5980762113533165)-- (7.,0.);


\draw (2.8,0.0) node[anchor=north west] {$2$};
\draw (5.8,0.0) node[anchor=north west] {$3$};

\draw (4.3,3) node[anchor=north west] {$1$};
\draw (6.8,0.06) node[anchor=north west] {$c$};
\draw (9.8,0.06) node[anchor=north west] {$b$};
\draw (8.3,2.96) node[anchor=north west] {$a$};

\draw (1.,2.2) node[anchor=north west] {$1\to b$};
\draw (1.,1.8) node[anchor=north west] {$2\to c$};
\draw (1.,1.4) node[anchor=north west] {$3\to a$};

\end{tikzpicture}
\]


Extenderla a una aplicación simplicial entre los complejos indicados en la siguiente figura 

\[
\begin{tikzpicture}[line cap=round,line join=round,>=triangle 45,x=1.0cm,y=1.0cm]
\clip(0.068,-1.) rectangle (12.334666666666664,3);
\draw [line width=1.pt] (3.,0.)-- (6.,0.);
\draw [line width=1.pt] (6.,0.)-- (4.5,2.5980762113533165);
\draw [line width=1.pt] (4.5,2.5980762113533165)-- (3.,0.);
\draw [line width=1.pt] (5.25,1.2990381056766587)-- (3.75,1.2990381056766582);
\draw [line width=1.pt] (3.75,1.2990381056766582)-- (4.5,0.);
\draw [line width=1.pt] (4.5,0.)-- (5.25,1.2990381056766587);
\draw [line width=1.pt] (7.,1.)-- (8.,0.);
\draw [line width=1.pt] (8.,0.)-- (10.,1.);
\draw [line width=1.pt] (10.,1.)-- (10.446666666666665,2.008);
\draw [line width=1.pt] (10.446666666666665,2.008)-- (8.004,2.6266666666666576);
\draw [line width=1.pt] (8.004,2.6266666666666576)-- (7.,1.);
\draw [line width=1.pt] (7.,1.)-- (10.446666666666665,2.008);
\draw [line width=1.pt] (10.446666666666665,2.008)-- (10.,1.);
\draw [line width=1.pt] (10.,1.)-- (7.,1.);
\draw (3.2,1.5706666666666609) node[anchor=north west] {$2$};
\draw (4.3,0.0) node[anchor=north west] {$1$};
\draw (5.38,1.5493333333333275) node[anchor=north west] {$3$};
\draw (6.6,1.1653333333333287) node[anchor=north west] {$c$};
\draw (10.030666666666665,1.2) node[anchor=north west] {$b$};
\draw (10.4,2.2) node[anchor=north west] {$a$};
\end{tikzpicture}
\]
\end{ejercicio}
\begin{solucion}
\end{solucion}

\newpage

\begin{ejercicio}{1.15}
Probar que la imagen de un subcomplejo por una aplicación simplicial es un subcomplejo y que la imagen inversa de un subcomplejo es subcomplejo.
\end{ejercicio}
\begin{solucion}
\end{solucion}

\newpage

\begin{ejercicio}{1.16}
Probar que una aplicación simplicial puede cubrir todos los vértices y no ser sobreyectiva.
\end{ejercicio}
\begin{solucion}
\end{solucion}

\newpage

\begin{ejercicio}{1.17}
Sean $|K| = |L| = [0,1]$, teniendo $K$ vértices en $0$, $1/3$ y $1$ y $L$ en $0$, $2/3$ y $1$.
Sea $f(x) = x^2$. Probar que $f$ de $K$ en $L$ no admite aproximaciń simplicial.

Análogamente, probar que no existe aproximación simplicial de $f$ de $sd K$ en $L$, y encontrar una aproximación simplicial de $f$ de $sd^2 K$ en $L$.
\end{ejercicio}
\begin{solucion}
\end{solucion}

\end{document}
