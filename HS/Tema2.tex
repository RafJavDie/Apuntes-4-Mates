\documentclass[HS.tex]{subfiles}
\begin{document}
\chapter{Homología simplicial}
\section{Álgebra homológica}
Nuestros objetos serán $\R$-espacio vectoriales de dimensión finita (o $R$-módulos con $R$ dominio de ideales principales).

\begin{defi}
Dado un diagrama de $\R$-e.v. y homomorfismos:
\[ M_{i-1} \xrightarrow{f_{i-1}} M_i \xrightarrow{f_i} M_{i+1} \]
se dice que la sucesión es \emph{exacta} en $M_i$ si $\ker f_i = \Ima f_{i-1}$. Si
\[ \cdots \rightarrow M_{i-2} \xrightarrow{f_{i-2}} \rightarrow M_{i-1} \xrightarrow{f_{i-1}} M_i \xrightarrow{f_i} M_{i+1} \xrightarrow{f_{i+1}} M_{i+2} \rightarrow \cdots \]
es exacto para todo $i \in \Z$, entonces diremos que es una \emph{sucesión exacta larga}.
\end{defi}

\begin{lemma}
Se verifica
\begin{enumerate}[a)]
\item $0 \rightarrow M_1 \xrightarrow{f_1} M_2$ es exacta sii $f_1$ es inyectiva.
\item $M_2 \xrightarrow{f_2} M_3 \rightarrow 0$ es exacto sii $f_2$ es sobreyectivo.
\item $0 \rightarrow M_1 \xrightarrow{f_1} M_2 \xrightarrow{f_2} M_3 \rightarrow 0$ es exacto sii $f_1$ es inyectiva, $f_2$ es sobreyectiva y $f_2$ induce un isomorfismo entre $\coker f_1$ y $M_3$. Este tipo de sucesiones es conocido como \emph{sucesión exacta corta}.
\end{enumerate}
\end{lemma}
\begin{dem}
\begin{enumerate}[a)]
\item $0 \rightarrow M_1 \xrightarrow{f_1} M_2$ es exacta sii $\ker f_1 = \{0\}$ sii $f_1$ es inyectiva.
\item $M_2 \xrightarrow{f_2} M_3 \rightarrow 0$ es exacto sii $\Ima f_2 = M_3$ sii $f_2$ es sobreyectivo.
\item $0 \rightarrow M_1 \xrightarrow{f_1} M_2 \xrightarrow{f_2} M_3 \rightarrow 0$ es exacto sii $f_1$ es inyectiva (apartado (a)), $f_2$ es sobreyectiva (apartado (b)) y $\ker f_2 = \Ima f_1$, es decir $\coker f_1 = M_2/\Ima f_1 = M_2/\ker f_2 \cong \Ima f_2 = M_3$ (por el primer teorema de isomorfía).
\end{enumerate}
\end{dem}

\begin{prop}
Si $0 \rightarrow V_1 \xrightarrow{f} V_2 \xrightarrow{g} V_3 \rightarrow 0$ es una sucesión exacta corta de $\R$-e.v. de dimensión finita, entonces $V_2 \cong V_1 \oplus V_3$.
\end{prop}
\begin{dem}
Sea $\{b_1,\dots,b_r\}$ una base de $V_1$ ($\dim V_1 = r$). Entonces $\{f(b_1), \dots, f(b_r)\}$ es base del subespacio vectorial $f(V_1) \subseteq V_2$. Extendiendo a una base $\{f(b_1), \dots, f(b_r),w_{r+1},\dots,w_n\}$ de $V_2$ ($\dim V_2 = n$). Comprobemos que $\{g(w_{r+1}),\dots,g(w_n)\}=:\mathcal{B}$ es una base de $V_3$.
\begin{itemize}
\item $\mathcal{B}$ es sistema generador: Sea $v \in V_3$, entonces existe $a \in V_2$ con $g(a)=v$ pues $g$ es sobreyectiva. Podemos expresar $a=\sum a_i f(b_i) + \sum \overline{a_j} w_j$. Entonces $v=g(a)= \sum \overline{a_j}g(w_j)$.
\item $\mathcal{B}$ es linealemnte independiente: Supongamos que existe $\sum λ_j g(w_j) = 0$. Entonces $\sum λ_j w_j \in \ker g = \Ima f$, pero $w_j \notin \Ima f$ $\forall j$, luego $λ_j = 0$ $\forall j$.
\end{itemize}
\end{dem}

Véase que hemos probado con toda sucesión exacta corta de espacios vectoriales \emph{escinde}. Usaremos este lema (demuéstrese como ejercicio):
\begin{lemma}[Splitting Lemma]
Se dice que $0 \rightarrow V_1 \xrightarrow{f} V_2 \xrightarrow{g} V_3 \rightarrow 0$ \emph{escinde} si se cumple uno de las siguientes condiciones equivalentes:
\begin{enumerate}
\item $V_2 \cong V_1 \oplus V_3$.
\item $\exists r : V_3 \to V_2$ tal que $g \circ r = id_{V_3}$.
\item $\exists t : V_2 \to V_1$ tal que $t \circ f = id_{V_1}$. 
\end{enumerate}
\end{lemma}

\begin{lemma}[Lema de los cinco]
Sea el diagrama conmutativo:
\[\begin{tikzcd}
	M_1 \arrow[r,"θ_1"] \arrow[d,two heads,"ψ_1"] & M_2 \arrow[r,"θ_2"] \arrow[d,"ψ_2","\cong"'] & M_3 \arrow[r,"θ_3"] \arrow[d,"ψ_3"] & M_4 \arrow[r,"θ_4"] \arrow[d,"ψ_4","\cong"'] & M_5 \arrow[d,hook,"ψ_5"]\\
	N_1 \arrow[r,"φ_1"] & N_2 \arrow[r,"φ_2"] & N_3 \arrow[r,"φ_3"] & N_4 \arrow[r,"φ_4"] & N_5
\end{tikzcd}\]
donde las filas son exactas, $ψ_2$ y $ψ_4$ son isomorfismos, $ψ_1$ es epimorfismos y $ψ_5$ es monomorfismo. Entonces $ψ_3$ es isomorfismo.
\end{lemma}
\begin{dem}
Lo demostramos por diagram-chasing\footnote{Es más fácil hacerlo que explicarlo}. En primer lugar la inyectividad.

Sea $m_3 \in M_3$ tal que $ψ_3(m_3) = 0$, luego $0=φ_3ψ_3(m)=ψ_4θ_3(m_3)$.
Como $ψ_4$ es isomorfismo, $θ_3(m_3)=0$.
Por exactitud, existe $m_2 \in M_2$ tal que $θ_2(m_2)=m_3$. Sea $n_2 = ψ_2(m_2)$.
Como $0=ψ_3θ_2(m_2)=φ_2ψ_2(m_2)$, por exactitud existe $n_1 \in N_1$ tal que $φ_1(n_1) = n_2$.
Como $ψ_1$ es sobreyectivo, existe $m_1 \in M_1$ tal que $ψ_1(m_1)=n_1$. 
Ahora bien, $ψ_2(θ_1(m_1))=φ_1(ψ_1(m_1))=ψ_2(m_2)$.
Como $ψ_2$ es inyectiva, esto quiere decir que $m_2 = θ_1(m_1)$
Luego $m_3 = θ_2(m_2)=θ_2(θ_1(m_1))=0$ por exactitud y $ψ_3$ es inyectiva.

Ahora la sobreyectividad.
Sea $n_3\in N_3$. Llamamos $n_4=\varphi_3(n_3)$. Como $\psi_4$ es sobreyectiva, existe $m_4$ tal que $\psi_4(m_4)=n_4$. Ahora, por la conmutatividad del diagrama, $\psi_5(\theta_4(m_4))=\varphi_4(\psi_4(m_4))=\varphi_4(n_4)=\varphi_4(\varphi_3(m_3))=0$, por exactitud. Luego, por inyectividad de $\psi_5$, tenemos que $\theta(m_4)=0$, que por exactitud significa que existe $m_3\in M_3$ tal que $m_4=\theta(m_3)$. Por conmutatividad del diagrama, tenemos que $\varphi_3(n_3)=\psi_4(\theta_3(m_3))=\varphi_3(\psi_3(m_3))$, luego $n_3-\psi_3(m_3)\in\ker{\varphi_3}$. Por exactitud, esto es equivalente a que exista $n_2\in N_2$ tal que $\varphi_2(n_2)=n_3-\psi_3(m_3)$. Como $\psi_2$, es sobreyectiva, existe $m_2\in M_2$ tal que $n_2=\psi_2(m_2)$. Por conmutatividad, $n_3-\psi(m_3)=\varphi_2(\psi_2(m_2))=\psi_3(\theta_2(m_2))$, luego $n_3=\psi(m_3+\theta_2(m_2))$, con lo que hemos encontrado su preimagen. \QED
\end{dem}

\begin{defi}
Un complejo de cadenas $\mathcal{C}$ es un diagrama del tipo siguiente:
\[ \cdots \rightarrow C_{n+1} \xrightarrow{δ_{n+1}} C_n \xrightarrow{δ_n} C_{n-1} \xrightarrow{δ_{n-1}} \cdots \]
donde $C_i$ son $\R$-e.v. llamados \emph{$n$-cadenas}, $δ_i$ son homomorfismos de $\R$-e.v. llamados \emph{operadores borde}, tal que $δ_n \circ δ_{n+1} = 0$ para todo $n \in \Z$. Habitualmente, denotamos $\mathcal{C}_* = \{(C_n,δ_n)\}_{n \in \Z}$.

Para todo $n \in \Z$, denotamos $Z_n := \ker δ_n \subseteq C_n$. A los elementos de $Z_n$ los llamamos \emph{$n$-ciclos}.
Definimos $B_n := \Ima δ_{n+1} \subseteq C_n$, cuyos elementos se llaman \emph{$n$-bordes}.
Como $δ_n \circ δ_{n+1} = 0$, tenemos que $B_n \subseteq Z_n$.
De esta manera, definimos $H_n(\mathcal{C}) = Z_n / B_n$ como el \emph{$n$-ésimo $\R$-e.v. de homología} del complejo de cadenas $\mathcal{C}$.
\end{defi}

\begin{nota}
Un complejo de cadenas $\mathcal{C}$ es exacta si y solo si $H_n(\mathcal{C})=0$ para todo $n\in\Z$. 
\end{nota}

\begin{defi}
Sean $\mathcal{C}=\{(C_n,δ_n)\}$ y $\mathcal{C}'=\{(C_n',δ_n')\}$ dos complejos de cadenas, se define $\mathcal{C}\oplus\mathcal{C}'$ como el el siguiente complejo: para cada $n \in \Z$, $(\mathcal{C} \oplus \mathcal{C}')_n = C_n \oplus C_n'$ y $d : C_n \oplus C_n' \to C_{n-1} \oplus C_{n-1}'$ definido como $d(c,c')=(δ_n(c),δ_n'(c'))$.
\end{defi}

\begin{defi}
Sean $\mathcal{C}_1=\{(C_n^1,δ_n^1)\}$ y $\mathcal{C}_2=\{(C_n^2,δ_n^2)\}$ dos complejos de cadenas. Un homomorfismo o morfismo de complejos de cadenas $f : \mathcal{C}_1 \to \mathcal{C}_2$ consiste en una familia $f = \{f_n : C_n^1 \to C_n^2\}_{n \in \Z}$ con $f_n$ homomorfismo de $\R$-e.v. tal que para todo $n \in \Z$, $δ_n^2 \circ f_n = f_{n-1} \circ δ_{n-1}^1$, es decir, el siguiente diagrama conmute:
\[\begin{tikzcd}
	C_n^1 \arrow[r,"f_n"] \arrow[d,"δ_n^1"] & C_n^2 \arrow[d,"δ_n^2"]\\
	C_{n-1}^1 \arrow[r,"f_{n-1}"] & C_{n-1}^2
\end{tikzcd}\]
Si cada uno de los $f_n$ es isomorfismo de espacios vectoriales, entonces $f$ es isomorfismo de complejo de cadenas.
\end{defi}

\begin{nota}
Es fácil comprobar que $f_n(Z_n(\mathcal{C}_1))\subseteq Z_n(\mathcal{C}_2),\forall n\in\Z$, es decir, $f$ lleva ciclos en ciclos. Asímismo, $f_n(B_n(\mathcal{C}_1))\subseteq B_n(\mathcal{C}_2),\forall n\in\Z$, es decir, $f$ lleva bordes en bordes.
\end{nota}

\begin{prop}
Si $f : \mathcal{C}_1 \to \mathcal{C}_2$ un homomorfismo de complejos de cadenas, entonces $f$ induce para cada $n \in \Z$ un homomorfismo:
\[ f_* : H_n(\mathcal{C}_1) \to H_n(\mathcal{C}_2) \]
\[ f_* ([z]) = [f_n(z)] \]
\end{prop}
\begin{dem}
En primer lugar tenemos que ver que la aplicación está bién definida. Sean $z,z'\in Z_n$ tales que $[z]=[z']$, es decir, $z-z'\in B^1_n=\Ima δ^1_{n+1}$. Nos preguntamos si $f_*([z])=f_*([z'])\in B^2_n$ sii $[f_n(z)]=[f_n(z')]\in B^2_n$ sii $f_n(z)-f_n(z')\in B^2_n=\Ima δ^2_{n+1}$. Como $z-z'\in\Ima δ^1_{n+1}$, existe $c\in C^1_{n+1}$ tal que $z-z'=δ^1_{n+1}(c)$, por lo que $f_n(z)-f_n(z')=f_n(z-z')=f_n(δ^1_{n+1}(c))$. Aplicando la definición de homomorfismo de cadenas, $f_n(δ^1_{n+1}(c))=δ^2_{n+1}(f_{n+1}(c))$. Por lo que claramente se tiene el resultado.

Probemos que $f_*$ es homomorfismo de complejos de cadenas. Sean $\alpha,\beta\in\R$ y $z_1,z_2\in Z_n$. Usamos las propiedades del cociente de espacios vectoriales y que $f_n$ es homomorfismo.
\[f_*(\alpha[z_1]+\beta[z_2])=f_*([\alpha z_1+\beta z_2])=[f_n(\alpha z_1+\beta z_2)]=[\alpha f_n(z_1)+\beta f_n(z_2)]=\]
\[=\alpha[f_n(z_1)]+\beta[f_n(z_2)]=\alpha f_*([z_1])+\beta f_*([z_2])\]
\QED
\end{dem}

\begin{propi}
\begin{enumerate}
\item Si $\mathcal{C}_1=\mathcal{C}_2=\mathcal{C}$ y $f=\{Id:C_n\to C_n, n\in\Z\}$, entonces $(Id)_*=Id:H_n(\mathcal{C})\to H_n(\mathcal{C})$. 
\item Si tenemos $f:\mathcal{C}\to\mathcal{C}'$ y $g:\mathcal{C}'\to\mathcal{C}''$ morfismos de complejos de cadenas, entonces $(g\circ f)_*=g_*\circ f_*$, donde la composición se entiende como $g_n\circ f_n$ para todo $n\in\Z$.
\end{enumerate}
\end{propi}

\begin{defi}
Sean $\mathcal{C}^1=\{(\mathcal{C}^1_n,\partial_n^1)\}$, $\mathcal{C}^2=\{(\mathcal{C}^2_n,\partial_n^2)\}$ y $\mathcal{C}^3=\{(\mathcal{C}^3_n,\partial_n^3)\}$, junto con morfismos $f:\mathcal{C}^1\to\mathcal{C}^2$ y $g:\mathcal{C}^2\to\mathcal{C}^3$. Si se cumple que
\[
0\to C_n^1\overset{f_n}{\to}C_n^2\overset{g_n}{\to}C_n^3\to 0
\]
es una sucesión exacta corta para todo $n\in\Z$, entonces decimos que
\[
0\to \mathcal{C}^1\overset{f}{\to}\mathcal{C}^2\overset{g}{\to}\mathcal{C}^3\to 0
\]
es una sucesión exacta corta de complejos de cadenas.
\end{defi}


\begin{prop}
Con la notación de la definición anterior, si 
\[
0\to \mathcal{C}^1\overset{f}{\to}\mathcal{C}^2\overset{g}{\to}\mathcal{C}^3\to 0
\]
es una sucesión exacta corta de complejo de cadenas, entonces existen homomorfismos
\[
\Delta:H_n(\mathcal{C}^3)\to H_{n-1}(\mathcal{C}^1),\ n\in\Z,
\]
de manera que se tiene una sucesión exacta larga de la manera siguiente
\[
\cdots\to H_n(\mathcal{C}^1)\overset{f_*}{\to}H_n(\mathcal{C}^2)\overset{g_*}{\to}H_n(\mathcal{C}^3)\overset{\Delta}{\to}H_{n-1}(\mathcal{C}^1)\overset{f_*}{\to}H_{n-1}(\mathcal{C}^2)\to\cdots
\]
llamada \emph{sucesión exacta larga en homología}.
\end{prop}

\begin{dem}
Tenemos
\[\begin{tikzcd}
0\arrow[r] & C_{n}^1\arrow[r, "f_{n}"]\arrow[d, "\partial_{n}^1"]& C_n^2\arrow[r, "g_{n}"]\arrow[d,"\partial_{n}^2"]& C_{n}^3\arrow[r]\arrow[d,"\partial_{n}^3"] & 0\\
0\arrow[r] & C_{n-1}^1\arrow[r, "f_{n-1}"]\arrow[d, "\partial_{n-1}^1"]& C_{n-1}^2\arrow[r, "g_{n-1}"] \arrow[d,"\partial_{n-1}^2"]& C_{n-1}^3\arrow[r]\arrow[d,"\partial_{n-1}^3"]&  0\\
0\arrow[r] & C_{n-2}^1\arrow[r, "f_{n-2}"]& C_{n-2}^2\arrow[r, "g_{n-2}"]& C_{n-2}^3\arrow[r] & 0
\end{tikzcd}
\]
donde todos los cuadrados son conmutativos. Definimos $\Delta$ de la siguiente manera: dado $[z_3]\in H_n(\mathcal{C}^3)$, es decir, $z_3\in Z_n(\mathcal{C}^3)\subseteq C_n^3$, y como $g$ es sobreyectiva, $\exists x_2\in C_n^2$ con $g_n(x_2)=z_3$. Por otro lado, $g_{n-1}(\partial_n^2(x_2))=\partial_n^3(g_n(x_2))=\partial_n^3(z_3)=0$, de donde se deduce que $\partial_n^2(x_2)\in\ker(g_{n-1})=\Ima(f_{n-1})$. Por tanto, existe un único (por inyectividad) $z_1\in C_{n-1}^1$ con $f_{n-1}(z_1)=\partial^2_n(x_2)$. Si probamos que $z_n\in Z_{n-1}(\mathcal{C}^1)$ entonces definiremos $\Delta([z_3])=[z_1]$.  

Tenemos $f_{n-2}(\partial^1_{n-1}(z_1)=\partial^2_{n-1}(f_{n-1}(z_1))=\partial^2_{n-1}(\partial^2_n(x_2))=0$. Como $f_{n-2}$ es inyectiva, entonces $\partial^1_{n-1}(z_1)=0$, con lo que tenemos lo que buscábamos.

Ahora tenemos que ver si realmente está bien definida.
Sea $z_3' \in Z_n(\mathcal{C}^3)$ otro representante de $[z_3] \in H_n(\mathcal{C}^3)$, es decir $[z_3]=[z_3']$.
Existe $ω_3 \in C_{n+1}^3$ tal que $z_3'-z_3 = \partial_{n+1}^3(ω_3)$.
Siguiendo el mismo procedimiento que usamos para $z_3$, obtenemos $x_2' \in C_n^2$ tal que $g_n(x_2')=z_3'$.
También existe un único $z_1' \in C_{n-1}^1$ tal que $f_{n-1}(z_1') = \partial_n^2(x_2')$.
Definimos $Δ([z_3'])=[z_1']$. Luego tenemos que ver si $[z_1] = [z_1']$, es decir, si $z_1' - z_1 \in B_{n-1}(\mathcal{C}^1)$.
\[ g_n(x_2'-x_2) = g_n(x_2')-g_n(x_2) = z_3-z_3 = \partial_{n+1}^3(ω_3) \]
Como $g_{n+1}$ es sobre, existe $θ_2 \in C_{n+1}^2$ tal que $g_{n+1}(θ_2)=ω_3$.
Ahora consideramos (usando conmutatividad del diagrama)
\[ g_n((x_2'-x_2)-\partial_{n+1}^2(θ_2)) = g_n(x_2'-x_2)-g_n(\partial_{n+1}^2(θ_2)) = δ_{n+1}^3(ω_3) - δ_{n+1}^3(g_{n+1}(θ_2)) = 0 \]
Entonces $x_2-x_2-\partial_{n+1}^2(θ_2) \in \ker g_n = \Ima f_n$, luego existe un único $α_1 \in C_n^1$ tal que $f_n(α_1)=x_2'-x_2-\partial_{n+1}^2(θ_2)$.
Observamos que:
\[ f_{n-1}(\partial_n^1(α_1)) = \partial_n^2(f_n(α_1)) = \partial_n^2(x_2'-x_2-\partial_{n+1}^2(θ_2)) = \partial_n^2(x_2')-\partial_n^2(x_2) = f_{n-1}(z_1'-z_1) \]
Por inyectividad de $f_{n-1}$, deducimos $z_1'-z_1 = \partial_n^1(α_1) \in B_{n-1}(\mathcal{C}^1)$.

Procedemos a ver que la sucesión es exacta. Hay que probar para todo $n \in \Z$:
\begin{enumerate}
\item $\Ima f_* = \ker g_*$. Tenemos que $g_n \circ f_n = 0$ para todo $n$, luego $g \circ f = 0$ como morfismo entre cadenas. En consecuencia, $g_* \circ f_* = (g \circ f)_* = 0$, finalmente $\Ima f_* \subseteq \ker g_*$.

Por otro lado, sea $[z_2] \in H_n(\mathcal{C}^2)$ tal que $g_*([z_2])=0$.
Como $g_*([z_2]) = [g_n(z_2)]$, entonces $g_n(z_2) \in B_n(\mathcal{C}^3)$.
Entonces, existe $x_3 \in C_{n+1}^3$ tal que $\partial_{n+1}^3(x_3)=g_n(z_2)$.
Por otro lado, por sobreyectividad, existe $x_2 \in C_{n+1}^2$ tal que $g_{n+1}(x_2) = x_3 \in C_{n+1}^3$.
Consideramos:
\[ g_n(z_2-\partial_{n+1}^2(x_2)) = g_n(z_2)-g_n(\partial_{n+1}^2(x_2)) = g_n(z_2)-\partial_{n+1}^3(g_{n+1}(x_2)) = g_n(z_2)-\partial_{n+1}^3(x_3) = 0 \]
Luego $z_2 - \partial_{n+1}^2(x_2) \in \ker g_n = \Ima f_n$.
Por lo tanto, existe un único $z_1 \in C_n^1$ tal que $f_n(z_1) = z_2 - \partial_{n+1}^2(x_2)$.
Si $z_1$ es un ciclo, ya tendríamos el resultado, pues $f_*([z_1])=[f_n(z_1)] = [z_2-\partial_{n+1}^2(x_2)] = [z_2]$.
Basta ver que $z_1$ es efectivamente un ciclo. Observando que:
\[ f_{n-1}(\partial_n^1(z_1)) = \partial_n^2(f_n(z_1)) = \partial_n^2(z_2-\partial_{n+1}^2(x_2)) = 0 \]
Por inyectividad de $f_{n-1}$, $\partial_n^1(z_1) = 0$ y $z_1 \in Z_n$.
\item $\Ima g_* = \ker Δ$.Considermos $g_*([x_2])$ con $x_2 \in Z_n(\mathcal{C}^2)$.
Entonces por definición de $Δ$ se tiene que $Δ([g_n(x_2)]) = 0$.

Sea $[z_3] \in H_n(\mathcal{C}^3)$ tal que $Δ([z_3]) = [0]$.
Por la construcción de $Δ$, esto quiere decir que existe $x_2 \in C_n^2$ tal que $g_n(x_2)=z_3$, $\partial_n^2(x_2)=0$ y existe $ω_1 \in C_n^1$ tal que $\partial_n^1(ω_1) = 0$.
Observamos que como $g_n \circ f_n = [0]$:
\[ g_*([x_2-f_n(ω_1)]) = [g_n(x_2 - f_n(ω_1))] = [g_n(x_2)] = [z_3] \]
Para argumentar esto falta ve que $x_2-f_n(ω_1)$ es un ciclo, pero:
\[ \partial_n^2(x_2-f_n(ω_1)) = \partial_n^2(x_2) - \partial_n^2f_n(ω_1) = \partial_n^2(x_2) - f_{n-1}\partial_n^1(ω_1) = 0 \]
\item $\Ima Δ = \ker f_*$. Consideramo $Δ([z_3]) = [z_1]$ con $z_3 \in Z_n(\mathcal{C}^3)$.
Por la construcción de $Δ$, esto quiere decir que existe $x_2 \in C_n^2$ tal que $g_n(x_2) = z_3$ y existe un único $z_1 \in C_{n-1}^1$ tal que $f_{n-1}(z_1) = \partial_n^2(x_2)$.
Luego:
\[ f_*(Δ([z_3])) = f_*([z_1]) = [f_{n-1}(z_1)] = [\partial_n^2(x_2)] = 0 \]

Por otro lado, supongamos que $f_*([z_1]) = 0$ entonces existe $x_2 \in C_n^2$ tal que $\partial_n^2(x_2)=f_{n-1}(z_1)$.
Tomamos $z_3 = g_n(x_2)$. Tenemos que:
\[ \partial_n^3(z_3) = \partial_n^3(g_n(x_2)) = g_{n-1}(\partial_n^2(x_2)) = g_{n-1}(f_{n-1}(z_1)) = 0 \]
Luego $z_3$ es un ciclo, además:
\[ Δ([z_3]) = [z_1] \Rightarrow [z_1] \in \Ima Δ \]
\end{enumerate}
\QED
\end{dem}

\begin{defi}
Dados dos complejos de cadenas $\mathcal{C} = \{(C_n,\partial_n)\}_{n \in \Z}$ y $\mathcal{C}' = \{(C_n',\partial_n')\}_{n \in \Z}$, podemos construir un tercer complejo de cadenas $\mathcal{C} \oplus \mathcal{C}'$ dado por:
\[ \left(\mathcal{C}_n \oplus \mathcal{C}'\right)_n = C_n \oplus C_n' \]
\[ \begin{matrix}\partial_n \oplus \partial_n' \colon & C_n \oplus C_n' \to C_{n-1} \oplus C_{n-1}'\\
& (x,y) \mapsto \left(\partial_n(x), \partial_n'(y)\right) \end{matrix} \]
\end{defi}

\begin{defi}
Dado un complejo de cadenas $\mathcal{C} = \{(C_n,\partial_n)\}_{n \in \Z}$ y dados subespacos vectoriales $C_n' \subseteq C_n$ $\forall n \in \Z$ cumpliendo que $\partial_n(C_n') \subseteq C_{n-1}'$, diremos que tenemos un \emph{subcomplejo de cadenas} $\mathcal{C}' = \{(C_n',\partial_n|_{C_n'})\}_{n \in \Z}$ de $\mathcal{C}$. Lo denotaremos $\mathcal{C}' \subseteq \mathcal{C}$.
\end{defi}

\begin{prop}
Sea $\mathcal{C}$ un complejo de cadenas y consideremos $\mathcal{C}^1, \mathcal{C}^2 \subseteq \mathcal{C}$ subcomplejos.
Entonces tenemos la secuencia exacta corta:
\[ 0 \to C_n^1 \cap C_n^2 \xrightarrow{i_n} C_n^1 \oplus C_n^2 \xrightarrow{j_n} C_n \to 0 \]
donde por:
\[ i_n \colon x \mapsto (x,-x) \]
\[ j_n \colon (x,y) \mapsto x+y\]
\end{prop}
La demostración es sencilla y queda como ejercicio.
\end{document}