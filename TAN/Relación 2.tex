\documentclass[twoside]{article}
\usepackage{../estilo-ejercicios}
\providecommand{\bo}[1]{\mathcal{O}\left(#1\right)}
%--------------------------------------------------------
\begin{document}

\title{Teoría Analítica de Números}
\author{Rafael González López}
\maketitle

\begin{ejercicio}{12}
Suponiendo la hipótesis de Riemann puede probarse que
$$
\theta(x):=\sum_{p\leq x}\log p = x + \mathcal{O}(x^{1/2}\log^2x).
$$
Admitiendo este resultado probar que si es cierta la hipótesis de Riemann, entonces existe una constante A tal que
$$
\sum_{p\leq x}\frac{\log p}{p} = \log x + A + \mathcal{O}(x^{-1/2}\log^2 x)
$$
\begin{sol}
\end{sol}
\end{ejercicio}
Sea $a_n = I_{\mathbb{P}}(n)\log n$, donde $I_{\mathbb{P}}(n)$ es la funcion indicatriz de los números primos. Entonces, suponiendo cierta la hipótesis de Riemann y el resultado anterior, además de aplicar la fórmula de sumación de Abel, obtenemos
\begin{align*}
\sum_{p\leq x}\frac{\log p}{p} &= \sum_{n\leq x}\frac{a_n}{n} = \frac{\theta(x)}{x} - \int_1^x \frac{-\theta(t)}{t^2}dt =\\
&= \frac{x + \mathcal{O}(x^{1/2}\log^2x)}{x} + \int_1^x \frac{t + \mathcal{O}(t^{1/2}\log^2t)}{t^2}dt\\
 &= 1+\mathcal{O}(x^{-1/2}\log^2 x) + \int_1^x \frac{1}{t} + \frac{\mathcal{O}(t^{1/2}\log^2 t)}{t^2}dt =\\
&=\log x + 1 + \mathcal{O}(x^{-1/2}\log^2 t)  + \int_1^x \frac{\mathcal{O}(t^{1/2}\log^2 t)}{t^2}dt
\end{align*}
Estamos muy cerca del resultado que buscamos, pero todavía nos falta trabajo. Tenemos que analizar el último término integral. Primeramente, $\mathcal{O}(t^{1/2}\log^2 t)$ es una función $U(t)$ de la que sabemos existen $t_0 \in \R$ y $C\in\R_{>0}$ tales que $|U(t)|\leq Ct^{1/2}\log^2 t$ $\forall t>t_0$. De hecho, $U(t)=\theta(t)-t$. Consideremos entonces
\begin{gather*}
\int_1^x \frac{\mathcal{O}(t^{1/2}\log^2 t)}{t^2}dt = \int_1^x \frac{U(t)}{t^2}dt = \int_1^\infty \frac{U(t)}{t^2}dt - \int_x^\infty \frac{U(t)}{t^2}dt
\end{gather*}
De hecho
$$
\gabs{\int_1^\infty \frac{U(t)}{t^2}dt} \leq \int_1^\infty \frac{\gabs{U(t)}}{t^2}dt \leq \int_1^\infty \frac{U(t)}{t^2}dt =\int_1^\infty  Ct^{-3/2}\log^2 t dt =16C<\infty
$$ Si tomamos $A = 1 + \int_1^\infty \frac{U(t)}{t^2}dt$ llegamos a que
$$
\sum_{p\leq x}\frac{\log p}{p} = \log x + A + \mathcal{O}(x^{-1/2}\log^2 t) - \int_x^\infty \frac{U(t)}{t^2}dt
$$
Entonces tenemos qué ver el orden del último sumando, pero
\begin{gather*}
S(x)= \int_x^\infty \frac{U(t)}{t^2}dt = \int_x^\infty \frac{\mathcal{O}(t^{1/2}\log^2 t)}{t^2}dt = \bo{\int_x^\infty  {t^{-3/2}\log^2 t}dt}=\\
=\bo{\left[-\frac{2(\log^2 t+4\log t + 8}{t^{1/2}}\right]_x^\infty }= \bo{\frac{2(\log^2 x+4\log x + 8)}{x^{1/2}}} = \mathcal{O}(t^{-1/2}\log^2 t)
\end{gather*}
Como queríamos probar. 
\end{document}