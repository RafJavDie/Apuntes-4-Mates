\documentclass[twoside]{article}
\usepackage{../estilo-ejercicios}

%--------------------------------------------------------
\begin{document}

\title{Algebra Conmutativa y Geometría Aplicada}
\author{Rafael González López}
\maketitle

\begin{ejercicio}{1}
Encontrar todos los enteros positivos tales que $\varphi(n)$ no sea divisible por 4.
\begin{sol}
Sea $n\in \N$. Sabemos que $n=p_1^{a_1}\cdots p_k^{a_k}$. En tal caso, 
\[
\varphi(n) = (p_1-1)p_1^{a_1-1}\cdots(p_k-1)p_k^{a_k-1}
\]
Fijémonos en el siguiente detalle: si $n$ tiene más de un divisor primo impar, es claro que se tendría que $4\mid \varphi(n)$, pues $p_j -1$ y $p_k-1$ serían dos factores pares de $\varphi(n)$. Por otra parte, si $p_i=2$, es claro que $0\leq a_i \leq 2$. 

Por lo que sabemos hasta ahora, $n = 2^a p^b$ donde $a\in\{0,1,2\}$, $p$ es un primo impar y $b\geq 0$. Distingamos casos en función de $a$:
\begin{itemize}
\item Si $a\neq 2$, $\varphi(n) = (p-1)p^{b-1}$ -no afecta que el 2 esté o no en la factorización de $n$. La condición del enunciado sí y solo sí $p \equiv 3 \mod 4$. Es decir, verifican el enunciado si $p$ es un primo de la forma $4k + 3$ y $b\geq 0$. El caso $b=0$ tenemos simplemente $n=1$ o $n=2$ (triviales).
\item Si $a=2$, entonces $n=2^2p^b$. Si $b\geq 1$ entonces $\varphi(n)=2(p-1)p^{b-1}$, que es divisible por 4. Por tanto, $b=0$. Nos queda pues que, el único caso, es que $n=4$.
\end{itemize}
Tenemos entonces que las soluciones son: 1, 2, 4, $p^k$ y $2p^k$ donde $p \equiv 3 \mod 4$.
\end{sol}
\end{ejercicio}

\newpage


\begin{ejercicio}{2}
Probar que para todo $k\geq 0$
\[
\gcd \left\{\binom{2k}{k},\binom{2k+1}{k}\right\}=\binom{2k}{k}\frac{1}{k+1}, \quad \gcd \left\{\binom{2k+1}{k},\binom{2k+2}{k+1}\right\}=\binom{2k+1}{k+1}
\]
\begin{sol}
Sea $n\in \N$. Sabemos que $n=p_1^{a_1}\cdots p_k^{a_k}$. En tal caso, 
\[
\varphi(n) = (p_1-1)p_1^{a_1-1}\cdots(p_k-1)p_k^{a_k-1}
\]
Fijémonos en el siguiente detalle: si $n$ tiene más de un divisor primo impar, es claro que se tendría que $4\mid \varphi(n)$, pues $p_j -1$ y $p_k-1$ serían dos factores pares de $\varphi(n)$. Por otra parte, si $p_i=2$, es claro que $0\leq a_i \leq 2$. 

Por lo que sabemos hasta ahora, $n = 2^a p^b$ donde $a\in\{0,1,2\}$, $p$ es un primo impar y $b\geq 0$. Distingamos casos en función de $a$:
\begin{itemize}
\item Si $a\neq 2$, $\varphi(n) = (p-1)p^{b-1}$ -no afecta que el 2 esté o no en la factorización de $n$. La condición del enunciado sí y solo sí $p \equiv 3 \mod 4$. Es decir, verifican el enunciado si $p$ es un primo de la forma $4k + 3$ y $b\geq 0$. El caso $b=0$ tenemos simplemente $n=1$ o $n=2$ (triviales).
\item Si $a=2$, entonces $n=2^2p^b$. Si $b\geq 1$ entonces $\varphi(n)=2(p-1)p^{b-1}$, que es divisible por 4. Por tanto, $b=0$. Nos queda pues que, el único caso, es que $n=4$.
\end{itemize}
Tenemos entonces que las soluciones son: 1, 2, 4, $p^k$ y $2p^k$ donde $p \equiv 3 \mod 4$.
\end{sol}
\end{ejercicio}
\end{document}