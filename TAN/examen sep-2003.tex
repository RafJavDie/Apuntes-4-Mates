\documentclass[twoside]{article}
\usepackage{../estilo-ejercicios}
\providecommand{\bo}[1]{\mathcal{O}\left(#1\right)}
%--------------------------------------------------------
\begin{document}

\title{Teoría Analítica de Números\\Examen, 5 de septiembre de 2003}
\author{Rafael González López\\Diego Pedraza López}
\maketitle

\begin{ejercicio}{1}
\begin{enumerate}[(a)]
	\item Partiendo de $Γ(1/2)=\sqrt{π}$ demuestra la fórmula de $Γ(n+1/2)$.
	\item ¿Qué se quiere decir cuando hablamos de que la probabilidad de que dos números naturales sean primos entre sí es $6/π^2$? ¿Podrían servir los primeros premios de la loteria para calcular una aproximación de $π$?
	\item ¿Qué relación tiene el teorema de Chebyshev con la estimación $p_n = \mathcal{O}(n \log n)$?
	\item ¿Qué relación puedes establecer entre la función de Möbius y la función $1/ζ(s)$?
\end{enumerate}
\end{ejercicio}
\begin{solucion}
\begin{enumerate}[(a)]
\item Tenemos que $Γ(s)=sΓ(s-1)$, luego para todo $n \in \N$:
	\[ Γ\left(\frac{1}{2}+n\right) = \left(\frac{1}{2}+n\right) Γ\left(\frac{1}{2}+n-1\right) = \dots = Γ\left(\frac{1}{2}\right)\prod_{i=0}^{n-1}\left(\frac{1}{2}+i\right) = \sqrt{π}\frac{1}{2^n}\prod_{i=0}^{n-1}\left(1+2i\right)\]
	\[ = \frac{\sqrt{π}}{2^n}(1 \cdot 3 \cdot 5 \cdots 2n) = \frac{\sqrt{π}}{2^n}\frac{(2n)!}{2 \cdot 4 \cdot 6 \cdots \cdot 2n} = \frac{\sqrt{π}}{2^n}\frac{(2n)!}{2^n n!}  = \frac{(2n)!}{n!2^{2n}}\sqrt{π} \]
	
\item Tenemos que dados dos números enteros menor que $x$, la probabilidad de que sean coprimos es $6/π^2+\bo{\log x/x}$.
	Para $x$ suficientemente grande, esto se aproxima a $6/π^2$.
	
Para aproximar $π$ por los primeros premios de la lotería, bastaría ir comprobando dos a dos si son coprimos.
Si hemos comprobado todos los pares de $n$ números y $m$ de ellos son coprimos, tenemos que $m/\binom{n}{2} \approx 6/π^2$.
Luego podemos aproximar $π$ usando que $π \approx \sqrt{6\binom{n}{2}/m}$.
Por ejemplo, con los números ganadores de la lotería diaría de la ONCE en 2017 tendríamos que $π \approx 3.07$ ($\Rightarrow$ tongo).

\item Recordemos que por el teorema de Chebyshev, para $x ≥ 3$:
\[ \frac{1}{2} \frac{x}{\log x} ≤ π(x) ≤ (1 + \log 4)\frac{x}{\log x} \]
Tomando $x = p_n$, de manera que $π(p_n)=n$:
\[ A \frac{p_n}{\log p_n} ≤ n ≤ B \frac{p_n}{\log p_n}\]
Tenemos que $p_n ≤ \frac{1}{A} n \log p_n$. Como existe $δ > 0$ tal que $\log x ≤ δ \sqrt{x}$ para $x≥1$, tenemos que $\frac{δ\sqrt{p_n}}{\log p_n} ≥ 1$. Luego:
\[ p_n ≤ \frac{1}{A} n \log p_n ≤ \frac{δ}{A}n\sqrt{p_n} \Rightarrow \sqrt{p_n} ≤ \frac{δ}{A}n \]
Sacando logaritmo a esta inecuación:
\[ \frac{1}{2} \log p_n ≤ \log \frac{δ}{A} + \log n\]
Luego:
\[ p_n ≤ \frac{1}{A} n \log p_n ≤ \frac{1}{A}n(\log \frac{δ}{A} + \log n) = \mathcal{O}(n\log n) \]

\item Como:
\[ \left(\sum_{n=1}^{∞} \frac{μ(n)}{n^s}\right) \left(\sum_{n=1}^{∞} \frac{1}{n^s}\right) = \sum_{n=1}^{∞} \left(\sum_{d\mid n} μ(d) \right)\frac{1}{n^s} = \frac{1}{1^s} = 1 \]
Entonces $\sum μ(n)/n^s = 1/ζ(s)$.
\end{enumerate}
\end{solucion}

\newpage

\begin{ejercicio}{2}
Sean $f$ y $g$ las funciones aritméticas tales que
\[ f(n) = 2^{ν_2(n)} \quad g(n) = 3^{ν_3(n)} \]
es decir $f(n)$ es la mayor potencia de $2$ que divide a $n$, y $g(n)$ la mayor potencia de $3$.
\begin{enumerate}[(a)]
\item Demostrar que $f$ y $g$ son multiplicativas.
\item Sea
\[ h(n) = \sum_{d \mid n} f(d) g(n/d) \]
¿Cuánto vale $h(p^α)$ si $p$ es un número primo?
\item Calcular $h(10!)$.
\end{enumerate}
\end{ejercicio}
\begin{solucion}
\begin{enumerate}[(a)]
\item Veamos que son completamente multiplicativas (que implica que son multiplicativas). Sean $n$ y $m$ naturales.
Entonces por el teorema fundamental de la aritmética: $n = \prod_p p^{ν_p(n)}$ y $m = \prod_p p^{ν_p(m)}$. Luego:
\[ nm = \prod_p p^{ν_p(n)+ν_p(m)}\]
Entonces $f(nm)=2^{ν_2(n)+ν_2(m)}=2^{ν_2(n)}\cdot 2^{ν_2(m)} = f(n)\cdot f(m)$. La demostración para $g$ es equivalente.
\item
\[ h(p^α) = \sum_{d \mid p^α} f(d) g(n/d) = \sum_{d=0}^{α} f(p^d) g(p^{α-d}) \]
Si $p=2$:
\[ h(2^α) = \sum_{d=0}^α f(2^d)\cdot 1 = \sum_{d=0}^α 2^d = 2^{α+1}-1 \]
Si $p=3$:
\[ h(3^α) = \sum_{d=0}^α g(3^d)\cdot 1 = \sum_{d=0}^α 3^d = \frac{1}{2}(3^{α+1}-1)\]
Para otro $p$:
\[ h(p^α) = \sum_{d=0}^α 1\cdot 1 = α+1\]

\item Como $h$ está definida como la convolución de dos funciones multiplicativas, $h$ es multiplicativa.
Todos los factores primos de $10!$ son $2$, $3$, $5$ y $7$, luego:
\[ h(10!) = h(2^{ν_2(10!)})\cdot h(3^{ν_3(10!)}) \cdot h(5^{ν_5(10!)}) \cdot h(7^{ν_7(10!)}) = \]
\[ = (2^{ν_2(10!)+1}-1)\left(\frac{1}{2}\left(3^{ν_3(10!)+1}-1\right)\right)(ν_5(10!)+1)(ν_7(10!)+1) \]
Por el teorema de Legendre:
\[ ν_2(10!) = \left\lfloor \frac{10}{2}\right\rfloor + \left\lfloor \frac{10}{4}\right\rfloor + \left\lfloor \frac{10}{8}\right\rfloor = 4+2+1=7 \]
\[ ν_3(10!) = \left\lfloor \frac{10}{3}\right\rfloor + \left\lfloor \frac{10}{9}\right\rfloor = 3+1 = 4 \]
\[ ν_5(10!) = \left\lfloor \frac{10}{5}\right\rfloor = 2 \]
\[ ν_7(10!) = \left\lfloor \frac{10}{7}\right\rfloor = 1 \]
Luego:
\[ h(10!) = (2^8-1)\cdot (3^5-1)\cdot 3 \cdot 2 \]

\newpage

\begin{ejercicio}{3}
Demostrar la relación asintótica
\[ \sum_{n≤x} \left(\log \frac{x}{n}\right)^2 = 2x + \bo{\log^2 x} \]
válida para $x ≥ 2$.
\end{ejercicio}
\begin{solucion}
\[ \sum_{n≤x} \left(\log \frac{x}{n}\right)^2 = \sum_{n≤x} \left(\log x - \log n\right)^2 = \sum_{n≤x} \left(\log^2 x + \log^2 n-2 \log x \log n\right)\]
\[ = x \log^2 x + \sum_{n≤x} \log^2 n - 2 \log x \sum_{n≤x} \log n\]
Por un lado: $\sum_{n≤x} \log n = x \log x - x + \mathcal{O}(\log x)$. Por otro lado, usando la fórmula de Abel con $a_n=1$ y $f(n)=\log^2 x$:
\[ \sum_{n≤x} \log^2 n = [x]\log^2 x - \int_1^x [t] 2 \log t \frac{dt}{t} = [x] \log^2 x - 2 (x \log x - x) + 2\int_1^x \{t\}\log t \frac{dt}{t} = \]
\[ = x \log^2 x -2x \log x +2x + \bo{\log^2 x} \]
Hemos usado que $[x]=x-\{x\}$ y que $\{x\}=\bo{1}$. Finalmente:
\begin{align*}
	\sum_{n≤x} \left(\log \frac{x}{n}\right)^2 & = x \log^2 x + ( x \log^2 x -2x \log x +2x) - 2 \log x (x \log x - x) + \bo{\log^2 x} =\\
	& = 2x + \bo{\log^2 x}
\end{align*}
\end{solucion}
\end{enumerate}
\end{solucion}
\end{document}