\documentclass[twoside]{article}
\usepackage{../estilo-ejercicios}
\providecommand{\bo}[1]{\mathcal{O}\left(#1\right)}
%--------------------------------------------------------
\begin{document}

\title{Teoría Analítica de Números\\Examen, 5 de septiembre de 2003}
\author{Rafael González López\\Diego Pedraza López}
\maketitle

\begin{ejercicio}{1}
\begin{enumerate}[(a)]
	\item Partiendo de $Γ(1/2)=\sqrt{π}$ demuestra la fórmula de $Γ(n+1/2)$.
	\item ¿Qué se quiere decir cuando hablamos de que la probabilidad de que dos números naturales sean primos entre sí es $6/π^2$? ¿Podrían servir los primeros premios de la loteria para calcular una aproximación de $π$?
	\item ¿Qué relación tiene el teorema de Chebyshev con la estimación $p_n = \mathcal{O}(n \log n)$?
	\item ¿Qué relación puedes establecer entre la función de Möbius y la función $1/ζ(s)$?
\end{enumerate}
\end{ejercicio}
\begin{solucion}
\begin{enumerate}[(a)]
	\item Tenemos que $Γ(s)=sΓ(s-1)$, luego para todo $n \in \N$:
	\[ Γ\left(\frac{1}{2}+n\right) = \left(\frac{1}{2}+n\right) Γ\left(\frac{1}{2}+n-1\right) = \dots = Γ\left(\frac{1}{2}\right)\prod_{i=0}^{n-1}\left(\frac{1}{2}+i\right) = \sqrt{π}\frac{1}{2^n}\prod_{i=0}^{n-1}\left(1+2i\right)\]
	\[ = \frac{\sqrt{π}}{2^n}(1 \cdot 3 \cdot 5 \cdots 2n) = \frac{\sqrt{π}}{2^n}\frac{(2n)!}{2 \cdot 4 \cdot 6 \cdots \cdot 2n} = \frac{\sqrt{π}}{2^n}\frac{(2n)!}{2^n n!}  = \frac{(2n)!}{n!2^{2n}}\sqrt{π} \]
	
	\item Tenemos que dados dos números enteros menor que $x$, la probabilidad de que sean coprimos es $6/π^2+\bo{\log x/x}$.
	Para $x$ suficientemente grande, esto se aproxima a $6/π^2$.
	
Para aproximar $π$ por los primeros premios de la lotería, bastaría ir comprobando dos a dos si son coprimos.
Si hemos comprobado todos los pares de $n$ números y $m$ de ellos son coprimos, tenemos que $m/\binom{n}{2} \approx 6/π^2$.
Luego podemos aproximar $π$ usando que $π \approx \sqrt{6\binom{n}{2}/m}$.
Por ejemplo, con los números ganadores de la lotería diaría de la ONCE en 2017 tendríamos que $π \approx 3.07$ ($\Rightarrow$ tongo).
\end{enumerate}
\end{solucion}
\end{document}