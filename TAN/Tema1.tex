\documentclass[TAN.tex]{subfiles}
\begin{document}

\chapter{Funciones aritméticas}
\section{Divisibilidad}
Consideramos conocidos los conjuntos:
\[ \N = \{1,2,3,4,\dots\} \text{ números naturales}\]
\[ \Z = \{\dots,-3,-2,-1,0,1,2,3,\dots,\} \text{ números enteros}\]
y las operaciones de suma y producto definidas en ellos con las propiedades usuales. El conjunto $\Z$, dotado con las operaciones usuales es un \textbf{anillo conmutativo}. El anillo $\Z$ es un \textbf{dominio de integridad}.

\section{Las funciones $d(n)$ y $σ(n)$}

\begin{prop} Si $f$ y $g$ son multiplicativas la función $f * g$ definida por
\[ f * g (n) = \sum_{d|n} f(d)g(n/d) \]
es también multiplicativa

\begin{dem}
\[ f*g(nm) = \sum_{c|nm}f(c)g\left(\frac{nm}{c}\right) = \sum_{a|n,b|m}f(ab)g\left(\frac{nm}{ab}\right) \]
Usando que $f$ y $g$ son multiplicativas:
\begin{align*}
	f*g(nm) & = \sum_{a|n,b|m} f(a)f(b)g(n/a)g(m/b)  = \sum_{a|n}f(a)g(n/a) \sum_{b|m}f(b)g(m/b) \\
	& = (f*g)(m) \cdot (f*g)(n)
\end{align*}
\end{dem}
\end{prop}

\section{Las funciones $φ(n)$ de Euler y $μ(n)$ de Möbius}
\begin{prop}\mbox{}
\begin{enumerate}[(a)]
	\item La funcion φ es multiplicativa
	\item $φ(n) = n \displaystyle\prod_{p|n} \left(1-\dfrac{1}{p}\right)$
\end{enumerate}
\end{prop}

\begin{dem}
Usando que $φ$ es multiplicativa:
\[ φ(p^a) = p^a - p^{a-1} = p^a(1-1/p) \]
Sea $n = p_1^{a_1}\cdots p_k^{a_k}$:
\[ φ(n) = φ(p_1^{a_1})\cdots φ(p_k^{a_k}) = p_1^{a_1}(1-1/p_1)\cdots p_k^{a_k}(1-1/p_k) = n \prod_{p|n} \left(1-\dfrac{1}{p}\right)\]
\end{dem}

\begin{prop}
Para todo $n \in \N$ se tiene $\displaystyle\sum_{d|n} φ(d) = n$.
\end{prop}

\begin{dem}
Sea $f(n) = \sum_{d|n} φ(d)$ y $g(n) = n$. Sea $h(n) = 1$. Obsérvese que $f = φ * h$. Como $φ$ y $h$ son multiplicativas, $f$ es multiplicativa. Como $g$ también es mulitplicativa, para probar que $f = g$ basta ver que $f(p^a)=g(p^a)$ para un $p$ primo y $a ≥ 1$.

\[ f(p^a) = \sum_{b|p^a} φ(b) = \sum_{k=0}^a φ(p^k) = 1+(p-1)+(p²-p)+\cdots+(p^a-p^{a-1}) = p^a = g(p^a) \]

Luego $f = g$.
\qed

Como demostración alternativa:
\[ n = \left|\left\{\frac{a}{n} \mid 1 ≤ a ≤ n\right\}\right|
= \left|\bigcup_{b|n} \left\{\frac{a}{b} \mid 1≤a≤b, a \perp b\right\}\right|
= \sum_{b|n} \left|\left\{\frac{a}{b} \mid 1 ≤ a ≤ b, a \perp b\right\}\right|
= \sum_{b|n} φ(b) \]
\end{dem}

\section{Series de Dirichlet}

\section{Convergencia de series de Dirichlet}

\section{Crecimiento de funciones multiplicativas}

\section{Sumación parcial}

\section{El orden medio de $d(n)$}

\section{Método de exclusión-inclusión}

\section{Algunas otras funciones aritméticas}
\end{document}
