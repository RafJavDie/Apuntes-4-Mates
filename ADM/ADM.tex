\documentclass[twoside]{article}
\usepackage{../estilo-ejercicios}
\newcommand{\x}{\underline{X}}
\renewcommand{\X}{\overline{\underline{X}}}

\usepackage{enumerate}
%--------------------------------------------------------
\begin{document}

\title{Análisis de Datos Multivariantes}
\author{Rafael, Diego}
\maketitle
\begin{ejercicio}{1} Sea $\x_1,\dotsc,\x_n$ una muestra aleatoria, entonces
$$
\sum_{i=1}^n (\x_i-\X) = 0
$$
\end{ejercicio}
\begin{solucion}
Basta tener en cuenta que 
$$
\sum_{i=1}^n (\x_i-\X) = \sum_{i=1}^n \x_i - \sum_{i=1}^n \X = \sum_{i=1}^n \x_i  - n \X = \sum_{i=1}^n \x_i  - n \frac{1}{n} \sum_{i=1}^n \x_i  =0
$$
\end{solucion}
\end{document}