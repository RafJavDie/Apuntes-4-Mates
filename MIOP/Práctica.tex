\documentclass[twoside]{article}
\usepackage{../estilo-ejercicios}

%--------------------------------------------------------
\begin{document}

\title{Modelos de Investigación Operativa\\ Práctica Obligatoria}
\author{Javier Aguilar Martín}
\maketitle



\begin{ejercicio}{1}
Una compañía desea introducir un nuevo producto al mercado y planea su estrategia de comercialización. Se ha tomado la decisión de introducir el producto en tres fases. La fase 1 incluye ofertas especiales para atraer a los compradores por primera vez. La fase 2 es una campaña para persuadir a estos compradores de primera vez a que continúen comprando el producto a precio normal. La fase 3 incluye una campaña para evitar que los clientes regulares cambien a una competencia que se sabe que se introducirá en el mercado.

Se cuenta con un presupuesto de 4 millones de euros para la campaña. El problema consiste en determinar cómo asignar este dinero de la manera más eficaza las tres fases. Sea $f_1$ la proporción de mercado inicial que se logra en la fase 1, $f_2$ la fracción de este mercado que se retiene en la fase 2, y $f_3$ la fracción del porcentaje de mercado que se retiene en la fase 3. Con los datos de la siguiente tabla, aplique programación dinámica para determinar la asignación de 4 millones para maximizar el porcentaje final del mercado para el nuevo producto, es decir, maximizar $f_1f_2f_3$. 
\begin{itemize}
\item[\textbf{a)}] Suponga que el dinero se debe gastar en cantidades enteras múltiplos de 1 millón en cada fase y que el mínimo permisible es 1 para la fase 1, y 0 para las fases 2 y 3.
\begin{center}
\begin{tabular}{c|lcc}
 & \multicolumn{3}{c}{\textbf{Efecto sobre el}}\\
\textbf{Millones de } & \multicolumn{3}{c}{\textbf{procentaje de mercado}} \\\cline{2-4}
\textbf{euros gastados} & $f_1\qquad$ & $f_2$ & $f_3$\\
\hline
0 & - & 0.2 & 0.3\\
1 & 0.2 & 0.4 & 0.5\\
2 & 0.3 & 0.5 & 0.6\\
3 & 0.4 & 0.6 & 0.7\\
4 & 0.5 & - & -
\end{tabular}
\end{center}

\item[\textbf{b)}] Suponga que se pueden gastar cualquier cantidad de presupuesto en cada fase, y que el efecto estimado al gastar una cantidad $x_i$ (en unidades de millones de dólares) en la fase $i$ ($i=1,2,3$) es:

\begin{tabular}{l}
$f_1=0.1x_1-0.01x2$\\
$f_2=0.4+0.1x_2$\\
$f_3=0.6+0.07x_3$
\end{tabular}

[\emph{Sugerencia}: después de obtener en forma analítica las funciones $f_2^*(s)$ y $f_3^*(s)$, obtenga $x_1^*$ de manera gráfica]
\end{itemize}
\end{ejercicio}
\begin{solucion}


\end{solucion}

\newpage

\begin{ejercicio}{2}
Consideremos el problema: $\min_{x\in[a_1,b_1]}f(x)$, donde $f$ es una función de $\R$ en $\R$ estrictamente convexa diferenciable. Fijar $n$ (número de iteraciones). Hacer $k=1$ y apliquemos el siguiente algoritmo:
\begin{enumerate}
\item Hacer $x_k=1/2(a_k+b_k)$ Si $f'(x_k)=0$ STOP.
\item Si $f'(x_k)>0$ hacer $a_{k+1}=a_k$ y $b_{k+1}=x_k$. Ir a 4.
\item Si $f'(x_k)<0$ hacer $a_{k+1}=x_k$ y $b_{k+1}=b_k$. Ir a 4. 
\item Hacer $k\leftarrow k+1$. Ir a 1 mientras no se alcance el número $n$ de iteraciones.
\end{enumerate}
\begin{itemize}
\item Probar que en cada iteración se reduce la longitud del intervalo que contiene la solución.
\item Probar que la solución óptima está en el intervalo $[a_n,b_n]$. 
\item ¿Cuál debe ser el número de iteraciones para asegurar una precisión de $\delta$?
\item Aplicar al problema $\min_{x\in[-3,5]}x^2+2x$ haciendo 5 iteraciones.
\end{itemize}
\end{ejercicio}
\begin{solucion}
\begin{itemize}
\item[]
\item Que el intervalo se reduce es claro, pues
$$\mu([a_{k+1},b_{k+1}]) = \begin{cases}
\mu([a_k,1/2(a_k+b_k)])\\
\mu([1/2(a_k+b_k),b_k])
\end{cases}
= \begin{cases}
1/2(a_k+b_k)-a_k\\
b_k - 1/2(a_k+b_k)
\end{cases} = \frac{b_k-a_k}{2}
$$
Como $b_k-a_k = \mu([a_k,b_k])$, por inducción llegamos a que
$$
\mu([a_{k+1},b_{k+1}]) = \frac{b_1-a_1}{2^k}
$$
\begin{lemma}
Existe una única solución del problema.

\begin{proof}
Como el intervalo $[a_1,b_1]$ es compacto (sea degenerado o no), tenemos la existencia por el Teorema de Weierstrass. La unicidad la tendremos a partir de que tanto el dominio como la función son convexos. Sea $x$ el mínimo global que nos asegura Weierstrass y sea $y$ otra mínimo global. Entonces $f(x)=f(y)$. Por ser estrictamente convexa, $f(\lambda x+(1-\lambda)y) < \lambda f(x)+(1-\lambda)f(y)=f(y)$, como el dominio es convexo, $\lambda x+(1-\lambda)y \in [a_1,b_1]$, luego $y$ no puede ser mínimo.
\end{proof}

\end{lemma}

Resta probar que en cada intervalo la única solución está dentro del intervalo. Veámoslo por inducción sobre la etapa fijado $n$. Para $k=1$ tenemos el propio intervalo, luego forzosamente ha de estar. Supongamos que la solución está en cada intervalo hasta la etapa $k\leq n-1$ y veamos que está en la etapa $k+1$. Distingamos casos
\begin{itemize}
\item Si $f'(x_k)>0$ entonces
\end{itemize}
\item Si el método no se ha detenido antes, sabemos por el apartado anterior que la solución óptima está en $[a_k,b_k]$ para todo $k\leq n$. En particular, también para $k=n$.
\item Sea $\delta>0$, $M=b_1-a_1>0$ y $x$ la solución óptima, entonces tras hacer $k$ iteraciones 
$$
|x-x_k|\leq b_{k+1}-a_{k+1} = \frac{b_1-a_1}{2^k}=\frac{M}{2^k}\leq\delta\\
$$
$$\frac{M}{2^k}\leq\delta \Leftrightarrow 2^k \geq \frac{M}{\delta} \Leftrightarrow k \geq \log_2  \frac{M}{\delta}
$$
Podemos asegurarlo a partir de $k=\lceil\log_2  \frac{M}{\delta}\rceil$.
\item Aplicamos las cinco iteraciones del método. $f'(x)=2x+2$.
\begin{enumerate}
\item $[a_1,b_1]=[-3,5]$, $x_1 = 1/2(5-3)= 1$, $f'(1)>0$. Hacemos $a_2=a_1$, $b_2=x_1$.
\item $[a_2,b_2]=[-3,1]$, $x_1 = 1/2(1-3)= 4-1$, $f'(-1)=0$. STOP. El mínimo se alcanza para $x_2 =1$.
\end{enumerate}
\end{itemize}

\end{solucion}

\newpage

\begin{ejercicio}{3}
La compañía $C$ debe servir a diez clientes cuyas respectivas demandas son $d_j, j=1,\dots,10$. Se dispone de cuatro camiones con capacidad $L_k$ y coste por día de operación $c_k,k=1,\dots,4$. Un camión no puede atender en un mismo día a más de cinco clientes, y no puede servir en el mismo día a los pares de clientes $(1,7),(2,6)$ y $(2,9)$. Formule el problema de minimizar el coste diario de atender a todos los clientes con los cuatro camiones disponibles.

Resuelva el problema, indicando la solución óptima, para los siguientes datos:
\begin{align*}
&d=(7,4,8,2,6,2,5,6,4,6),\\
&L=(30,20,10,10)\\
&c=(40,20,10,30)
\end{align*}
Resuélvalo utilizando Xpress.
\end{ejercicio}
\begin{solucion}
Definimos las variables de decisión $x_{kj}=$ cantidad de producto que el camión $k$ lleva al cliente $j$, para todo $k=1,\dots, 4$ y para todo $j=1,\dots, 10$. Vamos a definir también unas variables auxiliares que nos ayudarán a modelar el problema
\[
y_{kj}=\begin{cases}
1 & \text{si }x_{kj}\geq 1\\
0 & c.c.
\end{cases}
\]

Por tanto, tenemos que resolver el problema $\min \sum_{k=1}^4\sum_{j=1}^{10}c_ky_{kj}$ sujeto a las restricciones siguientes:
\begin{itemize}
\item Hay que suplir la demanda: $\sum_{k=1}^4 x_{kj}=d_j\ \forall j=1,\dots, 10$.
\item Límite de capacidad de los camiones: $x_{kj}\leq L_k\ \forall k=1,\dots, 4, \forall j=1,\dots, 10$. 
\item Límite de 5 servicios diarios: $\sum_{j=1}^{10} y_{kj}\leq 5\ \forall k=1,\dots, 4$.
\item Restricciones de parejas de clientes: $\forall k=1,\dots, 4$
\begin{align*}
&y_{k1}+y_{k7}\leq 1\\
&y_{k2}+y_{k6}\leq 1\\
&y_{k2}+y_{k9}\leq 1\\
\end{align*}
\item Restricciones adicionales para que $y_{kj}$ alcance los valores correctos: $\forall k=1,\dots, 4, \forall j=1,\dots, 10$
\begin{align*}
&y_{kj}\leq x_{kj}\\
&x_{kj}\leq L_k y_{kj}
\end{align*}
\item Restricciones de dominio: $\forall k=1,\dots, 4, \forall j=1,\dots, 10$
\begin{align*}
&x_{kj}\in\Z^+
&y_{kj}\in\{0,1\}
\end{align*}
\end{itemize}

En definitiva, el problema es el siguiente
\begin{align*}
\min\ & \sum_{k=1}^4\sum_{j=1}^{10}c_kx_{kj}\\
s.a.: & \sum_{k=1}^4 x_{kj}=d_j\\
      & x_{kj}\leq L_k\\
      &\sum_{j=1}^{10} y_{kj}\leq 5\\
      &y_{k1}+y_{k7}\leq 1\\
	  &y_{k2}+y_{k6}\leq 1\\
	  &y_{k2}+y_{k9}\leq 1\\
	  &y_{kj}\leq x_{kj}\\
	  &x_{kj}\leq L_k y_{kj}\\
	  &x_{kj}\in\Z^+\\
      &y_{kj}\in\{0,1\}
\end{align*}

\end{solucion}



\end{document}