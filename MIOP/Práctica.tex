\documentclass[twoside]{article}
\usepackage{../estilo-ejercicios}

%--------------------------------------------------------
\begin{document}

\title{Modelos de Investigación Operativa\\ Práctica Navideña}
\author{Rafael González López}
\maketitle



\begin{ejercicio}{1}
Una compañía desea introducir un nuevo producto al mercado y planea su estrategia de comercialización. Se ha tomado la decisión de introducir el producto en tres fases. La fase 1 incluye ofertas especiales para atraer a los compradores por primera vez. La fase 2 es una campaña para persuadir a estos compradores de primera vez a que continúen comprando el producto a precio normal. La fase 3 incluye una campaña para evitar que los clientes regulares cambien a una competencia que se sabe que se introducirá en el mercado.

Se cuenta con un presupuesto de 4 millones de euros para la campaña. El problema consiste en determinar cómo asignar este dinero de la manera más eficaza las tres fases. Sea $f_1$ la fracción de mercado inicial que se logra en la fase 1, $f_2$ la fracción de este mercado que se retiene en la fase 2, y $f_3$ la fracción del porcentaje de mercado que se retiene en la fase 3. Con los datos de la siguiente tabla, aplique programación dinámica para determinar la asignación de 4 millones para maximizar el porcentaje final del mercado para el nuevo producto, es decir, maximizar $f_1f_2f_3$. 
\begin{itemize}
\item[\textbf{a)}] Suponga que el dinero se debe gastar en cantidades enteras múltiplos de 1 millón en cada fase y que el mínimo permisible es 1 para la fase 1, y 0 para las fases 2 y 3.
\begin{center}
\begin{tabular}{c|lcc}
 & \multicolumn{3}{c}{\textbf{Efecto sobre el}}\\
\textbf{Millones de } & \multicolumn{3}{c}{\textbf{procentaje de mercado}} \\\cline{2-4}
\textbf{euros gastados} & $f_1\qquad$ & $f_2$ & $f_3$\\
\hline
0 & - & 0.2   & 0.3\\
1 & 0.2 & 0.4 & 0.5\\
2 & 0.3 & 0.5 & 0.6\\
3 & 0.4 & 0.6 & 0.7\\
4 & 0.5 & - & -
\end{tabular}
\end{center}

\item[\textbf{b)}] Suponga que se pueden gastar cualquier cantidad de presupuesto en cada fase, y que el efecto estimado al gastar una cantidad $x_i$ (en unidades de millones de dólares) en la fase $i$ ($i=1,2,3$) es:

\begin{tabular}{l}
$f_1=0.1x_1-0.01x_1^2$\\
$f_2=0.4+0.1x_2$\\
$f_3=0.6+0.07x_3$
\end{tabular}

[\emph{Sugerencia}: después de obtener en forma analítica las funciones $f_2^*(s)$ y $f_3^*(s)$, obtenga $x_1^*$ de manera gráfica]
\end{itemize}
\end{ejercicio}
\begin{solucion}
\
\begin{itemize}
\item[\textbf{a)}] 
Resolvemos el problema de programación dinámica. Distinguimos $3$ etapas con estados $(m,t)$ donde $m$ es la cantidad de dinero disponible y $t$ la etapa en la que nos encontramos. Nuestra función $f(m,t)$ nos devuelve el máximo de $\prod_{i=t}^3 f_i$ sabiendo que contamos con $m$ millones de euros en la etapa $t$. El objetivo será calcular $f(4,1)$. Si $g(x,t)$ nos da el efecto sobre el porcentaje de mercado en la fase $t$ sabiendo que invertimos $x$ millones de euros
$$
f(m,3) = g(m,3) \qquad f(m,t) = \max_{x_t\in\N,\;x_t\geq (2-t)} g(x_t,t)f(m-x_t,t+1) \quad t=1,2
$$

\underline{Etapa 3:} Como hay que gastar $1$ en la primera etapa, a lo sumo llegamos con $3$.
\begin{center}
\begin{tabular}{|c| c| c| c |}
\hline
$m$ & $x_3$ & $f(3,m)$ \\
\hline
$0$ & $0$ & 0.3   \\
\hline         
$1$ & $1$ & 0.5 \\
\hline        
$2$ & $2$ & 0.6  \\
\hline
$3$ & $3$  & 0.7   \\
\hline
\end{tabular}
\end{center}
\underline{Etapa 2:} Como hay que gastar $1$ en la primera etapa, a lo sumo llegamos con $3$.

\begin{center}
\begin{tabular}{|c| c| c| c | c| c|}
\hline
$m$ & $x_2$ & $g(x,2)$ & $f(m-x,3)$ &  $f(m,2)$ & $x^* $ \\
\hline
$0$ & $0$ & $0.2$ & $0.3$  & $0.06$& $\boxed{0}$\\
\hline
\hline
$1$ & $0$ & $0.2$ & $0.5$  & $0.1$& \\
\hline
 & $1$ & $0.4$ &$0.3$ & $0.12$ &$\boxed{1}$\\
\hline
\end{tabular} \qquad
\begin{tabular}{|c| c| c| c | c| c|}
\hline
$m$ & $x_2$ & $g(x,2)$ & $f(m-x,3)$ &  $f(m,2)$ & $x^* $ \\
\hline
$2$ & $0$ & $0.2$ & $0.6$  & $0.12$& \\
\hline
 & $1$    & $0.4$ &$0.5$ & $0.2$ &$\boxed{1}$\\
\hline
 & $2$    & $0.5$ &$0.3$ & $0.15$ &\\
\hline
\hline
$3$ & $0$ & $0.2$  & $0.7$  & $0.14$ & \\
\hline            
 & $1$    & $0.4$  &$0.6$ & $0.24ue$ &\\
\hline            
 & $2$    & $0.5$  &$0.5$ & $0.25$ &$\boxed{2}$\\
\hline
 & $3$ & $0.6$ & $0.3$ & $0.18$&\\
 \hline
\end{tabular}
\end{center}

\underline{Etapa 1:} 
\begin{center}

\begin{tabular}{|c| c| c| c | c| c|}
\hline
$m$ & $x_1$ & $g(x,1)$ & $f(m-x,2)$ &  $f(m,1)$ & $x^* $ \\
\hline
$4$ & $1$ & $0.2$  & $0.25$  & $0.05$ & \\
\hline            
 & $2$    & $0.3$  &$0.2$ & $0.06$ &$\boxed{2}$\\
\hline            
 & $3$    & $0.4$  &$0.12$ & $0.048$ &\\
\hline
 & $4$ & $0.5$     & $0.06$ & $0.03$&\\
 \hline
\end{tabular}
\end{center}
La solución óptima se alcanza invirtiendo $2$ millones en la Etapa 1, $1$ millón en la Etapa 2 y $1$ millón en la Etapa 3.
\item[\textbf{b)}] Resolvemos el problema de programación dinámica análogo al anterior. Distinguimos $3$ etapas con estados $(m,t)$ donde $m$ es la cantidad de dinero disponible y $t$ la etapa en la que nos encontramos. La función $f(m,t)$ devuelve el máximo de $\prod_{i=t}^3 f_i$ sabiendo que contamos con $m$ millones de euros en la etapa $t$. El objetivo será calcular $f(4,1)$. Si $g(x,t)$ nos da el efecto sobre el porcentaje de mercado en la fase $t$ sabiendo que invertimos $x$ millones de euros	
$$
f(m,3) = g(m,3) \qquad f(m,t)=\max_{0\leq x_t\leq m}g(x_t,t)f(m-x_t,t+1) \quad t=1,2
$$
Vamos a intentar dar una expresión clara del funcional en cada etapa
\begin{align*}
f(m,3)  &= \max_{0\leq x_3\leq m}0.6+0.07x_3 = 0.6+0.07m \\
f(m,2)  &= \max_{0\leq x_2\leq m}g(x_2,2)f(m-x_2,3)\\
		&= \max_{0\leq x_2\leq m}(0.4+0.1x_2)(0.6+0.07(m-x_2))\\
		&= \max_{0\leq x_2\leq m} - 0.007 x_2^2+ 0.007 m x_2 + 0.032 x_2 + 0.028 m   + 0.24
\end{align*}
Tenemos una parábola cóncava con vértice en $x_2=m/2 -16/7$ y ceros en $x_2=-4$ y $x_2=m+60/7$. Si $m\geq 32/7$, el vértice está dentro del intervalo considerado. En caso contrario, el máximo se alcanza en el extremo izquierdo del intervalo $(x_2=0)$. Por tanto
$$
f(m,2) = 
\begin{cases}
\dfrac{3(m+4)}{50} & 0\leq m \leq \dfrac{32}{7}\\
\dfrac{(7m+88)^2}{28000}& \dfrac{32}{7} \leq m\leq 4
\end{cases}
$$
Como $ \dfrac{32}{7} > 4$, la última condición podemos eliminarla. Vamos ahora a calcular la función que tenemos que maximizar en $f(4,1)$, a la que llamaremos $h(x_1)$, teniendo en cuenta que $x_1\geq 0$.
\begin{align*}
h(x_1) &= (0.1x_1-0.01x_1^2)\dfrac{3(8-x_1)}{50} & 0\leq 4-x_1 \leq \dfrac{32}{7}\\
&=(0.1x_1-0.01x_1^2)\dfrac{24-3x_1}{50} & 0 \leq x_1 \leq 4\\
\end{align*}
Por la gráfica de la función sabemos que el máximo se alcanza en el interior del intervalo. Analíticamente obtenemos que el máximo se alcanza en $x_1=6-2\sqrt{\frac{7}{3}}\approx 2.9449$ con un valor de $\dfrac{2(27+7\sqrt{21})}{1875}\approx0.063017$.
\end{itemize}
\end{solucion}

\newpage

\begin{ejercicio}{2}
Consideremos el problema: $\min_{x\in[a_1,b_1]}f(x)$, donde $f$ es una función de $\R$ en $\R$ estrictamente convexa diferenciable. Fijar $n$ (número de iteraciones). Hacer $k=1$ y apliquemos el siguiente algoritmo:
\begin{enumerate}
\item Hacer $x_k=1/2(a_k+b_k)$ Si $f'(x_k)=0$ STOP.
\item Si $f'(x_k)>0$ hacer $a_{k+1}=a_k$ y $b_{k+1}=x_k$. Ir a 4.
\item Si $f'(x_k)<0$ hacer $a_{k+1}=x_k$ y $b_{k+1}=b_k$. Ir a 4. 
\item Hacer $k\leftarrow k+1$. Ir a 1 mientras no se alcance el número $n$ de iteraciones.
\end{enumerate}
\begin{itemize}
\item Probar que en cada iteración se reduce la longitud del intervalo que contiene la solución.
\item Probar que la solución óptima está en el intervalo $[a_n,b_n]$. 
\item ¿Cuál debe ser el número de iteraciones para asegurar una precisión de $\delta$?
\item Aplicar al problema $\min_{x\in[-3,5]}x^2+2x$ haciendo 5 iteraciones.
\end{itemize}
\end{ejercicio}
\begin{solucion} Supondremos que el intervalo no es degenerado. Vamos a probar algunos lemas que podemos necesitar
\begin{lemma}
En las condiciones del enunciado, el problema tiene solución única.
\begin{proof}
Como el intervalo $[a_1,b_1]$ es compacto (sea degenerado o no), tenemos la existencia por el Teorema de Weierstrass. La unicidad la tendremos a partir de que tanto el dominio como la función son convexos. Sea $x$ el mínimo global que nos asegura Weierstrass y sea $y$ otro mínimo, que habrá de ser global también. Entonces $f(x)=f(y)$. Por ser estrictamente convexa y $x\neq y$, $\forall \lambda \in (0,1)$
$$f(\lambda x+(1-\lambda)y) < \lambda f(x)+(1-\lambda)f(y) = \lambda f(y)+(1-\lambda)f(y) = f(y) $$
Por tanto, $f(\lambda x+(1-\lambda)y) < f(y)$. Como el dominio es convexo, se sigue que $\forall \lambda \in[0,1]$ $\lambda x+(1-\lambda)y \in [a_1,b_1]$, luego $y$ no puede ser mínimo global. 
\end{proof}
\end{lemma}
\begin{lemma}
En las condiciones del enunciado, se tiene $\forall x,y\in[a_1,b_1]$
$$
(x-y)f'(y)\leq f(x)-f(y)
$$
\begin{proof}
Como $f$ es convexa $\forall t \in [0,1]$, $\forall x,y\in [a_1,b_1]$
\begin{gather*}
f(tx+(1-t)y)\leq tf(x)+(1-t)f(y)
\end{gather*} 
Si denotamos por $p(t)$ al miembro izquierdo y $s(t)$ al derecho entonces, entonces ambas son funciones diferenciables en $\R$ (por composición) tales que $p(0)=s(0)$, luego
$$
p'(0)=\lim_{h\downarrow 0}\frac{p(h)-p(0)}{h} = \lim_{h\downarrow 0}\frac{p(h)-s(0)}{h} \leq \lim_{h\downarrow 0}\frac{s(h)-s(0)}{h} = s'(0)
$$
Luego $\forall x,y\in [a_1,b_1]$
$$
(x-y)f'(y)\leq f(x)-f(y)
$$
\end{proof}
\end{lemma}
Nótese que por el resultado anterior si $f'(x)=0$ entonces el mínimo se alcanzará en $x$. Pasamos a resolver las cuestiones.
\begin{itemize}
\item Que el intervalo se reduce es claro, pues
$$\mu([a_{k+1},b_{k+1}]) = \begin{cases}
\mu([a_k,1/2(a_k+b_k)])\\
\mu([1/2(a_k+b_k),b_k])
\end{cases}
= \begin{cases}
1/2(a_k+b_k)-a_k\\
b_k - 1/2(a_k+b_k)
\end{cases} = \frac{b_k-a_k}{2}
$$
Como $b_k-a_k = \mu([a_k,b_k])$, por inducción llegamos a que
$$
\mu([a_{k+1},b_{k+1}]) = \frac{b_1-a_1}{2^k}
$$
Resta probar la única solución $x^*$ está dentro de cada intervalo. Supongamos que la solución está en $(a_k,b_k)$ y que $f(x_k)\neq0$.
\begin{itemize}
\item Si $f'(x_k)>0$, tenemos que ver que $x^*\in [a_k,x_k]$. Por reducción al absurdo, supongamos que $x^*\in [x_k,b_k]$. Como $x^*-x_k>0$ entonces $(x^*-x_k)f'(x_k)>0$. Tomando $x=x^*$, $y=x_k$ tenemos
\begin{gather*}0<(x^*-x_k)f'(x_k)\leq {f(x^*)-f(x_k)} \qquad f(x^*) > f(x_k)
\end{gather*}
Lo cuál es una contradicción con el hecho de que $x^*$ sea mínimo.
\item Si $f'(x_k)<0$, se sigue análogo al anterior teniendo en cuenta que si $x^*\in[a_k,x_k]$ entonces $x^*-x_k<0$, luego $(x^*-x_k)f'(x_k)>0$. 
\end{itemize}
\item Si el método no se ha detenido antes, sabemos por el apartado anterior que la solución óptima está en $[a_k,b_k]$ para todo $k\leq n$. En particular, también para $k=n$.
\item Sea $\delta>0$, $M=b_1-a_1>0$ y $x$ la solución óptima, entonces tras hacer $k$ iteraciones  
$$
|x-x_k|\leq b_{k+1}-a_{k+1} = \frac{b_1-a_1}{2^k}=\frac{M}{2^k}\leq\delta
$$
$$\frac{M}{2^k}\leq\delta \Leftrightarrow 2^k \geq \frac{M}{\delta} \Leftrightarrow k \geq \log_2  \frac{M}{\delta}
$$
Podemos asegurarlo a partir de $k=\lceil\log_2  \frac{M}{\delta}\rceil$.
\item Aplicamos las cinco iteraciones del método. $f'(x)=2x+2$.
\begin{enumerate} 
\item $[a_1,b_1]=[-3,5]$, $x_1 = 1/2(5-3)= 1$, $f'(1)>0$. Hacemos $a_2=a_1$, $b_2=x_1$.
\item $[a_2,b_2]=[-3,1]$, $x_1 = 1/2(1-3)= -1$, $f'(-1)=0$. STOP. El mínimo se alcanza para $x_2 =-1$.
\end{enumerate}
\end{itemize}

\end{solucion}

\newpage

\begin{ejercicio}{3}
La compañía $C$ debe servir a diez clientes cuyas respectivas demandas son $d_j, j=1,\dots,10$. Se dispone de cuatro camiones con capacidad $L_k$ y coste por día de operación $c_k,k=1,\dots,4$. Un camión no puede atender en un mismo día a más de cinco clientes, y no puede servir en el mismo día a los pares de clientes $(1,7),(2,6)$ y $(2,9)$. Formule el problema de minimizar el coste diario de atender a todos los clientes con los cuatro camiones disponibles.

Resuelva el problema, indicando la solución óptima, para los siguientes datos:
\begin{align*}
&d=(7,4,8,2,6,2,5,6,4,6),\\
&L=(30,20,10,10)\\
&c=(40,20,10,30)
\end{align*}
Resuélvalo utilizando Xpress.
\end{ejercicio}
\begin{solucion}
Definimos las variables de decisión $x_{kj}=$ cantidad de producto que el camión $k$ lleva al cliente $j$, para todo $k=1,\dots, 4$ y para todo $j=1,\dots, 10$. Vamos a definir también unas variables auxiliares que nos ayudarán a modelar el problema
\[
y_{kj}=\begin{cases}
1 & \text{si }x_{kj}\geq 1\\
0 & c.c.
\end{cases}
\]
\[
z_k=\begin{cases}
1 & \sum_{j=1}^{10}y_{kj}\geq 1\\
0 & c.c.
\end{cases}
\]

Las variables $y_{kj}$ representan si el camnión $k$ sirve al cliente $j$, y las variables $z_k$ representan si el camión $k$ realiza algún servicio. Por tanto, tenemos que resolver el problema $\min \sum_{k=1}^4c_kz_{k}$ sujeto a las restricciones siguientes:
\begin{itemize}
\item Hay que suplir la demanda: $\sum_{k=1}^4 x_{kj}=d_j\ \forall j=1,\dots, 10$.
\item Límite de capacidad de los camiones: $x_{kj}\leq L_k\ \forall k=1,\dots, 4, \forall j=1,\dots, 10$. 
\item Límite de 5 servicios diarios: $\sum_{j=1}^{10} y_{kj}\leq 5\ \forall k=1,\dots, 4$.
\item Restricciones de parejas de clientes: $\forall k=1,\dots, 4$
\begin{align*}
&y_{k1}+y_{k7}\leq 1\\
&y_{k2}+y_{k6}\leq 1\\
&y_{k2}+y_{k9}\leq 1\\
\end{align*}
\item Restricciones adicionales para que $y_{kj}$ y $z_k$ alcancen los valores correctos: $\forall k=1,\dots, 4, \forall j=1,\dots, 10$
\begin{align*}
&y_{kj}\leq x_{kj} & z_k\leq \sum_{j=1}^{10}y_{kj}\\
&x_{kj}\leq L_k y_{kj} & 5z_k\geq \sum_{j=1}^{10}y_{kj}
\end{align*}
\item Restricciones de dominio: $\forall k=1,\dots, 4, \forall j=1,\dots, 10$
\begin{align*}
&x_{kj}\in\Z^+ &y_{kj}\in\{0,1\}& &z_k\in\{0,1\}
\end{align*}
\end{itemize}

En definitiva, el problema es el siguiente
\begin{align*}
\min\ & \sum_{k=1}^4c_kz_{k}\\
s.a.: & \sum_{k=1}^4 x_{kj}=d_j\\
      & x_{kj}\leq L_k\\
      &\sum_{j=1}^{10} y_{kj}\leq 5\\
      &y_{k1}+y_{k7}\leq 1\\
	  &y_{k2}+y_{k6}\leq 1\\
	  &y_{k2}+y_{k9}\leq 1\\
	  &y_{kj}\leq x_{kj}\\
	  &x_{kj}\leq L_k y_{kj}\\
	  &z_k\leq \sum_{j=1}^{10}y_{kj}\\
	  &5z_k\geq \sum_{j=1}^{10}y_{kj}\\
	  &x_{kj}\in\Z^+\\
      &y_{kj}\in\{0,1\}\\
      &z_k\in\{0,1\}
\end{align*}

Cuya solución, calculada en Xpress es $x_{26}=2, x_{27}=5, x_{28}=6, x_{29}=4, x_{2,10}=6, x_{31}=7, x_{32}=4, x_{33}=8, x_{34}=2,x_{35}=6$ (el resto es 0), con un coste total de 30 unidades.

\end{solucion}



\end{document}