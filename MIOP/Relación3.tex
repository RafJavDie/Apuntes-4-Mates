\documentclass[twoside]{article}
\usepackage{../estilo-ejercicios}

%--------------------------------------------------------
\begin{document}

\title{Modelos de Investigación Operativa}
\author{Rafael González López, Javier Aguilar Martín}
\maketitle

\begin{ejercicio}{1}
Al principio de cada semana, una máquina se encuentra en uno de cuatro estados: excelente (E), bueno (B), regular (R) o malo (M). La ganancia semanal de una máquina en cada caso es la siguiente: excelente, 100; bueno, 80; regular, 50; malo,10, respectivamente. Después de observar la condición de una máquina al inicio de la semana, tenemos la opción de reemplazarla de forma instantánea por una máquina excelente, que cuesta 200. La máquina se deteriora con el tiempo como se ve en la siguiente tabla.

\hskip2.7cm Probabilidad de que la máquina inicie

\hskip3.7cm la semana próxima como

\begin{tabular}{ccccc}
\hline
Estado actual & Excelente & Bueno & Regular & Malo\\
\hline
Excelente & 0.7 & 0.3 & - & -\\
Bueno &      -  & 0.7 & 0.3 & -\\
Regular & - & - & 0.6 & 0.4\\
Malo & - & - & - & 1
\end{tabular}

\begin{itemize}
\item[\textbf{a)}] Determine el espacio de estados, los conjuntos de decisión, las probabilidades de transición y las recompensas.
\item[\textbf{b)}] Use el método de iteración de políticas para determinar la política más óptima con una tasa de depreciación $\alpha=$0.9.
\item[\textbf{c)}] Determine el problema de P.L. asociado.
\item[\textbf{d)}] Realice dos iteraciones de valor.
\end{itemize}
\begin{solucion}\
\begin{itemize}
\item[\textbf{a)}] $S=\{E,B,R,M\}$, $N=4$. $D(i)=\{R,NR\}$. 
$$
P^{NR} = 
\begin{pmatrix}
0.7 & 0.3 & 0 & 0\\
0   & 0.7 & 0.3& 0\\
0 & 0 & 0.6 & 0.4\\
0 & 0 & 0 & 1
\end{pmatrix} \qquad
P^R = 
\begin{pmatrix}
0.7 & 0.3 & 0 & 0\\
0.7 & 0.3 & 0 & 0\\
0.7 & 0.3 & 0 & 0\\
0.7 & 0.3 & 0 & 0
\end{pmatrix}
$$
Por tanto $R^{NR} = (100, 80, 50, 10)$, $R^R = (-100, -100, -100, -100)$. $R^R_M = -100$.
\item[\textbf{b)}] Si $\alpha = 0.9$ y sea $\delta = (NR,\ NR,\ R,\ R)$.
$\alpha=0'9$. Comprobamos la política $\delta=(NR,\ NR,\ R,\ R)$. 
$$v^\delta_E=R^\delta_E+\sum_{j=1}^N\alpha p^\delta_{Ej}v^\delta_j$$
Aplicamos esta fórmula a cada una de las componentes del vector de la política y obtenemos un sistema de ecuaciones
\begin{align*}
v^\delta_E=100+0'9(0'7v^\delta_E+0'3v^\delta_B)\\
v^\delta_B=80+0'9(0'7v^\delta_B+0'3v^\delta_R)\\
v^\delta_R=-100+0'9(0'7v^\delta_E+0'3v^\delta_B)\\
v^\delta_M=-100+0'9(0'7v^\delta_E+0'3v^\delta_B)
\end{align*}
La solución es la siguiente
\begin{align*}
v^\delta_E=687.8125\\
v^\delta_B=572.1875\\
v^\delta_R=487.8125\\
v^\delta_E=487.8125
\end{align*}
Además, teniendo en cuenta que $t_i^k= R_i^k + \sum_{j=1}^N \alpha p_{ij}^k v_j^\delta - v_i^\delta$, resulta $t_E^{NR} = 0$, $t_E^R <0$; $t_B^{R}<0$, $t_B^{NR}<0$, $t^{NR}_R >0$, $t^{NR}_M <0$. Si cambiamos nuestra política a $\delta = (NR, NR, NR, R)$, que es la óptima. \begin{align*}
v^\delta_E=690.23\\
v^\delta_B=575.50\\
v^\delta_R=492.35\\
v^\delta_E=490.23
\end{align*}
\item[\textbf{c)}]
\begin{align*}
\min\ & w_R+w_B+w_R+w_M\\
s.a.:\ & w_E - 0.9(0.7w_E+0.3w_B)\geq 100\\
&w_B-0.9(0.7w_B+0.3w_R)\geq 80\\
&w_B-0.9(0.7w_E+0.3w_B)\geq -100\\
&w_R-0.9(0.6w_R+0.4w_M)\geq 50\\
&w_R-0.9(0.7w_E+0.3w_B)\geq -100\\
&0.1w_M\geq 10\\
&w_M-0.9(0.7w_E+0.3w_B)\geq -100\\
&w_E,w_B,w_R,w_M\geq 0
\end{align*}
\end{itemize}
\end{solucion}
\end{ejercicio}

\newpage 
\begin{ejercicio}{2}
Cada año, al comenzar la estación para trabajar los jardines, un jardinero usa una prueba
química para determinar el estado del suelo. Dependiendo de los resultados de las pruebas, la
productividad para la nueva estación cae en uno de los tres estados: 1) bueno, 2) regular, 3) malo.
A través de los años el jardinero observó que las condiciones meteorológicas prevalecientes durante
el invierno (antes de empezar a trabajar los jardines) juegan un papel importante en la determinaci
ón de la condición del suelo, dejándolo igual o empeorándolo, pero nunca mejorándolo. En
este respecto, el estado del suelo en el año anterior es un factor importante para la productividad
del presente año. Usando los datos de las pruebas hechas por el jardinero, las probabilidades de
transición durante un periodo de un año, de un estado de productividad a otro, se representa con:
\[
\begin{pmatrix}
0. 2 & 0. 5 & 0. 3\\
0 & 0.5 & 0.5\\
0 & 0 & 1
\end{pmatrix}
\]
El jardinero puede alterar las probabilidades de transición aplicando fertilizante para mejorar las
condiciones del suelo, y se produce entonces la siguiente matriz de transición:
\[
\begin{pmatrix}
0.3 & 0.6 & 0.1\\
0.1 & 0.6 & 0.3\\
0.05 & 0.4 & 0.55
\end{pmatrix}
\]
Las ganancias que el jardinero obtiene en cada uno de los dos casos vienen dadas en las siguientes
matrices:

Beneficios si no utiliza fertilizantes: \hskip3cm Beneficios si utiliza fertilizantes:

\[
\begin{pmatrix}
7 & 6 & 3\\
0 & 5 & 1\\
0 & 0 & -1
\end{pmatrix}\hskip5.5cm
\begin{pmatrix}
6 & 5 & -1\\
7 & 4 & 0\\
6 & 3 & -2
\end{pmatrix}
\]
Use el método de iteración de políticas para determinar una la política óptima si hay un factor de
descuento $\alpha = 0,6$, determine el problema de P.L. asociado y realice dos iteraciones de valor.
\begin{solucion}
Etapas: 1,2,3,...
Estados: $S=\{1=Bueno,2=Regular,3=Malo\}$
$D(s)=\{F,NF\}, s\in S$. $\alpha=0.6$. Llamamos $P^{NF}$ y $P^F$ a las matrices de probabilidad de no usar fertilizante y de usarlo respectivamente. De igual modo llamamos $R^{NF}$ y $R^F$ a las de recompensas. Como las recompensas tenemos que darlas en forma de vector, calcularemos la esperanza de recompensa de cada estado, $R^F_i=\sum_{j=1}^r R_{ij}^Fp^F_{ij}$. La tabla de recompensas queda como sigue
\begin{tabular}{c|ccc}
$S\setminus s$ & 1 & 2 & 3\\
\hline
F & 4.7 & 3.1 & 0.4\\
NF & 5.3 & 3 & -1
\end{tabular}
%\begin{tabular}{|l|c|c|}\hline
%\diaghead{\theadfont Diag ColumnmnHead II}%
%  {Diag\\Column Head I}{Diag Column\\Head II}&
%\thead{Second\\column}&\thead{Third\\column}\\ \hline
%& foo & bar \\    \hline
%\end{tabular}

Ahora comenzamos con el algoritmo de evaluación de políticas. 
\begin{itemize}
\item Tenemos que elegir una política inicial, que la escogeremos basándonos en la recompensa esperada: $\delta=\{NF, F, F\}$.
\item  A continuación resolvemos la ecuación de evaluación en forma matricial $v^\delta=R^\delta+\alpha P^\delta v^\delta$ o vectorial $v^\delta_i=R^\delta_i+\alpha\sum_{j\in S}p_{ij}^\delta v_j^\delta$. $v_i^{\delta}$ es el beneficio o recompensa esperado siguiendo la política $\delta$ en el estado $i$.
\begin{align*}
&v^{NF}_1=5.3+0.6(0.2v_1^{NF}+0.5v_2^{F}+0.3v^F_3)\\
&v_2^F=3.1+0.6(0.1v_1^{NF}+0.6v_2^F+0.3v_3^F)\\
&v_3^F=0.4+0.6(0.05v_1^{NF}+0.4v_2^F+0.55v_3^F)
\end{align*}
La solución de este sistema es
\[
v_1^{NF}=8.98\quad v_2^F=6.63\quad v_3^F=3.38
\]
Ahora fijamos $t_1=t_2=t_3=0$.
\item $t_i=\max\{R_i^k+\alpha\sum_{j\in S}p_{ij}^kv_j^\delta-v_i^\delta, k\in D(i)\}$
Cuando elegimos como $k$ la decisión que habíamos tomado saldrá 0, así que basta calcular cada $t_i$ para las decisiones que no hemos tomado y comparemos. En caso de que salga menor o igual que cero no cambiamos de política (si fuera igual a cero podríamos cambiar pero no ganaríamos nada) y si es mayor que cero la cambiamos.
\begin{align*}
&t_1=\max\{4.7+0.6(0.3\cdot 8.98+0.6\cdot 6.63+0.1\cdot 3.38)-8.98<0,\ 0\}=0 \\
&t_2=\max\{0,\ 3+0.6(0.5\cdot 6.63+0.5\cdot 3.38)-6.63=-0.6\}=0\\
&t_3=\max\{0, smth<0\}=0
\end{align*}
\item No actualizamos ninguna política porque todos han dado 0. Por tanto la política que hemos determinado es óptima.
\end{itemize}

De forma general, el PL es
\begin{align*}
\min & \sum_{i=1}^N w_i\\
sa:  & w_i-\alpha\sum_{j=1}p_{ij}^kw_j\geq R_i^k, \forall i=1,\dots, N,\forall k\in D(i)
\end{align*}
cuya solución es el vector de beneficios esperado, y el dual
\begin{align*}
\max & \sum_{i=1}^N\sum_{k\in D(i)}R^k_ix_i^k\\
sa: & \sum_{k\in D(i)}x_i^k-\sum_{j=1}\sum_{k\in D(i)}\alpha p_{ij}^kx_j^k=1, \forall i=1,\dots, N\\
    & x^k_i\geq 0
\end{align*}

Sustituyendo los valores del problema obtenemos el PL
\begin{align*}
\min\ & w_1+w_2+w_3\\
sa: & w_1-0.6(0.3w_1+0.6w_2+0.1w_3)\geq 4.7 (F)\\
    & w_1-0.6(0.2w_1+0.5w_2+0.3w_3)\geq 5.3 (NF)\\
    & w_2-0.6(0.1w_1+0.6w_2+0.3w_3)\geq 3.1 (F)\\
    & w_2-0.6(0.5w_2+0.5w_3)\geq 3 (NF)\\
    & 0.4w_3\geq -1(F)\\
    & -0.03w_1-0.24w_2+0.63w_3\geq 0.4
\end{align*}
Dualizando después de simplificar obtenemos
\begin{align*}
\max\ & 5.3x_1^{NF}+4.7x_1^F+3x_2^{NF}+3.1x_2^F-x_3^{NF}+0.4x_3^F\\
sa:  & 0.88x_1^{NF}+0.82x_1^F-0.06x_2^F-0.03x_3^F=1\\
     & -0.3x_1^{NF}-0.36x_1^F+0.7x_2^{NF}+0.64x_2^F-0.24x_3^F=1\\
     & -0.18x_1^{NF}-0.06x_1^F-0.3x_2^{NF}-0.18x_2^F+0.4x_3^{NF}+0.63x_3^F=1\\
     & x_i^k\geq 0
\end{align*}

Para el último apartado, definimos para $n\geq 1$ $$v_i^n=\max_{k\in D(i)} R_i^k+\alpha\sum_{j=1}^N p_{ij}^k v_j^{n-1}, i=1,\dots, N$$
con $v_i^0$ arbitrario. Para simplificar los cálculos escogemos $v_i^0=0, i=1,2,3$. 
\begin{align*}
& v_1^1=\max\{4.7+0.6(0.3\cdot 0+0.6\cdot0+0.1\cdot 0)=4.7(F), 0.3+0.6\cdot0=5.3(NF)\}=5.3(NF)\\
& v_2^1=\max\{3,3.1\}=3.1(F)\\
& v_2^1=\max\{0.4,-1\}=0.4(NF)
\end{align*}
\begin{align*}
& v_1^2=\max\{4.7+0.6(0.3\cdot 5.3+0.6\cdot 3.1+0.\cdot 0.4)(F),5.3+0.6(0.2\cdot 5.3+0.5\cdot 3.1+0.3\cdot0.4 (NF)\}=6.93(NF)\\
& v_2^2=\max\{4.606(F),4.5(NF)\}\}=4.606(F)\\
& _3^2=\max\{1.435(F),-0.76(NF)\}=1.435(F)
\end{align*}
\end{solucion}

\end{ejercicio}

\newpage 

\begin{ejercicio}{3}

Cada año, una persona tiene la oportunidad de invertir en dos mutualidades diferentes, A y B.
Al final de cada año, liquida su inversión, recoge sus ganancias y reinvierte. Las ganancias anuales
de los fondos de mutualidad dependen de la reacción del mercado. En los últimos años el mercado
ha oscilado alrededor de los 2500 puntos, de acuerdo a las probabilidades que se dan en la siguiente
matriz:
\[
\begin{array}{cc}
  \begin{matrix}
 
\end{matrix} & \begin{matrix}
~~2400 & ~2500 & 2600~  & \\ 
\end{matrix} \\
\begin{matrix}
2400 \\
2500 \\
2600 \\
\end{matrix}
 & \begin{pmatrix}
0.3 & 0.5 & 0.2 \\
0.1 & 0.5 & 0.4 \\
0.2 & 0.4 & 0.4
\end{pmatrix}
\end{array}
\]
Cada año en que el mercado sube (o baja) 100 puntos, la inversión A tiene ganancias (o pérdidas)
de 20 u.m., mientras que la inversión B tiene ganancias (o pérdidas) de 10 u.m. Si el mercado sube
(o baja) 200 puntos en un año, las ganancias (o pérdidas) pasan a ser de 50 u.m. y 20 u.m. para A
y B respectivamente. Si el mercado no cambia, ninguno de los fondos tiene ganancias o pérdidas.
\begin{itemize}
\item[\textbf{a)}] Determine el espacio de estados, los conjuntos de decisión, las probabilidades de transición y
las recompensas.
\item[\textbf{b)}] Use el método de iteración de políticas para determinar una la política óptima con una tasa de
depreciación $\alpha = 0.9$.
\item[\textbf{c)}] Determine el problema de P.L. asociado.
\item[\textbf{d)}] Realice dos iteraciones de valor.
\end{itemize}

\begin{solucion}\
\begin{itemize}
\item[\textbf{a)}] Etapas: 1,2,3,... Estados $S=\{2400,2500,2600\}$. $P^A=P^B=$ la matriz del enunciado. Las tabla de ganancias y pérdidas son las siguientes

\begin{tabular}{c|ccc}
A & 2400 & 2500 & 2600\\
\hline
2400 & 0 & 20 & 50\\
2500 & -20 & 0 & 20\\
2600 & -50 & -20 & 0
\end{tabular}\quad
\begin{tabular}{c|ccc}
B & 2400 & 2500 & 2600\\
\hline
2400 & 0 & 10 & 20\\
2500 & -10 & 0 & 10\\
2600 & -20 & -10 & 0
\end{tabular}

Tenemos las recompensas  $R^A=(20,6,-18), R^B=(9,3,-8)$. Estas recompensas se otienen multpilicando el beneficio por su probabilidad, por ejemplo $20=0\cdot 0.03+20\cdot 0.5+50\cdot 0.2$
\item[\textbf{b)}] Vamos a comenzar con la política $\delta=(A,A,B)$. Tenemos que resolver el sistema
\begin{align*}
&v^A_1=20+0.9(0.3v^A_1+0.5v^A_2+0.2v^B_3)\\
&v^A_2=6+0.9(0.1v^A_1+0.5v^A_2+0.4v^B_3)\\
&v^B_3=-8+0.9(0.2v^A_1+0.4v^A_2+0.4v^B_3)
\end{align*}
cuya solución es 

$$v^A_1=53.5374\qquad v^A_2=33.7814\qquad v^B_3=21.5594$$

Ahora calculamos los $t_i^k, k\neq\delta(i)$ por si tenemos que modificar la política. 
\begin{align*}
&t_1^B=9+0.9(0.3\cdot 53.5374 +0.5\cdot 33.7814+0.2\cdot 21.5594)-53.5374<0\\
&t_2^B=3+0.9(0.1\cdot 53.5374+0.5\cdot  33.7814+0.4\cdot 21.5594)-33.7814<0\\
&t_3^A=-18+0.9(0.2\cdot 53.5374+ 0.4\cdot 33.7814+ 0.4\cdot 21.5594)-21.5594<0
\end{align*}
Por lo que la política que hemos elegido es óptima.
\item[\textbf{c)}]
\begin{align*}
\min\ & w_1+w_2+w_3\\
s.a.:\ & w_1-0.9(0.3w_1+0.5w_2+0.2w_3)\geq 20\\
       & w_2-0.9(0.1w_1+0.5w_2+0.4w_3)\geq 6\\
       & w_3-0.9(0.2w_1+0.4w_2+0.4w_3)\geq -8\\
       & w_1,w_2,w_3\geq 0
\end{align*}
\item[\textbf{d)}] Tomamos $v^0=(0,0,0)$.
\begin{align*}
&v^1_1=20+0.9(0.3v^0_1+0.5v^0_2+0.2v^0_3)=20\\
&  v^1_2=6+0.9(0.1v^0_1+0.5v^0_2+0.4v^0_3)=6\\
&  v^1_3=-8+0.9(0.2v^0_1+0.4v^0_2+0.4v^0_3)=-8
\end{align*} 
\begin{align*}
&v^2_1=20+0.9(0.3v^1_1+0.5v^1_2+0.2v^1_3)=26.66\\
&  v^2_2=6+0.9(0.1v^1_1+0.5v^1_2+0.4v^1_3)=7.62\\
&  v^2_3=-8+0.9(0.2v^1_1+0.4v^1_2+0.4v^1_3)=-5.12
\end{align*}
\end{itemize}

Resolución en Xpress:
\begin{verbatim}
model ModelName
uses "mmxprs"; 

!optional parameters section
parameters
 N=3
end-parameters

!sample declarations section
declarations
	w:array(1..N) of mpvar
	x:array(1..N,1..2) of mpvar
	v,vv:array(1..N) of real
end-declarations


!Problema de Programación Lineal asociado al PDM
forall(i in 1..N) w(i) is_free

w(1)-0.9*(0.3*w(1)+0.5*w(2)+0.2*w(3))>=20
w(2)-0.9*(0.1*w(1)+0.5*w(2)+0.4*w(3))>=6
w(3)-0.9*(0.2*w(1)+0.4*w(2)+0.4*w(3))>=-8
minimize(sum(i in 1..N)w(i))


(!
!Problema Dual
forall(i in 1..N,k in 1..2)x(i,k)>=0

x(1,1)+x(1,2)-0.9*(0.3*(x(1,1)+x(1,2))+0.1*(x(2,1)+x(2,2))+0.2*(x(3,1)+x(3,2)))=1
x(2,1)+x(2,2)-0.9*(0.5*(x(1,1)+x(1,2))+0.5*(x(2,1)+x(2,2))+0.4*(x(3,1)+x(3,2)))=1
x(3,1)+x(3,2)-0.9*(0.2*(x(1,1)+x(1,2))+0.4*(x(2,1)+x(2,2))+0.4*(x(3,1)+x(3,2)))=1
maximize(20*x(1,1)+9*x(1,2)+6*x(2,1)+3*x(2,2)-18*x(3,1)-8*x(3,2))
!)




!Método de iteraciones sucesivas
v(1):=0
v(2):=0
v(3):=0
vv(1):=0
vv(2):=0
vv(3):=0

!Puede cambiar el número de iteraciones, por defecto hemos puesto 100
forall(r in 0..100)do
  forall(i in 1..N)writeln("v(",i,")^",r," = ",v(i))
  v(1):=20+0.9*(0.3*vv(1)+0.5*vv(2)+0.2*vv(3))
  v(2):=6+0.9*(0.1*vv(1)+0.5*vv(2)+0.4*vv(3))
  v(3):=-8+0.9*(0.2*vv(1)+0.4*vv(2)+0.4*vv(3))
  forall(i in 1..N)vv(i):=v(i)
end-do


end-model
\end{verbatim}

\end{solucion}
\end{ejercicio}



\end{document}