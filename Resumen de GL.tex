\documentclass[twoside]{article}
\usepackage{amsmath,accents}%
\usepackage{amsfonts}%
\usepackage{amssymb}%
\usepackage{graphicx}
\usepackage{mathrsfs}
\usepackage[utf8]{inputenc}
\usepackage{amsfonts}
\usepackage{amssymb}
\usepackage{graphicx}
\usepackage{mathrsfs}
\usepackage{setspace}  
\usepackage{amsmath}
\usepackage{nccmath}
\usepackage[spanish]{babel}
\usepackage{multirow}

\renewcommand{\baselinestretch}{1,4}
\setlength{\oddsidemargin}{0.5in}
\setlength{\evensidemargin}{0.5in}
\setlength{\textwidth}{5.4in}
\setlength{\topmargin}{-0.25in}
\setlength{\headheight}{0.5in}
\setlength{\headsep}{0.6in}
\setlength{\textheight}{8in}
\setlength{\footskip}{0.75in}

\newtheorem{theorem}{Teorema}[section]
\newtheorem{acknowledgement}{Acknowledgement}
\newtheorem{algorithm}{Algorithm}
\newtheorem{axiom}{Axiom}
\newtheorem{case}{Case}
\newtheorem{claim}{Claim}
\newtheorem{propi}[theorem]{Propiedades}
\newtheorem{condition}{Condition}
\newtheorem{consec}[theorem]{Consecuencia}
\newtheorem{coro}[theorem]{Corolario}
\newtheorem{criterion}{Criterion}
\newtheorem{defi}[theorem]{Definición}
\newtheorem{example}[theorem]{Ejemplo}
\newtheorem{exercise}{Exercise}
\newtheorem{lemma}[theorem]{Lema}
\newtheorem{nota}[theorem]{Nota}
\newtheorem{problem}{Problem}
\newtheorem{prop}[theorem]{Proposición}
\newtheorem{remark}{Remark}

\newtheorem{dem}[theorem]{Demostración}

\newtheorem{summary}{Summary}
\numberwithin{equation}{section}

\providecommand{\abs}[1]{\lvert#1\rvert}
\providecommand{\norm}[1]{\lVert#1\rVert}
\providecommand{\ninf}[1]{\norm{#1}_\infty}
\providecommand{\numn}[1]{\norm{#1}_1}
\providecommand{\gabs}[1]{\left|{#1}\right|}
\newcommand{\bor}[1]{\mathcal{B}(#1)}
\newcommand{\erre}{\mathbb{R}}
\newcommand{\resi}{\varepsilon_L}
\providecommand{\conv}[1]{\overset{#1}{\longrightarrow}}
\providecommand{\convcs}{\xrightarrow{CS}}
% xrightarrow{d}[d]
\providecommand{\lrg}{\longrightarrow}
%--------------------------------------------------------
\begin{document}
\section{Curvas}
\subsection{Introducción}
\begin{defi}
Sea $\alpha:(a,b)\longrightarrow \erre$ $\alpha=\alpha(t)$ una CAPR y $\alpha(t_0)$ un punto de la misma. Se denomina parámetro natural (o longitud de arco) de $\alpha$ respecto del punto $\alpha(t_0)$ a la función escalar:
\begin{gather*}
s:(a,b)\longrightarrow \erre\\
s(t)=\int_{t_0}^t \abs{\alpha'(t)}dt
\end{gather*}
\end{defi}
\begin{theorem}[Caracterización del Parámetro Natural] Sea $\alpha:(a,b)\longrightarrow \erre$, $\alpha=\alpha(t)$ una CAPR. Entonces $t$ es el p.n. de $\alpha$ (salvo constante, es decir $t=s+cte$) si y solo sí $\abs{\alpha'(t)}=1$ $\forall t\in(a,b)$.
\end{theorem}
\subsection{Curvatura}
\begin{defi}
Sea $\alpha=\alpha(s)$ una CARPN, $\alpha:(a,b)\longrightarrow \erre^3$. Se denomina \textbf{curvatura} de $\alpha$ en cada punto de $\alpha(s)$ a la función escalar:
\begin{gather*}
k=k(s) \quad k:(a,b)\longrightarrow \erre\\
k(s)=\abs{\ddot{\alpha}(s)} \quad \forall s\in (a,b)
\end{gather*}
\end{defi}
\newpage
\subsection{Referencia de Frenet de curvas alabeadas}
\begin{defi}
Sea $\alpha(s)$ una CARPN. Se define el \textbf{vector normal principal} de $\alpha(s)$ en cada punto y se representa $N(s)$ como un vector unitario en la dirección de $\ddot{\alpha}(s)$. Es decir:
\begin{gather*}
N(s)=\frac{\ddot{\alpha}(s)}{\abs{\ddot{\alpha}(s)}}
\end{gather*}
En las condiciones anteriores se denomina binormal de $\alpha$ en cada punto y se representa por $B(s)$ al vetor:
\begin{gather*}
B(s)=\dot{\alpha}(s)\times\frac{\ddot{\alpha}(s)}{\abs{\ddot{\alpha}(s)}}=T(s)\times N(s)
\end{gather*}
\end{defi}
\begin{theorem} Expresiones de los vectores de la base de Frenet en el caso de que nuestra CAR no esté parametrizada naturalmente.
\begin{gather*}
T(t)=\frac{\alpha'(t)}{\abs{\alpha'(t)}} \qquad B(t)=\frac{\alpha'(t)\times\alpha''(t)}{\abs{\alpha'(t)\times\alpha''(t)}}\\
N(t)=\dfrac{[\alpha'(t)\times\alpha''(t)]\times\alpha'(t)}{\abs{[\alpha'(t)\times\alpha''(t)]\times\alpha'(t)}}
\end{gather*}
\end{theorem}
\subsection{Torsión}
\begin{defi}
Se define la torsión de un CARPN como: $\boxed{\tau(s)=-\dot{B}(s)N(s)}$.
\end{defi}
\begin{nota} No se puede pasar de $(2)$ a $(1)$ multiplicando por $-N(s)$ porque el producto escalar no es asociativo, es decir, $(-\dot{B}(s)N(s))(-N(s))\neq -\dot{B}(s)(N(s)N(s))$. A la inversa sí lo podemos hacer porque $\tau(s)$ es un escalar.
\end{nota}
\subsection{Consecuencias de la definición}
\begin{theorem}
Sea $\alpha(s)$ una CARPN. Entonces $\alpha(s)$ es plana si y solo si $\tau(s)=0$ $\forall s$.
\end{theorem}
\subsection{Ecuación de Frenet-Serret}
\begin{theorem}[Ecuaciones de Frenet-Serret] Sea $\alpha(s)$ una curva CARPN. En cada punto de la curva se tiene:
\begin{gather*}
\dot{T}(s)=k(s)N(s) \qquad \dot{N}(s)=-k(s)T(s)+\tau(s)B(s) \qquad \dot{B}(s)=-\tau(s)N(s)
\end{gather*}
\end{theorem}
\begin{theorem}
Normalmente calcularemos de la siguiente forma la curvatura y la torsión. 
\begin{gather*}
k(s)=\abs{\ddot{\alpha}(s)} \qquad k(t)=\frac{|\alpha'(t)\times\alpha''(t)|}{|\alpha'(t)|^3}\\
\tau(s)=\frac{\dot{\alpha}(s)\cdot(\ddot{\alpha}(s)\times\dddot{\alpha}(s))}{|\ddot{\alpha}(s)|^2} \qquad \tau(t)=\frac{\alpha'(t)\cdot(\alpha''(t)\times\alpha'''(t))}{|\alpha'(t)\times\alpha''(t)|^2}
\end{gather*}
\end{theorem}
\newpage
\section{Superficies}
\subsection{Introducción}
\begin{defi} Se denomina {\em superficie parametrizada regular o superficie inmersa} (SPR) en $\mathbb{R}^3$ a toda aplicación $\chi : U \, (abierto) \, \subseteq \mathbb{R}^2 \mapsto \mathbb{R}^3,$ tal que verifique las dos siguientes condiciones:
\begin{itemize}
\item C1 (Condición de Diferenciabilidad): $\chi \in \mathcal{C}^k$, para algún $k \in \mathbb{N}$.
\item C2 (Condición de Regularidad): $\frac{\partial \chi}{\partial u^1} \wedge \frac{\partial \chi}{\partial u^2} \neq \bar{0},$ \, para todo \, $(u^1, u^2) \in U$.
\end{itemize}
\end{defi}


\begin{defi} Se denomina {\em superficie simple} (SS) a toda superficie parametrizada regular que sea además inyectiva.
\end{defi}


\subsection{Vector Normal y Plano Tangente}

\begin{defi}
Sea $\chi: U \mapsto \mathbb{R}^3$ una superficie simple y $ p\in \chi(U)$ un punto de la misma. Se denomina {\em plano tangente} a $\chi$ en $p$ al plano que pasa por $p$ y tiene como vector normal al $\chi_1 \wedge \chi_2$ en $p.$
\end{defi}

\begin{defi}
Se denomina {\em vector unitario normal} a la superficie simple $\chi = \chi(u^1, u^2)$ en cada punto, y se representa por $ N(u^1, u^2),$ a un vector unitario ortogonal al plano tangente a $\chi$ en ese punto.
\begin{equation*}
N(u^1, u^2) = \frac{\chi_1(u^1, u^2) \wedge \chi_2(u^1, u^2)}{\mid \chi_1(u^1, u^2) \wedge \chi_2(u^1, u^2) \mid}
\end{equation*}
\end{defi}

\subsection{Curvas en una superficie}


\begin{defi} Sea $\chi : U \, (abierto) \subset \mathbb{R}^2 \mapsto \mathbb{R}^3,$ $\chi = \chi(u^1, u^2)$ una superficie simple. Se denomina {\em curva diferenciable} en $\chi$ a toda curva alabeada parametrizada regular, diferenciable, contenida entera\-mente en $\chi$, es decir a toda aplicación

$\alpha : (a,b) \, \subset \mathbb{R} \mapsto \mathbb{R}^3,$ $\alpha =
\alpha(t) = \chi(u^1(t), u^2(t)),$ para todo $ t \in (a,b).$
\end{defi}


\begin{defi} {\em Curvas paramétricas o líneas paramétricas de una superficie}

Consideremos el abierto $U$ de $\mathbb{R}^2,$ de coordenadas
$(u^1, u^2)$ y sea $\chi : U \, \mapsto \mathbb{R}^3$ una
superficie simple. Se denominan {\em líneas paramétricas o curvas
paramétricas} de $\chi$ a la imagen mediante la aplicación  $\chi$
de las líneas paramétricas $u^1 = cte$ y $u^2 = cte$ de $U,$ es decir,
las líneas paramétricas de $\chi$ son las curvas sobre la
superficie:
\begin{center} curvas \, $u^1$-param\'etricas: \,\, $\alpha =
\alpha(u^1) = \chi(u^1, u^2_0),$

curvas \, $u^2$-param\'etricas: \,\, $\beta = \beta(u^2) =
\chi(u^1_0, u^2).$
\end{center}
\end{defi}

\subsection{Vectores Tangentes en una superficie}

\begin{defi} Sea $\chi : U \, (abierto) \subset \mathbb{R}^2 \mapsto \mathbb{R}^3,$ $\chi = \chi(u^1, u^2)$ una superficie simple y $P \in \chi(U)$ un punto. Se denomina {\em vector tangente} a $\chi$ en $P$ a todo vector $X \in \mathbb{R}^3$ que sea tangente a una curva alabeada regular cualquiera, contenida en $\chi$ y que pase por $P$.
\end{defi}

\begin{defi}
Al conjunto de todos los vectores tangentes a la superficie $\chi$ en un punto $P$ de la misma lo denotaremos por $T_P(\chi),$ es decir:
\begin{center}
$T_P(\chi) = \{X \in \mathbb{R}^3 \,| \, X \, es \, un \, vector \, tangente \, a \, \chi \, en \, P \}$
\end{center}
\end{defi}

\begin{theorem}
Todo vector tangente $X$ a una superficie simple $\chi$ en un punto $P$ de la misma puede expresarse según $X = X^1 \, \chi_1 + X^2 \, \chi_2.$
\end{theorem}

\begin{dem}
Sea $X \in T_p(\chi)$ y sea $\alpha = \alpha(t) = \chi(u^1(t), u^2(t))$ una curva en $\chi$ tal que $P = \alpha(0)$ y $X = \alpha'(0).$
Entonces:
\begin{gather*}
\alpha'(t) = \frac{\partial \, \chi}{\partial \, u^1} \, \frac{d \, u^1}{d\, t} + \frac{\partial \, \chi}{\partial \, u^2} \, \frac{d \, u^2}{d\, t} = \chi_1 \, 
\frac{d \, u^1}{d\, t} + \chi_2 \,
\frac{d \, u^2}{d\, t}
\end{gather*}
Se tendrá que:

\begin{gather*}
X = \alpha'(0) = \chi_1 \, 
\left.\left(\frac{d \, u^1}{d\, t}\right)\right|_{t=0} + \chi_2 \,
\left.\left(\frac{d \, u^2}{d\, t}\right)\right|_{t=0}
\end{gather*}
Y llamando:

\begin{gather*}
X^1 = \left.\left(\frac{d \, u^1}{d\, t}\right)\right|_{t=0} \qquad  X^2 = \left.\left(\frac{d \, u^2}{d\, t}\right)\right|_{t=0}
\end{gather*}
Finalmente, $X = X^1 \, \chi_1 + X^2 \, \chi_2 = \sum^2_{i = 1} \, X^i \, \chi_1$.
\end{dem}
\subsection{Superficies de revolución}

\begin{defi} Se denominan {\em superficies de revoluci\'on} a aquellas superficies engendradas por una curva alabeada plana regular {\bf $\alpha = \alpha(v)$} (denominada curva generatriz) al girar alrededor de una recta contenida en su plano (denominada eje de giro).
\end{defi}

\begin{defi}
\begin{enumerate}\item[]
\item [a)] Se denominan {\em meridianos} (o generatrices) de una superficie de revoluci\'on a las curvas intersecci\'on de dicha superficie con planos
que pasan por el eje. Los meridianos de una de una superficie de revoluci\'on son curvas del mismo tipo que la generatriz.

\item [b)] Se denominan {\em paralelos} (o c\'irculos de latitud) de una superficie de revoluci\'on a las curvas intersecci\'on de dicha superficie con planos perpendiculares al eje. Los paralelos de una de una superficie de revoluci\'on son siempre circunferencias (de ahí su nombre).
\end{enumerate}
\end{defi}
\begin{nota} De las definiciones se siguen las siguientes equivalencias:
\begin{itemize}
\item Meridianos $\equiv$ generatrices $\equiv$ curvas v-param\'etricas $\equiv$ curvas {\bf $u = cte$}.
\item Paralelos $\equiv$ c\'irculos de latitud $\equiv$ curvas u-param\'etricas $\equiv$ curvas {\bf $v = cte$}.
\end{itemize} 
\end{nota}
\end{document}