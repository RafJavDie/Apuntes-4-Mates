\documentclass[twoside]{article}
\usepackage{../../estilo-ejercicios}

%--------------------------------------------------------
\begin{document}

\title{Ejercicios de Ideals, Varieties, and Algorithms (4ª Edición)}
\author{Diego Pedraza López, Javier Aguilar Martín, Rafael González López}
\maketitle

\begin{ejercicio}{7.2.1}
If $A, B ∈ GL(n, k)$ are invertible matrices, show that $AB$ and $A^{−1}$ are also invertible.
\end{ejercicio}
\begin{solucion}
Si $A$ y $B$ son invertibles podemos considerar $A^{-1}$ y $B^{-1}$ y el producto $B^{-1}A^{-1}$, que es claramente la inversa de $AB$. La inversa de $A^{-1}$ es trivialmente $A$.
\end{solucion}

\newpage

\begin{ejercicio}{7.2.2}
Suppose that $A ∈ GL(n, k)$ satisfies $A^m = I_n$ for some positive integer. If $m$ is the
smallest such integer, then prove that the set $C_m = \{I_n, A, A^2,\dots , A^{m−1}\}$ has exactly $m$
elements and is closed under matrix multiplication.
\end{ejercicio}
\begin{solucion}
$C_m$, por construcción, tiene a lo sumo $m$ elementos. Si tuviera menos, esto implicaría que $A^k=A^l$ para algunos $0\leq k<l\leq m$. En ese caso, $I_n=A^{l-k}$, lo cual contradice que que $m$ sea el menor entero que verifica esa propiedad. Para ver que es cerrado para la multiplicación, basta ver que si $l+k> m$ entonces $A^lA^k\in C_m$, pues en cualquier otro caso es trivial. Dividiendo $l+k$ por $m$ obtendremos que $A^lA^k=A^{l+k}=A^{qm+r}=A^{qm}A^r=A^r$ con $r<m$, por lo que se tiene el resultado. 
\end{solucion}

\newpage

\begin{ejercicio}{7.2.3}
Write down the six permutation matrices in $GL(3, k)$.
\end{ejercicio}
\begin{solucion}
\[
\begin{pmatrix}
1 & 0 & 0\\
0 & 1 & 0\\
0 & 0 & 1
\end{pmatrix}, \begin{pmatrix}
0 & 0 & 1\\
0 & 1 & 0\\
1 & 0 & 0
\end{pmatrix},\begin{pmatrix}
0 & 1 & 0\\
1 & 0 & 0\\
0 & 0 & 1
\end{pmatrix}, \begin{pmatrix}
1 & 0 & 0\\
0 & 0 & 1\\
0 & 1 & 0
\end{pmatrix}, \begin{pmatrix}
0 & 1 & 0\\
0 & 0 & 1\\
1 & 0 & 0
\end{pmatrix},
\begin{pmatrix}
0 & 0 & 1\\
1 & 0 & 0\\
0 & 1 & 0
\end{pmatrix}.
\]

\end{solucion}


\newpage

\begin{ejercicio}{7.2.4}
Let $M_{τ}$ be the matrix of the linear transformation taking $x_1,\dots , x_n$ to $x_{τ(1)},\dots , x_{τ(n)}$.
This means that if $e_1, \dots, e_n$ is the standard basis of $k^n$, then $M_τ \cdot(\sum_j x_je_j) =\sum_j x_{τ(j)}e_j$.
\begin{enumerate}[a.]
\item Show that $M_τ \cdot e_{τ(i)} = e_i$. Hint: Observe that
$\sum_j x_je_j =\sum_j x_{τ(j)}e_{τ(j)}$.
\item Prove that the $τ(i)$-th column of $M_τ$ is the $i$-th column of the identity matrix.
\item Prove that $M_τ \cdot M_ν = M_{τν}$, where $τν$ is the permutation taking $i$ to $τ(ν(i))$.
\end{enumerate}

\end{ejercicio}
\begin{solucion}\
\begin{enumerate}[a.]
\item Atendiendo a la observación, se tiene por definición y linealidad que $$ \sum_j x_{\tau(j)}M_{\tau}(e_{\tau(j)})= M_τ \cdot(\sum_j x_{τ(j)}e_{τ(j)})=M_τ \cdot(
\sum_j x_je_j) =\sum_j x_{τ(j)}e_j$$
Dado que las coordenadas son distintas para cada vector y son únicas, en este caso hay una correspondencia biunívoca entre las coordenada y el vector, por lo que necesariamente se tiene el resultado. 
\item Fila a fila, tenemos que $(M_τ)_i(x_1,\dots, x_n)'=x_{\tau(i)}=Id_{\tau(i)}(x_1\dots, x_n)$, de donde se deduce el resultado.  
\item Directamente comprobamos $$M_τ \cdot( M_ν \cdot(\sum_j x_je_j))=M_τ \cdot(\sum_j x_{\nu(j)}e_j)=\sum_j x_{\tau(\nu(j))}e_j= M_{ντ}\cdot(\sum_j x_je_j)$$
\end{enumerate}
\end{solucion}
\newpage

\begin{ejercicio}{7.2.5}
Consider a cube in $\R^3$ centered at the origin whose edges have length $2$ and are parallel to the coordinate axes.
\begin{enumerate}[a.]
\item Show that there are finitely many rotations of $\R^3$ about the origin which take the cube to itself and show that these rotations are closed under composition.
Taking the matrices representing these rotations, we get a finite matrix group $G \subseteq GL(3, \R^3)$.
\item Show that $G$ has $24$ elements.
Hint: Every rotation is a rotation about a line through the origin.
So you first need to identify the “lines of symmetry” of the cube. 
\item Write down the matrix of the element of $G$ corresponding to the $120^{\circ}$ counterclockwise rotation of the cube about the diagonal connecting the vertices $(-1, -1, -1)$ and $(1, 1, 1)$. 
\item Write down the matrix of the element of $G$ corresponding to the $90^{\circ}$ counterclockwise rotation about the $z$-axis.
\item Argue geometrically that $G$ is generated by the two matrices from parts (c) and (d).
\end{enumerate}
\end{ejercicio}
\begin{solucion}
\begin{enumerate}[a.]
\item Está claro que una rotación de $\R^3$ que deje invariante al cubo viene descrito por una permutación de sus $8$ vértices, luego $G \subseteq S_8$ y $G$ es finito.
Por supuesto, no toda permutación de $S_8$ vale.
Que la composición de rotaciones es rotación es consecuencia del Teorema de rotación de Euler\footnote{\url{https://en.wikipedia.org/wiki/Euler\%27s_rotation_theorem\#Euler's_theorem_(1776)}}.

\item En lugar de considerar los vértices, pensamos en las $4$ diagonales del cubo (no las diagonales de cara).
Las rotaciones del cubo son precisamente rotaciones de estas $4$ diagonales, luego $G \cong S_4$ y tiene $4!=24$ elementos.

\item Pensando en como se rotan los ejes bajo esta rotación queda claro que:
\[ A = \begin{pmatrix}0 & 1 & 0\\0 & 0 & 1\\1 & 0 & 0\end{pmatrix} \]

\item Pensando en como se rotan los ejes bajo esta rotación queda claro que:
\[ B = \begin{pmatrix}0 & -1 & 0\\1 & 0 & 0\\0 & 0 & 1\end{pmatrix} \]

\end{enumerate}
\end{solucion}
\newpage

\begin{ejercicio}{7.2.6}
In this exercise, we will use geometric methods to find some invariants of the rotation group $G$ of the cube (from Exercise \ref{ejer:7.2.5}). 
\begin{enumerate}[a.]
\item Explain why $x^2 + y^2 + z^2 \in \R[x, y, z]^G$.
Hint: Think geometrically in terms of distance to the origin.
\item Argue geometrically that the union of the three coordinate planes $\V(xyz)$ is invariant under $G$.
\item Show that $\I(\V(xyz)) = \gene{xyz}$ and conclude that if $f = xyz$, then for each $A \in G$, we have $f(A\cdot x) = axyz$ for some real number a.
\item Show that $f = xyz$ satisfies $f (A \cdot x) = \pm xyz$ for all $A \in G$ and conclude that $x^2y^2z^2 \in k[x, y, z]^G$.
Hint: Use part (c) and the fact that $A^m = I_3$ for some positive integer $m$.
\item Use similar methods to show that the polynomials
\[ \left((x+y+z)(x+y-z)(x-y+z)(x-y-z)\right)^2,\ \left((x^2-y^2)(x^2-z^2)(y^2-z^2)\right)^2 \]
are in $k[x,y,z]^G$.
Hint: The plane $x+y+z = 0$ is perpendicular to one of the diagonals of the cube.
\end{enumerate}
\end{ejercicio}
\begin{solucion}
\begin{enumerate}[a.]
\item Todas las rotaciones de $G$ son isometrías y dejan el centro fijo, por lo que para todo punto $(x,y,z)$, la distancia al centro es invariante por las rotaciones.
Entonces $\sqrt{x^2+y^2+z^2}$ es invariante por $G$, luego $x^2+y^2+z^2 \in \R[,y,z]^G$.

\item Como $G$ está generada por matrices que dejan invariante la unión de las rectas coordenadas, la unión de los planos coordenados son invariantes por $G$.

\item $\gene{xyz}$ es radical, pues no puede estar generado por un polinomio de grado menor:
\[ \I(\V(xyz)) = \sqrt{\gene{xyz}} = \gene{xyz} \]

Sea $g(\mathbf{x}) = f(A \cdot \mathbf{x})$.
Para todo punto $(x,y,z) \in \V(xyz)$, $A \cdot x$ también están en $\V(xyz)$, luego $g(\mathbf{x})$ se anula en $\V(xyz)$.
Entonces $g \in \I(\V(xyz)) = \gene{xyz}$, luego $g = hxyz$ para algún polinomio $h$.
Sin embargo, $g$ debe tener el mismo grado que $f$, luego $h=a$ para algún número real $a$.

\item El argumento anterior se puede generalizar para llegar a que $f(A^m \cdot \mathbf{x}) = a f(A^{m-1}\cdot \mathbf{x}) = a^m xyz$.
Para algún entero positivo $m$ se tiene que $A^M = I_3$, luego:
\[ xyz = f(I_3 \cdot \mathbf{x}) = f(A^m \cdot \mathbf{x}) = a^m xyz \]
Luego $a^m = 1$, entonces $a = \pm 1$.
Entonces, para $g = x^2y^2z^2 = f\cdot f$ tenemos que:
\[ g(A \cdot \mathbf{x}) = f(A \cdot x)\cdot f(A \cdot x) = (\pm xyz)^2 = x^2y^2z^2 = g(\mathbf{x})\]

\item Sin hacer.
\end{enumerate}

\end{solucion}
\newpage

\begin{ejercicio}{7.2.7}

\end{ejercicio}
\begin{solucion}

\end{solucion}
\newpage

\begin{ejercicio}{7.2.10}
Prove Proposition 9.
\end{ejercicio}
\begin{solucion}
\emph{If $G ⊆ GL(n, k)$ is a finite matrix group, then the set $k[x_1, \dots , x_n]^G$
is closed under addition and multiplication and contains the constant polynomials.}


Que contiene a los polinomios constantes es trivial. Si $f$ y $g$ son invariantes por $G$, entonces $f(x)+g(x)=f(Ax)+g(Ax)$ para toda $A\in G$ luego la suma es también invariante. Similarmente para la multiplicación, $f(x)g(x)=f(Ax)g(Ax)$. 
\end{solucion}

\newpage

\begin{ejercicio}{7.2.11}
Let $G ⊆ GL(n, k)$ be a finite matrix group. Then a polynomial
$f ∈ k[x_1, \dots , x_n]$ is invariant under $G$ if and only if its homogeneous components
are invariant.
\end{ejercicio}
\begin{solucion}
Si las componentes homogéneas son invariantes entonces $f$ es invariante por linealidad. Recíprocamente, al evaluar $f(Ax)$ no se modifica el grado de ningún monomio por ser una transformación lineal, de modo que para cada $m$, $f_m(Ax)=\sum_j c_jf_{m,j}(x)$ con $c_j\in k$. Pero para que $f(Ax)=f(x)$ debe cumplirse que $f(Ax)=\sum_m f_m(Ax)=\sum_m\sum_j c_jf_{m,j}(x)=f(x)=\sum_m\sum_j f_{m,j}$ por lo que necesariamente $c_j=1$ para todo $j$ y hemos terminado.

\end{solucion}

\newpage

\begin{ejercicio}{7.2.12}
In Example 13, we studied polynomials $f ∈ k[x, y]$ with the property that $f (x, y) =
f (−x,−y)$. If $f =\sum_{i,j} a_{ij}x^iy^j$, show that the above condition is equivalent to $a_{ij} = 0$
whenever $i + j$ is odd.
\end{ejercicio}
\begin{solucion}
Que $f(x,y)=f(-x,-y)$ significa
\[
\sum_{i,j} a_{ij}x^iy^j=\sum_{i,j} a_{ij}(-x)^i(-y)^j=\sum_{i,j} a_{ij}(-1)^ix^i(-1)^jy^j=\sum_{i,j} (-1)^{i+j}a_{ij}x^iy^j
\]
con lo que $a_{ij}=(-1)^{i+j}a_{ij}$, de donde se deduce el resultado. 
\end{solucion}
\newpage
\begin{ejercicio}{7.2.13}
In Example 13, we discovered the algebraic relation $x^2\cdot y^2 = (xy)^2$ between the invariants
$x^2$, $y^2$, and $xy$. We want to show that this is essentially the only relation. More precisely,
suppose that we have a polynomial $g(u, v, w) ∈ k[u, v, w]$ such that $g(x^2, y^2, xy) = 0$.
We want to prove that $g(u, v, w)$ is a multiple in $k[u, v, w]$ of $uv − w^2$ (which is the
polynomial corresponding to the above relation).
\begin{enumerate}[a.]
\item If we divide $g$ by $uv − w^2$ using lex order with $u > v > w$, show that the remainder
can be written in the form $uA(u, w) + vB(v, w) + C(w)$.
\item Show that a polynomial $r = uA(u, w) + vB(v, w) + C(w)$ satisfies $r(x^2, y^2, xy) = 0$
if and only if $r = 0$.
\end{enumerate}
\end{ejercicio}
\begin{solucion}
\begin{enumerate}[a.]
\item[]
\item Distingamos casos. Los monomios de $g$ son de la forma $u^av^bw^c$. Sabemos que si $f-g \in I$, entonces tienen el mismo resto al dividir por una base de Gröbner. Consideremos $I=\gene{uv-w^2}$. Si $a\geq 1$ y $b=0$ o al contrario, entonces es claro que tenemos el resultado. Si $a> b \geq 1$ entonces, $u^av^bw^c -  u^{a-b}w^{2b+c} = u^{a-b}w^c((uv)^{b}-w^{2b})$. Sabemos que 
$$
1-x^n = (1-x)\sum_{k=0}^{n-1}x^k \Rightarrow 1 - \left(\frac{w^2}{uv}\right)^b = (1-\frac{w^2}{uv})\sum_{k=0}^{b-1}\left(\frac{w^2}{uv}\right)^k 
$$
$$
(uv)^b - w^{2b} = (uv-w^2)\sum_{k=0}^{b-1}w^{k}(uv)^{b-1-k}
$$
Por tanto, $u^av^bw^c -  u^{a-b}w^{2b+c}\in I$ y se tiene el resultado. Análogamente se comprueba para $b>a\geq 1$.
\item Escribimos $uA(u,w) = \sum_{i,j} a_{ij} u^{i+1}w^j$, $uB(v,w) = \sum_{h,g} b_{hg} v^{h+1}w^g$ y $C(w)=\sum_{k}c_kw^k$. Sustituyendo en la suma obtenemos
$$
r(x^2,y^2,xy) = \sum_{i,j} a_{ij} y^{2i+2+j}x^j + \sum_{h,g} b_{hg} x^{2h+2+g}y^g+ \sum_{k}c_kx^ky^k
$$ 
Si tratamos de agrupar las potencias, observamos lo siguiente. En los primeros dos sumatorios no hay potencias de la forma $x^kw^k$, que son los que están en la tercera. Por tanto, $c_k = 0$ para todo $k$. Ahora bien, si $2i+2+j = g$ entonces no puede darse $2h+2+g=j$. Sumando ambas ecuaciones obtenemos $2i+2h + 4 = 0$. Como $i,h\geq 0$ no puede ser. Por tanto, como los coeficientes se igualan individualmente a $0$, tenemos que $r$ es el polinomio $0$.
\end{enumerate}
\end{solucion}
\newpage
\begin{ejercicio}{7.2.14}
Consider the finite matrix group $C_4 ⊆ GL(2,\C)$ generated by
$$A =\begin{pmatrix}
i & 0\\
0 & -i
\end{pmatrix}
∈ GL(2,\C).$$
\begin{enumerate}[a.]
\item Prove that $C_4$ is cyclic of order 4.
\item Use the method of Example 13 to determine $\C[x, y]^{C_4}$.
\item Is there an algebraic relation between the invariants you found in part (b)? Can you
give an example to show how uniqueness fails?
\item Use the method of Exercise 13 to show that the relation found in part (c) is the only
relation between the invariants.
\end{enumerate}
\end{ejercicio}
\begin{solucion}
\begin{enumerate}[a.]
\item[]
\item
\[
A^2=\begin{pmatrix}
-1 & 0\\
0 & -1
\end{pmatrix} \qquad A^3=\begin{pmatrix}
-i & 0\\
0 & i
\end{pmatrix} \qquad A^4=I
\]
\item Tenemos que
$$
R_{C_4}(f)=\frac{1}{4}(f(x,y)+f(xi,-yi)+f(-x,-y)+f(-ix,iy))
$$
Por el Teorema 5 de 7.3 tenemos
\begin{align*}
R_{C_4}(x)&=\frac{1}{4}(x+xi-x-xi) = 0  & 
R_{C_4}(x^2)&=\frac{1}{4}(x^2-x^2+x^2+x^2) = 0\\
R_{C_4}(x^3) &=\frac{1}{4}(x^3-x^3i-x^3+x^3i)=0 &
R_{C_4}(x^4) &=\frac{1}{4}(x^4+x^4+x^4+x^4)=x^4\\
R_{C_4}(y)&=\frac{1}{4}(y-yi-y+yi) = 0 &
R_{C_4}(y^2)&=\frac{1}{4}(y^2-y^2+y^2-y^2) = 0\\
R_{C_4}(y^3) &=\frac{1}{4}(y^3+y^3i+y^3-y^3i)=0&
R_{C_4}(y^4) &=\frac{1}{4}(y^4+y^4+y^4+y^4)=y^4\\
R_{C_4}(xy)&=\frac{1}{4}(xy+xy+xy +xy) = xy & 
R_{C_4}(xy^2)&=\frac{1}{4}(xy^2-xy^2i-xy^2+xy^2i) = 0\\
R_{C_4}(xy^3)&=\frac{1}{4}(xy^3+xy^3+xy^3+xy^3) = xy^3 & 
R_{C_4}(x^2y)&=\frac{1}{4}(x^2y+x^2yi -x^2y-x^2yi)=0 \\
R_{C_4}(x^3y)&=\frac{1}{4}(x^3y+x^3y+x^3y+x^3y)=x^3y &
R_{C_4}(x^2y^2)&=\frac{1}{4}(x^2y^2+x^2y^2+x^2y^2+x^2y^2)=x^2y^2
\end{align*}
Por tanto $k[x,y]^{G_4}=k[x^4,y^4,xy,xy^3,x^3y,x^2y^2]=k[x^4,y^4,xy,xy^3,x^3y]$

\item Se tiene 
$$
(x^3y)(xy^3)=(xy)^4=x^4y^4
$$
\item Por ejemplo, $(x^3y)(xy^3)x^4y^4 = (xy)^8$.
\end{enumerate}
\end{solucion}

\newpage
\begin{ejercicio}{7.2.15}
Consider
\[
V_4 =\left\{\pm\begin{pmatrix}
1 & 0\\
0 & 1
\end{pmatrix},\pm\begin{pmatrix}
0 & 1\\
1 & 0
\end{pmatrix}\right\}\subseteq GL(2, k)
\]
\begin{enumerate}[a.]
\item Show that $V_4$ is a finite matrix group of order 4.
\item Determine $k[x, y]^{V_4}$.
\item Show that any invariant can be written uniquely in terms of the generating invariants
you found in part (b).
\end{enumerate}
\end{ejercicio}
\begin{solucion}\
\begin{enumerate}[a.]
\item Trivial.
\item En el Ejemplo 12 aparece calculado $k[x, y]^{V_4}=k[x^2,y^2]$. 
\item Utilizaremos la Proposición 1 de la sección 7.4. Basta ver que $I_F = \{0\}$. Comprobamos qué ocurre si $f(x^2,y^2)=0$. 
$$
\sum_{i,j}c_{ij} x^{2i}y^{2j} = 0 \Rightarrow c_{ij} = 0 \; \forall i,j \Rightarrow f = 0
$$
\end{enumerate}
\end{solucion}

\end{document}

