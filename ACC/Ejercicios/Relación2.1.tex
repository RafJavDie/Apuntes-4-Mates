\documentclass[twoside]{article}
\usepackage{../../estilo-ejercicios}

%--------------------------------------------------------
\begin{document}

\title{Ejercicios de Ideals, Varieties, and Algorithms (4ª Edición)}
\author{Diego Pedraza López, Javier Aguilar Martín, Rafael González López}
\maketitle

\begin{ejercicio}{2.1.1}
Determine whether the given polynomial is in the given ideal $I \subseteq \R[x]$ using the method of Example 1.
\begin{enumerate}[a.]
\item $f(x) = x^2-3x+2$, $I = \gene{x-2}$.
\item $f(x) = x^5-4x+1$, $I = \gene{x^3-x^2+x}$.
\item $f(x) = x^2-4x+4$, $I = \gene{x^4-6x^2+12x-8,2x^3-10x^2+16x-8}$.
\item $f(x) = x^3-1$, $I = \gene{x^9-1,x^5+x^3-x^2-1}$.
\end{enumerate}
\end{ejercicio}
\begin{solucion}
\begin{enumerate}[a.]
\item $x^2-3x+2 = (x-1)(x-2) \Rightarrow x^2-3x+2 \in \gene{x-2}$.
\item $x^5-4x+1 = (x^2+x)(x^3-x^2+x)+(-x^2-4x+1) \Rightarrow x^5-4x+1 \notin \gene{x^3-x^2+x}$.
\item Buscamos un generador principal de $I$.
\[ x^4-6x^2+12x-8 = (x-\sqrt{5}+1)(x+\sqrt{5}+1)(x^2-2x+2) \]
\[ 2x^3-10x^2+16x-8 = 2(x-1)(x-2)^2 \]
Entonces $\gcd{x^4-6x^2+12x-8,2x^3-10x^2+16x-8}=1$, luego $I$ es el ideal total y $f \in I$.
\item Buscamos un generador principal de $I$.
\[ x^9-1 = (x-1)(x^2+x+1)(x^6+x^3+1) \]
\[ x^5+x^3-x^2-1 = (x^2+1)(x-1)(x^2+x+1) \]
Luego $\gcd(x^9-1,x^5+x^3-x^2-1) = (x-1)(x^2+x+1) = x^3-1$.
Entonces $f \in I$ evidentemente.
\end{enumerate}
\end{solucion}

\newpage

\begin{ejercicio}{2.1.2}
Find parametrization of the affine varieties defined by the following sets of equations.
\begin{enumerate}[a.]
\item In $\R^3$ or $\C^3$:
\begin{align*}
2x + 3y-z & = 9\\
x-y & = 1\\
3x + 7y - 2z & = 17
\end{align*}
\item In $\R^4$ or $\C^4$:
\begin{align*}
x_1 + x_2 - x_3 - x_4 & = 0\\
x_1 - x_2 + x_3 & = 0
\end{align*}
\item In $\R^3$ or $\C^3$:
\begin{align*}
y - x^3 & = 0\\
z - x^5 & = 0
\end{align*}
\end{enumerate}
\end{ejercicio}
\begin{solucion}
\begin{enumerate}[a.]
\item Tenemos que:
\[\begin{pmatrix}2 & 3 & -1 & -9\\1 & -1 & 0 & -1\\3 & 7 & -2 & -17\end{pmatrix} \to \begin{pmatrix}1 & -1 & 0 & -1\\0 & 5 & -1 & -7\\0 & 10 & -2 & -14\end{pmatrix} \to \begin{pmatrix}1 & -1 & 0 & -1\\0 & 5 & -1 & -7\\0 & 0 & 0 & 0\end{pmatrix} \to \begin{pmatrix}1 & 0 & -1/5 & -12/5\\0 & 1 & -1/5 & -7/5\\0 & 0 & 0 & 0\end{pmatrix}\]
Luego tomando $x_3 = t$, $x_2 = \frac{1}{5}(t+7)$ y $x_1 = -\frac{1}{5}(t+12)$.
\item Tenemos que:
\[\begin{pmatrix}1 & 1 & -1 & -1\\1 & -1 & 1 & 0\end{pmatrix} \to \begin{pmatrix}1 & 1 & -1 & -1\\0 & -2 & 2 & 1\end{pmatrix} \to \begin{pmatrix}1 & 0 & 0 & -1/2\\0 & 1 & -1 & -1/2\end{pmatrix}\]
Luego tomamos $x_3 = t$, $x_4 = u$, $x_1 = u/2$ y $x_2 = t+u/2$.
\item Si $x=t$, entonces $y=t^3$ y $z=t^5$.
\end{enumerate}
\end{solucion}
\end{document}
