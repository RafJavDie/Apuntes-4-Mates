\documentclass[twoside]{article}
\usepackage{../../estilo-ejercicios}

%--------------------------------------------------------
\begin{document}

\title{Ejercicios de Ideals, Varieties, and Algorithms (4ª Edición)}
\author{Diego Pedraza López, Javier Aguilar Martín, Rafael González López}
\maketitle

\begin{ejercicio}{3.2.1}
Prove the Geometric Extension Theorem (Theorem 2) using the Extension Theorem and
Lemma 1.
\end{ejercicio}
\begin{solucion}
Por el Lema 1, $\pi_1(V)\subseteq\V(I_l)$, luego $\V(I_l)=\pi_1(V)\cup W$ para un cierto conjunto $W$. Por otro lado, dados $c_i$ como los del teorema de extensión, dado $a=(a_{l+1},\dots, a_n)\in \V(I_l)$, si $a\notin \V(c_1,\dots, c_s)$, entonces $a=\pi(\overline{a})$ con $\overline{a}=(a_1,\dots, a_l,a_{l+1},a_n)\in V$, por lo que si $a\notin \pi_1(V)$ es porque $a\in \V(c_1,\dots, c_s)$, luego $a\in \V(I_l)\cap \V(c_1,\dots, c_s)$. Con esto, $W= \V(I_l)\cap \V(c_1,\dots, c_s)$, lo cual prueba el teorema. 
\end{solucion}

\newpage

\begin{ejercicio}{3.2.2}
In example (2), verify carefully that 
$\gene{(y−z)x^2+xy−1, (y−z)x^2+xz−1} =\gene{ 
xy−1, xz−1}$.
Also check that $y − z$ vanishes at all partial solutions in $V(I_1)$.
\end{ejercicio}
\begin{solucion}

\end{solucion}


\newpage

\begin{ejercicio}{3.2.3}
In this problem, we will prove part (ii) of Theorem 3 in the special case when $I =\gene{ f_1, f_2, f_3}$, where
\[f_1 = yx^3 + x^2,\]
\[f_2 = y^3x^2 + y^2,\]
\[f_3 = yx^4 + x^2 + y^2.\]
\begin{enumerate}[a.]
\item Find a Gröbner basis for $I$ and show that $I_1 = 
\gene{y^2}$.
\item Let $c_i$ be the coefficient of the highest power of $x$ in $f_i$. Then explain why $W =
\V(c_1, c_2, c_3) ∩ \V(I_1)$ does not satisfy part (ii) of Theorem 3.
\item Let $\tilde{I} = 
 \gene{f_1, f_2, f_3, c_1, c_2, c_3}$. Show that $\V(I) = \V(\tilde{I})$ and $\V(I_1) = \V(\tilde{I}_1)$.
\item Let $x^{N_i}$ be the highest power of $x$ appearing in $f_i$ and set $\tilde{f}_i = f_i − c_ix^{N_i}$ . Show that
$\tilde{I} = 
\gene{ \tilde{f}_1, \tilde{f}_2, \tilde{f}_3, c_1, c_2, c_3}$.
\item Repeat part (b) for $\tilde{I}$ using the generators from part (d) to find $\widetilde{W} \subsetneq \V(I_1)$ that satisfies
part (ii) of Theorem 3.
\end{enumerate}
\end{ejercicio}
\begin{solucion}
 
\end{solucion}


\newpage

\begin{ejercicio}{3.2.4}
To see how the Closure Theorem can fail over $\R$, consider the ideal
$I = 
\gene{x^2 + y^2 + z^2 + 2, 3x^2 + 4y^2 + 4z^2 + 5}$.
Let $V = \V(I)$, and let $π_1$ be the projection taking $(x, y, z)$ to $(y, z)$.
\begin{enumerate}[a.]
\item Working over $\C$, prove that $\V(I_1) = π_1(V)$.
\item Working over $\R$, prove that $V = ∅$ and that $\V(I_1)$ is infinite. Thus, $\V(I_1)$ may be much
larger than the smallest variety containing $π_1(V)$ when the field is not algebraically
closed.
\end{enumerate}
\end{ejercicio}
\begin{solucion}

\end{solucion}


\newpage

\begin{ejercicio}{3.2.5}
Suppose that $I ⊆ \C[x, y]$ is an ideal such that $I_1 \neq \{0\}$. Prove that $\V(I_1) = π_1(V)$,
where $V = \V(I)$ and $π_1$ is the projection onto the $y$-axis. Hint: Use part (i) of the Closure
Theorem. Also, the only varieties contained in $\C$ are either $\C$ or finite subsets of $\C$.
\end{ejercicio}
\begin{solucion}

\end{solucion}




\end{document}
