\documentclass[twoside]{article}
\usepackage{../../estilo-ejercicios}

%--------------------------------------------------------
\begin{document}

\title{Ejercicios de Ideals, Varieties, and Algorithms (4ª Edición)}
\author{Diego Pedraza López, Javier Aguilar Martín, Rafael González López}
\maketitle

\begin{ejercicio}{3.2.1}
Prove the Geometric Extension Theorem (Theorem 2) using the Extension Theorem and
Lemma 1.
\end{ejercicio}
\begin{solucion}
Probémoslo por doble contención. Por el Lema 1, $\pi_1(V)\subseteq\V(I_1)$, luego es claro que
$$
\pi_1(V)\cup (\V(I_1)\cap \V(c_1,\dots, c_s))\subset \V(I_1)
$$ Por otro lado, sean los $c_i$ como los del Teorema de Extensión y $a=(a_{l+1},\dots, a_n)\in \V(I_1)$. Si $a\in \V(c_1,\dotsc, c_s)$, y por tanto en $\V(c_1,\dotsc,c_s)\cap I_1$, o bien $a \notin \V(c_1,\dotsc,c_s)$. En el segundo caso, por el Teorema de Extensión, $a$ puede extenderse a una solución en $V$, por lo que $a\in \pi_1(V)$.
\end{solucion}

\newpage

\begin{ejercicio}{3.2.2}
In example (2), verify carefully that 
$$\gene{(y−z)x^2+xy−1, (y−z)x^2+xz−1} =\gene{ 
xy−1, xz−1}$$
Also check that $y − z$ vanishes at all partial solutions in $V(I_1)$.
\end{ejercicio}
\begin{solucion}
Tenemos que $x(y-z)\in \gene{xy−1, xz−1}$ pues 
$$x(y-z)=-z(xy-1)+y(xz-1)$$
Se sigue fácilmente que los polinomios
$$\{(y−z)x^2+xy−1, (y−z)x^2+xz−1\}\subset \gene{xy−1, xz−1} $$
Teniéndose trivialmente la contención del ideal. Análogamente
$$
(y−z)x^2+xy−1- (y−z)x^2-xz+1 = x(y-z)
$$
Multiplicando por $x$ y restando a los generadores se sigue que 
$$
\gene{xy-1,xz-1}\subset \gene{(y−z)x^2+xy−1, (y−z)x^2+xz−1}
$$
Calculando la base de Gröbner $G=\{xz-1,y-z\}$ es claro que $I_1=\gene{y-z}$ y el resultado es trivial.
\end{solucion}


\newpage

\begin{ejercicio}{3.2.3}
In this problem, we will prove part (ii) of Theorem 3 in the special case when $I =\gene{ f_1, f_2, f_3}$, where
\[f_1 = yx^3 + x^2,\]
\[f_2 = y^3x^2 + y^2,\]
\[f_3 = yx^4 + x^2 + y^2.\]
\begin{enumerate}[a.]
\item Find a Gröbner basis for $I$ and show that $I_1 = 
\gene{y^2}$.
\item Let $c_i$ be the coefficient of the highest power of $x$ in $f_i$. Then explain why $W =
\V(c_1, c_2, c_3) ∩ \V(I_1)$ does not satisfy part (ii) of Theorem 3.
\item Let $\tilde{I} = 
 \gene{f_1, f_2, f_3, c_1, c_2, c_3}$. Show that $\V(I) = \V(\tilde{I})$ and $\V(I_1) = \V(\tilde{I}_1)$.
\item Let $x^{N_i}$ be the highest power of $x$ appearing in $f_i$ and set $\tilde{f}_i = f_i − c_ix^{N_i}$ . Show that
$\tilde{I} = 
\gene{ \tilde{f}_1, \tilde{f}_2, \tilde{f}_3, c_1, c_2, c_3}$.
\item Repeat part (b) for $\tilde{I}$ using the generators from part (d) to find $\widetilde{W} \subsetneq \V(I_1)$ that satisfies part (ii) of Theorem 3.
\end{enumerate}
\end{ejercicio}
\begin{solucion}
\begin{enumerate}[a.]
\item[] 
\item Consideremos el ideal generado y calculemos con SAGE una base de Gröbner y obtenemos 
$G=\{x^2,y^2\}$
de donde se sigue el resultado por el Teorema de Eliminación.
\item Dichos coeficientes son $\{c_1,c_2,c_3\}=\{y,y^3,y\}$. Por tanto, $\V (c_1,c_2,c_3) \cap \V(I_1) = \{0\}$.
\item Es claro que una base de Gröbner de este nuevo ideal será $G=\{x^2,y\}$, de donde se deducen las igualdades.
\item La igualdad es una igualdad básica de teoría de ideales que no resiste mayor análisis, y los generadores se reducen a la base de Gröbner anterior.
\item O hay una pájara mental de cuidado o no se puede hacer.
\end{enumerate}
\end{solucion}


\newpage

\begin{ejercicio}{3.2.4}
To see how the Closure Theorem can fail over $\R$, consider the ideal
$I = 
\gene{x^2 + y^2 + z^2 + 2, 3x^2 + 4y^2 + 4z^2 + 5}$.
Let $V = \V(I)$, and let $π_1$ be the projection taking $(x, y, z)$ to $(y, z)$.
\begin{enumerate}[a.]
\item Working over $\C$, prove that $\V(I_1) = π_1(V)$.
\item Working over $\R$, prove that $V = ∅$ and that $\V(I_1)$ is infinite. Thus, $\V(I_1)$ may be much
larger than the smallest variety containing $π_1(V)$ when the field is not algebraically
closed.
\end{enumerate}
\end{ejercicio}
\begin{solucion}\
\begin{enumerate}[a.]
\item Como los $c_i$ son constantes, se tiene inmediatamente por el Corolario 4.
\item $V=\emptyset$ porque $x^2 + y^2 + z^2 + 2=0$ no tiene soluciones reales. Calculamos una base de Gröbner para $I_1$ sobre $\R$ y obtenemos
\[
G_1=\{y^2 + z^2 - 1\}
\]
Luego $\V(I_1)$ son los puntos reales de la circunferencia $y^2+z^2=1$, que son infinitos. 
\end{enumerate}
\end{solucion}


\newpage

\begin{ejercicio}{3.2.5}
Suppose that $I ⊆ \C[x, y]$ is an ideal such that $I_1 \neq \{0\}$. Prove that $\V(I_1) = π_1(V)$,
where $V = \V(I)$ and $π_1$ is the projection onto the $y$-axis. Hint: Use part (i) of the Closure
Theorem. Also, the only varieties contained in $\C$ are either $\C$ or finite subsets of $\C$.
\end{ejercicio}
\begin{solucion}
Si $I_1 \neq \{0\}$ entonce $\exists f \in k[y]$ tal que $f \in I_1$. Dado que $\gene{f}\subset I_1$, $\V(I_1) \subset \V(f)$, luego es finito, pues $f$ se anula en $n$ puntos. Ahora bien, si $\pi_1(V)$ es un conjunto finito de puntos es, en particular, una variedad en $\C$. Por el Teorema 3, $\V(I_1)$ es la menor variedad que contiene a $\pi_1(V)$, pero por ser variedad, la menor variedad que la contiene es ella misma.


\end{solucion}




\end{document}
