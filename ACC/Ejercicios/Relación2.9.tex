\documentclass[twoside]{article}
\usepackage{../../estilo-ejercicios}
\newcommand{\lex}{<_{lex}}
\newcommand{\grlex}{<_{grlex}}
\newcommand{\grevlex}{<_{grevlex}}

\newcommand{\PhantC}{\phantom{\colon}}%
\newcommand{\CenterInCol}[1]{\multicolumn{1}{c}{#1}}%

%--------------------------------------------------------
\begin{document}

\title{Ejercicios de Ideals, Varieties, and Algorithms (4ª Edición)}
\author{Diego Pedraza López, Javier Aguilar Martín, Rafael González López}
\maketitle

\begin{ejercicio}{2.9.1}
Let $S = (c_1, \dots , c_s)$ and $T = (d_1, \dots , d_s) ∈ (k[x_1, \dots , x_n])^s$ be syzygies on the leading
terms of $F = (f_1, \dots , f_s)$.
\begin{enumerate}[a.]
\item Show that $S + T = (c_1 + d_1, . \dots , c_s + d_s)$ is also a syzygy.
\item Show that if $g ∈ k[x_1,\dots , x_n]$, then $g  S = (gc_1, \dots , gc_s)$ is also a syzygy.
\end{enumerate}
\end{ejercicio}

\begin{solucion}

\end{solucion}

\newpage

\begin{ejercicio}{2.9.2}
Given any $G = (g_1, \dots , g_s) ∈ (k[x_1, \dots , x_n])_s$ , we can define a syzygy on $G$ to be an $s$-tuple
$S = (h_1,\dots , h_s) ∈ (k[x_1, \dots , xn_])_s$ such that
$\sum_i h_ig_i = 0$. [Note that the syzygies
we studied in the text are syzygies on $LT(G) = (LT(g_1), \dots , LT(g_s))$.]
\begin{enumerate}[a.]
\item Show that if $G = (x^2 − y, xy − z, y^2 − xz)$, then $(z,−y, x)$ defines a syzygy on $G$.
\item Find another syzygy on $G$ from part (a).
\item Show that if $S, T$ are syzygies on $G$, and $g ∈ k[x_1,\dots , x_n]$, then $S+T$ and $gS$ are also
syzygies on $G$.

\end{enumerate}
\end{ejercicio}
\begin{solucion}

\end{solucion}
\newpage

\begin{ejercicio}{2.9.3}
Let $M$ be an $m × (m + 1)$ matrix of polynomials in $k[x_1, \dots , x_n]$. Let $I$ be the ideal generated
by the determinants of all the $m × m$ submatrices of $M$ (such ideals are examples
of determinantal ideals).
\begin{enumerate}[a.]
\item Find a $2×3$ matrix $M$ such that the associated determinantal ideal of $2×2$ submatrices
is the ideal with generators $G$ as in part (a) of Exercise \ref{ejer:2.9.2}.
\item Explain the syzygy given in part (a) of Exercise \ref{eje:2.9.2} in terms of your matrix.
\item Give a general way to produce syzygies on the generators of a determinantal ideal.
Hint: Find ways to produce $(m + 1) × (m + 1)$ matrices containing $M$, whose determinants
are automatically zero.
\end{enumerate}
\end{ejercicio}
\begin{solucion}

\end{solucion}

\newpage

\begin{ejercicio}{2.9.4}
Prove that the syzygy $S_{ij}$ defined in (1) is homogeneous of multidegree $γ$.
\end{ejercicio}
\begin{solucion}

\end{solucion}
\newpage

\begin{ejercicio}{2.9.5}
Complete the proof of Lemma 4 by showing that the decomposition into homogeneous
components is unique. Hint: First show that if $S =
\sum_{α} S_{α}'$, where $S_{α}'$ has multidegree
$α$, then, for a fixed $i$, the $i$-th components of the $S_{α}'$ are either 0 or have multidegree equal
to $α − \textrm{multideg}( f_i)$ and, hence, give distinct terms as $α$ varies.
\end{ejercicio}
\begin{solucion}

\end{solucion}

\newpage

\begin{ejercicio}{2.9.6}
Suppose that $S$ is a homogeneous syzygy of multidegree $α$ in $S(G)$.
\begin{enumerate}[a.]
\item Prove that $S  G$ has multidegree $< α$.
\item Use part (a) to show that Corollary 7 follows from Theorem 6.
\end{enumerate}
\end{ejercicio}
\begin{solucion}

\end{solucion}

\newpage

\begin{ejercicio}{2.9.7}
Complete the proof of Proposition 8 by proving the formula expressing $S_{ij}$ in terms of
$S_{il}$ and $S_{jl}$.
\end{ejercicio}
\begin{solucion}\

\end{solucion}

\newpage

\begin{ejercicio}{2.9.8}
Let $G$ be a finite subset of $k[x_1, \dots , x_n]$ and let $f ∈ 
\gene{G}$. If $\overline{f}^G = r \neq 0$, then show that
$f →_{G'} 0$, where $G'
= G ∪ \{r\}$. This fact is used in the proof of Theorem 9.
\end{ejercicio}
\begin{solucion}


\end{solucion}

\newpage

\begin{ejercicio}{2.9.9}
In the proof of Theorem 9, we claimed that for every value of $B$, if $1 ≤ i < j ≤ t$ and
$(i, j) \not∈
B$, then condition (6) was true. To prove this, we needed to show that if the
claim held for $B$, then it held when $B$ changed to some $B'$. The case when $(i, j) \not∈
B'$ but
$(i, j) ∈ B$ was covered in the text. It remains to consider when $(i, j) \not∈B' ∪ B$. In this
case, prove that (6) holds for $B'$. Hint: Note that (6) holds for $B$. There are two cases
to consider, depending on whether $B'$ is bigger or smaller than $B$. In the latter situation,
$B'
= B \ \{(l,m)\}$ for some $(l,m) \neq (i, j)$.
\end{ejercicio}
\begin{solucion}

\end{solucion}

\newpage

\begin{ejercicio}{2.9.10}
In this exercise, we will study the ordering on the set $\{(i, j) | 1 ≤ i < j ≤ t\}$ described
in the proof of Theorem 9. Assume that $B = ∅$, and recall that $t$ is the length of $G$ when
the algorithm stops.
\begin{enumerate}[a.]
\item Show that any pair $(i, j)$ with $1 ≤ i < j ≤ t$ was a member of $B$ at some point during
the algorithm.
\item Use part (a) and $B = ∅$ to explain how we can order the set of all pairs according to
when a pair was removed from $B$.
\end{enumerate}
\end{ejercicio}
\begin{solucion}
\end{solucion}

\newpage

\begin{ejercicio}{2.9.11}
Consider $f:1 = x^3−2xy$ and $f_2 = x^2y−2y^2+x$ and use grlex order on $k[x, y]$. These polynomials
are taken from Example 1 of §7, where we followed Buchberger’s algorithm
to show how a Gröbner basis was produced. Redo this example using the algorithm of
Theorem 9 and, in particular, keep track of how many times you have to use the division
algorithm.
\end{ejercicio}
\begin{solucion}


\end{solucion}
\newpage

\begin{ejercicio}{2.9.12}
Consider the polynomials
$$x^{n+1} − yz^{n−1}w, xy^{n−1} − z^n, x^nz − y^nw,$$
and use grevlex order with $x > y > z > w$. Mora [see LAZARD (1983)] showed that the
reduced Gröbner basis contains the polynomial
$$z^{n^2+1} − y^{n^2}
w.$$
Prove that this is true when n is 3, 4, or 5. How big are the Gröbner bases?
\end{ejercicio}
\begin{solucion}
\end{solucion}

\newpage

\begin{ejercicio}{2.9.13}
In this exercise, we will look at some examples of how the term order can affect the
length of a Gröbner basis computation and the complexity of the answer.
\begin{enumerate}[a.]
\item Compute a Gröbner basis for $I = 
\gene{x^5 + y^4 + z^3 − 1, x^3 + y^2 + z^2 − 1}$ using lex and
grevlex orders with $x > y > z$. You will see that the Gröbner basis is much simpler
when using grevlex.
\item Compute a Gröbner basis for $I = 
\gene{x^5 + y^4 + z^3 − 1, x^3 + y^3 + z^2 − 1}$ using lex
and grevlex orders with $x > y > z$. This differs from the previous example by a
single exponent, but the Gröbner basis for lex order is significantly nastier (one of its
polynomials has 282 terms, total degree 25, and a largest coefficient of 170255391).
\item Let $I = 
\gene{x^4 − yz^2w, xy^2 − z^3, x^3z − y^3w}$ be the ideal generated by the polynomials
of Exercise \ref{ejer:2.9.12} with $n = 3$. Using lex and grevlex orders with $x > y > z > w$, show
that the resulting Gröbner bases are the same. So grevlex is not always better than
lex, but in practice, it is usually a good idea to use grevlex whenever possible.
\end{enumerate}
\end{ejercicio}
\begin{solucion}
\end{solucion}

\end{document}
