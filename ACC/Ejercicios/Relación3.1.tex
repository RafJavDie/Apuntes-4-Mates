\documentclass[twoside]{article}
\usepackage{../../estilo-ejercicios}

%--------------------------------------------------------
\begin{document}

\title{Ejercicios de Ideals, Varieties, and Algorithms (4ª Edición)}
\author{Diego Pedraza López, Javier Aguilar Martín, Rafael González López}
\maketitle

\begin{ejercicio}{3.1.1}
Let $I\subset k[x_1,\dotsc,x_n]$ be an ideal.
\begin{enumerate}[a.]
\item Prove that $I_l = I\cap k[x_{l+1},\dotsc,x_n]$ is an ideal of $k[x_{l+1},\dotsc,x_n]$.
\item Prove that ideal $I_{l+1} \subset k[x_{l+2},\dotsc,x_n]$ is the first elimination ideal $I_l \subset k[x_{l+1},\dotsc,x_n]$. This observation allows to use the Extension Theorem multiple times when eliminating more than one variable.
\end{enumerate}
\end{ejercicio}
\begin{solucion}
\begin{enumerate}[a.]
\item[]
\item Probemos que es un ideal
\begin{itemize}
\item Claramente $I_l \subset k[x_{l+1},\dotsc,x_n]$.
\item Si $f,g\in I_l$, entonces solo tienen términos en $x_{l+1},\dotsc,x_n$, luego la suma tendrá solo esos términos. Como $f+g\in I$ y $f+g\in k[x_{l+1},\dotsc,x_n]$ tenemos que $f+g\in I_l$.
\item Tomando $a\in k$ y $f \in I_l$, entonces $af$ solo tiene términos en $x_{l+1},\dotsc,x_n$, luego el producto tendrá solo esos términos. Se deduce, como en el apartado anterior, que $af\in I_l$.
\end{itemize}
\item Es inmediato si renombramos los índices de las variables y consideramos $I'=I_{l+1}$ que $I_1'=I_{l+1}$.
\end{enumerate}
\end{solucion}

\newpage

\begin{ejercicio}{3.1.2}
Consider the system of equations
\begin{align*}
x^2+2y^2&=3\\
x^2+xy+y^2&=3
\end{align*}
\begin{enumerate}[a.]
\item If $I$ is the ideal generated by these equations, find bases of $I ∩ k[x]$ and $I ∩ k[y]$.
\item Find all solutions of the equations.
\item Which of the solutions are rational, i.e., lie in $\Q^2$?
\item What is the smallest field $k$ containing $\Q$ such that all solutions lie in $k^2$?
\end{enumerate}
\end{ejercicio}
\begin{solucion}
\begin{enumerate}[a.]
\item[]
\item Consideremos $I=\gene{x^2+2y^2-3,x^2+xy+y^2-3}$. Claramente este ideal es igual a
$$
I=\gene{x^2+2y^2-3,xy-y^2}
$$
Calculamos una base de Gröbner para lex con $x>y$.
$$
S(f_1,f_2) = -y^2x + 2y^3-3y   = yf_2 + 3y^3-3y 
$$
Luego añadimos $y^3-y$ a nuestra base. Como $S(f_1,f_2)$ va a tener ahora resto 0, $LT(f_1)$ y $LT(f_3)$ son coprimos, solo tenemos que comprobar
$$
S(f_2,f_3) = xy - y^4 = f_2 + -yf_3
$$
Por tanto, $G=\{f_1,f_2,f_3\}$ es una base de Gröbner y por el Teorema 2 
$$I\cap k[y] = \gene{y^3-y}$$
Análogamente, calculamos una base de Gröbner para $y>x$ y obtenemos
$$
\{2y^2 + x^2 - 3, -2yx - x^2 + 3, -x^4 + 2x^2 - 3, -y^2 + yx, 6y +3x^3 - 9x\}
$$
Por lo que 
$$
I\cap k[x] = \{-x^4+2x^2-3\}
$$
\item Es claro que mejor eliminar $y$. Obtenemos $y(y^2-1)=0$, por lo que $y=0$ e $y=\pm 1$. Sustituimos en las otras 2 y obtenemos
$$
x^2 - 3 = 0 \qquad x^2-1 = x-1 = 0 \qquad x^2 -1 = -x +1 = 0
$$
De donde obtenemos las soluciones $(\pm \sqrt{3},0),(1,1),(-1,-1)$.
\item Tan solo está en $\Q^2$ las soluciones $(1,1)$ y $(-1,-1)$.
\item $k=Q(\sqrt{3}) = \{a+b\sqrt{3}\mid a,b\in \Q\}$.
\end{enumerate}
\end{solucion}


\newpage

\begin{ejercicio}{3.1.3}

\end{ejercicio}
\begin{solucion}
\begin{enumerate}[a.]
\end{enumerate}
\end{solucion}


\newpage

\begin{ejercicio}{3.1.4}

\end{ejercicio}
\begin{solucion}
\begin{enumerate}[a.]
\end{enumerate}
\end{solucion}


\newpage

\begin{ejercicio}{3.1.5}

\end{ejercicio}
\begin{solucion}
\begin{enumerate}[a.]
\end{enumerate}
\end{solucion}


\newpage

\begin{ejercicio}{3.1.6}

\end{ejercicio}
\begin{solucion}
\begin{enumerate}[a.]
\end{enumerate}
\end{solucion}


\newpage

\begin{ejercicio}{3.1.7}

\end{ejercicio}
\begin{solucion}
\begin{enumerate}[a.]
\end{enumerate}
\end{solucion}


\newpage

\begin{ejercicio}{3.1.8}

\end{ejercicio}
\begin{solucion}
\begin{enumerate}[a.]
\end{enumerate}
\end{solucion}


\newpage

\begin{ejercicio}{3.1.9}

\end{ejercicio}
\begin{solucion}
\begin{enumerate}[a.]
\end{enumerate}
\end{solucion}


\end{document}
