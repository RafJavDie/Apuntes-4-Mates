\documentclass[twoside]{article}
\usepackage{../../estilo-ejercicios}

%--------------------------------------------------------
\begin{document}

\title{Ejercicios de Ideals, Varieties, and Algorithms (4ª Edición)}
\author{Diego Pedraza López, Javier Aguilar Martín, Rafael González López}
\maketitle

\begin{ejercicio}{7.2.1}
If $A, B ∈ GL(n, k)$ are invertible matrices, show that $AB$ and $A^{−1}$ are also invertible.
\end{ejercicio}
\begin{solucion}
Si $A$ y $B$ son invertibles podemos considerar $A^{-1}$ y $B^{-1}$ y el producto $B^{-1}A^{-1}$, que es claramente la inversa de $AB$. La inversa de $A^{-1}$ es trivialmente $A$.
\end{solucion}

\newpage

\begin{ejercicio}{7.2.2}
Suppose that $A ∈ GL(n, k)$ satisfies $A^m = I_n$ for some positive integer. If $m$ is the
smallest such integer, then prove that the set $C_m = \{I_n, A, A^2,\dots , A^{m−1}\}$ has exactly $m$
elements and is closed under matrix multiplication.
\end{ejercicio}
\begin{solucion}
$C_m$ no puede tener más de $m$ elementos. Si tuviera menos esto implicaría que $A^k=A^l$ para algunos $0\leq k<l\leq m$. En ese caso, $I_n=A^{l-k}$, lo cual contradice que que $m$ sea el menor entero que verifica esa propiedad, por lo que hay exactamente $m$ elementos. Para ver que es cerrado para la multiplicación, basta ver que si $l+k> m$ entonces $A^lA^k\in C_m$, pues en cualquier otro caso es trivial. Dividiendo $l+k$ por $m$ obtendremos que $A^lA^k=A^{l+k}=A^{qm+r}=A^{qm}A^r=A^r$ con $r<m$, por lo que se tiene el resultado. 
\end{solucion}

\newpage

\begin{ejercicio}{7.2.3}
Write down the six permutation matrices in $GL(3, k)$.
\end{ejercicio}
\begin{solucion}
\[
\begin{pmatrix}
1 & 0 & 0\\
0 & 1 & 0\\
0 & 0 & 1
\end{pmatrix}, \begin{pmatrix}
0 & 0 & 1\\
0 & 1 & 0\\
1 & 0 & 0
\end{pmatrix},\begin{pmatrix}
0 & 1 & 0\\
1 & 0 & 0\\
0 & 0 & 1
\end{pmatrix}, \begin{pmatrix}
1 & 0 & 0\\
0 & 0 & 1\\
0 & 1 & 0
\end{pmatrix}, \begin{pmatrix}
0 & 1 & 0\\
0 & 0 & 1\\
1 & 0 & 0
\end{pmatrix},
\begin{pmatrix}
0 & 0 & 1\\
1 & 0 & 0\\
0 & 1 & 0
\end{pmatrix}.
\]

\end{solucion}
\newpage

\begin{ejercicio}{7.2.4}

\end{ejercicio}
\begin{solucion}

\end{solucion}

\newpage

\begin{ejercicio}{7.2.5}
Let $M_{τ}$ be the matrix of the linear transformation taking $x_1,\dots , x_n$ to $x_{τ(1)},\dots , x_{τ(n)}$.
This means that if $e_1, \dots, e_n$ is the standard basis of $k^n$, then $M_τ \cdot(\sum_j x_je_j) =\sum_j x_{τ(j)}e_j$.
\begin{enumerate}[a.]
\item Show that $M_τ \cdot e_{τ(i)} = e_i$. Hint: Observe that
$\sum_j x_je_j =\sum_j x_{τ(j)}e_{τ(j)}$.
\item Prove that the $τ(i)$-th column of $M_τ$ is the $i$-th column of the identity matrix.
\item Prove that $M_τ \cdot M_ν = M_{τν}$, where $τν$ is the permutation taking $i$ to $τ(ν(i))$.
\end{enumerate}

\end{ejercicio}
\begin{solucion}\
\begin{enumerate}[a.]
\item Atiendiendo a la observación, se tiene por definición que $M_τ \cdot(\sum_j x_{τ(j)}e_{τ(j)})=M_τ \cdot(
\sum_j x_je_j) =\sum_j x_{τ(j)}e_j$ y por otro lado $M_τ \cdot(\sum_j x_{τ(j)}e_{τ(j)})=\sum_j x_{τ(j)}e_{τ(j)}$. Por la unicidad de las coordenadas se tiene el resultado. 
\item Fila a fila, tenemos que $(M_τ)_i(x_1,\dots, x_n)'=x_{\tau(i)}=Id_{\tau(i)}(x_1\dots, x_n)$, de donde se deduce el resultado.  
\item Por un lado $M_{ντ}\cdot(\sum_j x_je_j) =\sum_j x_{\nu(τ(j))}e_j$ y por otro $M_τ (\cdot M_ν \cdot(\sum_j x_je_j))=M_τ \cdot(\sum_j x_{\nu(j)}e_j)=\sum_j x_{\tau(\nu(j))}e_j$.
\end{enumerate}
\end{solucion}
\newpage

\begin{ejercicio}{7.2.6}

\end{ejercicio}
\begin{solucion}

\end{solucion}
\newpage

\begin{ejercicio}{7.2.7}

\end{ejercicio}
\begin{solucion}

\end{solucion}
\newpage

\begin{ejercicio}{7.2.10}
Prove Proposition 9.
\end{ejercicio}
\begin{solucion}
\emph{If $G ⊆ GL(n, k)$ is a finite matrix group, then the set $k[x_1, \dots , x_n]^G$
is closed under addition and multiplication and contains the constant polynomials.}


Que contiene a los polinomios constantes es trivial. Si $f$ y $g$ son invariantes por $G$, entonces $f(x)+g(x)=f(Ax)+f(Ax)$ para toda $A\in G$ luego la suma es también invariante. Similarmente para la multiplicación, $f(x)g(x)=f(Ax)g(Ax)$. 
\end{solucion}

\newpage

\begin{ejercicio}{7.2.11}
Let $G ⊆ GL(n, k)$ be a finite matrix group. Then a polynomial
$f ∈ k[x_1, \dots , x_n]$ is invariant under $G$ if and only if its homogeneous components
are invariant.
\end{ejercicio}
\begin{solucion}
Si las componentes homogéneas son invariantes entonces $f$ es invariante por linealidad. Recíprocamente, al evaluar $f(Ax)$ no se modifica el grado de ningún monomio por ser una transformación lineal, de modo que para cada $m$, $f_m(Ax)=\sum_j c_jf_{m,j}(x)$ con $c_j\in k$. Pero para que $f(Ax)=f(x)$ debe cumplirse que $f(Ax)=\sum_m f_m(Ax)=\sum_m\sum_j c_jf_{m,j}(x)=f(x)=\sum_m\sum_j f_{m,j}$ por lo que necesariamente $c_j=1$ para todo $j$ y hemos terminado.

\end{solucion}
\end{document}
