\documentclass[twoside]{article}
\usepackage{../../estilo-ejercicios}

%--------------------------------------------------------
\begin{document}

\title{Ejercicios de Ideals, Varieties, and Algorithms (4ª Edición)}
\author{Diego Pedraza López, Javier Aguilar Martín, Rafael González López}
\maketitle

\begin{ejercicio}{1.3.1}
Parametrize all solutions of linear equations
\begin{align*}
x+2y-2z+w&=-1\\
x+y+z-w&=2
\end{align*}
\end{ejercicio}
\begin{solucion}
Simplemente consideramos $(t,u,v,-1-t-2u+2v)$ y $(t,u,v,-2+t+u+v)$
\end{solucion}

\newpage

\begin{ejercicio}{1.3.2}
Use a trigonometric identity to show that
\begin{align*}
x&=\cos(t)\\
y&=\cos(2t)
\end{align*}
parametrizes a portion of a parabola. Indicate exactly what portion of the parabola is covered.
\end{ejercicio}
\begin{solucion}
Sabemos que $\cos(2t)=\cos^2(t)-\sin^2(t) = 2\cos^2(t)-1$. Por tanto, es claro que se parametriza la parábola $y=2x^2-1$ en el intervalo $[-1,1]$.
\end{solucion}

\newpage

\begin{ejercicio}{1.3.3}
Given $f\in k[x]$, find a parametrization of $\V(y-f(x))$.
\end{ejercicio}
\begin{solucion}
Trivialérrimo. Tomamos $x=t$ y $y=f(t)$.
\end{solucion}

\newpage

\begin{ejercicio}{1.3.4}
Consider parametric representation
\begin{align*}
x&=\frac{t}{1+t}\\
y&=1-\frac{1}{t^2}
\end{align*}
\begin{enumerate}[a)]
\item Find the equation of the affine variety determined by the above parametric equations.
\item Show that the above equations parametrize all points of the variety found in part
(a) except for the point $(1, 1)$.
\end{enumerate}
\end{ejercicio}
\begin{solucion}
\begin{enumerate}[a)]
\item[]
\item Es claro que si $x=1$ entonces $1+t=t$, luego $1=0$. Por tanto podemos suponer que $x\neq 1$.
$$
x=\frac{t}{1+t} = \frac{1+t}{1+t}-\frac{1}{1+t} = 1 -\frac{1}{1+t}\qquad 1+t= \frac{1}{1-x} \qquad
t = \frac{x}{1-x}
$$
$$
y = 1-\frac{1}{t^2} = 1 - \frac{(1-x)^2}{x^2} = \frac{2x-1}{x^2}
$$
Luego tenemos $\V(x^2y-2x+1)$. Efectivamente es fácil ver que la parametrización está contenida en $\V(x^2y-2x+1)$. \item Si $x\neq 1$ entonces podemos escribir 
$t=x(1-x)^{-1}$ y procedemos inversamente al apartado anterior para ver que verifican la parametrización. Para ver que la parametrizació no cubre al $(1,1)$, vemos que no hay soluciones si impones que
$$
1=y=1-\frac{1}{t^2} \qquad 0 = \frac{1}{t^2}
$$
\end{enumerate}
\end{solucion}

\newpage


\begin{ejercicio}{1.3.5}
This problem will be concerned with the hyperbola $x^2 − y^2 = 1$.
\begin{enumerate}[a)]
\item Just as trigonometric functions are used to parametrize the circle, hyperbolic functions are used to parametrize the hyperbola. Show that the point
\begin{align*}
x&=\cosh(t)\\
y&=\sinh(t)
\end{align*}
always lies on $x^2 − y^2 = 1$. What portion of the hyperbola is covered?
\item Show that a straight line meets a hyperbola in $0$, $1$, or $2$ points, and illustrate your
answer with a picture. Hint: Consider the cases $x = a$ and $y = mx + b$ separately.
\item Adapt the argument given at the end of the section to derive a parametrization of the hyperbola. Hint: Consider nonvertical lines through the point $(−1, 0)$ on the hyperbola.
\item The parametrization you found in part (c) is undefined for two values of t. Explain how this relates to the asymptotes of the hyperbola.
\end{enumerate}
\end{ejercicio}
\begin{solucion}
\begin{enumerate}[a)]
\item[]
\item Es inmediato a partir de la igualdad $\cosh^2(t)
-\sinh^2(t)=1$. De hecho, la cubre completamente.
\item Si $x=a$ entonces tenemos los casos donde $a\in(-1,1)$ donde corta en 0 puntos, los casos $a=-1,1$, donde corta en un punto, y el resto de casos donde se cortan en dos puntos. Si $y=mx+b$ entonces $(1-m^2)x^2-2mbx-(b^2+1)=0$. La ecuación sobre $x$ da como mucho dos soluciones. La ecuación $y=mx+b$ nos da un $y$ por cada $x$. Luego como mucho la recta corta a la hipérbola en $2$ puntos.
\item Observemos que cada recta no vertical partiendo del $(-1,0)$ corta al eje $OY$ en un punto $(0,t)$ para $t\in(-\infty,+\infty)$. Cuando la recta sea paralela a alguna de las asíntotas no habrá punto de corte distinto de $(-1,0)$. Veremos que para $t\in(1,+\infty)$, la recta corta a la hipérbola por la parte inferior izquierda, para $t\in(0,1)$ corta en la parte superior derecha, para $t\in (-1,0)$ corta en la parte inferior derecha (para $t=0$ corta claramente en $(1,0)$) y para $t\in (-\infty,-1)$ corta en la parte superior izquierda.

Sea entonces la recta $r\equiv (-1,0)+\lambda (1,t)$. Tenemos que imponer que esta recta corte a $x^2-y^2=1$, luego sustituimos y obtenemos
\[
(-1+\lambda)^2-\lambda^2t^2=1\Rightarrow (1-t^2)\lambda^2-2\lambda=0.
\]
Como vemos, para $t=\pm 1$ la única solución sería $\lambda=0$, lo que nos da solo el punto $(-1,0)$. En otro caso podemos eliminar esta solución y llegar a 
\[
(1-t^2)\lambda-2=0\Rightarrow \lambda= \frac{2}{1-t^2}.
\]
Sustituyendo ahora en $r$ esto nos da el punto $(-1+\frac{2}{1-t^2}, \frac{2t}{1-t^2})$, lo cual agrupando nos proporciona la parametrización
\begin{align*}
&x=\frac{1+t^2}{1-t^2}\\
&y= \frac{2t}{1-t^2}
\end{align*}
\item Los valores $t=1$ y $t=-1$ hacen que la recta sea paralela respectivamente a alguna de las asíntotas y ocurre lo que hemos explicado al principio del anterior apartado.
\end{enumerate}
\end{solucion}

\newpage
\begin{ejercicio}{1.3.6}
The goal of this problem is to show that the sphere $x^2 + y^2 + z^2 = 1$ in $3$-dimensional space can be parametrized by
\begin{align*}
x&=\frac{2u}{u^2+v^2+1}\\
y&=\frac{2v}{u^2+v^2+1}\\
z&=\frac{u^2+v^2-1}{u^2+v^2+1}
\end{align*}
The idea is to adapt the argument given at the end of the section to 3-dimensional space.
\begin{enumerate}[a)]
\item Given a point $(u, v, 0)$ in the $(x, y)$-plane, draw the line from this point to the “north pole” $(0, 0, 1)$ of the sphere, and let $(x, y, z)$ be the other point where the line meets the sphere. Draw a picture to illustrate this, and argue geometrically that mapping
$(u, v)$ to $(x, y, z)$ gives a parametrization of the sphere minus the north pole.
\item Show that the line connecting $(0, 0, 1)$ to $(u, v, 0)$ is parametrized by $(tu, tv, 1 − t)$,
where $t$ is a parameter that moves along the line.
\item Substitute $x = tu$, $y = tv$ and $z = 1−t$ into the equation for the sphere $x^2+y^2+z^2 = 1$.
Use this to derive the formulas given at the beginning of the problem.
\end{enumerate}
\end{ejercicio}
\begin{solucion}
Es trivial, pues es la proyección estereográfica.
\end{solucion}

\newpage

\begin{ejercicio}{1.3.7} Adapt the argument of the previous exercise to parametrize the “sphere” $x^2_1
+\cdots+x^{2}_n = 1$ in n-dimensional affine space. Hint: There will be n − 1 parameters.
\end{ejercicio}
\begin{solucion}
Análogo al anterior.
\end{solucion}

\newpage

\begin{ejercicio}{1.3.8}
Consider the curve defined by $y^2 = cx^2 −x^3$, where $c$ is some constant.
\end{ejercicio}

Our goal is to parametrize this curve.
\begin{enumerate}[a.]
\item Show that a line will meet this curve at either 0, 1, 2, or 3 points. Illustrate your answer
with a picture. Hint: Let the equation of the line be either $x = a$ or $y = mx + b$.
\item Show that a nonvertical line through the origin meets the curve at exactly one other
point when $m^2 \neq c$. Draw a picture to illustrate this, and see if you can come up with
an intuitive explanation as to why this happens.
\item Now draw the vertical line $x = 1$. Given a point $(1, t)$ on this line, draw the line
connecting $(1, t)$ to the origin. This will intersect the curve in a point $(x, y)$. Draw a
picture to illustrate this, and argue geometrically that this gives a parametrization of
the entire curve.
\item Show that the geometric description from part (c) leads to the parametrization
\begin{align*}
&x = c − t^2,
&y = t(c − t^2).
\end{align*}
\end{enumerate}
\begin{solucion}
Vamos a pasar por alto los dibujos. 
\begin{enumerate}[a.]
\item Si consideramos la recta $x=a$, cortará en los puntos que verifiquen $y^2=ca^2-a^3$. Que, dependiendo del signo de $ca^2-a^3$, serán 0, 1 o 2 puntos.  Si consideramos la recta $y=mx+b$, cortará en los puntos que verifiquen $(mx+b)^2=cx^2-x^3$. Es una ecuación de tercer grado, luego como mucho tendrá 3 soluciones.
\item Una recta no vertical que pase por el origen es de la forma $y=mx$. Sustituyendo obtenemos $m^2x^2=cx^2-x^3=(m^2-c)x^2=x^3$. Si $m^2\neq c$ podemos eliminar la solución $x=0$, luego nos queda $m^2-c^2=x$. Luego efectivamente tenemos una única solución adicional. La explicación es que para $m^2=c$ sale una recta tangente en 0 y en cualquier otro caso es secante, luego da otra solución. 
\item Vamos a construir la parametrización descrita. La recta que conecta el origen con $(1,t)$ es $r\equiv (\lambda, \lambda t)$. Sustituyendo en la ecuación de la curva elíptica
\[
\lambda^2t^2=c\lambda^2-\lambda^3\Rightarrow \lambda^2(t^2-c+\lambda)=0
\]
Para $\lambda=0$ obtenemos el origen, así que descartamos esta solución y nos quedamos con
\[
\lambda=c-t^2
\]
Lo cual da lugar a la parametrización buscada
\begin{align*}
&x= c-t^2\\
&y=t(c-t^2)
\end{align*}
Esta parametrización cubre todos los punto. Además cualquier punto de la curva elíptica distinto del origen definen una recta que pasa por un punto de la forma $(1,t)$ y por el $(0,0)$. 
\item Ya está resuelto antes.
\end{enumerate}
\end{solucion}

\newpage

\begin{ejercicio}{1.3.9}
The strophoid is a curve that was studied by various mathematicians, including Barrow (1630–1677), Jean Bernoulli (1667–1748), and Maria Agnesi (1718–1799). A trigonometric parametrization is given by
\begin{align*}
x&=a\sin(t)\\
y&=a\tan(t)(1+\sin(t))
\end{align*}
where $a$ is a constant. If we let $t$ vary in the range $−4.5 ≤ t ≤ 1.5$.
\begin{enumerate}[a.]
\item Find the equation in x and y that describes the strophoid. Hint: If you are sloppy, you will get the equation $(a^2 − x^2)y^2 = x^2(a + x)^2$. To see why this is not quite correct, see what happens when $x = −a$.
\item Find an algebraic parametrization of the strophoid
\end{enumerate}
\end{ejercicio}
\begin{solucion}
\begin{enumerate}[a.]
\item[]
\item Dado que $1\notin \sin([-4.5,1.5])$, deducimos que $x\neq a$ $\forall t$ en nuestro intervalo. Por tanto, podemos considerar
$$
y=\frac{\sin(t)}{\sqrt{1-\sin^2(t)}}(1+\sin(t)) =\frac{x}{\sqrt{1-\frac{x^2}{a^2}}}\left(1+\frac{x}{a}\right)=\frac{x}{\sqrt{a^2-{x^2}}}(a+x) = x\sqrt{\frac{a+x}{a-x}}
$$
Luego satisface $\V(y^2(a-x)-x^2(a+x))$. 
\item La idea para encontrar nuestra parametrización algebraica es simplemente utilizar las que conocemos del seno y el coseno.
\begin{align*}
x&=a\frac{2t}{1+t^2}\\
y&=a\frac{2t}{(1+t)(1-t)}\frac{(1+t)^2}{1+t^2} = a\frac{2t(1+t)^2}{1-t^4}
\end{align*}
\end{enumerate}
\end{solucion}
\newpage

\begin{ejercicio}{1.3.10}
Around 180 B.C.E., Diocles wrote the book On Burning-Glasses. One of the curves he considered was the cissoid and he used it to solve the problem of the duplication of the cube [see part (c) below]. The cissoid has the equation $y^2(a + x) = (a − x)^3$, where a is
a constant. This gives the following curve in the plane:
\begin{enumerate}[a.]
\item Find an algebraic parametrization of the cissoid.
\item Diocles described the cissoid using the following geometric construction. Given a circle of radius $a$ (which we will take as centered at the origin), pick $x$ between $a$ and $−a$, and draw the line $L$ connecting $(a, 0)$ to the point $P = (−x,\sqrt{a^2 − x^2})$ on the
circle. This determines a point $Q = (x, y)$ on $L$. Prove that the cissoid is the locus of all such points Q.
\item The duplication of the cube is the classical Greek problem of trying to construct $\sqrt[3]{2}$ using ruler and compass. It is known that this is impossible given just a ruler and  compass. Diocles showed that if in addition, you allow the use of the cissoid, then
one can construct $\sqrt[3]{2}$. Here is how it works. Draw the line connecting $(−a, 0)$ to $(0, a/2)$. This line will meet the cissoid at a point $(x, y)$. Then prove that
$$
2=\left(\frac{a-x}{y}\right)^3
$$
which shows how to construct $\sqrt[3]{2}$ using ruler, compass, and cissoid.
\end{enumerate}
\end{ejercicio}
\begin{solucion}
\begin{enumerate}[a.]
\item[]
\item Consideremos la recta que pasa por $(a,0)$ y $(-a,t)$, es decir $(a+2a\lambda,-\lambda t)$. Esta recta corta a la cisoide en
$$
\lambda^2t^2(2a+2a\lambda)=(-2a\lambda)^3 \qquad 2a\lambda^2(t^2(1+\lambda)+4a^2\lambda)=0
$$
Si $\lambda \neq 0$, entonces $\lambda = \dfrac{-t^2}{t^2+4a^2}$, es decir, el punto 
\begin{align*}
x&=\frac{4a^3-at^2}{t^2+4a^2}\\
y&=\frac{t^3}{t^2+4a^2}
\end{align*}
La solución obtenida para $\lambda = 0$ podemos descartarla porque la obtenemos para $t=0$.
\item  Construyamos la recta descrita en el enunciado. $r\equiv (a+\lambda (-x-a),\lambda\sqrt{a^2-x^2})$. Ahora vemos cuál es el punto $Q$, imponiendo que la primera coordenada sea justamente $x$.
\[
a+\lambda (-x-a)=x\Leftrightarrow \lambda=\frac{x-a}{-x-a}.
\]
Nótese que por la elección de $x$, el cociente está bien definido. Por tanto, la coordenada $y$ es
\[
y=\frac{x-a}{-x-a}\sqrt{a^2-x^2}=\frac{x-a}{-(x+a)}\sqrt{(a-x)(a+x)}=(a-x)\sqrt{\frac{a-x}{a+x}}
\]

A partir de aquí es inmediato que los puntos de esta nueva curva cumplen la ecuación de la cisoide y recíprocamente.

\item Tomamos la recta que pasa por $(-a,0)$ y $(0,a/2)$, es decir, $y=(a+x)/2$. Vemos que para el punto en la intersección:
\[ \left(\frac{a-x}{y}\right)^3 = \frac{(a-x)^3}{y^3} = \frac{(a-x)^3}{y\frac{(a-x)^3}{(a+x)}} = \frac{a+x}{y} = \frac{a+x}{\frac{a+x}{2}} = 2 \]
\end{enumerate}
\end{solucion}

\newpage

\begin{ejercicio}{1.3.11}
In this problem, we will derive the parametrization
\begin{align*}
x &= t(u^2 − t^2),\\
y &= u,\\
z &= u^2 − t^2,
\end{align*}
of the surface $x^2 − y^2z^2 + z^3 = 0$ considered in the text.
\begin{enumerate}[a.]
\item Adapt the formulas in part (d) of Exercise 8 to show that the curve $x^2 = cz^2 − z^3$ is parametrized by
\begin{align*}
z &= c − t^2,\\
x &= t(c − t^2).
\end{align*}
\item Now replace the c in part (a) by $y^2$, and explain how this leads to the above parametrization of $x^2 − y^2z^2 + z^3 = 0$.
\item Explain why this parametrization covers the entire surface $\V(x^2 − y^2z^2 + z^3)$. Hint: See part (c) of Exercise 8.
\end{enumerate}
\end{ejercicio}
\begin{solucion}
\begin{enumerate}[a.]
\item[]
\item La recta que conecta el origen con $(1,t)$ es $(\lambda,\lambda t)$. Sustituyendo 
$$
\lambda^2t^2 = \lambda^2(c-\lambda)
$$
Como $\lambda =0$ nos da el origen, tenemos $t^2=c-\lambda$, luego $\lambda = c-t^2$. Luego $z=c-t^2$, $x=t(c-t^2)$.
\item Si $y^2 = c$ luego podemos parametrizar
\begin{align*}
x &= t(u^2 − t^2),\\
y &= u,\\
z &= u^2 − t^2,
\end{align*}
\item Finalmente, veamos que la parametrización cubre completamente la superficie, pero eso es claro pues para cada $y$ fijo, la curva está completamente cubierta por la parametrización como vemos en el aparatado $(c)$ del Ejercicio 8.
\end{enumerate}
\end{solucion}

\newpage

\begin{ejercicio}{1.3.12}
Consider the variety $V=\V(y-x^2,z-x^4) \subset \R^3$.
\begin{enumerate}[a.]
\item Draw a picture of $V$.
\item Parametrize V in a way similar to what we did with the twisted cubic.
\item Parametrize the tangent surface of V.
\end{enumerate}
\begin{solucion}
\begin{enumerate}[a.]
\item[]
\item Los dibujos son para nenazas.
\item Claramente es $(t,t^2,t^4)$. 
\item $r(t)+ur'(t) = (t,t^2,t^4)+u(1,2t,4t^3)$.
\end{enumerate}
\end{solucion}

\end{ejercicio}


\newpage

\begin{ejercicio}{1.3.13}
The general problem of finding the equation of a parametrized surface will be studied in Chapters 2 and 3. However, when the surface is a plane, methods from calculus or linear algebra can be used. For example, consider the plane in $\R^3$ parametrized by
\begin{align*}
x&=1+u-v\\
y&=u+2v\\
z&=-1-u+v
\end{align*}
Find the equation of the plane determined this way.
\begin{solucion}
Tenemos que el plano está generado por $(1,1,-1)$ y $(-1,2,1)$ luego
$$
\begin{pmatrix}
1 & -1 & x_1\\
1 & 2 & x_2\\
-1& 1 & x_3
\end{pmatrix} \Longrightarrow
\begin{pmatrix}
1 & -1 & x_1\\
1 & 2 & x_2\\
0 &0  & x_1 + x_3
\end{pmatrix}
$$
Luego nuestra ecuación es $x+z=d$ con $d$ tal que $1+u-v -1-y+v = d = 0$. Por tanto, el plano que buscamos es simplemente $x+z=0$.
\end{solucion}

\end{ejercicio}

\end{document}