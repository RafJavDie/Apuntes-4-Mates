\documentclass[twoside]{article}
\usepackage{../../estilo-ejercicios}
\newcommand{\lex}{<_{lex}}
\newcommand{\grlex}{<_{grlex}}
\newcommand{\grevlex}{<_{grevlex}}

\newcommand{\PhantC}{\phantom{\colon}}%
\newcommand{\CenterInCol}[1]{\multicolumn{1}{c}{#1}}%

%--------------------------------------------------------
\begin{document}

\title{Álgebra, Combinatoria y Computación, Práctica 3A}
\author{Javier Aguilar Martín}
\maketitle

\begin{ejercicio}{1}
Consideremos los ideales de $\Q[x_1,x_2]$ dados por 
\[
I=\gene{x_1^2+x_2^3-1, x_1-x_1x_2+3},J=\gene{x_1^2x_2-1}.
\]
Calcule un conjunto de generadores de $I\cap J$.
\end{ejercicio}

\begin{solucion}
Sean $f_1=x_1^2+x_2^3-1$, $f_2=x_1-x_1x_2+3$ y $f_3=x_1^2x_2-1$. Para calcular los generadores de $I\cap J$ calculamos una base de Gröbner para el ideal $\gene{wI,(1-w)J}=\gene{wf_1,wf_2,(1-w)f_3}$ y a continuación calculamos la base de Gröbner del primer ideal de eliminación. En SAGE:
\begin{verbatim}
Q.<w,x1,x2>=PolynomialRing(QQ,order='lex')
I=ideal(x1^2+x2^3-1,x1-x1*x2+3)
J=ideal(x1^2*x2-1)
IcapJ=ideal(w*(x1^2+x2^3-1),w*(x1-x1*x2+3), (1-w)*(x1^2*x2-1))
IcapJ.groebner_basis()
\end{verbatim}
Los dos últimos polinomios de la base de Gröbner son los del ideal que buscamos:
\[
x_1^3x_2 + 1/3x_1^2x_2^5 - 1/3x_1^2x_2^4 - 1/3x_1^2x_2^2 + 1/3x_1^2x_2 - x_1 - 1/3x_2^4 + 1/3x_2^3 + 1/3x_2 - 1/3
\]
\[
x_1^2x_2^6 - 2x_1^2x_2^5 + x_1^2x_2^4 - x_1^2x_2^3 + 2x_1^2x_2^2 + 8x_1^2x_2 - x_2^5 + 2x_2^4 - x_2^3 + x_2^2 - 2x_2 - 8
\]

\end{solucion}

\begin{ejercicio}{2}
Obtenga los restos $r,s$ a partir de su DNI módulo 15 como en la práctica anterior. Para los valores encontrados, determine el ideal $I$ de la casilla $r$ y el ideal $J$ de la casilla $s$. Calcula un conjunto de generadores de los ideales $I:J$, $I:J^{\infty}$ en $\Q[x_1,x_2]$.
\end{ejercicio}
\begin{solucion}
Mis números son $r=0$, $s=6$. Estos números determinan los ideales $I=\gene{5x_1x_2^2-x_2^2,-x_1^2+3x_2+3}=\gene{f_1,f_2}$ y $J=\gene{2x_1x_2^2-2x_2^2,-5x_1^2+5x_2+5}=\gene{g_1,g_2}$. Por la proposición 2
\[
I:J=I:\gene{g_1,g_2}=(I:\gene{g_1}):(I:\gene{g_2}).
\]
Calculamos ahora una base de los dos ideales por el mismo procedimiento que en el ejercicio anterior. En primer lugar tenemos que calcular la base de $I\cap\gene{g_1}$, que es
\[
\{x_1^2x_2^2 - 6/5x_1x_2^2 + 1/5x_2^2, x_1x_2^3 + 74/75x_1x_2^2 - x_2^3 - 74/75x_2^2\}
\]
Por lo que la base de $I:\gene{g_1}$ se obtiene dividiendo estos polinomios por $g_1$, es decir
\[
\{1/2x_1 - 1/10,1/2x_2 + 37/75\}
\]
Asimismo, la de $I\cap\gene{g_2}$ es
\[
\{x_1^4 - 4x_1^2x_2 - 4x_1^2 + 3x_2^2 + 6x_2 + 3, x_1^3x_2^2 - 1/5x_1^2x_2^2 - x_1x_2^3 - x_1x_2^2 + 1/5x_2^3 + 1/5x_2^2,
\]
\[
 x_1^2x_2^3 + 74/75x_1^2*x_2^2 - x_2^4 - 149/75x_2^3 - 74/75x_2^2\},
\]
lo que nos da como base de $I:\gene{g_2}$
\[
\{-1/5x_1x_2^2 + 1/25x_2^2,-1/5x_2^3 - 74/375x_2^2\}.
\]
Finalmente calculamos la intersección de estos dos ideales, cuya base es
\[
\{x_1x_2^2 - 1/5x_2^2, x_2^3 + 74/75x_2^2\}
\]

Procedemos ahora a calcular la base de $I:J^{\infty}$. Igual que antes
\[
I:J^{\infty}=(I:\gene{g_1}^{\infty})\cap (I:\gene{g_2}^{\infty})
\]
Por la proposición 3, para $I:\gene{g_1}^{\infty}$, calculamos la base de $\gene{f_1,f_2,1-wg_1}\cap \Q[x_1,x_2]$:
\[
 \{x_1 - 1/5, x_2 + 74/75\}
\]
y para $I:\gene{g_2}^{\infty}$ hacemos lo propio con $\gene{f_1,f_2,1-wg_2}\cap \Q[x_1,x_2]$:
\[
\{x_1^2 - 3x_2 - 3, x_1x_2^2 - 1/5x_2^2, x_2^3 + 74/75x_2^2\}.
\]
Finalmente, calculamos la intersección de estos dos últimos, cuyo resultado es el ideal
\[
\gene{x_1^2 - 3x_2 - 3, x_1x_2^2 - 1/5x_2^2, x_2^3 + 74/75x_2^2}.
\]
\end{solucion}

\end{document}
