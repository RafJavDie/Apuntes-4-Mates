\documentclass[twoside]{article}
\usepackage{../../estilo-ejercicios}
\newcommand{\lex}{<_{lex}}
\newcommand{\grlex}{<_{grlex}}
\newcommand{\grevlex}{<_{grevlex}}

\newcommand{\PhantC}{\phantom{\colon}}%
\newcommand{\CenterInCol}[1]{\multicolumn{1}{c}{#1}}%

%--------------------------------------------------------
\begin{document}

\title{Ejercicios de Ideals, Varieties, and Algorithms (4ª Edición)}
\author{Diego Pedraza López, Javier Aguilar Martín, Rafael González López}
\maketitle

\begin{ejercicio}{2.5.1}
Let $I =\langle 
g_1, g_2, g_3\rangle ⊆ \R[x, y, z]$, where $g_1 = xy^2 − xz + y$, $g_2 = xy − z^2$ and
$g_3 = x − yz^4$. Using the lex order, give an example of $g ∈ I$ such that $LT(g) \not∈
\langle
LT(g1), LT(g2), LT(g3)\rangle$.
\end{ejercicio}
\begin{solucion}
Se comprueba que el máximo común divisor de los 3 polinomios es 1, por lo que $1\in I$ pero claramente $Lt(1)=1\notin \langle
LT(g1), LT(g2), LT(g3)\rangle$.
\end{solucion}
\newpage

\begin{ejercicio}{2.5.2}
For the ideals and generators given in Exercises 5, 6, and 7 of §3, show that $LT(I)$ is
strictly bigger than 
$\gene{LT( f_1),\dots , LT( f_s)}$. Hint: This should follow directly from what
you did in those exercises.
\end{ejercicio}
\begin{solucion}

\end{solucion}
\newpage

\begin{ejercicio}{2.5.3}
To generalize the situation of Exercises 1 and 2, suppose that $I = 
\gene{ f_1,\dots, fs}$ is an ideal
such that 
$\gene{LT( f_1),\dots , LT( f_s)}$ is strictly smaller than 
$\gene{LT(I)}$.

\begin{enumerate}[a.]
\item Prove that there is some $f ∈ I$ whose remainder on division by $f_1,\dots , f_s$ is nonzero.
Hint: First show that $LT( f ) \not∈ 
\gene{LT( f_1),\dots , LT( f_s)}$ for some $f ∈ I$. Then use
Lemma 2 of §4.
\item What does part (a) say about the ideal membership problem?
\item How does part (a) relate to the conjecture you were asked to make in Exercise 8 of
§3?
\end{enumerate}
\end{ejercicio}
\begin{solucion}
\begin{enumerate}[a.]
\item 
\item 
\end{enumerate}
\end{solucion}

\newpage

\begin{ejercicio}{2.5.4}
If $I ⊆ k[x_1,\dots , x_n]$ is an ideal, prove that 
$\gene{LT(g) | g ∈ I\setminus \{0\}} = 
\gene{LM(g) | g ∈ I\setminus \{0\}}$.
\end{ejercicio}
\begin{solucion}

\end{solucion}
\newpage

\begin{ejercicio}{2.5.5}
Let $I$ be an ideal of $k[x_1,\dots , x_n]$. Show that $G = {g_1,\dots , g_t} ⊆ I$ is a Gröbner basis of
$I$ if and only if the leading term of any element of $I$ is divisible by one of the $LT(g_i)$.
\end{ejercicio}
\begin{solucion}

\end{solucion}

\newpage

\begin{ejercicio}{2.5.6}
Corollary 6 asserts that a Gröbner basis is a basis, i.e., if $G = {g_1,\dots , g_t} ⊆ I$ satisfies
$\gene{LT(I)} = 
\gene{LT(g_1),\dots , LT(g_t)}$, then $I =\gene{ 
g_1,\dots , g_t}$. We gave one proof of this in the
proof of Theorem 4. Complete the following sketch to give a second proof. If $f ∈ I$, then
divide $f$ by $(g_1,\dots, g_t)$. At each step of the division algorithm, the leading term of the
polynomial under the division will be in 
$\gene{LT(I)}$ and, hence, will be divisible by one of
the $LT(g_i)$. Hence, terms are never added to the remainder, so that $f =
\sum_{i=1}^t a_i g_i$ when
the algorithm terminates.
\end{ejercicio}
\begin{solucion}

\end{solucion}

\newpage

\begin{ejercicio}{2.5.7}
If we use grlex order with $x > y > z$, is $\{x^4y^2 −z^5, x^3y^3 −1, x^2y^4 −2z\}$ a Gröbner basis
for the ideal generated by these polynomials? Why or why not?
\end{ejercicio}
\begin{solucion}

\end{solucion}

\newpage

\begin{ejercicio}{2.5.8}
Repeat Exercise 7 for $I = 
\gene{x − z^2, y − z^3}$ using the lex order. Hint: The difficult part of
this exercise is to determine exactly which polynomials are in 
$\gene{LT(I)}$.
\end{ejercicio}
\begin{solucion}

\end{solucion}

\newpage

\begin{ejercicio}{2.5.9}
Let $A = (a_{ij})$ be an $m × n$ matrix with real entries in row echelon form and let $J ⊆
\R[x_1,\dots , xn]$ be an ideal generated by the linear polynomials
$\sum^n_{j=1} a_{ij}x_j$ for $1 ≤ i ≤ m$.
Show that the given generators form a Gröbner basis for $J$ with respect to a suitable
lexicographic order. Hint: Order the variables corresponding to the leading 1’s before
the other variables.
\end{ejercicio}
\begin{solucion}

\end{solucion}

\newpage

\begin{ejercicio}{2.5.10}
Let $I ⊆ k[x_1, \dots , x_n]$ be a principal ideal (that is, $I$ is generated by a single $f ∈ I$—
see §5 of Chapter 1). Show that any finite subset of $I$ containing a generator for $I$ is a
Gröbner basis for $I$.
\end{ejercicio}
\begin{solucion}
\end{solucion}

\newpage

\begin{ejercicio}{2.5.11}
Let $f ∈ k[x_1,\dots , x_n]$. If $f \not∈ 
\gene{x_1,\dots , x_n}$, then show 
$\gene{x_1,\dots , xn, f }
 = k[x_1,\dots, x_n]$.
\end{ejercicio}
\begin{solucion}
Si  $f \not∈ 
\gene{x_1,\dots , x_n}$ entonces es una constante, luego se tiene el resultado. 
\end{solucion}

\newpage

\begin{ejercicio}{2.5.12}
Show that if we take as hypothesis that every ascending chain of ideals in $k[x_1,\dots , x_n]$
stabilizes, then the conclusion of the Hilbert Basis Theorem is a consequence. Hint: Argue
by contradiction, assuming that some ideal $I ⊆ k[x_1,\dots , x_n]$ has no finite generating
set. The arguments you gave in Exercise 12 should not make any special use of properties
of polynomials. Indeed, it is true that in any commutative ring $R$, the following two
statements are equivalent:
\begin{enumerate}[(i)]
\item Every ideal $I ⊆ R$ is finitely generated.
\item Every ascending chain of ideals of $R$ stabilizes.
\end{enumerate}
\end{ejercicio}
\begin{solucion}
$(i)\Rightarrow(ii)$ Análogo al teorema 7 pero para un anillo cualquiera.

$(ii)\Rightarrow (i)$ Dado un ideal $I=\langle x_1,x_2,\dots\rangle$ podemos considera la cadena ascendente
\[
\gene{x_1}\subseteq\gene{x_1,x_2}\subseteq\cdots.
\]
Como existe un $N$ para el que se estabiliza, $I=\gene{x_1,\dots,x_N}$. 
\end{solucion}

\newpage

\begin{ejercicio}{2.5.13}
Let
$$V_1 ⊇ V_2 ⊇ V_3 ⊇ \cdots$$
be a descending chain of affine varieties. Show that there is some $N ≥ 1$ such that
$V_N = V_{N+1} = V_{N+2} = \cdots$ . Hint: Use the ACC and Exercise 14 of Chapter 1, §4.
\end{ejercicio}
\begin{solucion}
Consecuencia inmediata del Nullstellensatz y la ascending chain condition.
\end{solucion}


\newpage

\begin{ejercicio}{2.5.14}
Let $f_1, f_2,\dots ∈ k[x_1,\dots, x_n]$ be an infinite collection of polynomials. Prove that there is
an integer $N$ such that $f_i ∈ 
\gene{f_1,\dots, f_N}$ for all $i ≥ N + 1$. Hint: Use $f_1, f_2,\dots$ to create
an ascending chain of ideals.
\end{ejercicio}
\begin{solucion}
Análogo a la segunda implicación de \ref{ejer:2.5.12}
\end{solucion}

\newpage

\begin{ejercicio}{2.5.15}
Given polynomials $f_1, f_2,\dots ∈ k[x_1,\dots , x_n]$, let $\V( f_1, f_2,\dots) ⊆ k^n$ be the solutions of
the infinite system of equations $f_1 = f_2 = \dots = 0$. Show that there is some $N$ such that
$\V( f_1, f_2,\dots) = \V( f_1,\dots , f_N)$.
\end{ejercicio}
\begin{solucion}
Consecuencia inmediata de \ref{ejer:2.5.13}
\end{solucion}

\newpage

\begin{ejercicio}{2.5.16}
In Chapter 1, §4, we defined the ideal $\I(V)$ of a variety $V ⊆ k^n$. In this section, we
defined the variety of any ideal (see Definition 8). In particular, this means that $\V(\I(V))$
is a variety. Prove that $\V(\I(V)) = V$. Hint: See the proof of Lemma 7 of Chapter 1, §4.
\end{ejercicio}
\begin{solucion}
\end{solucion}

\newpage

\begin{ejercicio}{2.5.17}
Consider the variety $V = \V(x^2 − y, y + x^2 − 4) ⊆ \C^2$. Note that $V = \V(I)$, where
$I =\gene{ 
x^2 − y, y + x^2 − 4}$.
\begin{enumerate}[a.]
\item Prove that $I =\gene{ 
x^2 − y, x^2 − 2}$.
\item Using the basis from part (a), prove that $\V(I) = \{(±
√
2, 2)\}$.
\end{enumerate}
\end{ejercicio}
\begin{solucion}
\end{solucion}

\newpage

\begin{ejercicio}{2.5.18}
When an ideal has a basis where some of the elements can be factored, we can use the
factorization to help understand the variety.
\begin{enumerate}[a.]
\item Show that if $g ∈ k[x_1,\dots , x_n]$ factors as $g = g_1g_2$, then for any $f$, we have $\V( f , g) =
\V( f , g_1) ∪ \V( f , g_2)$.
\item Show that in $\R^3$, $\V(y − x^2, xz − y^2) = \V(y − x^2, xz − x^4)$.
\item Use part (a) to describe and/or sketch the variety from part (b).
\end{enumerate}
\end{ejercicio}
\begin{solucion}
\end{solucion}

\end{document}
