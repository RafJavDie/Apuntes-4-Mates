\documentclass[twoside]{article}
\usepackage{../../estilo-ejercicios}
\newcommand{\lex}{<_{lex}}
\newcommand{\grlex}{<_{grlex}}
\newcommand{\grevlex}{<_{grevlex}}

\newcommand{\PhantC}{\phantom{\colon}}%
\newcommand{\CenterInCol}[1]{\multicolumn{1}{c}{#1}}%

%--------------------------------------------------------
\begin{document}

\title{Ejercicios de Ideals, Varieties, and Algorithms (4ª Edición)}
\author{Diego Pedraza López, Javier Aguilar Martín, Rafael González López}
\maketitle
(``
\begin{ejercicio}{2.3.1}
Compute the remainder on division of the given polynomial $f$ by the ordered set $F$ (by hand).
Use the grlex order, then the lex order in each case.
\begin{enumerate}[a.]
\item $f = x^7y^2 + x^3y^2 - y + 1$, $F = (xy^2-x, x-y^3)$.
\item Repeat part (a) with the order of the pair $F$ reversed.
\end{enumerate}
\end{ejercicio}
\begin{solucion}
Usando grlex para ordenar los monomios:

%% Este formato de división será reemplazada por uno más similar al que se haga en clase.
\[
\begin{array}{rrr}
   q_1\colon  & \CenterInCol{x^6+x^2} & r \\
   q_2\colon  & \CenterInCol{0}\\
xy^2-x\PhantC & \multirow{2}*{$\sqrt{x^7y^2+x^3y^2-y+1}$}\\
-y^3+x\PhantC &\\
              & x^7y^2 - x^7\\\cline{2-2}
              & x^7+x^3y^2-y+1\\
              & x^3y^2-y+1 & \to x^7\\
              & x^3y^2-x^3 \\\cline{2-2}
              & -x^3-y+1\\
              & -y+1 & \to x^3\\
              & 1 & \to -y\\
              & 0 & \to 1
\end{array}
\]

Luego $x^7y^2+x^3y^2-y+1 = (x^6+x^2)(xy^2-x) + 0 (-y^3+x) + (x^7+x^3-y+1)$.

Usando lex sale después de $17$ pasos:
\[ x^7y^2 + x^3y^2- y + 1 = q_1(xy^2-x)+q_2(x-y^3)+(2y^3-y-1)\]
donde
\[ q_1 = x^6 + x^5y + x^4y^2 + x^4 + x^3y + x^2y^2 + 2x^2 + 2xy + 2y^2 + 2\]
\[ q_2 = x^6 + x^5y + x^4 + x^3y + 2x^2 + 2xy + 2 \]

Reordenando $F$ como $F = (x-y^3,xy^2-x)$ no se producen cambios en la división usando grlex, pues $LT(-y^3+x) = -y^3$ nunca divide al polinomio dividendo en ningún paso en (a).
Por otro lado, la división lex nos da:
\[ x^7y^2+x^3y^2-y+1 = q_1(x-y^3)+0(xy^2-x)+(y^{10}-y+1)\]
con $q_1 = x^6y+x^5y^4+x^4y^7+x^3y^{10}+x^2y^{13}+xy^{16}+y^{19}+x^2y+xy^4+y^7$.
\end{solucion}
\newpage

\begin{ejercicio}{2.3.2}
Compute the remainder on division:
\begin{enumerate}[a.]
\item $f = xy^2z^2 + xy - yz$, $F = (x-y^2, y-z^3, z^2-1)$.
\item Repeat part (a) with the order of the set $F$ permuted cyclically.
\end{enumerate}
\end{ejercicio}
\begin{solucion}
Como no nos dan un orden, escogemos $grlex$.
\[
\begin{array}{rrr}
   q_1\colon  & \CenterInCol{-xz^2} & r \\
   q_2\colon  & \CenterInCol{0}\\
     q_3\colon  & \CenterInCol{x^2}\\
-y^2+x\PhantC & \multirow{2}*{$\sqrt{xy^2z^2 + xy - yz}$}\\
-z^3+y\PhantC &\\
z^2-1\PhantC &\\
              & xy^2z^2-x^2z^2\\\cline{2-2}
              &x^2z^2 +xy - yz \\
              & x^2z^2-x^2\\\cline{2-2}
              & x^2+xy-yz\\
              & xy-yz& \to x^2\\
              & -yz & \to xy\\
              & 0& \to -yz
\end{array}
\]
Luego tenemos que
$$
xy^2z^2 + xy - yz = -xz^2(x-y^2) + 0(y-z^3)+x^2(z^2-1)+(x^2+xy-yz)
$$
El apartado $b)$ se deja como ejercicio para el lector. 
\end{solucion}
\newpage

\begin{ejercicio}{2.3.3}
Using a computer algebra system, check your work from Exercises \ref{ejer:2.3.1} and \ref{ejer:2.3.2}.
\end{ejercicio}
\begin{solucion}
Ofrecemos un código en sagemath en el archivo adjunto `utils.sage'.
\end{solucion}

\newpage

\begin{ejercicio}{2.3.4}
Let $f = q_1 f_1 + \dots q_s f_s + r$ be the output of the divison algorithm.
\begin{enumerate}[a.]
\item Complete the proof begun in the text that $multideg(f) ≥ multideg(q_i f_i)$ provided that $q_i f_i \neq 0$.
\item Prove that $multideg(f) ≥ multideg(r)$ when $r \neq 0$.
\end{enumerate}
\begin{solucion}
\begin{enumerate}[a.]
\item[]
\item Hemos visto en el texto (Páginas 65-66) que $q_i =LT(p)/LT(f_i)$ para algún valor de $p$ que sabemos verifica $LT(p)\leq LT(f)$. Esto es, $$multideg(q_i) = multideg(p)-multideg(f_i) \leq multideg(f)-multideg(f_i)$$ 
O equivalentemente
$$multideg(f) \geq multideg(q_i)+multidef(f_i)$$
Ahora basta tener en cuenta $multideg(f_iq_i) = multideg(f_i)+multideg(q_i)$ por el Lema 8 (Página 60).
\item Vamos a distinguir dos casos. 
\begin{itemize}
\item Si $r=f$ no hay nada que probar.
\item Si $r\neq f$ entonces $\sum_i q_i f_i \neq 0$. Sea $g=-\sum_i q_i f_i$. Entonces, por el Lema 8 tenemos que
$$
multideg(r)=multideg(f+g) \leq \max{\{multideg(f),multideg(g)\}}=multideg(f)
$$
En la última igualdad hemos utilizidado el primer apartado y una sencilla inducción en la suma de los $q_if_i$.

\end{itemize}

\end{enumerate}
\end{solucion}
\end{ejercicio}

\newpage

\begin{ejercicio}{2.3.5}
We will study the division of $f = x^3-x^2y-x^2z+x$ by $f_1 = x^2y-z$ and $f_2 = xy-1$.
\begin{enumerate}[a.]
\item Compute using grlex order:
\[ r_1 = \text{remainder of $f$ on division by $(f_1,f_2)$} \]
\[ r_2 = \text{remainder of $f$ on division by $(f_2,f_1)$} \]
Your results should be \emph{different}.
Where in the division algorithm did the difference occur?
\item Is $r = r_1-r_2$ in the ideal $\gene{f_1,f_2}$?
If so, find an explicit expression $r = Af_1 + Bf_2$..
If not, say why not.
\item Compute the remainder of $r$ on division by $(f_1,f_2)$.
Why could you have predicted your answer before doing the division?
\item Find another polynomial $g \in \gene{f_1,f_2}$ such that the remainder on division of $g$ by $(f_1,f_2)$ is nonzero.
Hint: $(xy+1)f_2 = x^2y^2-1$, whereas $yf_1 = x^2y^2-yz$.
\item Does the division algorithm give us a solution for the ideal membership problem for the ideal $\gene{f_1,f_2}$?
Explain your answer.
\end{enumerate}
\end{ejercicio}
\begin{solucion}
\begin{enumerate}[a.]
\item[]
\item Computamos la división: $r_1 = x^3-x^2z+x-z$ y $r_2 = x^3-x^2z$. 
La diferencia radica en que $x^2y$ del dividendo es divisible por $LT(f_1)$ y $LT(f_2)$ y $f_1 \neq f_2$, por lo que es relevante el orden de los divisores.
\item $r=r_1-r_2 = x-z$. Tenemos que $f = q_1f_1+q_2f_2 + r_1$ y $f = q_1'f_1+q_2'f_2 + r_2$. Luego:
\[ r=r_1-r_2 = (-q_1+q_1')f_1+(-q_2+q_2')f_2 \]
Como $q_1 = -1$, $q_2 = q_1' = 0$ y $q_2' = -x$, deducimos que $r = f_1 - x f_2$.
\item Tenemos que la división da por resto $x-z$ (aunque $r \in \gene{f_1,f_2}$).
Podriamos haber predicho este resultado observando que todos los términos de $r$ son no divisibles por los términos líderes de $f_1$ y $f_2$, una de las propiedades del resto.
\item Definimos $g = -yf_1 + (xy+1)f_2 = yz-1$, que no es divisible por los términos líderes de $f_1$ y $f_2$.
\item No. Desearíamos que el resto de la división de un elemento de $\gene{f_1,f_2}$ por $(f_1,f_2)$ fuera $0$, pero no es el caso en $r$ ó en $g$.
\end{enumerate}
\end{solucion}

\newpage

\begin{ejercicio}{2.3.6}
 Using the grlex order, find an element $g$ of 
 $\gene{f_1, f_2} = \langle 2xy^2 −x, 3x^2y−y−1\rangle ⊆ \R[x, y]$
whose remainder on division by $( f_1, f_2)$ is nonzero. Hint: You can find such a $g$ where
the remainder is $g$ itself.
\end{ejercicio}
\begin{solucion}
Cualquier polinomio cuyos términos sean no divisibles por los términos líderes de $f_1$ y $f_2$ vale. Por dar uno en concreto, $3x^2-y^2-y$.
\end{solucion}

\newpage

\begin{ejercicio}{2.3.7}
 Answer the question of Exercise \ref{ejer:2.3.6} for 
 $\gene{f_1, f_2, f_3} = 
\langle x^4y^2 − z,$ $ x^3y^3 − 1, x^2y^4 − 2z\rangle
⊆ \R[x, y, z]$. Find two different polynomials $g$ (not constant multiples of each other).
\end{ejercicio}
\begin{solucion}
Análogamente al caso anterior, escogemos un par de polinomios cuyos términos líderes no son divisibles por los términos líderes de $f_i$. Por ejemplo: $xy$, $xz$.
\end{solucion}

\newpage

\begin{ejercicio}{2.3.8}
Try to formulate a general pattern that fits the examples in Exercises \ref{ejer:2.3.5}(c)(d), \ref{ejer:2.3.6}, and \ref{ejer:2.3.7}.
What condition on the leading term of the polynomial $g = A_1 f_1 +\dots+ A_s f_s$ would
guarantee that there was a nonzero remainder on division by $( f_1,\dots, f_s)$? What does
your condition imply about the ideal membership problem?
\end{ejercicio}
\begin{solucion}
El patrón es lo que se ha dicho en \ref{ejer:2.3.6}. Se implica que ningún elemento de grado estrictamente menor que algún elemento de grado mínimo en el ideal pertenece al ideal.
\end{solucion}

\newpage

\begin{ejercicio}{2.3.9}
The discussion around equation (2) of Chapter 1, §4 shows that every polynomial $f ∈
\R[x, y, z]$ can be written as
$$f = h_1(y − x^2) + h_2(z − x^3) + r,$$
where $r$ is a polynomial in $x$ alone and $\V(y−x^2, z−x^3)$ is the twisted cubic curve in $\R^3$.
\begin{enumerate}[a.]
\item Give a proof of this fact using the division algorithm. Hint: You need to specify
carefully the monomial ordering to be used.
\item Use the parametrization of the twisted cubic to show that $z^2 − x^4y$ vanishes at every
point of the twisted cubic.
\item Find an explicit representation
$$z^2 − x^4y = h_1(y − x^2) + h_2(z − x^3)$$
using the division algorithm.
\end{enumerate}
\end{ejercicio}
\begin{solucion}
\begin{enumerate}[a.]
\item[]
\item Usando lex con $z>y>x$, cualquier polinomio que tenga $z$ o $y$ sería divisible por el término líder de alguno de los generadores, luego el polinomio debe contener solo la variable $x$.
\item Es claro sustituyendo un punto de la forma $(t,t^2, t^4)$ en el polinomio. 
\item $h_1=-x^4$, $h_2=z+x^3$.
\end{enumerate}
\end{solucion}

\newpage

\begin{ejercicio}{2.3.10}
Let $V ⊆ \R^3$ be the curve parametrized by $(t, t^m, t^n)$, $n,m ≥ 2$.
\begin{enumerate}[a.]
\item Show that $V$ is an affine variety.
\item Adapt the ideas in Exercise \ref{ejer:2.3.9} to determine $\I(V)$.
\end{enumerate}
\end{ejercicio}
\begin{solucion}
\begin{enumerate}[a.]
\item[]
\item A estas alturas creo que está claro que $V=\V(y-x^m, z-x^n)$. 
\item Se tiene que $\I(V)={\gene{y-x^m,z-x^n}}$. Basta utilizar el Ejercicio \ref{ejer:2.3.9} y el argumento del ejemplo de la Página 33.
\end{enumerate}
\end{solucion}

\newpage

\begin{ejercicio}{2.3.11}
In this exercise, we will characterize completely the expression
$$f = q_1 f_1 + \dots + q_s f_s + r$$
that is produced by the division algorithm (among all the possible expressions for $f$ of
this form). Let $LM( f_i) = x^{α(i)}$ and define
\begin{align*}
&Δ_1 = α(1) + \Z^n_{≥0},\\
&Δ_2 = (α(2) + \Z^n_{≥0}) \setminus Δ_1,\\
&\;\vdots\\
&Δ_s = (α(s) + \Z^n_{≥0}) \setminus
\left(\bigcup^{s−1}_{i=1}
Δ_i
\right)
,\\
&\overline{Δ} = \Z^n_{≥0} \setminus
\left(\bigcup^{s}_{i=1}
Δ_i
\right)
.
\end{align*}
(Note that $\Z^n_{≥0}$ is the disjoint union of the $Δ_i$ and $\overline{Δ}$.)
\begin{enumerate}[a.]
\item Show that $β ∈ Δ_i$ if and only if $x^{α(i)}$ divides $x^β$ and no $x^{α(j)}$ with $j < i$ divides $x^β$.
\item Show that $γ ∈ \overline{Δ}$ if and only if no $x^{α(i)}$ divides $x^γ$.
\item Show that in the expression $f = q_1 f_1 + \dots + q_s f_s + r$ computed by the division
algorithm, for every $i$, every monomial $x^β$ in $q_i$ satisfies $β + α(i) ∈ Δ_i$, and every
monomial $x^γ$ in $r$ satisfies $γ ∈ \overline{Δ}$.
\item Show that there is exactly one expression $f = q_1 f_1 + \dots + q_s f_s + r$ satisfying the
properties given in part (c).
\end{enumerate}
\end{ejercicio}
\begin{solucion}\mbox{}
\begin{enumerate}[a.]
\item Por definición $β \in Δ_i$ sii $β \in α(i) + \Z_{≥0}^n$ y $β \notin \bigcup_{j=1}^{i-1} Δ_j$.
Lo primero se da si y sólo si $β = α(i) + δ$ para algún $δ \in \Z_{≥0}$, es decir, $x^β = x^{α(i)}\cdot x^δ$.
En definitiva, si y sólo si $x^{α(i)}$ divide a $x^β$.
Supongamos que $x^β$ fuera divisble entre $x^{α(j)}$ para $j < i$.
Esto se da si y sólo si existe $γ$ tal que $x^β = x^{α(j)} \cdot x^γ$, es decir, $β = α(j) + γ \in α(j) + \Z_{≥0}^n \subset Δ_j$.
Esto demuestra las dos implicaciones.
\item $γ$ no está en $\overline{Δ}$ si y sólo si $γ \in Δ_i$ para algún $i$, si y sólo si $γ$ es divisible por $x^{α(i)}$ por el apartado anterior.
\item Claramente, $β+α(i) \in α(i) + \Z_{≥0}^n$.
Hay que ver que $β+α(i) \notin Δ_j$ para cualquier $j < i$.
Por el algoritmo de la división, $x^β = LM(p)/LM(f_i)$ para el primer $f_i$ donde sea posible esta división.
En otras palabras, $β = multideg(p) - α(i)$ para el primer $i$ donde sea posible.
Si existiera $x^{β+α(i)}$ fuera divisible por un $x^{α(j)}$ con $j < i$, tendríamos que hay $δ$ tal que $δ+α(j)=β+α(i)=multideg(f)$.
Esto no es posible, pues la primera $j$ donde tenemos $multideg(f)-α(j) \in \Z_{≥0}^n$ es en $j=i$.
Por el apartado (a), esto se resume en decir que $β+α(i) \in Δ_i$.

Por el algoritmo de la división, todos los monomios del resto no son divisibles por $x^{α(i)}$, luego todos los monomios del resto están en $\overline{Δ}$ por el apartado (b).

\item Tenemos que el resultado es cierto para $n=1$. Para $s=1$ se puede usar un razonamiento análogo al caso univariante\footnote{\url{https://math.stackexchange.com/questions/1972421/proof-of-uniqueness-of-division-theorem-for-polynomials}} reemplazando el grado por el multigrado. Teniendo en cuenta que $k[x_1,\dots, x_n]=k[x_1,\dots,x_{n-1}][x_n]$ se puede hacer una doble inducción en $s$ y en $n$.
\end{enumerate}
\end{solucion}

\newpage

\begin{ejercicio}{2.3.12}
Show that the operation of computing remainders on division by $F = (f_1,\dots,f_s)$ is linear over $k$.
That is, if the remainder on division of $g_i$ by $F$ is $r_i$, $i=1,2$, then, for any $c_1,c_2 \in k$, the remainder on division of $c_1 g_1 + c_2 g_2$ is $c_1 r_1 + c_2 r_2$.
Hint: Use Exercise \ref{ejer:2.3.11}
\end{ejercicio}
\begin{solucion}
Sean $g_i = q_i^1 f_1 + \dots q_i^s f_s + r_1$.
Se tiene que:
\[ c_1 g_1 + c_2 g_2 = (c_1 q_1^1+ c_2 q_2^1) f_1 + \dots + (c_1 q_1^s + c_2 q_2^s) f_s + (c_1 r_1 + c_2 r_2) \]
Como los monomios de $c_1 q_1^i+ c_2 q_2^i$ son monomios de $q_1^i$ ó $q_2^i$, son monomios que están $Δ_i$. Igualmente, los monomios de $c_1 r_1 + c_2 r_2$ están en $\overline{Δ}$.

Por el apartado (d) del ejercicio \ref{ejer:2.3.12}, esto quiere decir que la expresión anterior es el resultado de la división.
En particular, el resto de la división es efectivamente $c_1 r_1 + c_2 r_2$.
\end{solucion}

\end{document}
