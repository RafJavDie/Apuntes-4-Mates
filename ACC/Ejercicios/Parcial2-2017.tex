\documentclass[twoside]{article}
\usepackage{../../estilo-ejercicios}

%--------------------------------------------------------
\begin{document}

\title{2º Parcial - 16/5/2017}
\author{Diego Pedraza López, Javier Aguilar Martín, Rafael González López}
\maketitle

\begin{ejercicio}{1}
Supongamos que $f, g \in \C[x]$ son polinomios de grado positivo.
El propósito de este ejercicio es construir un polinomio cuyas raíces sean la suma de una raíz de $f$ más una raíz de $g$.
\begin{enumerate}
\item Probar que un número complejo $γ \in \C$ se puede escribir $γ = α + b$, con $f(α) = g(β) = 0$, si y sólo si el sistema de ecuaciones $f(x) = g(y-x) = 0$ tiene solución con $y = γ$.
\item Demostrar que $γ$ es raíz de $Res(f(x), g(y-x), x)$ si y sólo si $γ = α + β$, con $f(α) = g(β) = 0$.
\item Usar lo anterior para construir un polinomio con coeficientes en $\Q$ que tenga a $\sqrt{2} + \sqrt{3}$ como raíz.
\item Modificar la construcción anterior para crear un polinomio cuyas raíces sean todas las diferencias de una raíz de $f$ menos un raíz de $g$.
\end{enumerate}
\end{ejercicio}
\begin{solucion}\mbox{}
\textbf{NO ENTRA}.

%\begin{enumerate}
%\item Lo demostramos por doble implicación.
%Supongamos primero que $γ = α + β$ con $f(α) = g(β) = 0$.
%Entonces, claramente $β = γ - α$.
%Luego $f(x) = g(y-x) = 0$ tiene solución cuando $x = α$ e $y = γ$.

%Por otro lado, si $f(x) = g(γ-x) = 0$ tiene solución con $x = α$, entonces tomando $β = γ-x$ tenemos $f(α) = g(β) = 0$.
%\end{enumerate}
\end{solucion}

\newpage

\begin{ejercicio}{2}
El objetivo de este ejercicio es demostrar la identidad
\[ 0 = h_k(x_k, \dots, x_n) + \sum_{i=1}^k (-1)^i h_{k-i}(x_k,\dots,x_n) σ_i(x_1,\dots,x_n) \]
Poniendo $σ_0 = 1$, la identidad se escribe más compacta
\[ 0 = \sum_{i=0}^k (-1)^i h_{k-i}(x_k,\dots,x_n) σ_i(x_1,\dots,x_n) \]
Si $S \subset \{1,\dots,k-1\}$, notaremos por $x^S$ al producto de las variables correspondientes y por $|S|$ al número de elementos de $S$.
\begin{enumerate}
\item Probar que
\[ σ_i(x_1,\dots,x_n) = \sum_{S \subset \{1,\dots,k-1\}} x^S σ_{i-|S|}(x_k,\dots,x_n), \]
tomando $σ_j = 0$ si $j < 0$.
\item Probar que
\begin{align*}
\sum_{i=0}^k (-1)^i h_{k-i}(x_k,\dots,x_n) σ_i(x_1,\dots,x_n)\\
= \sum_{S \subset \{1,\dots,k-1\}} x^S \left(\sum_{i=|S|} (-1)^i h_{k-i}(x_k,\dots,x_n) σ_{i-|S|}(x_k,\dots,x_n)\right)
\end{align*}
\item Usar la identidad $0 = \sum_{i=0}^k (-1)^i h_{k-i}(x_1,\dots,x_n) σ_i(x_1,\dots,x_n)$ para concluir que la suma dentro de los paréntesis es cero para todo $S$.
Esto termina la prueba de la identidad deseada.
\end{enumerate}
\end{ejercicio}
\begin{solucion}
Ver solución en Relación 7.3, ejercicio 11.
\end{solucion}
\end{document}
