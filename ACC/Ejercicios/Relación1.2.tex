\documentclass[twoside]{article}
\usepackage{../../estilo-ejercicios}

%--------------------------------------------------------
\begin{document}

\title{Ejercicios de Ideals, Varieties, and Algorithms (4ª Edición)}
\author{Diego Pedraza López}
\maketitle

\begin{ejercicio}{1.2.1}
Sketch the following affine varieties in $\R^2$.
\begin{enumerate}[a.]
\item $\V(x^2+4y^2+2x-16y+1)$.
\item $\V(x^2-y^2)$.
\item $\V(2x+y-1,3x-y+2)$.
\end{enumerate}
\end{ejercicio}
\begin{solucion}
\begin{enumerate}[a.]
\item Elipse. Véase $x^2+4y^2+2x-16y+1=0$ es equivalente a $(x+1)^2+(2y-4)^2=4^2$.
\item Par de rectas $x+y=0$ y $x-y=0$.
\item Punto $(-0.2,1.4)$.
\end{enumerate}
\end{solucion}

\begin{ejercicio}{1.2.2}
In $\R^2$, sketch $\V(y^2-x(x-1)(x-2))$.
For which $x$'s is it possible to solver for $y$?
How many $y$'s correspond to each $x$?
What symmetry does the curve have?.
\end{ejercicio}
\begin{solucion}
Véase que para $x<0$, $-x(x-1)(x-2)>0$, luego no existe solución en $x<0$.
Lo mismo ocurre para $x$ en $(1,2)$.
Además, obsérvese que la curva tiene simetría respecto del eje de abcisa.
Estamos ante una curva elíptica.
\end{solucion}

\begin{ejercicio}{1.2.3}
In the plane $\R^2$, draw a picture to illustrate
\[ \V(x^2+y^2-4) \cap \V(xy-1) = \V(x^2+y^2-4,xy-1) \]
and determine the points of intersection. Note that this is a special case of Lemma 2.
\end{ejercicio}
\begin{solucion}
De $xy-1=0$, obtenemos $x=1/y$. Entonces $x^2+1/x^2=4$, luego $x^4-4x^2+1=0$.
\[ x^2 = \frac{4 \pm \sqrt{16-4}}{2} = 2 \pm \sqrt{3} \]
\end{solucion}

\begin{ejercicio}{1.2.4}
Sketch the following affine varieties in $\R^3$.
\begin{enumerate}[a.]
\item $\V(x^2+y^2+z^2-1)$.
\item $\V(x^2+y^2-1)$.
\item $\V(x+2,y-1.5,z)$.
\item $\V(xz^2-xy)$.
\item $\V(x^4-zx,x^3-yx)$.
\item $\V(x^2+y^2+z^2-1, x^2+y^2+(z-1)^2-1)$.
\end{enumerate}
\end{ejercicio}
\begin{solucion}
\begin{enumerate}[a.]
\item Esfera centrada en el origen de radio $1$.
\item Cilindro con eje en el eje $z$ de radio $1$.
\item Punto $(-2,1.5,0)$.
\item Unión de plano $x=0$ y cilindro parabólico $y=z^2$.
\item Unión del plano $x=0$ y la curva $(t^3,t^3,t)$.
\item Circunferencia intersección de dos esferas.
\end{enumerate}
\end{solucion}

\begin{ejercicio}{1.2.5}
Use the proof of Lemma 2 to sketch $\V((x-2)(x^2-y),y(x^2-y),(z+1)(x^2-y))$ in $\R^3$.
\end{ejercicio}
\begin{solucion}
Tenemos que:
\[ \V((x-2)(x^2-y),y(x^2-y),(z+1)(x^2-y)) = \V(x^2-y) \cup \V(x-2,y,z+1) \]
Luego tenemos un cilindro parabólico $y=x^2$ y el punto $(2,0,-1)$.
\end{solucion}

\begin{ejercicio}{1.2.6}
Let us show that all finite subsets of $k^n$ are affine varieties.
\begin{enumerate}[a.]
\item Prove that a single point $(a_1,\dots,a_n) \in k^n$ is an affine variety.
\item Prove that every finite subset of $k^n$ is an affine variety.
\end{enumerate}
\end{ejercicio}
\begin{solucion}
Tenemos que $\{(a_1,\dots,a_n)\} = \V(x_1-a_1,x_2-a_2,\dots,x_n-a_n)$, luego todo punto es una variedad afín.
Todo subconjunto finito de $k^n$ es unión de puntos.
Como los puntos son variedades afines y la unión de variedades afines es variedad afín, todo subconjunto finito de $k^n$ es variedad afín.
\end{solucion}

\begin{ejercicio}{1.2.7}
One of the pretties examples from polar coordinates is the four-leaved rose.
This curve is defined by the poar equation $r = \sin(2θ)$. We will show that this curve is an affine variety.
\begin{enumerate}[a.]
\item Using $r^2=x^2+y^2$, $x=r \cos θ$ and $y = r \sin θ$, show that the four-leaved rose is contained in the affine variety $\V((x^2+y^2)^3-4x^2y^2)$.
\item Now argue carefully that $\V((x^2+y^2)^3-4x^2y^2)$ is contained in the four-leaved rose.
This is trickier that it seems since $r$ can be negative in $r = \sin(2θ)$.
\end{enumerate}
Combining parts $(a)$ and $(b)$, we have proved that the four-leaved rose is the affine variety $\V((x^2+y^2)^3-4x^2y^2)$.
\end{ejercicio}
\begin{solucion}
\begin{enumerate}[a.]
\item Tenemos que
\[ (x^2+y^2)^3-4x^2y^2=r^6-4(r\cosθ)^2(r\sinθ)^2=r^6-4r^4\cos^2θ\sin^2θ \]
Como $\sin(2θ)=2\sinθ\cosθ$:
\[ r^6-4r^4\cos^2θ\sin^2θ = r^6-r^4 (\sin2θ)^2 = r^6 - r^4 r^2 = 0 \]
Luego la rosa de cuatro hojas está contenida en $\V((x^2+y^2)^3-4x^2y^2)$.
\end{enumerate}
\end{solucion}
\end{document}