\documentclass{article}
\usepackage{amsmath,accents}%
\usepackage{amsfonts}%
\usepackage{amssymb}%
\usepackage{comment}
\usepackage{graphicx}
\usepackage{mathrsfs}
\usepackage[utf8]{inputenc}
\usepackage{amsfonts}
\usepackage{amssymb}
\usepackage{graphicx}
\usepackage{mathrsfs}
\usepackage{setspace}  
\usepackage{amsthm}
\usepackage{nccmath}
\usepackage[spanish]{babel}
\usepackage{multirow}
\theoremstyle{plain}

\renewcommand{\baselinestretch}{1,4}
\setlength{\oddsidemargin}{0.5in}
\setlength{\evensidemargin}{0.5in}
\setlength{\textwidth}{5.4in}
\setlength{\topmargin}{-0.25in}
\setlength{\headheight}{0.5in}
\setlength{\headsep}{0.6in}
\setlength{\textheight}{8in}
\setlength{\footskip}{0.75in}

\newtheorem{theorem}{Teorema}[section]
\newtheorem{acknowledgement}{Acknowledgement}
\newtheorem{algorithm}{Algorithm}
\newtheorem{axiom}{Axiom}
\newtheorem{case}{Case}
\newtheorem{claim}{Claim}
\newtheorem{propi}[theorem]{Propiedades}
\newtheorem{condition}{Condition}
\newtheorem{conjecture}{Conjecture}
\newtheorem{coro}[theorem]{Corolario}
\newtheorem{criterion}{Criterion}
\newtheorem{defi}[theorem]{Definición}
\newtheorem{example}[theorem]{Ejemplo}
\newtheorem{exercise}{Ejercicio}
\newtheorem{lemma}[theorem]{Lema}
\newtheorem{nota}[theorem]{Nota}
\newtheorem{sol}{Solución}
\newtheorem*{sol*}{Solución}
\newtheorem{prop}[theorem]{Proposición}
\newtheorem{remark}{Remark}

\newtheorem{dem}[theorem]{Demostración}

\newtheorem{summary}{Summary}

\providecommand{\abs}[1]{\lvert#1\rvert}
\providecommand{\norm}[1]{\lVert#1\rVert}
\providecommand{\ninf}[1]{\norm{#1}_\infty}
\providecommand{\numn}[1]{\norm{#1}_1}
\providecommand{\gabs}[1]{\left|{#1}\right|}
\newcommand{\bor}[1]{\mathcal{B}(#1)}
\newcommand{\R}{\mathbb{R}}
\newcommand{\Q}{\mathbb{Q}}
\newcommand{\Z}{\mathbb{Z}}
\newcommand{\F}{\mathbb{F}}
\newcommand{\X}{\chi}
\providecommand{\Zn}[1]{\Z / \Z #1}
\newcommand{\resi}{\varepsilon_L}
\newcommand{\cee}{\mathbb{C}}
\providecommand{\conv}[1]{\overset{#1}{\longrightarrow}}
\providecommand{\gene}[1]{\langle{#1}\rangle}
\providecommand{\convcs}{\xrightarrow{CS}}
% xrightarrow{d}[d]
\setcounter{exercise}{0}
\newcommand{\cicl}{\mathcal{C}}
%--------------------------------------------------------
\begin{document}

\title{Ejercicios semanales TCYC - Tercera entrega }
\author{Rafael González López}
\maketitle
\begin{exercise} Sea $\alpha(s)$ una curva parametrizada naturalmente en una superficie $\chi : U \rightarrow \R^3$. Probar
que la normal intrínseca S de $\alpha$ es paralela a lo largo de $\alpha$ si y solo si $\alpha$ es una geodésica.
\end{exercise}
\begin{sol*}
Supongamos que $\alpha$ es geodésica. En tal caso $n(s)=\pm N(s)$. Entonces:
\[
b(s) = t(s)\times n(s)= \pm t(s)\times N(s) = \pm S(s) \Rightarrow S'(s) = \pm \tau(s)k(s) \Rightarrow S'(s) \parallel N(s)
\]
Recíprocamente, si $S'(s)||N(s)$ entonces $t(s)S'(s)$, de lo que deducimos que $t'(s)S(s)=0$, pero $t'(s) = n(s)k(s)$. Por tanto, $n(s)$ es perpendicular a $t(s)$ y $S(s)$, luego $n(s)\parallel N(s)$.
\end{sol*}


\newpage
\begin{exercise}(a). Probar que una curva regular $\alpha(s)$, $s=p.n.$ con $\tau(s)\neq 0$, $\forall s\in (a,b)$ es geodésica en la superficie $\X:(a,b)\times\R \rightarrow\R^3$ dada por:
\[
\X(s,v)=\alpha(s)+v t(s) + v\frac{k(s)}{\tau(s)}b(s)
\]
Sean $X(s),Y(s)$ los campos vectoriales $\X_1$ y $\X_2$ a lo largo de $\alpha$ respectivamente. Estudiar bajo qué condiciones cada uno de estos campo es paralelo a a lo largo de $\alpha$.
\end{exercise}
\begin{sol*}
\begin{itemize}
\item[]
\item Veamos el primer apartado:
\begin{gather*}
\chi_1(s,v) = t(s) +v(k(s)n(s)+v\left(\frac{k(s)}{\tau(s)}b(s)\right)' \qquad \chi_2(s,v) = t(s)+\frac{k(s)}{\tau(s)}b(s)\\
\X_1(s,0)\times\X_2(s,0) = \left(0,-\frac{k(s)}{\tau(s)},0\right) \Rightarrow n(s) \parallel N(s)
\end{gather*}
\item Veamos ahora, por casos, el segundo aparatado:
\begin{itemize}
\item $\chi_1(s)$ es paralelo si y solo sí $X_1'(s) = t'(s) = k(s)n(s) \parallel N(s)$. Por tanto, como $\alpha$ es geodésica, se cumple siempre.
\item $\chi_2(s)$ es paralelo si y solo sí $X_2'(s) = \left(\dfrac{k(s)}{\tau(s)}\right)'b(s)\parallel N(s)$, esto es, al ser $\alpha$ geodésica, si y solo si $\dfrac{k(s)}{\tau(s)}\equiv cte$, es decir, si y solo sí es un hélice.
\end{itemize}
\end{itemize}
\end{sol*}

\newpage
\begin{exercise} Sea $\alpha(s)$ una curva regular parametrizada naturalmente sobre una superficie. Encontrar qué condición ha de verificar la función $\lambda(s)$ para que se verifique lo siguiente: ``$\alpha(s)$ es una geodésica si y sólo si $t(s) + \lambda(s)S(s)$ es paralelo a lo largo de $\alpha(s)$".
\end{exercise}
\begin{sol*}
La condición es que $\lambda(s)\equiv cte$. Veámoslo.
\begin{itemize}
\item Supongamos que $\alpha(s)$ es geodésica. En tal caso $S(s)$ y $t(s)$ son campos paralelos a lo largo de $\alpha$. Como los campos paralelos forman un $R$-e.v. deducimos que $t(s)+\lambda S(s)$ también es un campo paralelo.
\item Supongamos que $t(s)+\lambda S(s)$ es un campo paraleo. En tal caso:
\begin{gather*}
(t(s)+\lambda S(s))' = t'(s) + \lambda(N'(s)\times t(s) + N(s)\times t'(s))= \\
K_g(s) S(s) + K_n(s)N(s)+ \lambda \tau_g(s) N(s) - \lambda K_g(s)t(s) \parallel N(s)
\end{gather*}
Por tanto $K_g(s) = 0$, luego es geodésica.
\end{itemize}
\end{sol*}

\newpage
\begin{exercise}
Sea M una superficie, $\alpha:(a,b)\rightarrow \R^3$ una curva regular en M, y sea $X(t)$ un campo vectorial paralelo a lo largo de $\alpha$. Sea $Y(t)$ un campo vectorial a lo largo de $\alpha$ con $|Y(t)|=cte$ y $\theta(t)=\widehat{(X(t),Y(t))} = cte$. Probar que $Y(t)$ es paralelo a lo largo de $\alpha$. (b) Sean $Z(t),W(t)$ campos paralelos no colineales a lo largo de $\alpha$ y $F(t)=\lambda Z(t)+\mu(t) W(t)$ un campo arbitrario a lo largo de $\alpha$. Estudiar condiciones para que $F(t)$ sea paralelo.
\end{exercise}
\begin{sol*}
\begin{itemize}
\item[]
\item Consideramos $Q(t)$ un campo paralelo que sea perpendicular a $X(t)$ en cada punto y que además $|X(t)|=|Q(t)|$. En tal caso, dadas las hipótesis sobre $Y(t)$, existe un escalar $a\in\R$ tal que:
\[
Y(t) = a(X(t)cos(\theta) + Y(t)sin(\theta))
\]
Por tanto, $Y(t)$ es un campo paralelo.
\item Derivando directamente:
\begin{gather*}
F'(t) = (a(t)\mu(t)+b(t)\lambda(t))N(t) + \mu'(t)W(t)+\lambda'(t)Z(t) = s(t)N(t)\\
 \mu'(t)W(t)+\lambda'(t)Z(t) = 0 \Leftrightarrow \mu'(t)=\lambda'(t)=0
\end{gather*}
\end{itemize}
\end{sol*}

\newpage
\begin{exercise}
Sea $\X:U\rightarrow \R^3$ una superficie simple, $\alpha$ una curva en $\X$ con $k\neq 0$ y $b_T$ la proyección sobre el plano tangente a $\X$. del vector binormal $b$. (a) Probar que $b_T = -\frac{k_n}{k}S$. (b) Probar que son equivalente las condiciones:
\begin{enumerate}
\item $b_T = b$
\item $\alpha$ es geodésica.
\item $b_T \neq 0$ y $b_T$ es paralelo a lo largo de $\alpha$.
\end{enumerate}
\end{exercise}
\begin{sol*}
Dado que el plano tangente tiene una base ortonormal $\{t,S\}$ y que $tb=0$, deducimos que $b_T \parallel S$. De hecho,
\[
b_T = (bS)S = (b N\times t)S = (N t \times b)S = -(N n)S = -\frac{k_n}{k}S
\]
Pasemos a demostrar las equivalencias.
\begin{itemize}
\item Si $b_T = b$ entonces $b\parallel S$, luego multiplicando vectorialmente por $t$ nos da la caracterización de geodésica.
\item Si $\alpha$ es geodésica entonces $n\parallel N$, luego $b\parallel S$, dado que $b_T || b$, han de ser iguales y, en particular $b_T\neq 0$. Además, usando que $b_T'(s) = b'(s)=S'(s)=-K_g(s)t(s)+ \tau_g(s)N(s)$. Al ser $\alpha$ geodéisca, $K_g\equiv 0$. Deducimos que $b\parallel N$.
\item Si $b_T$ es campo paralelo entonces $b_T'\parallel N$.
\begin{gather*}
b_T' = (b'S+bS')S+(bs)S' \perp S,t \Rightarrow 
\begin{cases}
b'S+bS' = 0\\
(bS)(S't) = 0 \overset{b_T\neq 0}{\Rightarrow} S't = 0 \Rightarrow t'S = 0
\end{cases}\\
t'S = 0 \Rightarrow nS=0 \overset{nt=0}{\Rightarrow} n\parallel N \Rightarrow b\parallel S \Rightarrow b=b_T
\end{gather*}
\end{itemize} 
\end{sol*}


\newpage
\begin{exercise}
Sea $\X:U\rightarrow\R^3$ una superficie simple, $\alpha$ una curva regular en $\X$. Sean $t(s)$ y $S(s)$ los vectores tangente y normal intrínseca de $\alpha$ respectivamente, s natural de $\alpha$. Se considera un campo vectorial a lo largo de $\alpha$: $X(s)=\lambda(s)t(s)+\mu(s)S(s)$. (a): Encontrar las condiciones que deben verificar $\lambda(s)$ y $\mu(s)$ para que $X(s)$ sea paralelo a lo largo de $\alpha$. ¿Qué ocurre si $\alpha$ es geodésica? (b) Aplicar el estudio realizado en (a) al caso en que $\X$ sea un plano, comprobando que X es paralelo si y solo sí X es constante.
\end{exercise}
\begin{sol*}
\end{sol*}
\end{document}