\documentclass[twoside]{article}
\usepackage{../estilo-ejercicios}
\newcommand{\x}{{\mathbf{x}}}
\newcommand{\y}{{\mathbf{y}}}
%--------------------------------------------------------
\begin{document}

\title{Geometría Aplicada}
\author{Javier Aguilar Martín, Rafael González López}
\maketitle

\begin{ejercicio}{diapositiva 67}
Sean $P_{0,(1)}=(0,0),P_{1,(1)}=(-1,1), P_{2,(1)}=(1,3),P_{0,(2)}=(1,3),P_{1,(2)}=(2,4),P_{2,(2)}=(5,1),P_{0,(3)}=(5,1),P_{1,(3)}=(6,0),P_{2,(3)}=(7,5)$ los puntos de control de tres curvas de Bézier simples. Construir la curva compuesta de clase $\mathcal{C}^1$. 
\end{ejercicio}
\begin{solucion}
Comprobamos primero que es $\mathcal{G}^1$ ya que es condición necesaria.  
\begin{gather*}
P_{2,(1)}-P_{1,(1)}=(2,2)\parallel (1,1)=P_{1,(2)}-P_{0,(2)}\\
P_{2,(2)}-P_{1,(2)}=(3,-3)\parallel (1,-1)=P_{1,(3)}-P_{2,(3)}
\end{gather*}
La curva compuesta será de la forma 
\[
C(u)=\begin{cases}
C_1(\frac{u-u_0}{u_1-u_0}) & u\in[u_0,u_1]\\
C_2(\frac{u-u_1}{u_2-u_1}) & u\in[u_1,u_2]\\
C_3(\frac{u-u_2}{u_3-u_2}) & u\in[u_2,u_3]\\
\end{cases}
\]
Para que la curva sea $\mathcal{C}^1$
\begin{gather*}
\frac{u_2-u_1}{u_1-u_0}=\frac{||P_{1,(2)}-P_{0,(2)}||}{||P_{2,(1)}-P_{1,(1)}}=\frac{1}{2}\\
\frac{u_3-u_2}{u_2-u_1}=\frac{||P_{1,(3)}-P_{0,(3)}||}{||P_{2,(2)}-P_{1,(2)}}=\frac{1}{3}
\end{gather*}
Escogemos por ejemplo $u_0=0,u_1=6,u_2=9,u_3=10$, o si queremos tenerla parametrizada en el $[0,1]$, elegimos $u_0=0,u_1=0.6,u_2=0.9,u_3=1$. En el examen recuerda comprobar que efecticamente es $\mathcal{C}^1$ al reparametrizarla.
\end{solucion}

\newpage

\begin{ejercicio}{3.1}
Hallar los coeficientes del polinomio $p(t)= 1-t+2t^2+t^3$ en las bases de Bernstein $\{B_i^3(t)\}_{i=0}^3$ y $\{B_i^4(t)\}_{i=0}^4$.
\end{ejercicio}
\begin{solucion}
Tomamos coordenadas de los $B_i^k(t)$ respecto de la base $\beta_1 =\{1,t,t^2,t^3\}$ y la base $\beta_2 = \{1,t,t^2,t^3,t^4\}$.
\begin{align*}
B_0^3(t) &= 1 - 3t +3t^2 -t^3 & B_0^4(t) &= 1 -4t+6t^2-4t^3+t^4\\
B_1^3(t) & = 3t-6t^2+3t^3 & B_1^4(t) &= 4t-12t^2+12t^3-4t^4\\
B_2^3(t) &= 3t^2 -3t^3 & B_2^4(t) &= 6t^2-12t^3+6t^4\\
B_3^3(t) &= t^3 & B_3^4(t) &=4t^3-4t^4\\
 & & B_4^4(t) &= t^4
\end{align*}
Lo que nos deja las siguientes matrices de cambio de base con respecto a las bases canónicas
$$
M_1 = 
\begin{pmatrix}
1 & 0 & 0 & 0\\
-3 & 3 & 0 &0\\
3 & -6 & 3 & 0\\
-1 & 3 & -3 & 1
\end{pmatrix} \qquad \qquad  M_2 = 
\begin{pmatrix}
1  & 0   & 0   & 0  & 0\\
-4 & 4   & 0   & 0  & 0\\
6  & -12 & 6   & 0  & 0\\
-4 & 12  & -12 & 4  & 0\\
1  & -4  & 6   & -4 & 1
\end{pmatrix}
$$ 
Pasamos a calcular directamente las coeficientes
\begin{gather*}
M_1^{-1}\begin{pmatrix}
1\\
-1\\
2\\
1
\end{pmatrix}
= 
\begin{pmatrix}
1\\
2/3\\
1\\
3
\end{pmatrix} \qquad \qquad M_2^{-1}  \begin{pmatrix}
1\\
-1\\
2\\
1\\
0
\end{pmatrix} = 
 \begin{pmatrix}
1\\
3/4\\
5/6\\
3/2\\
3
\end{pmatrix} 
\end{gather*}
\end{solucion}

\newpage

\begin{ejercicio}{3.2}
Si denotamos por $\{x_0,\dots,x_n\}$ los coeficientes de un polinomio en la base de Bernstein $\{B^n_i(t)\}_{i=0}^n$, demostrar que $\sum_{i=0}^nx_iB_i^n(t)$ es un polinomio de grado $n-1$ si y solo si $\sum_{i=0}^n(-1)^i\binom{n}{i}x_i=0$.
\end{ejercicio}
\begin{solucion}
Por definición, 
\begin{gather*}
\sum_{i=0}^nx_iB_i^n(t)=\sum_{i=0}^nx_i\binom{n}{i}(1-t)^{n-i}=\sum_{i=0}^nx_i\binom{n}{i}\sum_{j=0}^{n-i}\binom{n-i}{j}1^j(-t)^{n-i-j}=\\
\sum_{i=0}^nx_i\binom{n}{i}\sum_{j=0}^{n-i}\binom{n-i}{j}(-1)^{n-i-j}(-t)^{n-j}
\end{gather*}
El coeficiente de $t^n$ será $\sum_{i=0}^nx_i\binom{n}{i}(-1)^{n-i}=0\Leftrightarrow\sum_{i=0}^n(-1)^i\binom{n}{i}x_i=0$.
\end{solucion}

\end{document}