\documentclass[12pt,a4paper]{amsart}
\usepackage[spanish]{babel}
\usepackage[utf8]{inputenc}
\usepackage{tikz-cd}  
\usetikzlibrary{babel} 
\usepackage{pgf,tikz}
%\usepackage{mathrsfs}
\usetikzlibrary{arrows}
\usetikzlibrary{cd} 
\usepackage[]{graphicx,wrapfig}
%\usepackage[all]{xy}
%\CompileMatrices
%\OnlyOutlines
%\ShowOutlines
\usepackage{amsmath,amssymb,varioref,enumerate}
%\usepackage{emlines2}
%\usepackage[dvipsone]{graphicx}
%\usepackage{epsfig}
%\usepackage{psfrag}
%\newdir{ >}{{}*!/-5pt/\dir{>}}
%\externaldocument{}
%begin numlast%\newcommand{\lga}{\longrightarrow}
\newcommand{\lgaf}{\longleftarrow}
\newcommand {\sub}{\subset}
\newtheorem{dummy}{realdumb}[section]
\newtheorem{theorem}[dummy]{Theorem}
\newtheorem{lemma}[dummy]{Lemma}
\newtheorem{corollary}[dummy]{Corollary}
\newtheorem{proposition}[dummy]{Proposition}
\newtheorem*{theoremun}{Theorem}
\newtheorem*{corollaryun}{Corollary}
\theoremstyle{definition}               %%Change Theoremstyle
\newtheorem{definition}[dummy]{Definition}
\newtheorem{conjecture}[dummy]{Conjecture}
\newtheorem{question}[dummy]{Question}
\newtheorem{example}[dummy]{Example}
%\newtheorem{nothing}[dummy]{Ejercicio}
\newtheorem{nothing}{Ejercicio}
%\theoremstyle{remark}
\newtheorem{remark}[dummy]{Remark}
\newenvironment{display}{\refstepcounter{dummy} $$}%
{\leqno{\rm ({\thedummy})} $$} \numberwithin{equation}{dummy}
\theoremstyle{plain}
%end numlast
\DeclareMathOperator{\co}{co} \DeclareMathOperator{\Diag}{Diag}
\DeclareMathOperator{\Ar}{Ar} \DeclareMathOperator{\Hom}{Hom}
\DeclareMathOperator{\coker}{coker} \DeclareMathOperator{\im}{Im}
\DeclareMathOperator{\R}{\mathbb R}
\DeclareMathOperator{\N}{\mathbb N}
\newcommand{\loc}[2]{{\mathcal#1}^{-1}{\mathcal#2}}
\newcommand{\cat}[1]{\mathcal#1}
\newcommand{\scat}[1]{\overset{\sim}{\mathcal#1}}
\newcommand{\ccat}[2]{{\mathcal #1}^{\wedge #2}}
\newcommand{\complex}[1]{C({\mathcal#1})}
\newcommand{\ccomp}[2]{C\left({\mathcal#1}^{\wedge#2}\right)}
\newcommand{\quot}[2]{\left.{\mathcal#1}\!\right/\negmedspace{\mathcal#2}}
\newcommand{\cquot}[2]{C\left(\left.{\mathcal#1}\!\right/\negmedspace{\mathcal#2}\right)}
\newcommand{\ccquot}[3]{C\left(\left.
{\mathcal#1}^{\wedge#3}\right/\negmedspace{\mathcal#2}^{\wedge#3}\right)}
\newcommand{\qquot}[3]{{\mathcal#1}^{\wedge#3}\negmedspace
\left.\right/\negmedspace{\mathcal#2}^{\wedge#3}}
\newenvironment{solucion}{\begin{trivlist}
	\item[\hskip \labelsep {\textit{Solución}.}\hskip \labelsep]}{\end{trivlist}}
\title{Geometr\'{\i}a y Topolog\'{\i}a de Superficies 2016/17. Relaci\'on 4.}
%\newcommand{\cat}[1]{\mathcal#1}
\begin{document}

\maketitle



\begin{nothing} Probar que la yuxtaposici\'on de caminos satisface la siguiente propiedad de cancelaci\'on: si $f_0*g_0 \sim f_1* g_1$ y $g_0 \sim g_1$, entonces $f_0 \sim f_1$.
\end{nothing}
\begin{solucion}
Como $g_0 \sim g_1\Leftrightarrow \overline{g_0}\sim\overline{g_1}$, usamos la compatibilidad de la yuxtaposición con la equivalencia de caminos.
\[
f_0*g_0 \sim f_1* g_1\Leftrightarrow f_0*g_0*\overline{g_0}\sim f_1*g_1*\overline{g_1}\Leftrightarrow f_0\sim f_1.
\]
\end{solucion}

\vspace{0.1cm}

\begin{nothing} \label{s1}Probar que todo lazo $\alpha$ en $x_0\in X$ define una aplicaci\'on continua $f_\alpha: S^1 \to X$ con $f_\alpha((1,0)) = x_0$ y rec\'{\i}procamente. M\'as a\'un, si $\alpha \sim \alpha'$ entonces $f_{\alpha} \simeq f_{\alpha'} \mbox{ rel. } (0,1)$.
\end{nothing}
\begin{solucion}
Observemos que $S^1$ se puede considerar como $[0,1]/\sim$ con la relación generada por $0\sim 1$. Por tanto, si denotamos por $\pi$ a la aplicación cociente e $I=[0,1]$ obtenemos el siguiente diagrama conmutativo:
\[
\begin{tikzcd}
I\arrow[r,"\alpha"]\arrow[d,"\pi"'] & X\\
S^1\arrow[ur,"\alpha\circ\pi^{-1}"']
\end{tikzcd}
\]
Por lo que $f_\alpha=\alpha\circ\pi^{-1}$. Por otro lado, como $\alpha\sim\alpha'$ implica que la homotopía es relativa al $\{0,1\}$, en el cociente esto implica que es relativa a $[0]=[1]=(0,1)$. 
\end{solucion}

\vspace{0.1cm}

\begin{nothing} Probar que para un espacio conexo por caminos $X$ las tres condiciones siguientes son equivalentes:
(a) Toda aplicaci\'on continua $f: S^1\to X$ es homot\'opica a una constante; (b) toda aplicaci\'on continua $f: S ^1 \to X$ se extiende al disco $D^2$;
(c) $\pi_1(X,x_0) = 0$ para todo $x_0\in X$.
\end{nothing}
\begin{solucion}
$\boxed{(a)\Rightarrow(b)}$ Por hipótesis existe una homotopía $H:S^1\times I\to X$ cumpliendo $H(x,0)=f(x)$ y $H(x,0)=x_0\in X$. Podemos pensar en esta homotopía como una aplicación del cilindro a $X$ en la que toda la tapa superior tiene la misma imagen. Por tanto podemos factorizar a través del cono, que sabemos que es homeomorfo mediante un homemomorfismo que llamaremos $g$ a $D^2$.
\[
\begin{tikzcd}
S^\times I\arrow[r,"H"]\arrow[d,"\pi"']& X\\
CS^1\arrow[ur,"\tilde{\pi}"]\arrow[d,"g"',"\cong"]\\
D^2\arrow[uur,"\tilde{\pi}\circ g^{-1}"']
\end{tikzcd}
\]
Por tanto $f$ se extiende mediante $\tilde{\pi}\circ g^{-1}$.\\
$\boxed{(b)\Rightarrow(c)}$ Sea $i:S^1\hookrightarrow D^2$ la inclusión y volvamos a considerar el homeomorfismo $g$ del apartado anterior. Supongamos que $f$ se extiende mediante una función $h$. Entonces obtenemos el siguiente diagrama conmutativo:
\[
\begin{tikzcd}
CS^1\arrow[ddr,"g^{-1}\circ h"]\\
D^2\arrow[u,"g"]\arrow[dr,"h"]\\
S^1\arrow[u,"i",hookrightarrow]\arrow[r,"f"]& X
\end{tikzcd}
\]
Dado un lazo, podemos extenderlo a $D^2$ y llevarlo al cono, contrayéndolo sobre el vértice. Por lo tanto el lazo es homotópico al constante. Como $X$ es conexo por caminos, esta construcción no depende del punto base y por tanto $\pi_1(X,x_0)=0$.\\
$\boxed{(c)\Rightarrow(a)}$ Se tiene que $f:S^1\to X$ identifica a un lazo en $X$, y como $\pi_1(X,x_0)=0$ concluimos que $f\sim x_0$.
\end{solucion}

\vspace{0.1cm}

\begin{nothing}
Probar que $\pi_1(X,x_0) = 0$ si y s\'olo para todo $x\in X$ dos caminos cualesquiera entre $x_0$ y $x$ son equivalentes.
\end{nothing}
\begin{solucion}
$\boxed{\Rightarrow}$ Sea $\gamma_1$ y $\gamma_2$ dos caminos entre $x_0$ y $x$. Entonces $\gamma_1*\overline{\gamma_2}$ es un lazo en $x_0$. Como $\pi_1(X,x_0)=0$ se tiene
\[
\gamma_1*\overline{\gamma_2}\sim c_{x_0}\Leftrightarrow \gamma_1\sim\gamma_2,
\]
como queríamos demostrar.\\
$\boxed{\Leftarrow}$ Tomando $x=x_0$ tenemos que todos los lazos son equivalentes. En particular son equivalentes al lazo constante $c_{x_0}$, por lo que se deduce que $\pi_1(X,x_0)=0$.
\end{solucion}

\vspace{0.1cm}

\begin{nothing} Sea $A\subseteq X$ la componente conexa por caminos del punto $x_0\in X$. Probar que la inclusi\'on
$i: A\to X$ induce un isomorfismo $i_*: \pi_1(A,x_0) \to \pi_1(X,x_0)$.
\end{nothing}



\begin{nothing}
Probar que si $X = A\cup B$ con $A$ y $B$ abiertos tales que $A$, $B$ y $A\cap B$ son conexos por caminos
 y $A$ y $B$ simplemente conexos entonces  $X$ es simplemente conexo. Aplicar este resultado para deducir
 que $\pi_1(S^n) = 0$ para todo $n \geq 2$.
\end{nothing}

\begin{nothing}
Dar una familia infinita de espacios conexos por caminos simplemente conexos que no sean homeomorfos.
\end{nothing}
\begin{solucion}
La familia de hiperesferas $\{S^n\}_{n\geq 2}$, que son simplemente conexas como se prueba en el apartado anterior y por el teorema de invariancia del dominio no son homeomorfas. 
\end{solucion}

\vspace{0.1cm}

\begin{nothing} Un espacio conexo por caminos $X$ se dice que es {\it $1$-simple} si
dados $x_0, x_1\in X$, dos caminos de $X$ cualesquiera entre $x_0$ y $x_1$ inducen el mismo
isomorfismo entre $\pi_1(X,x_1)$ y $\pi_1(X,x_0)$.
Probar que $X$ es 1-simple si y s\'olo si $\pi_1(X)$ es abeliano.
\end{nothing}


\begin{nothing}
Sea $[S^1,X]$ el conjunto de las clases de homotop\'{\i}a (no necesariamente relativa) de aplicaciones de $S^1$ en $X$. De acuerdo con el problema \ref{s1} hay definida una aplicaci\'on $\varphi: \pi_1(X,x_0) \to [S^1,X]$, $[\alpha] \mapsto [f_{\alpha}]$. Probar que si $X$ es conexo por caminos se tiene que $\varphi$ es sobreyectiva. M\'as a\'un, si $\varphi ([\alpha]) = \varphi([\beta])$, $[\alpha]$ y $[\beta]$ son elementos conjugados en $\pi_1(X,x_0)$.
Como consecuencia, si $X$ es conexo por caminos existe una biyecci\'on entre $[S^1,X]$ y las clases por conjugaci\'on de $\pi_1(X,x_0)$.
\end{nothing}




\end{document}

\begin{nothing} Consideremos el espacio $X = \{a,b,c,d\}$ con la topolog\'{\i}a $\mathcal{T} = \{\emptyset, X, \{c\},\{d\}, \{c,d\}, \{a,c,d\}, \{b,c,d\}\}$.
\par
 Veamos que $\pi_1(X,d)$ no es el grupo trivial. Para ello, sea $\alpha: [0,1]\to X$ la aplicaci\'on $\alpha(t) = d$ si $t \in [0,\frac{1}{4}) \cup (\frac{3}{4},1]$, $\alpha(t) = c$ si $t\in (\frac{1}{4},\frac{3}{4})$, $f(\frac{1}{4}) = a$ y $f(\frac{3}{4}) = b$.  Probar que $\alpha$ es un lazo en $d$ que no es equivalente al lazo constante. Para esto \'ultimo, suponer lo contrario y si $H$ es una homotop\'{\i}a, probar que $t_0 = \mbox{sup }\{t; c\in H(I\times \{t\})\}$ es un m\'aximo y llegar a la contradicci\'on de que debe ser $t_0 = 1$.
\end{nothing} 

\begin{nothing} Sea $H: X\times I \to X$ una homotop\'{\i}a con $H(x,0) =  H(x,1) = x$ para todo $x\in X$. Si $\gamma$ es el lazo en $x_0$ $\gamma(t) = H(x_0,t)$, probar que la clase $[\gamma]$ conmuta con todo elemento de $\pi_1(X,x_0)$.
\end{nothing}