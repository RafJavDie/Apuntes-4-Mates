%\documentclass[12pt,a4paper,spanish]{amsart}
%\usepackage[margin=1.5cm]{geometry}
%%\usepackage[spanish]{babel}
%%\usepackage[all]{xy}
%%\CompileMatrices
%%\OnlyOutlines
%%\ShowOutlines
%\usepackage{amsmath,amssymb,varioref,enumerate,showkeys}
%%\usepackage{emlines2}
%\usepackage[dvipsone]{graphicx}
%\usepackage{epsfig}
%%\usepackage{psfrag}
%%\newdir{ >}{{}*!/-5pt/\dir{>}}
%%\externaldocument{}
%%begin numlast
%\newcommand{\lga}{\longrightarrow}
%\newcommand{\lgaf}{\longleftarrow}
%\newcommand {\sub}{\subset}
%\newtheorem{dummy}{realdumb}[section]
%\newtheorem{theorem}[dummy]{Theorem}
%\newtheorem{lemma}[dummy]{Lemma}
%\newtheorem{corollary}[dummy]{Corollary}
%\newtheorem{proposition}[dummy]{Proposition}
%\newtheorem*{theoremun}{Theorem}
%\newtheorem*{corollaryun}{Corollary}
%\theoremstyle{definition}               %%Change Theoremstyle
%\newtheorem{definition}[dummy]{Definition}
%\newtheorem{conjecture}[dummy]{Conjecture}
%\newtheorem{question}[dummy]{Question}
%\newtheorem{example}[dummy]{Example}
%%\newtheorem{nothing}[dummy]{Ejercicio}
%\newtheorem{nothing}{Ejercicio}
%%\theoremstyle{remark}
%\newtheorem{remark}[dummy]{Remark}
%\newenvironment{display}{\refstepcounter{dummy} $$}%
%{\leqno{\rm ({\thedummy})} $$} \numberwithin{equation}{dummy}
%\theoremstyle{plain}
%%end numlast
%\DeclareMathOperator{\co}{co} \DeclareMathOperator{\Diag}{Diag}
%\DeclareMathOperator{\Ar}{Ar} \DeclareMathOperator{\Hom}{Hom}
%\DeclareMathOperator{\coker}{coker} \DeclareMathOperator{\im}{Im}
%\DeclareMathOperator{\R}{\mathbb R}
%\DeclareMathOperator{\N}{\mathbb N}
%\newcommand{\loc}[2]{{\mathcal#1}^{-1}{\mathcal#2}}
%\newcommand{\cat}[1]{\mathcal#1}
%\newcommand{\scat}[1]{\overset{\sim}{\mathcal#1}}
%\newcommand{\ccat}[2]{{\mathcal #1}^{\wedge #2}}
%\newcommand{\complex}[1]{C({\mathcal#1})}
%\newcommand{\ccomp}[2]{C\left({\mathcal#1}^{\wedge#2}\right)}
%\newcommand{\quot}[2]{\left.{\mathcal#1}\!\right/\negmedspace{\mathcal#2}}
%\newcommand{\cquot}[2]{C\left(\left.{\mathcal#1}\!\right/\negmedspace{\mathcal#2}\right)}
%\newcommand{\ccquot}[3]{C\left(\left.
%{\mathcal#1}^{\wedge#3}\right/\negmedspace{\mathcal#2}^{\wedge#3}\right)}
%\newcommand{\qquot}[3]{{\mathcal#1}^{\wedge#3}\negmedspace
%\left.\right/\negmedspace{\mathcal#2}^{\wedge#3}}
%\title{Geometr\'{\i}a y Topolog\'{\i}a de Superficies 2016/17\\ Relaci\'on 6.}
\documentclass{article}
\usepackage{amsmath,accents}%
\usepackage{amsfonts}%
\usepackage{amssymb}%
\usepackage{comment}
\usepackage{graphicx}
\usepackage{mathrsfs}
\usepackage[utf8]{inputenc}
\usepackage{amsfonts}
\usepackage{amssymb}
\usepackage{graphicx}
\usepackage{mathrsfs}
\usepackage{setspace}
\usepackage{amsthm}
\usepackage{nccmath}
\usepackage[spanish]{babel}
\usepackage{multirow}
\usepackage{hyperref}
\usepackage{tikz-cd}
\usepackage{pgf,tikz}
\usetikzlibrary{arrows}
\usetikzlibrary{cd}
\usetikzlibrary{babel}
\theoremstyle{plain}
\hypersetup{colorlinks=true,citecolor=red, linkcolor=blue}

\renewcommand{\baselinestretch}{1,4}
\setlength{\oddsidemargin}{0.5in}
\setlength{\evensidemargin}{0.5in}
\setlength{\textwidth}{5.4in}
\setlength{\topmargin}{-0.25in}
\setlength{\headheight}{0.5in}
\setlength{\headsep}{0.6in}
\setlength{\textheight}{8in}
\setlength{\footskip}{0.75in}

\theoremstyle{definition}

\newtheorem{theorem}{Teorema}[section]
\newtheorem{acknowledgement}{Acknowledgement}
\newtheorem{algorithm}{Algorithm}
\newtheorem{axiom}{Axiom}
\newtheorem{case}{Case}
\newtheorem{claim}{Claim}
\newtheorem{propi}[theorem]{Propiedades}
\newtheorem{condition}{Condition}
\newtheorem{conjecture}{Conjecture}
\newtheorem{coro}[theorem]{Corolario}
\newtheorem{criterion}{Criterion}
\newtheorem{defi}[theorem]{Definición}
\newtheorem{example}[theorem]{Ejemplo}
\newtheorem{exercise}{Ejercicio}
\newtheorem{lemma}[theorem]{Lema}
\newtheorem{nota}[theorem]{Nota}
\newtheorem{sol}{Solución}
\newtheorem*{sol*}{Solución}
\newtheorem{prop}[theorem]{Proposición}
\newtheorem{remark}{Remark}

\newtheorem{dem}[theorem]{Demostración}

\newtheorem{summary}{Summary}

\providecommand{\abs}[1]{\lvert#1\rvert}
\providecommand{\norm}[1]{\lVert#1\rVert}
\providecommand{\ninf}[1]{\norm{#1}_\infty}
\providecommand{\numn}[1]{\norm{#1}_1}
\providecommand{\gabs}[1]{\left|{#1}\right|}
\newcommand{\bor}[1]{\mathcal{B}(#1)}
\newcommand{\R}{\mathbb{R}}
\newcommand{\Q}{\mathbb{Q}}
\newcommand{\Z}{\mathbb{Z}}
\newcommand{\F}{\mathbb{F}}
\newcommand{\C}{\mathbb{C}}
\newcommand{\X}{\chi}
\providecommand{\Zn}[1]{\Z / \Z #1}
\newcommand{\resi}{\varepsilon_L}
\newcommand{\cee}{\mathbb{C}}
\providecommand{\conv}[1]{\overset{#1}{\longrightarrow}}
\providecommand{\gene}[1]{\langle{#1}\rangle}
\providecommand{\convcs}{\xrightarrow{CS}}
% xrightarrow{d}[d]
\setcounter{exercise}{0}
\newcommand{\cicl}{\mathcal{C}}

\begin{document}
\title{Relación 6 - Geometría y Topología de superficies }
\author{Javi, Rafa, Diego}
\maketitle

\begin{exercise}

Calcula el grupo fundamental del cilindro $\mathbb{R}\times S^1$.

\end{exercise}



\begin{exercise}

Calcula el grupo fundamental de la suma puntual de dos copias de $S^2$.

\end{exercise}

\begin{exercise}

Calcula el grupo fundamental de la suma puntual de dos copias de $S^1$, y generaliza el resultado al caso de $n$ copias, para un $n$ cualquiera.

\end{exercise}

\begin{exercise}

Calcula el grupo fundamental de la banda de Moebius, y tambi\'en el de la botella de Klein.

\end{exercise}


\begin{exercise}

Calcula el grupo fundamental del \emph{sombrero bobo} (i.e. un tri\'angulo relleno con los lados identificados seg\'un la palabra $aa^{-1}a$), y del plano proyectivo real.

\end{exercise}

\begin{exercise}

Calcula el grupo fundamental de de cualquier superficie compacta sin borde.

\end{exercise}

\begin{exercise}

Calcula el grupo fundamental del toro menos un disco.

\end{exercise}

\begin{exercise}

Calcula el grupo fundamental del toro con una membrana.

\end{exercise}

\begin{exercise}

Calcula el grupo fundamental de la esfera $S^2$ con un di\'ametro.

\end{exercise}

\begin{exercise}

Sean $A$ un conjunto finito de puntos del plano, y $B$ un conjunto finito de puntos del espacio. Calcula los grupos fundamentales de $\mathbb{R}^2\setminus A$ y $\mathbb{R}^3\setminus B$.

\end{exercise}

\begin{exercise}

Calcula el grupo fundamental de los siguientes espacios:

\begin{itemize}

\item $\mathbb{R}^3$ menos una recta.

\item $\mathbb{R}^3$ menos una circunferencia.

\item $\mathbb{R}^3$ menos la uni\'on del eje OZ y la circunferencia unidad del eje $xy$.

\item $\mathbb{R}^3$ menos la uni\'on puntual de dos circunferencias.

\item $\mathbb{R}^3$ menos la uni\'on de dos circunferencias coplanarias disjuntas.

\item $\mathbb{R}^3$ menos la uni\'on de dos circunferencias disjuntas pero engarzadas como eslabones.



\end{itemize}

\end{exercise}



\begin{exercise}

Calcular el grupo fundamental de la uni\'on puntual de un toro con la uni\'on puntual de dos circunferencias.



\end{exercise}


\begin{exercise}

Dados dos toros por las identificaciones $aba^{-1}b^{-1}$ y $cdc^{-1}d^{-1}$, calcular el grupo fundamental del cociente de la uni\'on disjunta de esos toros por la identificaci\'on $a=c$.

\end{exercise}

\begin{exercise}

Consideramos el subespacio $Y$ de $\mathbb{R}^2$ dado por la uni\'on de $X=\{(x,\textrm{sen }1/x);0<x\leq 1\}$ y un arco entre $(0,-1)$ y $(1,0)$ disjunto con $X$. Calcular el grupo fundamental de $Y$.

\end{exercise}

\begin{exercise}
Se sabe que las identificaciones del pr\'{\i}metro de  un pol\'{\i}gono de diez lados dadas por los c\'odigos $abcb^{-1}daec^{-1}e^{-1}d^{-1}$ y $abcbdaece^{-1}d^{-1}$ son
superficies. Determinar sus grupos fundamentales. A partir de la abelianizaci\'on de los mismos determinar el modelo de cada superficie.
\end{exercise}

\begin{exercise}
Usar el abelianizado del grupo fundamental para determinar qu\'e superficies son las representadas por los c\'odigos
$$
a_1a_2\dots a_n a^{-1}_1a^{-1}_2\dots a^{-1}_{n-1}a_n \mbox{ y } a_1a_2\dots a_na^{-1}_1a^{-1}_2\dots a^{-1}_{n-1}a^{-1}_n.
$$
\end{exercise}

\begin{exercise}
Determinar con la ayuda del grupo fundamental el modelo de la superficie dada por  el c\'odigo $abcd^{-1}ad^{-1}b^{-1}c^{-1}$.
\end{exercise}


\end{document}
