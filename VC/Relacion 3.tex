\documentclass[twoside]{article}
\usepackage{../estilo-ejercicios}

\usepackage{enumerate}
%--------------------------------------------------------
\begin{document}

\title{Variable Compleja - Relación 1}
\author{Rafael González López}
\maketitle

\begin{ejercicio}{5}
Demostrar que el polinomio $p(z)=2x^6-6z^5+4z^4-z^2+z-1$ tiene sus ceros en el disco cerrado de radio $3$.
\begin{solucion}
Vamos a aplicar el Teorema de Rouché tal y como se enuncia en el libro Complex Analysis (página 91), que establece lo siguiente.
\begin{theorem}
Sean $f$ y $g$ dos funciones holomorfas en un abierto que contiene a la circunferencia $C$ y su interior. Si 
$$|g(z)|<|f(z)| \qquad \forall z \in C
$$
Entonces $f$ y $f+g$ tienen el mismo número de ceros en el  círculo $C$ y su interior. 
\end{theorem}
Consideremos $f(z)=2z^6-6z^5+4z^4$, $g(z)=p(z)-f(z)$ y $C$ la circunferencia de centro el origen y radio $R=3$. Dado que $f(z)=2z^4(z-1)(z-2)$, $f$ tiene sus 6 raíces en el interior de $C$. Tenemos además que $\forall z \in C$ 
\begin{align*}
|g(z)|&=|-z^2+z-1|\leq |z^2|+|z|+1 = 3^2+3+1=13\\
|f(z)|&=|2z^6-6z^5+4z^4|=|2z^4||z^2-3z+2|=162|z-1||z-2|\\
&\geq 162||z|-1|||z|-2| = 324 > 13 \geq |g(z)|
\end{align*}
Basta aplicar el teorema para tener el resultado.
\end{solucion}
\end{ejercicio}
\end{document}