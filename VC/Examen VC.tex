\documentclass[twoside]{article}
\usepackage{../estilo-ejercicios}


\usepackage{enumerate}
%--------------------------------------------------------
\begin{document}

\title{Variable Compleja - Relación 6}
\author{Rafael González López}
\maketitle


\begin{ejercicio}{1}
Contesta breve y razonadamente a las siguientes cuestiones:
\begin{enumerate}[a)]
\item Sea $f\colon\Omega\to \C$ holomorfa. Quiero encontrar una primitiva $F$ de $f$, ¿qué condición en $\Omega$ garantiza la existencia de $F$?
\item Pon un ejemplo de función entera  de orden $\rho$ no entero. 
\item Si desarrollamos $\frac{1}{\cos z}$ entorno al punto $z=2+i$, ¿cuál es el radio de convergencia?
\item ¿Puedes dar algún ejemplo de una función holomorfa $f\colon\Omega\to\C$ y un compacto 
$K\subset\Omega$ de manera que $f$ no se pueda aproximar uniformemente en $K$ por polinomios?
\item ¿Qué aplicación tiene el teorema de Riemann para el problema de Dirichlet?
\end{enumerate}
\end{ejercicio}
\begin{solucion}
\begin{enumerate}[a)]
\item[]
\item La condición más general que hemos estudiado es que $\Omega$ sea simplemente conexo.
\item La función $\cos \sqrt{z}$.
\item Basta encontrar la distancia entre $2+i$ y la discontinuidad más próxima de la función a dicho punto. Un análisis heurístico revela que esta es $z=\pi/2$, luego $R=\sqrt{1 + (2 - \pi/2)^2}$.
\item Tomamos $f(z)=1/z$ definida en $\Omega = \C\setminus\{0\}$ y el compacto $K=\{z\mid 1\leq |z|\leq 2\}$.
\item Supongamos que planteamos el problema de Dirichlet en una región simplemente conexa $\Omega$ distinta de $\C$. Por el Teorema de Riemann sabemos que existe $F\func{ \mathbb{D}}{\Omega}$. Aunque en general no tiene por qué ocurrir, bajo ciertas hipótesis puede suponer que $F$ se extiende continuamente a una función $F\func{\overline{\mathbb{D}}}{\overline{\Omega}}$. Dado que sabemos resolver explícitamente el problema de Dirichlet en el disco y la composición de una función harmónica y una analítica es harmónica, podemos resolver fácilmente el problema original.
\end{enumerate}
\end{solucion}

\newpage
\begin{ejercicio}{2}
Se define la sucesión $(F_n)$ como los coeficientes del desarrollo de potencias 
\[(1-z-z^2)^{-1}=\sum_{n=0}^\infty F_n z^n.\]
\begin{enumerate}[a)]
\item Demostrar que existen y son únicos los números $F_n$ definidos de ese modo.
\item Calcula $F_n$ para $0\le n\le 6$. 
\item ¿Puedes encontrar una recurrencia para $F_n$?
\end{enumerate}
\end{ejercicio}
\begin{solucion}
\begin{enumerate}[a)]
\item[]
\item Notemos que las raíces del polinomio son $A= -(1+\sqrt{5})/2$ y $B=(\sqrt{5}-1)/2$. Sabemos que
\begin{align*}
\frac{1}{1-z-z^2} &= \frac{1/\sqrt{5}}{z-A}-\frac{1/\sqrt{5}}{z-B}\\
&= \frac{1/(\sqrt{5}B)}{1-z/B} - \frac{1/(\sqrt{5}A)}{1-z/A}\\
& = 
\frac{1}{\sqrt{5}}\sum_{n=0}^\infty z^n\left(\frac{1}{B^{n+1}}-\frac{1}{A^{n+1}}\right)
\end{align*}
La existencia y unicidad es consecuencia de que $f$ sea holomorfa en un entorno del origen.
\item $F_0 = 0$. $F_1 = 1$, $F_2 = 1$, $F_3 = 2$, $F_4=3$, $F_5=5$ y $F_6 = 8$.
\item Claramente sigue la sucesión de Fibonacci, descrita por $a_0 =0$, $a_1=1$ y $a_n = a_{n-1}+a_{n-2}$ para $n\geq 2$.
\end{enumerate}
\end{solucion}

\newpage
\begin{ejercicio}{3}
Consideremos la integral 
$\displaystyle{F(z)=\int_0^{+\infty}\frac{\cos t}{(z+t)^2}\,dt}$.
\begin{enumerate}[a)]
\item Sea $\varepsilon>0$ un número real fijado. Demostrar que $F$  está bien definida y es analítica en la región $\Re z>\varepsilon$.
\item Sea $\varepsilon>0$ y $R>0$ dos números reales fijados. Probar que $F(z)$ está bien definida y es analítica en la región $\Omega^+_{R,\varepsilon}=\{z\in\C: |\Re z|\le R, \Im z>\varepsilon\}.$
\item Demostrar también que $F$ está bien definida y es holomorfa en la región\newline 
$\Omega^-_{R,\varepsilon}=\{z\in\C: |\Re z|\le R, \Im z<-\varepsilon\}.$
\item ¿Dónde podemos decir que $F$ es holomorfa?
\end{enumerate}
\end{ejercicio}
\begin{solucion}
En los tres primeros casos vamos a aplicar el Teorema de Analiticidad de Integrales paramétricas. En los tres primeros apartados las dos primeras condiciones del teoremas son claras, luego basta probar la tercera.
\begin{enumerate}[a)]
\item Para $z$ con $\Re(z)>\varepsilon$, la función del integrando es continua en 0, luego basta estudiar el comportamiento en el infinito $|z+t| \geq |\Re(z+t)| \geq \varepsilon+t$. 
$$
\abs{\frac{\cos t}{(z+t)^2}} \leq \frac{1}{(t+\varepsilon)^2} \qquad \int_0^\infty \frac{1}{(t+\varepsilon)^2} dt < \infty
$$
\item Nuevamente, el integrando está bien definido en $0$, pues $$|z+t|^2 = |t+x + iy|^2 \geq t^2+x^2-2xt + \varepsilon^2 \geq t^2 - R^2-2Rt + \varepsilon^2$$
Es fácil ver que el último polinomio no tiene raíces reales. Por tanto,
$$
\abs{\frac{\cos t}{(z+t)^2}} \leq \frac{1}{(t-R)^2+\varepsilon^2} \qquad \int_0^\infty \frac{1}{(t-R)^2+\varepsilon^2} dt < \infty
$$
\item Este apartado es análogo al anterior.
\item En $\C\setminus (-\infty,0]$.
\end{enumerate}
\end{solucion}

\newpage
\begin{ejercicio}{4}
Consideremos la función $f(z)=e^{2\pi z^2}-1$.
\begin{enumerate}[a)]
\item Determinar los ceros de $f$ y la multiplicidad de cada uno de ellos. 
\item ¿Cuál es el orden de $f$?
\item Escribir el producto de Hadamard para $f(z)$.  Simplificar lo mas posible el resultado. 
\end{enumerate}
\end{ejercicio}
\begin{solucion}
\begin{enumerate}[a)]
\item[]
\item Tomando logaritmo en $e^{2\pi z^2} = 1$ tenemos directamente que
$$
2\pi z^2 = 2\pi i k \qquad z^2 = ik \qquad z = \pm \frac{1+i}{\sqrt{2}}\sqrt{k} 
$$
Denotemos
$$
a_k = \begin{cases}
\frac{1+i}{\sqrt{2}}\sqrt{k} & k\geq 0\\
-\frac{1+i}{\sqrt{2}}\sqrt{-k} & k\leq 0
\end{cases}
\qquad
b_k = ia_k
$$
Que son precisamente los ceros de $f$. La multiplicidad de cualquiera de ellos es $1$ salvo que $a_0=b_0=0$ que naturalmente es doble. 
\item Trivialmente 2.
\item Lo de simplificar igual está complicado, pero vamos a ver qué sale
\begin{align*}
f(z) & = e^{2\pi z^2}-1\\
&= e^{Az^2+Bz+C}z^2\prod_{\substack{n=1}}^\infty E_2(z/a_{n}E_2(z/a_{-n})\prod_{\substack{n=1}}^\infty E_2(z/b_{-n})E_2(z/b_{-n})\\
&= e^{Az^2+Bz+C}z^2\prod_{n=1}^\infty E_2(z/a_n)E_2(-z/a_{n})\prod_{n=1}^\infty E_2(z/b_n)E_2(-z/b_n)\\
&= e^{Az^2+Bz+C}z^2\prod_{n=1}^\infty (1-z^2/a_n^2)e^{z^2/a_n^2}\prod_{n=1}^\infty  (1-z^2/b_n^2)e^{z^2/b_n^2}\\
&= e^{Az^2+Bz+C}z^2\prod_{n=1}^\infty (1-z^2/a_n^2)e^{z^2/a_n^2}\prod_{n=1}^\infty  (1+z^2/a_n^2)e^{-z^2/a_n^2}\\
&= e^{Az^2+Bz+C}z^2\prod_{n=1}^\infty (1+iz^2/n)e^{-iz^2/n}\prod_{n=1}^\infty  (1-iz^2/n)e^{iz^2/n}\\
&= e^{Az^2+Bz+C}z^2\prod_{n=1}^\infty (1-(z^2i)^2/n^2)
\end{align*}
Sabemos que 
$$
\frac{\sin \pi z}{z\pi} =\prod_{i=1}^\infty\left(1-\frac{z^2}{n^2}\right)
$$
Por tanto
$$
f(z) = e^{Az^2+Bz+C}\sin(\pi z^2i) = ie^{Az^2+Bz+C}\sinh(\pi z^2)  
$$
Veamos los coeficientes del polinomio. Como $f(z)$ es par es claro que $B=0$. Dividiendo por $z^2$ y tomando límite tenemos que $C=\log (-2i)$, luego
$$
f(z)=2 e^{Az^2}\sinh(\pi z^2) 
$$
Sabemos que
$$
f(z)=e^{2\pi z^2} -1 = e^{\pi z^2}(e^{\pi z^2}-e^{-\pi z^2}) = 2e^{\pi^2 z^2} \sinh \pi z^2
$$
Luego $A=\pi$. Otra forma de ver esto es la siguiente.
\begin{align*}
e^{2\pi z^2} -1 &= 2\pi z^2 + 2\pi^2 z^4 + \dotsc\\
e^{Az^2} &= 1 + Az^2 + \frac{A^2}{2}z^4 + \dotsc\\
\prod_{n=1}^\infty (1+z^4/n^2) &= 1 + \dotsc \\
2e^{Az^2}z^2  \prod_{n=1}^\infty (1+z^4/n^2) &= 2Az^2 + \dotsc
\end{align*}
Luego $2A = 2\pi$.
\end{enumerate}
\end{solucion}

\newpage


\end{document}