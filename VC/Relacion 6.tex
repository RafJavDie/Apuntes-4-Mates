\documentclass[twoside]{article}
\usepackage{../estilo-ejercicios}


\usepackage{enumerate}
%--------------------------------------------------------
\begin{document}

\title{Variable Compleja - Relación 6}
\author{Rafael González López}
\maketitle


\begin{ejercicio}{1}
Demostrar que la función $w(z)$ definida por la integral
$\displaystyle{w(z)=\frac{1}{\pi i}\int_{-\infty}^{+\infty}\frac{e^{-t^2}}{t-z}\,dt}$
es analítica para $\Im(z)>0$. 
\end{ejercicio}
\begin{solucion}
Como ya a estas alturas de la película uno se espera, consideremos $a>0$ y la banda $\Im(z)>a$. En tal caso aplicamos el teorema de analiticidad de integrales paramétricas. Las dos primeras condiciones son evidentes, pasemos a la tercera. Notemos que $t \in \R$ e $\Im(z)>a$, luego $|t-z| > \Im(t-z) = a>0$. Por tanto
$$
\abs{\frac{e^{-t^2}}{t-z}}\leq \frac{e^{-t^2}}{a} \qquad \int_{-\infty}^\infty \frac{e^{-t^2}}{a} = \frac{\sqrt{\pi}}{a}
$$ 
\end{solucion}

\newpage
\begin{ejercicio}{2}
Sea $f(z)$ la función definida  por la serie de potencias
$\displaystyle{f(z)=\sum_{n=1}^\infty \frac{z^n}{n^2}}$.
\begin{enumerate}[a)]
\item ¿Dónde vale la representación anterior?

Sea $\Omega$ el plano complejo con un corte a lo largo del eje real desde $1$ a $\infty$, es decir,
$\Omega=\C\smallsetminus[1,\infty)$. 

\item Demostrar que para $|z|<1$ se tiene 
$\displaystyle{f(z)=-\int_0^z \log(1-\zeta)\frac{d\zeta}{\zeta}}$,
tomando un camino de integración contenido en $\Omega$ con extremos $0$ y $z$. 
\item  Demostrar que la integral en (b) permite extender $f(z)$ a todo $\Omega$ como función 
holomorfa.
\end{enumerate}
\end{ejercicio}
\begin{solucion}
\begin{enumerate}[a)]
\item[]
\item Por la fórmula de Hadamard tenemos que $R^{-1} = \limsup \sqrt[n]{1/n^2} = 1$. Por tanto, en principio vale en $|z|<1$. En $|z|=1$ podemos aplicar el criterio M de Weierstrass para ver que también es válida la representación.
\item Sabemos que en $\Omega$ la función $1-z$ no se anula y podemos definir $\log(1-z)$ que verifica las propiedades habituales. Definimos por tanto 
$$
h(z)=-\int_0^{z}\log(1-\zeta)\frac{d\zeta}{\zeta}
$$
Como $\Omega$ es simplemente conexo, la integral no depende del camino escogido. Ahora bien, en $|z|<1$ podemos derivar la serie anterior término a término y definir un logaritmo de manera natural como una serie de potencias, de manera que
$$
f'(z)=\sum_{n=1}^\infty \frac{z^{n-1}}{n} = \frac{1}{z}\sum_{n=1}^\infty \frac{z^{n}}{n} = -\frac{1}{z}\log(1-z)
$$
Como $f(0)=0$ tenemos que
$$
f(z)=-\int_0^{z}\log(1-\zeta)\frac{d\zeta}{\zeta}
$$
donde la integral es cualquier camino entre $0$ y $z$ contenido en $|z|<1$. Es claro que, por ejemplo, si $z\in \R$ y $-1<z<1$ entonces $f(z)=h(z)$, luego coinciden en $|z|<1$. 
\item Consecuencia inmediata del apartado anterior. Hemos probado que $h(z)$ coincide con $f(z)$ en el disco unidad y además extiende $f(z)$ a $\Omega$.
\end{enumerate}
\end{solucion}

\newpage
\begin{ejercicio}{3}
Sea $\sum_{n=0}^\infty c_n z^n$ una serie de potencias con  radio de convergencia  $R$. Suponemos que $0<R<\infty$.  
\begin{enumerate}[a)]
\item  Demostrar que 
$\displaystyle{f(z)=\sum_{n=0}^\infty \frac{c_n}{n!}z^n}$
es una función  entera. 
\item  Dado $0<r<R$, probar que existe una constante $M=M(r)$ tal que 
$|f(z)|\le Me^{|z|/r}.$
\item  Si existen constantes $A$ y $B$ tales que $|f(z)|\le A e^{B|z|}$, demostrar que 
$BR\ge 1$.
\item Demostrar que el orden de $f(z)$ es 1.
\end{enumerate}
\end{ejercicio}
\begin{solucion}
\begin{enumerate}[a)]
\item[]
\item Denotemos por $R'$ el radio de la nueva serie. Sabemos que
$$
\frac{1}{R} = \limsup_n \abs{c_n}^{1/n} \qquad\frac{1}{R'} = \limsup_n \abs{c_n\frac{1}{n!}}^{1/n}
$$
Como estamos trabajando con sucesiones de reales positivos, tenemos que $1/R\geq 0$. Además, siempre se tiene que
$$
0\leq \frac{1}{R'}=\limsup_n \abs{c_n\frac{1}{n!}}^{1/n} \leq \limsup_n \abs{c_n}^{1/n}\limsup_n\abs{\frac{1}{n!}}^{1/n} = \frac{1}{R}\limsup_n \frac{e}{n} = 0
$$
Por tanto, $R=\infty$.
\item Si $0<r<R$ entonces $1/R < 1/r$. Como $1/R$ es el límite superior de $|c_n|^{1/n}$, tenemos que $1/r^n > |c_n|$ salvo para una cantidad finita. Multiplicando por una constante $M$ apropiada, lo tenemos para todo $n$. Por tanto
$$
|f(z)| = \abs{\sum_{n=0}^\infty \frac{c_n}{n!}z^n}\leq \sum_{n=0}^\infty \abs{\frac{c_n}{n!}z^n} \leq \sum_{n=0}^\infty M{\frac{(|z|/r)^n}{n!}} = Me^{|z|/r}
$$
\item Sean $A$ y $B$ que verifiquen la desigualdad. Denotamos por $C_R$ la circunferencia de centro $0$ y radio $R$. Entonces
$$
\abs{\frac{c_n}{n!}} = \abs{\frac{1}{2\pi i}\int_{C_R}\frac{f(z)}{z^{n+1}}dz} \leq \frac{1}{2\pi} \abs{\int_0^{2\pi}\frac{f(Re^{it})}{R^{n+1}e^{(n+1)it}}Rie^{it}dz} \leq \frac{1}{2\pi}\int_0^{2\pi}\frac{Ae^{B|Re^{it}|}}{R^{n}}dz = \frac{Ae^{BR}}{R^n}
$$
Optimicemos esta función como función de $R$. Aplicando logaritmo y derivando obtenemos
$$
\log A + BR - n \log R \Rightarrow B - \frac{n}{R} = 0 \Rightarrow R = \frac{n}{B}
$$
Deducimos que
$$
|c_n| \leq n!\frac{Ae^n}{(n/B)^n} \Rightarrow |c_n|^{1/n} \leq \frac{A^{1/n}e}{n/B}n!^{1/n} \to B 
$$
Por tanto $1/R \leq B$ o equivalentemente $BR\geq 1$.
\item En el apartado b) ya hemos probado que el orden es $\leq 1$. Supongamos que $\exists r$ con $0<r<1$ tal que $\exists A,B>0$ con $|f(z)|\leq Ae^{B|z|^r}$. Entonces podemos escribir
$$
|f(z)|\leq Ae^{B|z||z|^{r-1}}
$$
Como $r-1<0$ para $|z|$ suficentemente grande tenemos que $|z|^{r-1}<\varepsilon$. Multiplicando por una constante $A'(\varepsilon)$ tenemos que $\forall z \in \C$
$$
|f(z)|\leq A'(\varepsilon)e^{B \varepsilon|z|}
$$
Tomando $\varepsilon = 1/(2RB)$ la cota debería seguir verificándose, pero $RB\varepsilon = 1/2 < 1$ lo cual contradice lo probado en $c)$.
\end{enumerate}
\end{solucion}

\newpage
\begin{ejercicio}{4}
Sea $f$ la función
\[ f(z)\ =\frac{1}{z + 3}+2iz^3+z^5.\]
Determinar el número de raíces que,
contando multiplicidades, tiene $f$ en las regiones:
\begin{enumerate}[a)]
\item El disco abierto $\Omega_1$ de centro 0 y radio 2; es decir, $\Omega_1=\{z\in\C : |z| < 2\}$.
\item El disco unidad abierto $\mathbb{D}$. 
\item La corona $\Omega_2=\{z\in\C : 1/2 < |z| < 1\}$.
\end{enumerate}
\end{ejercicio}
\begin{solucion}
Aplicaremos en cada caso el Teorema de Rouché.
\begin{enumerate}[a)]
\item En $|z|=2$ tenemos que $|z+3|\geq |z|-3 = 1$. Por tanto, si tomamos $g(z)=z^5$ tenemos
$$
|f(z)-g(z)| = \abs{\frac{1}{z+3}+2iz^3} \leq 1 + 2|z|^3 = 1+2^4 < 2^5 = |g(z)|
$$
Luego tiene $5$ raíces.
\item En $|z|=1$ tenemos que $|z+3| \geq 3-1 = 2$, luego $|z-3|^{-1}\leq 1/2$. Tomamos $g(z)=2iz^3$.
$$
\abs{f(z)-g(z)} \leq 1/2+|z|^5 = 3/2 < 2 = |2iz^3|  = |g(z)|
$$
Luego tiene 3 soluciones. 
\item En $|z|=1/2$ tenemos $|z+3| \leq 3+1/2 = 7/2 $.  Tomamos $g(z)=1/(z+3)$.
$$
\abs{f(z)-g(z)} \leq 2|z|^3+|z|^5 = 9/32 < 2/7 \leq \abs{\frac{1}{z+3}} = |g(z)|
$$
Luego no tiene raíces en $|z|\leq 1/2$.
\end{enumerate}
\end{solucion}

\newpage
\begin{ejercicio}{5}
Definimos una función por medio de la integral
\[F(z)=\int_{-\infty}^\infty \frac{e^t}{1+e^{2t}}e^{izt}\,dt.\]
Determinar el mayor abierto $G$ en que está definida la función $F(z)$ y demostrar que $F\colon G\to \C$ es holomorfa. 
\end{ejercicio}
\begin{solucion}
Primeramente, tengamos ne cuenta que el denominador no se anula para ningún $t\in \R$. Vamos a tratar de aplicar el teorema de analiticidad de integrales paramétricas. Las dos primeras hipótesis se verifican para todo $z\in \C$ y $t\in \R$. Veamos dónde podemos acotar la integral.
$$
\abs{\frac{e^{t(1+iz)}}{1+e^{2t}}} = \frac{|e^{t(1+iz)}|}{1+e^{2t}} = \frac{e^{t(1-\Im(z))}}{1+e^{2t}}
$$
Si $1-\Im(z)-2\geq 0$ entonces el límite $t\to\infty$ del integrando no tiende a $0$. Por tanto, sabemos que $\Im(z)>-1$. Igualmente, si $1-\Im(z)\leq0$ el límite cuando $t\to-\infty$ no tiende a $0$, luego $\Im(z)<1$. Por tanto, nuestro candidato es
$$
G = \{z \in \C \mid -1 < \Im(z)<1\}
$$
Tomando $-1<a<b<1$ es fácil probar que es holomorfa en la banda $a<\Im(z)<b$. Si $t\geq 0$
$$
 \frac{e^{t(1-\Im(z))}}{1+e^{2t}} \leq \frac{e^{t(1-a)}}{1+e^{2t}} \qquad \int_0^\infty \frac{e^{t(1-a)}}{1+e^{2t}} dt < \infty 
$$
Si $t<0$ entonces
$$
 \frac{e^{t(1-\Im(z))}}{1+e^{2t}} \leq \frac{e^{t(1-b)}}{1+e^{2t}} \qquad \int_{-\infty}^0 \frac{e^{t(1-b)}}{1+e^{2t}} dt < \infty
$$
\end{solucion}

\newpage
\begin{ejercicio}{6}
Para $n\in\N$, y $z\in\C$, denotamos $f_n(z)=z^n + n^{z-1}$. 
Sea $\Omega$ el conjunto formado por los $z\in\C$ para los que la serie
$\sum_n f_n(z)$ converge absolutamente.
\begin{enumerate}[a)]
\item Probar:  $\lim_{n\to\infty}{|f_n(z)|\over |z^n|}=1$, si $|z|>1$;\quad  y \quad
$\lim_{n\to\infty}{|f_n(z)|\over |n^{z-1}|}=1$, si $|z|<1$.
\item Determinar el conjunto $\Omega$, y verificar que es una
regi\'on.
\item Si denotamos por $F(z)=\sum_{n=1}^\infty f_n(z)$, demostrar que $F$ es holomorfa en $\Omega $.
\end{enumerate}
\end{ejercicio}
\begin{solucion}
\begin{enumerate}[a)]
\item[]
\item Naturalmente $|f_n(z)|/|z^n| = |1 + n^{z-1}/z^n|$. Tenemos además que
$$
1-\frac{|n^{z-1}|}{|z|^n} \leq \frac{|f_n(z)|}{z^n}\leq 1 + \frac{|n^{z-1}|}{|z|^n}
$$
Dado que $|n^{z-1}|/|z|^n\to 0$ cuando $n\to \infty$ cuando $|z|>1$ tenemos el primer límite. Para el segundo, tenemos que $|f_n(z)|/|n^{z-1}| = |1 + z^n/n^{z-1}|$ 
$$
1-\frac{|z|^n}{|n^{z-1}|} \leq \frac{|f_n(z)|}{z^n}\leq 1 + \frac{|z|^n}{|n^{z-1}|}
$$
Como $|z|^n/|n^{z-1}| \to 0$ cuando $n\to \infty$ y $|z|<1$, tenemos el resultado.
\item El primer límite nos indica que en $|z|>1$ la serie original diverge, pues $\sum z^n$ diverge para $|z|>1$, pues en particular $f_n(z) \not \to 0$. En $|z|=1$ tenemos
$$
|f_n(z)| \geq n^{z-1}-|z|^n  = n^{z-1} - 1 \not\to 0
$$ 
Consideremos ahora $|z|<1$. Si $\Re(z)>0$ entonces es claro que también diverge por comparación, pues $|n^{z-1}| = n^{x-1} \to \infty$. Si $\Re(z)=0$ entonces diverge pues $|z^n+n^{z-1}| > n^{-1} - |z|^n \to \infty$. Por tanto, 
$$
\Omega = \{z\in \C \mid |z|<1,\; \Re(z)<0\}
$$
\item Consideremos $\Omega_{r,\varepsilon} = \{z\in \C \mid |z|<r<1,\; \Re(z)<-\varepsilon<0\}$. En esta región tenemos
$$
|z^n+n^{z-1}| < r^n + n^{-\varepsilon-1} \qquad \sum_{n=1}^\infty r^n + n^{-\varepsilon-1} < \infty
$$
Por el criterio $M$ de Weierstrass tenemos que la serie converge absolutamente. Basta hacer $r\to 1$ y $\varepsilon\to 0$.
\end{enumerate}
\end{solucion}

\newpage
\begin{ejercicio}{7}
Determinar todos los isomorfismos $\varphi$ entre el semiplano superior $\Omega=\{ z\in \C : \Im z >0 \}$ y el disco unidad $\mathbb{D}$ que verifican $\varphi(i)=0$. Si $K$ es el disco de centro $0$ y radio $1/2$, quién es $\varphi^{-1}(K)$?
\end{ejercicio}
\begin{solucion}
Hagamos la siguiente cosideración. Sabemos los isomorfismos del semiplano al disco forman un grupo isomorfo al de los automorfismos del disco. Veamos esto. Sea
$$
f(z) = i\frac{1-z}{1+z}
$$
La aplicación $f(z)$ es un isomorfismo del disco al semiplano.
Definimos la operación $\phi+\psi = \phi \circ f \circ \psi$. Entonces es claro que los isomorfismos del semiplano verifican trivialmente la asociatividad, la clausura y
$$
\psi + f^{-1} = f^{-1} + \psi = \psi \qquad \phi + (f^{-1} \circ \psi^{-1} \circ f^{-1}) = (f^{-1} \circ \psi^{-1} \circ f^{-1}) + \phi = f^{-1}
$$
por lo que tenemos identidad y elemento inverso. Ahora veamos que este grupo es isomorofo al de los automorfismos del disco.  Consideremos la aplicación
$$
F(\phi) =  \phi \circ f
$$
Entonces verifica
$$
F(\phi + \psi) =(\phi \circ f \circ \psi) \circ f= (\phi\circ f)\circ (\psi\circ f) = F(\phi)\circ G(\psi)
$$
Además, $F$ tiene inversa dada por $G(h) = h \circ f^{-1}$, donde $h$ es un automorfimo del disco. Por tanto, tenemos el isomorfismos entre los grupos. Ahora bien, fijémonos que si $\phi,\psi$ son dos isomorfismos del semiplano en el disco tales que $\phi(i)=\psi(i)=0$ entonces
$$
\phi + (f^{-1}\circ \psi^{-1} \circ f^{-1})(i) = \phi \circ \psi^{-1} \circ f^{-1} (i) = \phi\circ \psi^{-1}(0) =\phi(i)=0
$$
Luego los isomorfismos considerados forman un subgrupo. En el disco, estos se corresponden con los automorfismos tales que $F^{-1}(h)(i) = h \circ f^{-1}(i)= h(0) = 0$, luego son los automorfismos que dejan fijo el origen. Sabemos que estos son precisamente los giros. Por tanto, los isomorfismos del semiplano al disco que llevan $i$ a $0$ son
$$
\phi_\alpha(z) = e^{i\alpha}\frac{i-z}{i+z}
$$
\newpage
A mí me apetecía escribir todo lo anterior por mero ejercicio intelectual, pero otra forma de llegar al mismo resultado sería la siguiente. Si $\phi$ es un isomorifsmo del semiplano al disco, entonces $h = f \circ \phi$ es un automorfismo del disco. Hemos probado en Teoría que los automorfismos del disco son transformaciones bilineales. Por tanto, $\phi = f^{-1}\circ h$ es una transformación bilineal. Como $\phi(i)=0$ tenemos que $z-i$ es un factor. Como $\phi$ lleva el eje real en la circunferencia unidad y $\phi$ es una transformación bilineal, el simétrico de $i$ respecto del eje real ($-i$) tiene que ir en el simétrico de $0$ respecto de la circunferencia ($\infty$), por lo que 
$$
\phi(z) = a \frac{z-i}{z+i}
$$
Como $|a| = 1$ por ir al disco, tenemos que $a = e^{it}$ para algún $t \in [0,2\pi)$.
\newline
Para la segunda parte consideremos $k$ el disco $|z|<1/2$. Como la inversa de un giro es un giro, que deja invariante al disco, $\phi^{-1}(K)=f(K)$. Sabemos que
$$
f(0) =i \qquad f(1/2)=i/3 \qquad f(-1/2) = 3i
$$
Notemos que el eje real va en el eje imaginario. Además, $f(i/2)$ no es puramente imaginario, $f(K)$ es un disco. Tenemos dos punto, resta determinar el centro. Como el eje real es perpendicular a $|z|=1/2$, el eje imaginario es perpendicular al borde de $f(K)$, luego el centro está en el eje imaginario, pero ya tenemos que dos puntos del borde que están allí, luego el centro es $5i/3$ y el radio es $4/3$.
\end{solucion}

\newpage
\begin{ejercicio}{8}
Sea $\Omega$ la región angular $\Omega=\{ z\in \C : |\arg(z)| < \pi /4 \}$:
\begin{enumerate}[a)]
\item Dar un isomorfismo $f$ entre $\Omega $ y el disco unidad $\mathbb{D}$, con $f(1)=0$.
Comprobar que $|f(2)|=3/5$.
\item  Sea $g\colon \Omega\to \mathbb{D}$ holomorfa con $g(1)=0$ y  $g(2)=3/5$. Probar que $g$ es un 
isomorfismo entre $\Omega $ y el disco unidad $\mathbb{D}$.
\end{enumerate}
\end{ejercicio}
\begin{solucion}
\begin{enumerate}[a)]
\item[]
\item Sabemos que la función $z^2$ es inyectiva en $\Omega$ y $z^2(\Omega)$ es precisamente el semiplano derecho. Componiendo con el giro $iz$ tenemos semiplano superior y llevamos este al disco, teniendo finalmente la aplicación
$$
f(z) = \frac{i-z^2i}{i+z^2i} = \frac{1-z^2}{1+z^2}
$$
Notemos que $f(1)=0$ y $|f(2)|= \abs{\frac{1-4}{1+4}} = 3/5$.
\item Notemos que la composición $h = g \circ  f^{-1}$ verifica que $h$ es holomorfa, $h(0)=0$ y $|h(3/5)| = |3/5|$ luego, por el Lema de Schwarz, $h$ es una rotación. De hecho, como $h(3/5)=3/5$ $h=id$ y $f=g$.
\end{enumerate}
\end{solucion}

\newpage

\begin{ejercicio}{9}
Se define, para cada $z\in \C$, y cada $\alpha >1/2$, la función
$$
Q_\alpha(z)=
\prod_{n=1}^\infty \Bigl(1+{z\over n^\alpha}\Bigl)\exp(-z/ n^\alpha).
$$
\begin{enumerate}[a)]
\item Probar que, para todo $\alpha>1/2$, la función $Q_\alpha$ está bien definida y es entera.
\item Probar que existe $A>0$ tal que $(z+1)Q_1(z+1) = AQ_1(z)$, para todo $z\in \C$.
\item Calcular $g(z)=Q_1(z)\cdot Q_1(-z)$, expresándola como combinación de funciones elementales.
\end{enumerate}
\end{ejercicio}
\begin{solucion}
\begin{enumerate}[a)]
\item[]
\item Sea $\alpha>1/2$ y consideremos $|z|<R$. Sabemos que $n^{\alpha}\to \infty$, luego $\exists n_0=n_0(\alpha,R) \in \N$ tal que $\forall n \geq n_0$ $n^\alpha > 2R$. Por tanto, $|z/n^\alpha|<1/2$. Vamos a ver que la suma de los logaritmos converge utilizando el criterio M de Weierstrass. Para $n\geq n_0$
\begin{align*}
\abs{-\frac{z}{n^\alpha}+\log\left(1+\frac{z}{n^\alpha}\right)}&=\abs{-\frac{z}{n^\alpha}+\sum_{k=1}^\infty \frac{(-1)^k}{k}\left(\frac{z}{n^\alpha}\right)^k}\\
&=\abs{\sum_{k=2}^\infty \frac{(-1)^k}{k}\left(\frac{z}{n^\alpha}\right)^k}\\
&\leq\sum_{k=2}^\infty \frac{1}{k}\abs{\frac{z}{n^\alpha}}^k\\
&= \frac{|z|^2}{n^{2\alpha}}\sum_{k=2}^\infty \frac{1}{k}\abs{\frac{z}{n^\alpha}}^{k-2} \\
&\leq \frac{R^2}{n^{2\alpha}}\sum_{k=2}^\infty \frac{1}{k}(1/2)^{k-2}\\
&=\frac{4R^2}{n^{2\alpha}}(-\log(1/2)) \leq \frac{3R^2}{n^{2\alpha}}
\end{align*}
Como $\alpha>1/2$, $2\alpha>1$, por lo que la serie $\sum 3R^2/n^{2\alpha}$ converge. Por tanto, $Q_\alpha(z)$ está bien definida y es holomorfa en $|z|<R$ para $\alpha>1/2$. Como $R$ era arbitrario, deducimos que es entera.
\item Notemos que si $1+ (z+1)/n = 0$ entonces $z = -(n+1)$. 
\begin{align*}
(z+1)Q_1(z+1)&=(z+1)\prod_{n=1}^\infty \Bigl(1+{z+1\over n}\Bigl)e^{-(z+1)/ n}\\
&=(z+1)\prod_{n=1}^\infty\left(1+\frac{z}{n+1}\right)\left(1+\frac{1}{n}\right)e^{-z/n - 1/n}\\
&=\prod_{n=1}^\infty \left(1+\frac{z}{n}\right)e^{-z/n}\prod_{m=1}^\infty \left(1+\frac{1}{m}\right)e^{-1/m}\\
&=Q_1(z)Q_1(1)
\end{align*}
\item Comprobamos directamente
\begin{align*}
Q_1(z)Q_1(-z) &= \prod_{n=1}^\infty \left(1+\frac{z}{n}\right)e^{z/n}\prod_{n=1}^\infty \left(1-\frac{z}{n}\right)e^{-z/n}\\
&= \prod_{n=1}^\infty \left(1-\frac{z^2}{n^2}\right)\\
&=\frac{\sin \pi z}{\pi z}
\end{align*}
\end{enumerate}
\end{solucion}

\newpage
\begin{ejercicio}{10}
Sea $f\colon\overline{\mathbb{D}} \to \C$ continua, holomorfa en $\mathbb{D}$, y con $|f(z)|=1$, 
para todo $z\in \partial\mathbb{D}$. Probar:
\begin{enumerate}
\item Si $f$ no tiene ceros en $\mathbb{D}$, entonces $f$ es constante.
\item En todo caso $f$ tiene a lo sumo un n\'umero finito de ceros en $\mathbb{D}$.
\item Existen $c\in\partial\mathbb{D}$, $k\ge 0$, $N\ge 0$, y $a_1$, 
$a_2$, \dots, $a_N\in\mathbb{D}$ (no necesariamente distintos), tales que
$$
f(z)=c z^k \prod_{n=1}^N \biggl({z-a_n\over 1-\overline{a_n} z}\biggr)\,,\qquad
\hbox{para todo $z\in \overline{\mathbb{D}}$.}
$$

\end{enumerate}
\end{ejercicio}
\begin{solucion}
\begin{enumerate}[a)]
\item[]
\item Supongamos que $f$ no tiene ceros en $\mathbb{D}$. Si $f$ no es constante, por su módulo alcanza el máximo en la frontera y $\exists z \in \mathbb{D}$ tal que $|f(z)|<1$. Además, existe, es holomorfa y está bien definida $g=1/f$, pero $|g(z)|=1/|f(z)| >1$ y $|g(z)|$ debería alcanzar su máximo en la frontera, donde tiene módulo 1. Por tanto, si $f$ no tiene ceros en $\mathbb{D}$ entonces es constante.
\item Es una consecuencia del Teorema de Weierstrass: Una sucesión acotada tiene al menos un punto de acumulación. Si el conjunto de ceros en $\mathbb{D}$ fuese infinito, habría uno de acumulación, luego $f$ debería ser la función nula.	
\item Vamos a aplicar el apartado anterior. Sean $a_1,\dotsc,a_N$ los ceros de $f$ en $0<|z|<1$ y $k$ el orden del $0$ de $f$ (si no es $f(0)\neq 0$ entonces $k=0$). Entonces la siguiente función es continua en $\overline{D}$ y holomorfa en $\mathbb{D}$
$$
h(z)=\frac{f(z)}{z^k}\prod_{n=1}^N \left(\frac{1-\overline{a_n}z}{z-a_n}\right)
$$
Por construcción, $h(z)$ no se anula en $\mathbb{D}$ y en $|z|=1$ tenemos
$$
|h(z)| = \abs{\frac{f(z)}{z^k}\prod_{n=1}^N \left(\frac{1-\overline{a_n}z}{z-a_n}\right)} = \frac{|f(z)|}{|z|^k}\prod_{n=1}^N \abs{\frac{1-\overline{a_n}z}{z-a_n}} = \prod_{n=1}^N \abs{\frac{1-\overline{a_n}z}{z-a_n}}=1 
$$
Por tanto, $h(z) = e^{it}$ para algún $t\in [0,2\pi)$ y 
$$
f(z) = e^{it}z^k \prod_{n=1}^N \biggl({z-a_n\over 1-\overline{a_n} z}\biggr)
$$
\end{enumerate}
\end{solucion}

\newpage
\begin{ejercicio}{11}
Supongamos que la función analítica $f(z) = \sum_{n\ge 0}
c_n z^n$  definida en el disco unidad $\mathbb{D}$ es un isomorfismo de tal disco sobre una región $G$ de \'area $S$. Demostrar: 
\begin{enumerate}[a)]
\item $S = \pi \cdot \sum_{n=1}^\infty n \vert c_n\vert^2$.
\item ({\sl Propiedad extremal de la representación sobre
un disco}) Si $f'(0) = 1$, el área de $G$ no es menor que la
del disco.
\end{enumerate}
\end{ejercicio}
\begin{solucion}
\begin{enumerate}[a)]
\item[]
\item 
\item Sabemos que $c_n = \frac{f^{n)}(0)}{n!}$. En particular $c_1 = f'(0) = 1$. Por tanto
$$
S = \pi \sum_{n=1}^\infty n|c_n|^2 \geq \pi \cdot 1 \cdot 1 = \pi 
$$
Como $\pi$ es el área de $\mathbb{D}$ se tiene el resultado.
\end{enumerate}
\end{solucion}

\newpage


\end{document}