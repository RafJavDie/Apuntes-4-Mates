\documentclass[GTS.tex]{subfiles}
%\usepackage{amsmath,amssymb}
%\usepackage[utf8]{inputenc}
%\usepackage[spanish]{babel}
%\usepackage[]{graphicx}
%\usepackage{enumerate}
%\usepackage{amsthm}
%\usepackage{tikz-cd}
%\usetikzlibrary{babel}
%\usepackage{pgf,tikz}
%\usepackage{mathrsfs}
%\usetikzlibrary{arrows}
%\usetikzlibrary{cd}
%\usepackage[spanish]{babel}
%\usepackage{fancyhdr}
%\usepackage{titlesec}
%\usepackage{floatrow}
%\usepackage{makeidx}
%\usepackage[tocflat]{tocstyle}
%\usetocstyle{standard}
%\usepackage{color}
%\usepackage{hyperref}
%\hypersetup{colorlinks=true,citecolor=red, linkcolor=blue}
%%\usepackage{ntheorem}
%
%
%\renewcommand{\baselinestretch}{1,4}
%\setlength{\oddsidemargin}{0.25in}
%\setlength{\evensidemargin}{0.25in}
%\setlength{\textwidth}{6in}
%\setlength{\topmargin}{0.1in}
%\setlength{\headheight}{0.1in}
%\setlength{\headsep}{0.1in}
%\setlength{\textheight}{8in}
%\setlength{\footskip}{0.75in}
%
%\newtheorem{teorema}{Teorema}[section]
%\newtheorem{defi}[teorema]{Definición}
%\newtheorem{coro}[teorema]{Corolario}
%\newtheorem{lemma}[teorema]{Lema}
%\newtheorem{ej}[teorema]{Ejemplo}
%\newtheorem{ejs}[teorema]{Ejemplos}
%\newtheorem{observacion}[teorema]{Observación}
%\newtheorem{observaciones}[teorema]{Observaciones}
%\newtheorem{prop}[teorema]{Proposición}
%\newtheorem{propi}[teorema]{Propiedades}
%\newtheorem{nota}[teorema]{Nota}
%\newtheorem{notas}[teorema]{Notas}
%\newtheorem*{dem}{Demostración}
%\newtheorem{ejer}[teorema]{Ejercicio}
%\newtheorem{consec}[teorema]{Consecuencia}
%\newtheorem{consecs}[teorema]{Consecuencias}
%
%\providecommand{\abs}[1]{\lvert#1\rvert}
%\providecommand{\sen}[1]{sen #1}
%\providecommand{\norm}[1]{\lVert#1\rVert}
%\providecommand{\ninf}[1]{\norm{#1}_\infty}
%\providecommand{\numn}[1]{\norm{#1}_1}
%\providecommand{\gabs}[1]{\left|{#1}\right|}
%\newcommand{\bor}[1]{\mathcal{B}(#1)}
%\newcommand{\R}{\mathbb{R}}
%\newcommand{\N}{\mathbb{N}}
%\newcommand{\Q}{\mathbb{Q}}
%\newcommand{\C}{\mathbb{C}}
%\newcommand{\Pro}{\mathbb{P}}
%\newcommand{\Tau}{\mathcal{T}}
%\newcommand{\verteq}{\rotatebox{90}{$\,=$}}
%\newcommand{\vertequiv}{\rotatebox{110}{$\,\equiv$}}
%\providecommand{\lrg}{\longrightarrow}
%\providecommand{\func}[2]{\colon{#1}\longrightarrow{#2}}
%\newcommand*{\QED}{\hfill\ensuremath{\blacksquare}}
%\newcommand*\circled[1]{\tikz[baseline=(char.base)]{
%            \node[shape=circle,draw,inner sep=1.5pt] (char) {#1};}}
%\newcommand*{\longhookarrow}{\ensuremath{\lhook\joinrel\relbar\joinrel\rightarrow}}
%
%\newenvironment{solucion}{\begin{trivlist}
%\item[\hskip \labelsep {\textit{Solución}.}\hskip \labelsep]}{\end{trivlist}}
%
%
%\def\quot#1#2{%
%    \raise1ex\hbox{$#1$}\Big/\lower1ex\hbox{$#2$}%
%}
%\def\quott#1#2{%
%    \hbox{$#1$}\Big/\lower1ex\hbox{$#2$}%
%}
%
%\makeatletter
%\renewcommand\tableofcontents{%
%  \null\hfill\textbf{\Large\contentsname}\hfill\null\par
%  \@mkboth{\MakeUppercase\contentsname}{\MakeUppercase\contentsname}%
%  \@starttoc{toc}%
%}
%
%\pagestyle{fancy}
%\fancyhf{}
%\rhead{Topología de Superficies (Grado en Matemáticas)}
%\lhead{Curso 2016/2017}
%\cfoot{\thepage}

\begin{document}
%\title{Topología de Superficies}
%\author{Antonio Rafael Quintero Toscano\\ Javier Aguilar Martín}
%\date{Curso 2016/2017}
%\maketitle

\renewcommand\chaptername{\Huge Tema}

\titleformat{\chapter}[display]
    {\normalfont\huge\bfseries}{\chaptertitlename\ \thechapter}{10pt}{\Huge}
\titlespacing*{\chapter}{0pt}{-1cm}{10pt}



%\tableofcontents






\chapter{Introducción}
Los espacios que vamos a tratar están contenidos en algún espacio euclídeo, es decir $X\subseteq\R^n$ (espacio euclídeo n-dimensional). No obstante, recordaremos algunos conceptos generales del curso de Topología.
\section{Espacios Topológicos}
\begin{defi}Una \textbf{topología} es una colección $\Tau$ de subconjuntos de $X$ verificando las siguientes propiedades:
\begin{enumerate}
\item $\emptyset,X\in\Tau$.
\item Si $A_1,...,A_n\in\Tau$ entonces $\underset{i=1}{\overset{n}{\bigcap}} A_i\in\Tau$.
\item Si $\{A_\alpha\}_{\alpha\in\Lambda}\subset\Tau$ entonces $\underset{\alpha\in\Lambda}{\bigcup} A_i\in\Tau$.
\end{enumerate}
Al par $(X,\Tau)$ se le denomina \textbf{espacio topológico}. A los conjuntos de $\Tau$ se les llama \textbf{abiertos} de $(X,\Tau)$ y a sus complementarios \textbf{cerrados} de $(X,\Tau)$.
\end{defi}
\begin{nota} Por abuso de lenguaje a menudo se identificará $(X,\Tau)$ con $X$.
\end{nota}

\begin{ej} Si $(X,d)$ es un espacio métrico, su topología asociada es
\[
\Tau_d=\{A\subseteq X\mid\forall a\in A\ \exists\varepsilon>0\ \textit{tal\ que } B_d(a,\varepsilon)\subseteq A\}\cup\{\emptyset\}.
\]
Algunos ejemplos de distancia son los siguientes:
\begin{enumerate}
\item[-] $d_e\equiv$distancia euclídea, $d_e((x,y),(x',y'))=\sqrt{(x-x')^2+(y-y')^2}$.
\item[-] $d_{max}\equiv$distancia del máximo $d_{max}((x,y),(x',y'))=\max\{\abs{x-x'},\abs{y-y'}\}$.
\item[-] $d_{taxi}\equiv$distancia taxi $d_{taxi}((x,y),(x',y'))=\abs{x-x'}+\abs{y-y'}$ (Taxicab Geometry).
\end{enumerate}
\end{ej}

\begin{defi}Sean $x\in B\subseteq X$. Diremos que $x$ es \textbf{punto interior} de $B$ (o que $B$ es \textbf{entorno} de $x$) si $\exists U\in\Tau$ tal que $x\in U\subseteq B$. Diremos que es \textbf{adherente} a $B$ si para todo $U\in\Tau$ con $x\in U$, $U\cap B\neq\emptyset$.
\end{defi}

\begin{prop} Un conjunto $A$ es abierto si y solo si todos sus puntos son interiores. Un conjunto $B$ es cerrado si y solo si coincide con el conjunto de sus puntos adherentes.
\end{prop}

\begin{defi}
Dado un espacio topológico $(X,\Tau)$, se llama \textbf{topología restricción o relativa } a $A\subseteq X$ a la topología sobre $A$
\[
\Tau_A=\{G\cap A\mid\forall G\in \Tau\}.
\]
A $(A,\Tau_A)$ se le llama \textbf{subespacio topológico} de $(X,\Tau)$. Nótese que si $A$ es un conjunto abierto de $X$ entonces $\Tau_A$ está formada por los abiertos de $\Tau$ que están contenidos en $A$.
\end{defi}

\begin{defi}Una aplicación $f\func{(X,\Tau)}{(X',\Tau')}$ entre dos espacios topológicos diremos que es \textbf{continua} si $\forall x$ adherente a $A$ en $(X,\Tau)$, $f(x)$ es adherente a $f(A)$ en $(X',\Tau')$.
\end{defi}
\begin{prop}Si $U$ en $(X',\Tau')$ es un abierto y $f$ es continua entonces $f^{-1}(U)$ es abierto en $(X,\Tau)$.
\end{prop}

\vspace{0.5cm}

En espacios métricos se puede usar el criterio $\varepsilon-\delta$.\\
{\large \bf Criterio $\varepsilon-\delta$ en espacios métricos}

Se dice que una aplicación $f\func{(X,d)}{(X',d')}$ entre espacios métricos es continua si $\forall x\in X, \forall\varepsilon>0\ \exists\delta>0$ tal que $d(x,x')<\delta\Rightarrow d(f(x),f(x'))<\varepsilon$.

\vspace{0.5cm}

\begin{defi}Diremos que $f\func{(X,\Tau)}{(X',\Tau')}$ es \textbf{homeomorfismo} (equivalencia topológica) si f es biyectiva, continua y con inversa continua.
\end{defi}

\begin{defi} Sea $(X,\Tau)$ un espacio topológico. Diremos que $\{x_n\}$ \textbf{converge} a $x$ si $\forall U\in\Tau$ con $x\in U$, $\exists n_0$ tal que $x_n\in U\ \forall n\geq n_0$.
\end{defi}

La convergencia se hereda por aplicaciones continuas, es decir, si $\{x_n\}_{n\geq 1}$ converge a $x$ entonces $\{f(x_n)\}_{n\geq 1}$ converge a $f(x)$.

Para la convergencia también hay un criterio propio de espacios métricos.\\
{\large \bf Criterio de convergencia en espacios métricos}\\
La sucesión $\{x_n\}$ converge en el espacio métrico $(X,d)$ a $x$ si para todo $\varepsilon>0$ existe $n_0$ tal que $d(x,x_n)<\varepsilon\ \forall n\geq n_0$.

\begin{defi}
Dado un espacio topológico $X$ y un punto $x\in X$, denotemos por $\mathcal{V}(x)$ al conjunto de entornos de $x$. Decimos entonces que un subconjunto $\mathcal{B}(x)\subseteq\mathcal{V}(x)$ es una \textbf{base de entornos} si $\forall V\in\mathcal{V}(x)\ \exists B\in\mathcal{B}(x)$ con $B\subseteq V$. 
\end{defi}

\begin{defi} Sea $X$ un espacio topológico. Decimos que la colección de abiertos $\{U_\alpha\}_{\alpha\in\Lambda}$ es una \textbf{base} si para todo $U\subseteq X$ abierto existe una familia de índices $J\subseteq\Lambda$ tal que podemos escribir $U=\bigcup_{\alpha\in J}U_\alpha$.
\end{defi}

\begin{defi}
Un espacio topológico $X$ se dice \textbf{primero numerable} (1º N) si todo punto posee una base numerable de entornos. Se dice \textbf{segundo numerable} (2º N) si existe una base para la topología de $X$ de cardinal numerable.
\end{defi}

\begin{prop}
Los homeomorfismos preservan los axiomas de numerabilidad. Estos axiomas son además son hereditarios.
\end{prop}

\section{Compacidad}
\begin{defi}Sea un espacio $X$ y $A\subseteq X$. Un \textbf{recubrimiento} por abiertos de $A$ es una colección de abiertos $\{U_\alpha\}_{\alpha\in\Lambda}$ tal que $A\subseteq\underset{\alpha\in\Lambda}{\bigcup}U_\alpha$.
\end{defi}
\begin{defi} Si para todo recubrimiento $\{U_\alpha\}_{\alpha\in\Lambda}$ de $A$ existen $U_{\alpha_1},\dots,U_{\alpha_n}$ (subrecubrimiento finito) tal que $A\subseteq\underset{i=1}{\overset{n}{\bigcup}}U_i$ decimos que $A$ es \textbf{compacto}.
\end{defi}
\begin{defi} Un espacio $X$ se dice que tiene la propiedad de separación de Hausdorff si dados $x,y\in X$ con $x\neq y$ existen abiertos $U,V$ con $x\in U$, $y\in V$ y $U\cap V=\emptyset$.
\end{defi}
Si el espacio $X$ es compacto entonces todos sus subconjuntos cerrados son también compactos.
\begin{prop} En todo espacio $X$ con propiedad de Hausdorff los conjuntos compactos son cerrados.
\end{prop}

\begin{prop}
Cualquier aplicación continua de un compacto a un Hausdorff es cerrada.
\end{prop}

\begin{teorema}[Tychonoff]
El producto de compactos es compacto.
\end{teorema}

\section{Conexión}
\begin{defi} Sea $(X,\Tau)$ un espacio topológico. Diremos que $C\subseteq X$ es \textbf{conexo} si y solo si dados $A,B\in\Tau$ cumpliendo $A\cap B\cap C=\emptyset, C\subseteq A\cup B$ se tiene que o bien $C\subseteq A$ o bien $C\subseteq B$.
\end{defi}
\begin{defi} $(X,\Tau)$ será \textbf{conexo por caminos} si $\forall x,y\in X\ \exists\alpha\func{([0,1],eucl\acute{\imath}dea)}{(X,\Tau)}$ continua tal que $\alpha(0)=x$ y $\alpha(1)=y$.
\end{defi}

\begin{defi}
Un espacio $X$ se dice \textbf{localmente conexo (por caminos)} si para todo $x\in X$ y todo entorno $N$ de $x$ existe otro entorno de $N'\subseteq N$ tal que $N'$ es conexo (por caminos).
\end{defi}
\begin{prop} Todo espacio conexo por caminos es conexo. Recíprocamente, todo espacio conexo y localmente conexo por caminos es conexo por caminos.
\end{prop}

\begin{defi}
Dado un espacio $X$, se llama \textbf{componente conexa (por caminos)} de $x$ al mayor subespacio conexo (por caminos) de $X$ que contiene a $x$.
\end{defi}

\begin{prop}
Si $X$ es localmente conexo por caminos entonces la componente conexa de $x$ coincide con su componente conexa por caminos y es un conjunto abierto y cerrado de $X$.
\end{prop}

\begin{defi}Dado un subconjunto $A\subseteq X$ conexo (por caminos) diremos que $x\in A$ es un \textbf{punto de corte} si $A\setminus\{x\}$ no es conexo (por caminos). El número de componentes conexas (por caminos) de   $A\setminus\{x\}$ de llama \textbf{orden de corte} de $x$.
\end{defi}


\section{Espacio Cociente}
Sea $(X,\Tau)$ un espacio topológico y $\mathcal{R}$ una relación de equivalencia. En el espacio cociente $X/\mathcal{R}$ consideramos la siguiente topología: sea $\pi\func{X}{X/\mathcal{R}}$ la aplicación proyección $\pi(x)=[x]$. Se define la siguiente topología:
\begin{equation*}
\Tau_\pi=\left\{G\subseteq X/\mathcal{R};\pi^{-1}(G)\in\Tau\right\}
\end{equation*}
Esta topología se llama \textbf{topología cociente} para la relación $\mathcal{R}$. Al par $\left(X/\mathcal{R},\Tau_\pi\right)$ se le llama \textbf{espacio cociente} por la relación $\mathcal{R}$.
\begin{lemma}$\Tau_\pi$ es topología de $\left(X/\mathcal{R},\Tau_\pi\right)$
\end{lemma}
\begin{dem}
Tenemos que verificar las tres propiedades de una topología.
\begin{enumerate}
\item $\pi^{-1}(\emptyset)=\emptyset$ que es abierto de $X$ $\Rightarrow \emptyset\in\Tau_\pi$\\ $\pi^{-1}(X/\mathcal{R})=X$ que es abierto de $X$ $\Rightarrow X/\mathcal{R}\in\Tau_\pi$.
\item Sean $A_1,\dots,A_n\in\Tau_\pi$ entonces $\pi^{-1}(A_i)$ es abierto de $X$ $\forall i=1\dots n$.\\ Como $\pi^{-1}\left(\underset{i=1}{\overset{n}{\bigcap}}A_i\right)=\underset{i=1}{\overset{n}{\bigcap}}\pi^{-1}(A_i)$, que es abierto de $X$ por ser intersección finita de abiertos, entonces se tiene que $\underset{i=1}{\overset{n}{\bigcap}}A_i\in\Tau_\pi$.
\item Sea $\{A_\alpha\}_{\alpha\in\Lambda}\subseteq\Tau_\pi$. Se tiene que $\pi^{-1}(A_i)$ es abierto de $X$ $\forall\alpha\in\Lambda$.\\ Como $\pi^{-1}\left(\underset{\alpha\in\Lambda}{\bigcup}A_i\right)=\underset{\alpha\in\Lambda}{\bigcup}\pi^{-1}(A_i)$, que es abierto de $X$ se tiene que ${\underset{\alpha\in\Lambda}{\bigcup}}A_i\in\Tau_\pi$. $\QED$
\end{enumerate}
\end{dem}

\vspace{1cm}

\begin{ej}\

\begin{center}
\definecolor{zzttqq}{rgb}{0.6,0.2,0.}
\begin{tikzpicture}[line cap=round,line join=round,>=triangle 45,x=1.0cm,y=1.0cm]
\clip(-0.5,-0.5) rectangle (7,2.5);

\draw [line width=0.4pt,color=zzttqq,fill=zzttqq,fill opacity=0.17] (5.530362528716437,1.0064423707474248) circle (0.3cm);
\draw (0.,2.)-- (0.,0.);
\draw (0.,0.)-- (2.,0.);
\draw (2.,0.)-- (2.,2.);
\draw (2.,2.)-- (0.,2.);
\draw [shift={(0.,1.)},color=zzttqq,fill=zzttqq,fill opacity=0.1]  (0,0) --  plot[domain=-1.5707963267948966:1.5707963267948966,variable=\t]({1.*0.38*cos(\t r)+0.*0.38*sin(\t r)},{0.*0.38*cos(\t r)+1.*0.38*sin(\t r)}) -- cycle ;
\draw [shift={(2.,1.)},color=zzttqq,fill=zzttqq,fill opacity=0.1]  (0,0) --  plot[domain=1.5707963267948966:4.71238898038469,variable=\t]({1.*0.36666666666667*cos(\t r)+0.*0.36666666666667*sin(\t r)},{0.*0.36666666666667*cos(\t r)+1.*0.36666666666667*sin(\t r)}) -- cycle ;
\draw [rotate around={-0.48554583000832496:(5.526666666666675,0.026666666666669225)}] (5.526666666666675,0.026666666666669225) ellipse (0.8420037767665578cm and 0.30013575461824116cm);
\draw [rotate around={0.:(5.526666666666656,2.0333333333333354)}] (5.526666666666656,2.0333333333333354) ellipse (0.8400454719920218cm and 0.2946726159144658cm);
\draw (4.686621194674634,2.0333333333333354)-- (4.686883926578783,0.05453599091983475);
\draw (6.363470124970542,2.059197076229566)-- (6.367927797199578,0.008065619937139841);
\draw [dash pattern=on 3pt off 3pt] (5.540007254636147,1.738697877916328)-- (5.513506126542233,-0.2733444708766622);
\draw (-0.45,0.) node[anchor=north west] {$X=[0,1]\times[0,1]$};
\draw (-0.45,1.14) node[anchor=north west] {$a$};
\draw (1.9933333333333343,1.1533333333333378) node[anchor=north west] {$a$};
\draw (5.513333333333333,1.22) node[anchor=north west] {$a$};
\draw (5.8,1.1533333333333378) node[anchor=north west] {$G$};
\draw (0.4,0.8) node[anchor=north west] {$\pi^{-1}(G)$};
\draw (3.273333333333334,1.42) node[anchor=north west] {$\pi$};
\draw [->] (0.5,0.5) -- (0.2,0.65);
\draw [->] (1.65,0.4) -- (1.9,0.65);
\draw [->] (2.4066666666666676,0.9933333333333372) -- (4.513333333333334,1.0066666666666704);
\begin{scriptsize}
\draw [fill=black] (0.,1.) circle (1.5pt);
\draw [fill=black] (2.,1.) circle (1.5pt);
\draw [fill=black] (5.530362528716437,1.0064423707474248) circle (1.5pt);
\end{scriptsize}
\end{tikzpicture}

Se identifican los lados verticales del cuadrado para obtener el cilindro.
\end{center}

\begin{center}
\definecolor{ffffff}{rgb}{1.,1.,1.}
\begin{tikzpicture}[line cap=round,line join=round,>=triangle 45,x=1.0cm,y=1.0cm]
\clip(-2.3533333333333335,-0.1) rectangle (8.5,2.4);
\fill[,fill=black,fill opacity=0.09] (0.,2.) -- (0.,0.) -- (2.,0.) -- (2.,2.) -- cycle;
\fill[dash pattern=on 3pt off 3pt,color=ffffff,fill=ffffff,fill opacity=0.69] (1.66,1.6266666666666658) -- (0.3,1.6266666666666658) -- (0.3,0.2666666666666663) -- (1.66,0.26666666666666594) -- cycle;
\draw (0.,2.)-- (0.,0.);
\draw (0.,0.)-- (2.,0.);
\draw (2.,0.)-- (2.,2.);
\draw (2.,2.)-- (0.,2.);
\draw [dash pattern=on 3pt off 3pt,color=ffffff] (1.66,1.6266666666666658)-- (0.3,1.6266666666666658);
\draw [dash pattern=on 3pt off 3pt,color=ffffff] (0.3,1.6266666666666658)-- (0.3,0.2666666666666663);
\draw [dash pattern=on 3pt off 3pt,color=ffffff] (0.3,0.2666666666666663)-- (1.66,0.26666666666666594);
\draw [dash pattern=on 3pt off 3pt,color=ffffff] (1.66,0.26666666666666594)-- (1.66,1.6266666666666658);
\draw [dash pattern=on 3pt off 3pt] (0.3,1.6266666666666658)-- (1.66,1.6266666666666658);
\draw [dash pattern=on 3pt off 3pt] (0.3,1.6266666666666658)-- (0.3,0.2666666666666663);
\draw [dash pattern=on 3pt off 3pt] (0.3,0.2666666666666663)-- (1.66,0.26666666666666594);
\draw [dash pattern=on 3pt off 3pt] (1.66,0.26666666666666594)-- (1.66,1.6266666666666658);
\draw [line width=0.4pt] (5.,1.) circle (1.cm);
\draw [rotate around={-0.058050415307076636:(5.001552872580047,1.000920107326479)},dash pattern=on 1pt off 1pt] (5.001552872580047,1.000920107326479) ellipse (1.0001703094848937cm and 0.3077009729140826cm);
\draw [shift={(5.589352280475849,1.8078761597534094)},line width=0.4pt,dash pattern=on 3pt off 3pt,fill=black,fill opacity=0.15]  plot[domain=2.7559887379791794:5.439758589086505,variable=\t]({1.*0.48443921801696804*cos(\t r)+0.*0.48443921801696804*sin(\t r)},{0.*0.48443921801696804*cos(\t r)+1.*0.48443921801696804*sin(\t r)});
\draw [shift={(5.,1.)},line width=0.4pt,fill=black,fill opacity=0.13]  plot[domain=0.5141997407372897:1.429845345702226,variable=\t]({1.*1.*cos(\t r)+0.*1.*sin(\t r)},{0.*1.*cos(\t r)+1.*1.*sin(\t r)});
\draw [->] (2.18,1.04) -- (3.86,1.0533333333333328);
\draw (2.0733333333333333,2.133333333333332) node[anchor=north west] {$\pi^{-1}(G)$};
\draw (5.326666666666667,2.3) node[anchor=north west] {$G$};
\draw (2.9533333333333336,1.52) node[anchor=north west] {$\pi$};

\begin{scriptsize}

\draw [fill=black] (5.589352280475849,1.8078761597534094) circle (1.5pt);
\end{scriptsize}
\end{tikzpicture}

\end{center}
El borde del cuadrado se identifica a un punto. El área sombreada es lo que corresponde a un casquete esférico.
\end{ej}



\begin{defi}Sea $(X,\Tau)$ un espacio topológico, $Y$ un conjunto y $f\func{X}{Y}$ una aplicación sobreyectiva entre conjuntos. Se llama \textbf{topología final} de $f$ a $\Tau_f=\{H ; f^{-1}(H)$ es abierto de $X\}$.
\end{defi}
\begin{observacion}  Con $\Tau_f$, $f$ se vuelve continua por definición. En particular, la topología cociente es la topología final de la proyección canónica $\pi\func{X}{X/\mathcal{R}}$, luego es continua. De hecho, toda topología final puede verse como una topología cociente como se verá en la siguiente proposición.
\end{observacion}
\begin{prop}\label{146} Dada una aplicación sobreyectiva $f\func{X}{Y}$ se define $\mathcal{R}_f$ sobre $X$ como $x\mathcal{R}_f x'\Leftrightarrow f(x)=f(x')$. Se cumple entonces que $\left(X/\mathcal{R}_f,\Tau_\pi\right)$ y $(Y,\Tau_f)$ son homeomorfos.
\end{prop}
\begin{dem}
\[
\begin{tikzcd}
X \ar[r, "f"]\arrow[d,"\pi "'] & \left(Y,\Tau_f\right)\arrow[dl,shift left=1ex,"\tilde{f} "]\\
\left(X/\mathcal{R}_f,\Tau_\pi\right) \arrow[ur,"\tilde{f}^{-1} "]
\end{tikzcd}
\]
Sea $\tilde{f}([x])=f(x)$. Entonces $\tilde{f}$ es biyectiva pues tiene por inversa a $\tilde{f}^{-1}(y)=[z]$ con $z$ tal que $f(z)=y$.\\
Veamos que $\tilde{f}$ es continua:

Sea $W$ abierto de $(Y,\Tau_f)$, entonces $f^{-1}(W)$ es abierto de $X$. Tenemos que ver que $\tilde{f}^{-1}(W)$ es abierto de $\Tau_\pi$, lo cual es cierto si y solo si $\pi^{-1}(\tilde{f}^{-1}(W))$ es abierto de $X$. Como $\tilde{f}\circ\pi=f$, $\pi^{-1}(\tilde{f}^{-1}(W))=f^{-1}(W)$, que es abierto de $X$.\\
Veamos que $\tilde{f}^{-1}$ es continua:

Sea $\Omega$ abierto de $(X/\mathcal{R}_f,\Tau_\pi)$, por lo que $\pi^{-1}(\Omega)$ es abierto de $X$. Se tiene que $(\tilde{f}^{-1})^{-1}(\Omega)=\tilde{f}(\Omega)$ es abierto de $(Y,\Tau_f)\Leftrightarrow f^{-1}(\tilde{f}(\Omega))$ es abierto de $X$. Por tanto bastará comprobar que $\pi^{-1}(\Omega)=f^{-1}(\tilde{f}(\Omega))$. Lo haremos por doble incluión.
\begin{enumerate}
\item[$\boxed{\subseteq}$] Sea $x\in\pi^{-1}(\Omega)\Rightarrow\pi(x)=[x]\in\Omega\Rightarrow\tilde{f}(\Omega)\ni\tilde{f}([x])=f(x)\in\tilde{f}(\Omega)\Rightarrow x\in f^{-1}(\tilde{f}(\Omega))$.
\item[$\boxed{\supseteq}$] Sea $x\in f^{-1}(\tilde{f}(\Omega))\Rightarrow f(x)=\tilde{f}([x])\in\tilde{f}(\Omega)\Rightarrow\tilde{f}^{-1}(f(x))\in\Omega\Rightarrow [x]\in\Omega\Rightarrow x\in\pi^{-1}([x])\subseteq\pi^{-1}(\Omega)\Rightarrow x\in\pi^{-1}(\Omega)$. $\QED$
\end{enumerate}
\end{dem}

\vspace{1cm}

\begin{defi} A $\mathcal{R}_f$ en la proposición \ref{146} se le llama la \textbf{relación inducida por $f$}. En general, una relación $\sim$ sobre $X$ se dice \textbf{compatible} con $f$ si $x\sim y\Rightarrow f(x)=f(y)$
\end{defi}

\begin{prop}[Propiedad universal]\label{149}
Sea $\mathcal{R}$ una relación de equivalencia sobre el espacio topológico $X$ y sea $f\func{X}{Z}$ una aplicación continua y cerrada (o abierta) compatible con $\mathcal{R}$. Se define $\tilde{f}\func{X/\mathcal{R}}{Z}$ como $\tilde{f}([x])=f(x)$, entonces $\tilde{f}\func{(X/\mathcal{R},\Tau_\pi)}{Z}$ es continua.
\end{prop}
\begin{dem}
\[
\begin{tikzcd}
X \ar[r, "f"]\arrow[d,"\pi "'] & Z\\
\left(X/\mathcal{R}_f,\Tau_\pi\right) \arrow[ur, dashrightarrow, "\tilde{f} "']
\end{tikzcd}
\]
Sea $W$ abierto de $Z$. Como $f$ es continua, $f^{-1}(W)$ es abierto de $X$. Además $\tilde{f}^{-1}(W)$ es abierto de $X/\mathcal{R}\Leftrightarrow\pi^{-1}(\tilde{f}^{-1}(W))$ es abierto de $X$. Como $f=\tilde{f}\circ\pi$, $\pi^{-1}(\tilde{f}^{-1}(W))=f^{-1}(W)$, que ya sabíamos que era abierto. $\QED$
\end{dem}

\vspace{0.4cm}

Un caso especialmente útil de la propiedad universal es el siguiente.
\begin{prop} Sea $f\func{X}{Y}$ continua y sobreyectiva. Sea $\mathcal{R}_f$ la relación inducida por $f$. Supongamos que $Y$ tiene la propiedad de separación de Hausdorff y supongamos además que $X$ es compacto. Entonces $\tilde{f}\func{X/\mathcal{R}_f}{Y}$ es homeomorfismo.
\end{prop}

\begin{dem}
Vamos a ver que $\tilde{f}$ cumple las tres propiedades para ser homeomorfismo.
\begin{enumerate}
\item[a)] Por la proposición \ref{149}, $\mathcal{R}_f$ compatible con $f\Rightarrow\tilde{f}$ continua.
\item[b)] $\tilde{f}$ es sobreyectiva porque lo es $f$. Veamos que es inyectiva:
\begin{equation*}
\tilde{f}([x])=\tilde{f}([x'])\Rightarrow f(x)=f(x')\Rightarrow x'\mathcal{R}_f x\Rightarrow [x]=[x'].
\end{equation*}
\item[c)] Para ver que $\tilde{f}^{-1}$ es continua basta ver que $\tilde{f}$ lleva cerrados en cerrados. Sea $F$ un cerrado de $X/\mathcal{R}_f\Rightarrow\pi^{-1}(F)$ es cerrado de $X$. Por ser $X$ compacto, esto implica que $\pi^{-1}(F)$ es compacto y por ser $f$ continua $f(\pi^{-1}(F))$ es compacto. Por lo tanto, al ser $Y$ Hausdorff, $\tilde{f}(F)=f(\pi^{-1}(F))$ es cerrado. $\QED$
\end{enumerate}

\end{dem}

\begin{coro}[Versión para espacios métricos] Sea $f\func{X}{Y}$ continua y sobreyectiva entre espacios métricos compactos. Entonces $\tilde{f}\func{X/\mathcal{R}_f}{Y}$ es homeomorfismo.
\end{coro}

\begin{ej}\

\begin{tikzpicture}[line cap=round,line join=round,>=triangle 45,x=1.0cm,y=1.0cm]
\clip(-0.1,-0.5) rectangle (11.74,2.1);
\draw (0.,0.)-- (0.,2.);
\draw (0.,0.)-- (2.,0.);
\draw (0.,2.)-- (2.,2.);
\draw (2.,2.)-- (2.,0.);
\draw (1.34,2.)-- (1.34,0.);
\draw (2.246666666666667,1.686666666666667) node[anchor=north west] {$\textit{Sea } X=[0,1]\times[0,1] \textit{ y } $};
\draw (2.233333333333334,1.26) node[anchor=north west] {$p\mathcal{R}q\Leftrightarrow \textit{están en la misma vertical.} $};
\draw (1.34,1.42) node[anchor=north west] {$p$};
\draw (1.1,0) node[anchor=north west] {$[p]$};
\begin{scriptsize}
\draw [fill=black] (1.34,1.26) circle (1.5pt);
\end{scriptsize}
\end{tikzpicture}

Vamos a ver que el cociente es $Y=[0,1]$. Basta tomar $f\func{X}{Y}\mid f(t,s)=t$ continua. Sean $p=(t,s)$ y $q=(t',s')$. Solo tenemos que comprobar lo siguiente
\[
p\mathcal{R}_f q\Leftrightarrow f(p)=f(q)\Leftrightarrow t=t'\Leftrightarrow p\mathcal{R}q
\]
\end{ej}

\vspace{0.2cm}

\begin{ej}[Cilindro] Sea $X=[0,1]\times [0,1]$ y $\mathcal{R}$ definida como $p\mathcal{R}q\begin{cases}
p=q\\
\acute{o}\\
p=(0,t),q=(1,t)
\end{cases}$. Vamos a considerar el cilindro $Y=S^1\times[0,1]$ y $f\func{X}{Y}\mid f(t,s)=(\cos(2\pi t), \sen(2\pi t), s)$. Sean $p=(t,s)$ y $q=(t',s')$. Se tiene por definición que $p\mathcal{R}_f q\Leftrightarrow f(p)=f(q)$, esto es,

\[
f(p)=f(q)\Leftrightarrow\left\{
\begin{array}{c}
\cos(2\pi t)=\cos(2\pi t')\\
\sen(2\pi t)= \sen(2\pi t')\\
s=s'
\end{array}\right. \Leftrightarrow\left\{\begin{array}{c}
s=s'\textit{ y } t=t'\\
\acute{o}\\
s=s',t=0,t'=1\textit{ o }s=s',t=1,t'=0
\end{array}\right.
\]

Por tanto $\mathcal{R}$ es la relación inducida por $f$, luego, aplicando la proposición \ref{146}, $X/\mathcal{R}$ y el cilindro son homeomorfos.
\end{ej}

\begin{ej}[Toro]
La ecuación paramétrica de la superficie tórica $T\subset\R^3$ generada al girar la circunferencia $(x-2)^2+z^2=1$ alrededor del eje $OZ$ es
\[
\begin{array}{l}
x=(2+\cos\theta)\cos\varphi\\
y=(2+\cos\theta)\sen\varphi\\
z=\sen\theta\\
0\leq\theta,\varphi\leq 2\pi

\end{array}
\]
Sea $X=[0,1]\times[0,1]$, buscamos una relacion $\mathcal{R}$ tal que $X/\mathcal{R}\cong T$. Para ello definimos
\begin{gather*}
f\func{X}{T}\\
f(t,s)=((2+\cos(2\pi t))\cos(2\pi s), (2+\cos(2\pi t)\sen(2\pi s), \sen(2\pi t))
\end{gather*}
Se tiene que $f$ es continua y que $T$ es un compacto Hausdorff. Vamos a buscar una expresión explícita de la relación $\mathcal{R}_f$ (esta es la relación que buscamos)
\begin{gather*}
f(s,t)=f(s',t')\Leftrightarrow\begin{cases}
\sen(2\pi t)=\sen(2\pi t')\\
(2+\cos(2\pi t))\cos(2\pi s)=(2+\cos(2\pi t'))\cos(2\pi s')\\
(2+\cos(2\pi t)\sen(2\pi s)=(2+\cos(2\pi t')\sen(2\pi s')
\end{cases}\Leftrightarrow\begin{cases}
(t,s)=(t',s')\\
\acute{o}\\
s=0,s'=1, t=t'\\
\acute{o}\\
s=1,s'=0, t=t'\\
\acute{o}\\
s=s', t=0,t'=1\\
\acute{o}\\
s=s',t=1,t'=0
\end{cases}
\end{gather*}
\end{ej}

\begin{ej}[Banda de Möbius]
Consideremos la circunferencia dada por
\begin{gather*}
\begin{array}{l}
x=2\cos\theta\\
y=2\sen\theta\\
z=0
\end{array}
\end{gather*}
Vamos a considerar el segmento $V_\theta$  centrado en el punto de esa circunferencia de ángulo $\theta$ de longitud $2$ girado un ángulo $\frac{\theta}{2}$ respecto al eje $OY$. Al a unión $M=\underset{0\leq\theta\leq\pi}{\bigcup}V_\theta$ se le llama \textbf{banda de Möbius}. De esta forma, las ecuaciones paramétricas de nuestra banda de Möbius M son
\begin{gather*}
\begin{array}{l}
x=2\cos\theta +\lambda \cos\theta \sen\frac{\theta}{2}=\cos\theta(2+\lambda \sen\frac{\theta}{2})\\
y=2\sen\theta +\lambda \sen\theta \sen\frac{\theta}{2}=\sen\theta(2+\lambda \sen\frac{\theta}{2})\\
z=\lambda \cos\frac{\theta}{2}\\
\end{array}\\
0\leq\theta\leq 2\pi,\ |\lambda|\leq 1
\end{gather*}
Queremos encontrar $\mathcal{R}_f$ sobre $X=[0,1]\times[0,1]$ tal que $X/\mathcal{R}_f\cong M$. Para ello definimos
\[
f(t,s)=\left((2+(2s-1)sen(\pi t))\cos(2\pi t),(2+(2s-1)\sen(\pi t))\sen(2\pi t),(2s-1)\cos(\pi t) \right).
\]
Se puede comprobar que
\[
(t,s)\mathcal{R}_f (t',s')\Leftrightarrow\begin{cases}
(t,s)=(t',s')\\
\acute{o}\\
t=0,t'=1, s=1-s'\\
\acute{o}\\
t=1,t'=0,s'=1-s
\end{cases}
\]
\end{ej}

\vspace{0.2cm}

\begin{nota} Los resultados obtenidos en los dos últimos ejemplos son los utilizados para construir los modelos topológicos que veremos más adelante.
\end{nota}

\begin{defi}
Dados dos espacios topológicos disjuntos $X$ e $Y$, sean $x_0\in X, y_0\in Y$. Consideremos sobre $X\sqcup Y$ la relación de equivalencia según la cual $x_0\sim y_0$ y todos los demás puntos de $X$ e $Y$ solo están relacionados consigo mismos. Definimos entonces la \textbf{unión por un punto} como el cociente $(X\sqcup Y)/\sim$ y la denotamos $X\vee Y$.
\end{defi}

\begin{ej}\

\begin{tikzpicture}[line cap=round,line join=round,>=triangle 45,x=1.0cm,y=1.0cm]
\clip(-4.132863999999993,-1.1) rectangle (8.133802666666664,1.326794666666668);
\draw(-1.,0.) circle (1.cm);
\draw(1.,0.) circle (1.cm);
\draw [fill=black] (0,0) circle (2pt);
\draw (0,0) node[anchor=north west] {$x_0=y_0$};
\draw (-3.2,1) node[anchor=north west] {$S^1\vee S^1$};

\end{tikzpicture}
\end{ej}

\begin{ej} Se define en $\R$ la relación $x\sim x'\Leftrightarrow x-x'=2k\pi,k\in\mathbb{Z}$. Se cumple que $\R/\sim\cong S^1$. El modelo es una circunferencia de radio 1 centrada en el origen. Basta comprobar la compatibilidad de $f(t)=(\sen{t},\cos{t})$, que se deja como ejercicio.
\end{ej}

\begin{defi} Sean $X$ e $Y$ espacios topológicos y sea $A\subseteq X$ un supespacio. Sea $f\func{A}{Y}$ una función continua. Se define el \textbf{espacio de adjunción} $X\sqcup_f Y$ como el cociente $X\sqcup Y/\sim$ bajo la relación generada por $x\sim f(x)\ \forall x\in A$. A este espacio también se le conoce con el nombre de espacio de pegamiento.
\end{defi}

\section{Suspensión de espacios}
\begin{defi}Se define la \textbf{topología producto} sobre $X\times Y$ donde $(X,\Tau_1), (Y,\Tau_2)$ son espacios topológicos como
\[
\Tau_1*\Tau_2=\{\emptyset\}\cup\{H\subseteq X\times Y:\forall(x,y)\in X\times Y\ \exists U\in\Tau_1,\ \exists V\in\Tau_2\mid(x,y)\in U\times V\subseteq H\}
\]
\end{defi}

\begin{ejer}Probar que si $\Tau=$ topología euclídea de $\R^n$ y $\Tau'=$ topología euclídea de $\R^m$, entonces $\Tau*\Tau'$ coincide con la topología euclídea de $\R^{n+m}=\R^n\times\R^m$.
\end{ejer}



\begin{defi} Llamamos \textbf{cilindro} de un espacio $X$ a $X\times I$ con la topología producto, siendo $I$ un intervalo.
\end{defi}
\begin{tikzpicture}[line cap=round,line join=round,>=triangle 45,x=1.0cm,y=1.0cm]
\clip(-2.16,-0.5) rectangle (7,2.5);
\draw [rotate around={0.:(1.78,0.)}] (1.78,0.) ellipse (0.833069093729879cm and 0.29258180894909247cm);
\draw (0.9469309062701207,0.)-- (0.95,2.);
\draw (2.613069093729879,0.)-- (2.61333,2.);
\draw [rotate around={0.24485226804443055:(1.78,2.0033333333333347)}] (1.78,2.0033333333333347) ellipse (0.8204566483246158cm and 0.25443663389722054cm);
\draw [->] (2.813333333333334,0.18) -- (2.8,1.7666666666666708);
\draw (2.6,0.26) node[anchor=north west] {$0$};
\draw (2.6,2.18) node[anchor=north west] {$1$};
\draw (2.946666666666667,1.1) node[anchor=north west] {$t$};
\end{tikzpicture}
\begin{defi} A partir del cilindro, definimos el \textbf{cono} de un espacio $X$, denotado $CX$,  con la relación
\begin{gather*}
(x,t)\mathcal{R}(x',t')\Leftrightarrow\left\{\begin{array}{c}
t=t'=1\\
\acute{o}\\
(x,t)=(x',t')
\end{array}\right.
\end{gather*}
\end{defi}
\begin{tikzpicture}[line cap=round,line join=round,>=triangle 45,x=1.0cm,y=1.0cm]
\clip(-1.4,-0.3) rectangle (12,2.5);
\draw [rotate around={0.:(1.78,0.)}] (1.78,0.) ellipse (0.833069093729879cm and 0.29258180894909247cm);
\draw (0.9469309062701207,0.)-- (0.95,2.);
\draw (2.613069093729879,0.)-- (2.61333,2.);
\draw [rotate around={0.24485226804443055:(1.78,2.0033333333333347)}] (1.78,2.0033333333333347) ellipse (0.8204566483246158cm and 0.25443663389722054cm);
\draw [->] (3.,1.) -- (4.,1.);
\draw [rotate around={0.:(5.253333333333333,0.)}] (5.253333333333333,0.) ellipse (0.8078289758137795cm and 0.30834484437598847cm);
\draw (5.293333333333335,1.9933333333333372)-- (4.445504357519554,0.);
\draw (5.293333333333335,1.9933333333333372)-- (6.061162309147113,0.);
\draw (3.36,0.9533333333333374) node[anchor=north west] {$\pi$};
\draw (6.346666666666668,1.6333333333333375) node[anchor=north west] {$CX= \quott{X\times I}{\mathcal{R}}$};
\begin{scriptsize}
\draw [fill=black] (5.293333333333335,1.9933333333333372) circle (2.5pt);
\end{scriptsize}
\end{tikzpicture}

\begin{ej} Sea $X=S^1$, entonces $CS^1\cong B^2$ (círculo de radio 1). Para probarlo buscamos una aplicación $f\func{S^1\times I}{B^2}$ continua y sobreyectiva tal que $\mathcal{R}_f=\mathcal{R}$ sea la relación compatible con $f$. Escogemos $f(x,t)=(1-t)x\in\R^2$. Claramente es continua y $f(S^1\times I)=B^2$. Veamos que $\mathcal{R}_f=\mathcal{R}$:
\begin{gather*}
(x,t)\mathcal{R}_f(x',t')\underset{def}{\Leftrightarrow}f(x,t)=f(x',t')\Leftrightarrow (1-t)x=(1-t')x'\Leftrightarrow\left\{\begin{array}{c}
t=t'=1 \\
\acute{o}\\
x=x', t=t'
\end{array}\right.
\end{gather*}
\end{ej}

\begin{defi} Definimos en $X\times I$ la \textbf{suspensión} de un espacio, denotada $\Sigma X$, a través de la relación $\mathcal{R}$ tal que
\begin{gather*}
(x,t)\mathcal{R}(x',t')\Leftrightarrow\left\{\begin{array}{c}
t=t'=1\\
\acute{o}\\
t=t'=0\\
\acute{o}\\
(x,t)=(x',t')\textit{ si }t,t'\neq 0,1
\end{array}\right.
\end{gather*}
\end{defi}
\begin{tikzpicture}[line cap=round,line join=round,>=triangle 45,x=1.0cm,y=1.0cm]
\clip(-2,-0.8) rectangle (10,2.3);
\draw [rotate around={0.:(1.78,0.)}] (1.78,0.) ellipse (0.833069093729879cm and 0.29258180894909247cm);
\draw (0.9469309062701207,0.)-- (0.95,2.);
\draw (2.613069093729879,0.)-- (2.61333,2.);
\draw [rotate around={0.24485226804443055:(1.78,2.0033333333333347)}] (1.78,2.0033333333333347) ellipse (0.8204566483246158cm and 0.25443663389722054cm);
\draw [->] (3.,1.) -- (4.,1.);
\draw [rotate around={0.:(5.253333333333333,0.)}] (5.253333333333333,0.) ellipse (0.8078289758137795cm and 0.30834484437598847cm);
\draw (5.293333333333335,1.9933333333333372)-- (4.445504357519554,0.);
\draw (5.293333333333335,1.9933333333333372)-- (6.061162309147113,0.);
\draw [->] (6.146666666666668,0.9933333333333373) -- (7.32,0.98);
\draw [rotate around={0.:(8.74666666666669,1.)}] (8.74666666666669,1.) ellipse (0.7921557309162034cm and 0.26457435800196405cm);
\draw (8.746666666666668,2.0333333333333377)-- (7.954510935750486,1.);
\draw (8.746666666666668,2.0333333333333377)-- (9.527841570453878,1.0439000191034111);
\draw (8.746666666666668,-0.033333333333337656)-- (7.954510935750486,1.);
\draw (8.746666666666668,-0.033333333333337656)-- (9.527841570453882,0.956099980896595);
\draw (1.3,-0.32666666666666294) node[anchor=north west] {$X\times I$};
\draw (4.75,-0.3) node[anchor=north west] {$CX$};
\draw (8.4,-0.22) node[anchor=north west] {$\Sigma X$};
\begin{scriptsize}
\draw [fill=black] (5.293333333333335,1.9933333333333372) circle (2.5pt);
\draw [fill=black] (8.746666666666668,2.0333333333333377) circle (2.5pt);
\draw [fill=black] (8.746666666666668,-0.033333333333337656) circle (2.5pt);
\end{scriptsize}
\end{tikzpicture}

\begin{ej} Sea $X=S^1$, entonces $\Sigma X=\Sigma S^1\cong S^2$. Se prueba de forma análoga al ejemplo anterior con $f(x_1,x_2,t)=(\sqrt{1-(2t-1)^2}x_1,\sqrt{1-(2t-1)^2}x_2,2t-1)$. Se pueden generalizar las fórmulas a dimensiones arbitrarias
\[
CS^n\cong B^{n+1}\qquad \Sigma S^n\cong S^{n+1}.
\]
\end{ej}



\section{Espacios Proyectivos}
El espacio proyectivo $\Pro_n\R$ se entiende como el conjunto de direcciones de $\R^{n+1}$. Más formalmente, se define $\Pro_n\R=\R^{n+1}\setminus\{0\}/\sim$, donde dados $v,w\in\R^{n+1}\setminus\{0\}$, $v\sim w\Leftrightarrow\exists\lambda\neq 0\mid v=\lambda w$. Con esta estructura tenemos la proyección $\pi\func{\R^{n+1}\setminus\{0\}}{\R^{n+1}\setminus\{0\}/\sim}=\Pro_n\R$. Por lo tanto, dotaremos a $\Pro_n\R$ con la topología cociente en la proyección $\pi$.
\begin{ej}
\item[$\boxed{\Pro_1\R}$] Sea $f\func{\R^2\setminus\{0\}}{S^1}$ definida por $f(x)=\dfrac{x}{\norm{x}}$ y sea $g\func{S^1}{\R^2\setminus\{0\}}$ la inclusión $g(y)=y$. En $S^1$ consideramos la relación $\mathcal{R}$ definida por $y\mathcal{R}y'\Leftrightarrow y=\pm y'$. Sea entonces la aplicación cociente $\pi'\func{S^1}{S^1/\mathcal{R}}$. Teníamos que $v\sim w\Leftrightarrow v=\lambda w$ por lo que
\begin{equation*}
f(v)=\frac{v}{\norm{v}}=\frac{\lambda w}{\norm{\lambda w}}=\frac{\lambda}{|\lambda|}\frac{w}{\norm{w}}=\pm f(w)
\end{equation*}

\begin{figure}[h!]
	\includegraphics[scale=0.3]{circ}
\end{figure}

Luego $v\sim w\Leftrightarrow f(v)\mathcal{R}f(w)\Leftrightarrow\pi'(f(v))=\pi'(f(w))$. Entonces, tenemos el siguiente diagrama
\[
\begin{tikzcd}
\R^2\setminus\{0\} \ar[r, "f"]\arrow[d,"\pi "] & S^1 \arrow[r,"g"]\arrow[d,"\pi' "] & \R^2\setminus\{0\}\arrow[d,"\pi "]\\
\Pro_1\R \arrow[r, dashrightarrow, "\tilde{f} "] & S^1/\mathcal{R}\arrow[r, dashrightarrow, "\tilde{g} "] & \Pro_1\R
\end{tikzcd}
\]
Las aplicaciones $\tilde{f}([v])=[f(v)]$ y $\tilde{g}([x])=[g(x)]$ son continuas por la proposición \ref{149}. Además
\begin{gather*}
\tilde{g}(\tilde{f}([v]))=[g(f(v))]=\left[\frac{v}{\norm{v}}\right]=[v]\Rightarrow\tilde{g}\circ\tilde{f}=Id\\
\tilde{f}(\tilde{g}([v]))=[f(g(v))]=[v]\Rightarrow\tilde{f}\circ\tilde{g}=Id
\end{gather*}
Por todo ello, $\tilde{f}$ es homeomorfismo.\\
Tenemos entonces que todos los puntos de la parte de abajo de $S^1$ tienen un único representante arriba (salvo los extremos $(0,1)$ y $(0,-1)$, que representan el mismo punto de $\Pro_1\R$), así que si los identificamos $S^1/\mathcal{R}\cong S^1$.

\begin{figure}[h]
	\centering
	\includegraphics[scale=0.4]{s1}
\end{figure}

Para ello definimos $h\func{S^1}{S^1}$ de modo que si $e^{i\theta}\in S^1$ con $0\leq\theta\leq 2\pi$ entonces $h(e^{i\theta})=e^{i2\theta}$. Observemos lo siguiente:
\begin{gather*}
e^{i2\theta}=e^{i2\psi}\Leftrightarrow\left\{\begin{array}{c}
cos(2\theta)=cos(2\psi)\\
sen(2\theta)=sen(2\psi)
\end{array}\right\}\Leftrightarrow\begin{array}{c c}
\textit{o bien } & 2\theta=2\psi\Rightarrow\theta=\psi\\
\textit{o bien } & 2\theta=2\psi+2k\pi\Rightarrow\theta=\psi+\pi
\end{array}
\end{gather*}
Podemos definir entonces $\mathcal{R}_h$ de modo que dos puntos de la circunferencia están relacionados si y solo si son el mismo o son opuestos. Luego, con $\tilde{h}$ como homeomorfismo, $\Pro_1\R=S^1/\mathcal{R}\cong S^1$.\\
\item[$\boxed{\Pro_2\R}$] Análogamente al ejemplo anterior, definimos $f\func{\R^3\setminus\{0\}}{S^2}\mid f(v)=\dfrac{v}{\norm{v}}$ y $g\func{S^2}{\R^3\setminus\{0\}}\mid g(x)=x$. Tendremos pues el siguiente diagrama donde para $x,y \in S^2$, $x\mathcal{R}y\Leftrightarrow x=\pm y$
\[
\begin{tikzcd}
\R^3\setminus\{0\} \ar[r, "f"]\arrow[d,"\pi "] & S^2 \arrow[r,"g"]\arrow[d,"\pi' "] & \R^3\setminus\{0\}\arrow[d,"\pi "]\\
\Pro_2\R \arrow[r, dashrightarrow, "\tilde{f} ","\cong"'] & S^2/\mathcal{R}\arrow[r, dashrightarrow, "\tilde{g} "] & \Pro_2\R
\end{tikzcd}
\]
Además, si denotamos por $E^+$ a la semiesfera superior y sobre ella la relación $x\mathcal{R}'y\Leftrightarrow x=y$ ó $x,y\in S^1$(el borde de la semiesfera), se tiene un homeomorfismo

\begin{equation*}
\Pro_2\R\overset{\tilde{f}}{\cong} S^2/\mathcal{R}\overset{(1)}{\cong} E^+/\mathcal{R}'
\end{equation*}

Para encontrar el homeomorfismo $(1)$ consideramos las aplicaciones $k:E^+\longhookarrow S^2$ y $h\func{S^2}{E^+}$, donde $k$ es la inclusión y $h(x,y,z)=(x,y,|z|)$. Entonces es inmediato que $\mathcal{R}'$ es compatible con $\pi\circ k$ y que $\mathcal{R}$ lo es con $\pi\circ h$, por lo que tenemos el siguiente diagrama
\[
\begin{tikzcd}
E^+ \ar[r, "k"]\arrow[d,"\pi "] & S^2 \arrow[r,"h"]\arrow[d,"\pi' "] & E^+\arrow[d,"\pi "]\\
E^+/\mathcal{R}' \arrow[r, dashrightarrow, "\tilde{k} "] & S^2/\mathcal{R}\arrow[r, dashrightarrow, "\tilde{h} "] & E^+/\mathcal{R}'
\end{tikzcd}
\]
donde $\tilde{h}$ y $\tilde{k}$ son continuas y además $\tilde{h}\circ\tilde{k}=Id$ y $\tilde{k}\circ\tilde{h}=Id$, por lo que son homeomorfismos.

En lugar de $E^+$ podemos tomar la bola cerrada $B^2$, pues ambos espacios son homeomorfos, y la relación correspondiente a $\mathcal{R}'$  sobre $B^2$,  $p\mathcal{R}''q\Leftrightarrow p=q$ ó $p,q\in S^1$ y $p=\pm q$.
Vamos ahora al caso general.\\
\item[$\boxed{\Pro_n\R}$] Tendremos en general el cociente $B^n/\sim$ con $p\sim q\Leftrightarrow p=q$ ó $p,q\in S^{n-1}$ y $p=\pm q$. De esta forma $\Pro_n\R\cong B^n/\sim$. Nótese que en el borde de la bola aparece $\Pro_{n-1}\R$.
\begin{center}
\begin{tikzpicture}[line cap=round,line join=round,>=triangle 45,x=1.0cm,y=1.0cm]
\clip(-5.4066666666666645,-2.5) rectangle (8.78,1.4);
\draw [line width=1.2pt,fill=black,fill opacity=0.1] (0.,0.) circle (1.cm);
\draw [line width=1.2pt,] (0.24565880350762515,0.9693563597868443)-- (0.12666666666666662,1.16);
\draw [line width=1.2pt,] (0.24565880350762515,0.9693563597868443)-- (0.06,0.7866666666666654);
\draw [line width=1.6pt,] (-0.24886078369838055,-0.9685392662855894)-- (-0.04666666666666665,-0.826666666666667);
\draw [line width=1.2pt,] (-0.24886078369838055,-0.9685392662855894)-- (-0.06,-1.1733333333333333);
\draw (0.23333333333333325,1.4) node[anchor=north west] {$\Pro_1\mathbb{R}$};
\draw (-0.32666666666666655,0.2933333333333324) node[anchor=north west] {$\Pro_2\mathbb{R}$};
\draw (-1.5,0.25) node[anchor=north west] {$a$};
\draw (1.1533333333333329,0.25) node[anchor=north west] {$a$};
\begin{scriptsize}
\draw [fill=black] (-1.,0.) circle (2.5pt);
\draw [fill=black] (1.,0.) circle (2.5pt);
\end{scriptsize}
\end{tikzpicture}
\end{center}
\end{ej}

\begin{ejer}(Importante) $\Pro_2\R$ se puede ``ver'' en  $\R^4$. Sea $f\func{S^2}{\R^4}\mid f(x,y,z)=(xy,yz,xz,x^2 +2y^2 +3z^2)$
\begin{enumerate}
\item Probar que $f(p)=f(q)\Leftrightarrow p=\pm q$.
\item Deducir que $\Pro_2\R$ es homeomorfo a un subespacio de $\R^4$.
\end{enumerate}
\begin{solucion}\
\begin{enumerate}
\item $\boxed{\Leftarrow}$ Sean $(x,y,z)=\pm(x',y',z')\Rightarrow f(x,y,z)=(xy,yz,xz,x^2 +2y^2 +3z^2)=f(x',y',z')$.\\
$\boxed{\Rightarrow}$ Para esta implicación diferenciaremos varios casos.
\begin{itemize}
\item $x,y,z\neq 0$\\
$f(x,y,z)=f(x',y',z')\Rightarrow\left\{\begin{array}{c}
xy=x'y'\\
yz=y'z'\\
xz=x'z'\\
\end{array}\right.\Rightarrow x',y',z'\neq 0$\\
Esto implica
\begin{gather*}
\frac{x}{x'}=\frac{y'}{y};\ \frac{y}{y'}=\frac{z'}{z};\\
\frac{x}{x'}=\frac{z'}{z}=\frac{y}{y'}=\frac{x'}{x}\Rightarrow x^2=(x')^2\Rightarrow x=\pm x'
\end{gather*}
Por lo que o bien
\[
x=x'\Rightarrow y=y',z=z'
\]
o bien
\[
x=-x'\Rightarrow y=-y',z=-z'
\]
\item $x=0$, $y,z\neq 0$\\
En primer lugar, como $(x,y,z)\in S^2\Rightarrow x^2+y^2+z^2=1\Rightarrow x^2+2y^2+3z^2=2+z^2$. Ahora
\begin{gather*}
f(x,y,z)=f(x',y',z')\Rightarrow\left\{\begin{array}{c}
0=xy=x'y'\\
0=xz=x'z'\\
0\neq yz=y'z'
\end{array}\right.
\end{gather*}
De las dos últimas expresiones deducimos que $x'=0$, luego
\[
2+(z')^2=2+z^2\Rightarrow z=\pm z'\Rightarrow y=\pm y'.
\]
\item $x=y=0$, $z\neq 0$\\
Como $(x,y,z)\in S^2\Rightarrow z=\pm 1$. Entonces
\begin{gather*}
xy=x'y'=0\\
yz=y'z'=0\\
xz=x'z'=0\\
0+0+3z^2=3=(x')^2+2(y')^2+3(z')^2=1+(y')^2+2(z')^2\Rightarrow 2=(y')^2+2(z')^2
\end{gather*}
Ahora bien, $z'\neq 0$ porque si lo fuera, $|y'|=\sqrt{2}>1$. Aplicando esto a las igualdades anteriores y despejando $z'$ de esta última, llegamos a que
\[
x'=0, y'=0, z'=\pm 1
\]
\end{itemize}
\item $X=f(S^2)\subseteq\R^4$ es un compacto Hausdorff y $f\func{S^2}{X}$ es sobreyectiva. Por la propiedad universal esto significa que $\tilde{f}\func{S^2/\mathcal{R}_f}{X}$ es homeomorfismo. \qed
\end{enumerate}
\end{solucion}
\end{ejer}

\begin{ejer} Probar que el cociente $\R/\mathcal{R}$ no es Hausdorff, donde $x\mathcal{R}y\Leftrightarrow x=y$ ó $x,y\in\Q$.
\begin{solucion}
$\boxed{RA}$ Sean $[x],[y]\in\R/\mathcal{R}$ con $[x]\in U$ abierto, $[y]\in V$ abierto, y $U\cap V=\emptyset$. Por definición de topología cociente, si denotamos $\pi\func{\R}{\R/\mathcal{R}}$ a la proyección canónica, entonces $\pi^{-1}(U)$ y $\pi^{-1}(V)$ son abiertos. Además, $U\cap V=\emptyset\Rightarrow \pi^{-1}(U\cap V)=\pi^{-1}(U)\cap\pi^{-1}(V)=\emptyset$. Se tiene
\begin{itemize}
\item $\pi^{-1}(U)$ abierto $\Rightarrow \exists\ q\in\Q\mid q\in\pi^{-1}(U)\Rightarrow [q]\in U$.
\item $\pi^{-1}(V)$ abierto $\Rightarrow \exists\ q'\in\Q\mid q'\in\pi^{-1}(V)\Rightarrow [q']\in V$.
\end{itemize}
Pero $[q]=[q']\in U\cap V$, con lo que hemos llegado a una contradicción. \qed
\end{solucion}
\end{ejer}

\vspace{0.2cm}

\begin{nota} Si $A\subseteq X$ y $\mathcal{R}$ es la relación $x\mathcal{R}y\Leftarrow x=y$ ó $x,y\in A$, se suele escribir $X/A$.
\end{nota}

\begin{ejer} Probar que $\R^2/D^2$ es homeomorfo a $\R^2$, siendo $D^2$ la bola unidad cerrada. Ayuda: usar la función $f\func{\R^2}{\R^2}$ definida a trozos como $f(D^2)=0$ y $f(x)=\dfrac{||x||-1}{||x||}x\ \forall x\in\R^2\setminus D^2$.
\end{ejer}

\newpage

\section{Modelos Topológicos}
Antes de comenzar con la clasificación de superficies es importante familiarizarse con los modelos usados para representarlas. Estos modelos son una especie de ``instrucciones"\ sobre cómo se construye la superficie. Resultan especialmente importantes cuando tratamos con superficies que no pueden ser representadas en 3 dimensiones. Veamos algunos ejemplos.

\vspace{0.4cm}

\begin{ej}[Modelo de una caja]\

\definecolor{zzttqq}{rgb}{0.6,0.2,0.}
\begin{tikzpicture}[line cap=round,line join=round,>=triangle 45,x=1.0cm,y=1.0cm]
\clip(-1.58,-2.5) rectangle (13.753333333333336,4.5);
\fill [color=zzttqq,fill=zzttqq, fill opacity=0.1](0.,0.)-- (0.,2.)--(2,2)--(4,2)--(4,4)--(6,4)--(6,2)--(8,2)--(8,0)--(6,0)--(6,-2)--(4,-2)--(4,0)--(2,0)--cycle;
\draw (0.,0.)-- (0.,2.);
\draw (0.,0.)-- (2.,0.);
\draw (2.,0.)-- (2.,2.);
\draw (0.,2.)-- (2.,2.);
\draw (2.,0.)-- (4.,0.);
\draw (4.,0.)-- (4.,2.);
\draw (2.,2.)-- (4.,2.);
\draw (4.,0.)-- (6.,0.);
\draw (6.,0.)-- (6.,2.);
\draw (4.,2.)-- (6.,2.);
\draw (6.,0.)-- (8.,0.);
\draw (8.,0.)-- (8.,2.);
\draw (6.,2.)-- (8.,2.);
\draw (4.,2.)-- (4.,4.);
\draw (4.,4.)-- (6.,4.);
\draw (6.,4.)-- (6.,2.);
\draw (4.,0.)-- (4.,-2.);
\draw (4.,-2.)-- (6.,-2.);
\draw (6.,-2.)-- (6.,0.);
\draw [->] (4.,2.) -- (4.,3.1);
\draw [->] (4.,2.) -- (3.0066666666666673,2.);
\draw [->] (4.,4.) -- (5.046666666666668,4.);
\draw [->] (6.,2.) -- (6.,3.1);
\draw [->] (6.,2.) -- (7.153333333333335,2.);
\draw [->] (6.,0.) -- (7.006666666666668,0.);
\draw [->] (6.,0.) -- (6.,-1.1666666666666674);
\draw [->] (8.,0.) -- (8.,1.0333333333333339);
\draw [->] (2.,2.) -- (0.8733333333333335,2.);
\draw [->] (0.,0.) -- (0.,1.14);
\draw [->] (2.,0.) -- (0.9,0.);
\draw [->] (4.,0.) -- (2.9666666666666672,0.);
\draw [->] (4.,0.) -- (4.,-1.153333333333334);
\draw [->] (4.,-2.) -- (5.18,-2.);
\draw (3.6066666666666674,3.2866666666666684) node[anchor=north west] {$a$};
\draw (3.0466666666666673,2.446666666666668) node[anchor=north west] {$a$};
\draw (4.886666666666668,4.446666666666669) node[anchor=north west] {$c$};
\draw (0.9266666666666669,2.46) node[anchor=north west] {$c$};
\draw (5.02,-2.02) node[anchor=north west] {$g$};
\draw (0.9533333333333335,-0.08666666666666671) node[anchor=north west] {$g$};
\draw (3.5,-0.7666666666666672) node[anchor=north west] {$f$};
\draw (3.033333333333334,-0.02) node[anchor=north west] {$f$};
\draw (6.14,3.2466666666666684) node[anchor=north west] {$b$};
\draw (6.926666666666668,2.5) node[anchor=north west] {$b$};
\draw (8.22,1.193333333333334) node[anchor=north west] {$d$};
\draw (-0.5,1.2466666666666675) node[anchor=north west] {$d$};
\draw (6.86,0.0066666666666666706) node[anchor=north west] {$e$};
\draw (6.206666666666668,-0.7933333333333338) node[anchor=north west] {$e$};
\end{tikzpicture}

\end{ej}

\begin{ej}[Plano proyectivo]\

\definecolor{qqffqq}{rgb}{0.,1.,0.}
\definecolor{zzttqq}{rgb}{0.6,0.2,0.}
\begin{tikzpicture}[line cap=round,line join=round,>=triangle 45,x=1.0cm,y=1.0cm]
\clip(-1.82,-1.1) rectangle (13.513333333333335,1.1);
\draw[fill=qqffqq, opacity=0.1](1,0) circle (1cm);
\draw [shift={(0.,0.)},color=zzttqq,fill=zzttqq,fill opacity=0.1]  (0,0) --  plot[domain=-1.3918053133444221:1.3941368213526866,variable=\t]({1.*0.3560735948210689*cos(\t r)+0.*0.3560735948210689*sin(\t r)},{0.*0.3560735948210689*cos(\t r)+1.*0.3560735948210689*sin(\t r)}) -- cycle ;
\draw [shift={(2.,0.)},color=zzttqq,fill=zzttqq,fill opacity=0.1]  (0,0) --  plot[domain=1.7474558322371072:4.533397966934215,variable=\t]({1.*0.3514841123014168*cos(\t r)+0.*0.3514841123014168*sin(\t r)},{0.*0.3514841123014168*cos(\t r)+1.*0.3514841123014168*sin(\t r)}) -- cycle ;
\draw [->] (2.4866666666666672,0.) -- (3.233333333333334,0.);
\draw [shift={(1.,0.)},line width=2.8pt,color=qqffqq]  plot[domain=3.141592653589793:3.4995746804907433,variable=\t]({1.*1.*cos(\t r)+0.*1.*sin(\t r)},{0.*1.*cos(\t r)+1.*1.*sin(\t r)});
\draw [shift={(1.,0.)},line width=2.8pt,color=qqffqq]  plot[domain=0.:0.3534072430575585,variable=\t]({1.*1.*cos(\t r)+0.*1.*sin(\t r)},{0.*1.*cos(\t r)+1.*1.*sin(\t r)});
\draw [shift={(1.,0.)},line width=1.2pt]  plot[domain=-2.783610626688843:0.,variable=\t]({1.*1.*cos(\t r)+0.*1.*sin(\t r)},{0.*1.*cos(\t r)+1.*1.*sin(\t r)});
\draw [shift={(1.,0.)},line width=1.2pt]  plot[domain=0.3534072430575585:3.141592653589793,variable=\t]({1.*0.9995039532145752*cos(\t r)+0.*0.9995039532145752*sin(\t r)},{0.*0.9995039532145752*cos(\t r)+1.*0.9995039532145752*sin(\t r)});
\draw [shift={(4.,0.)},color=zzttqq,fill=zzttqq,fill opacity=0.1]  (0,0) --  plot[domain=1.5893127288629036:4.730905382452701,variable=\t]({1.*0.360061723103754*cos(\t r)+0.*0.360061723103754*sin(\t r)},{0.*0.360061723103754*cos(\t r)+1.*0.360061723103754*sin(\t r)}) -- cycle ;
\draw [shift={(4.313333333333334,0.)},color=zzttqq,fill=zzttqq,fill opacity=0.1]  (0,0) --  plot[domain=-1.5707963267948966:1.5707963267948966,variable=\t]({1.*0.34666666666666646*cos(\t r)+0.*0.34666666666666646*sin(\t r)},{0.*0.34666666666666646*cos(\t r)+1.*0.34666666666666646*sin(\t r)}) -- cycle ;
\draw [line width=2.pt,color=qqffqq] (4.,0.)-- (3.993333333333334,0.36);
\draw [line width=2.pt,color=qqffqq] (4.313333333333334,0.)-- (4.313333333333334,-0.34666666666666646);
\draw [->] (5.,0.) -- (6.,0.);
\draw [shift={(6.833333333333335,0.)},color=zzttqq,fill=zzttqq,fill opacity=0.1]  (0,0) --  plot[domain=1.5385494443596441:4.745709976262935,variable=\t]({1.*0.413548331180555*cos(\t r)+0.*0.413548331180555*sin(\t r)},{0.*0.413548331180555*cos(\t r)+1.*0.413548331180555*sin(\t r)}) -- cycle ;
\draw [shift={(7.153333333333335,0.)},color=zzttqq,fill=zzttqq,fill opacity=0.1]  (0,0) --  plot[domain=-1.5707963267948966:1.5707963267948966,variable=\t]({1.*0.4*cos(\t r)+0.*0.4*sin(\t r)},{0.*0.4*cos(\t r)+1.*0.4*sin(\t r)}) -- cycle ;
\draw [->] (8.,0.) -- (9.,0.);
\draw [shift={(9.726666666666668,0.)},color=zzttqq,fill=zzttqq,fill opacity=0.1]  (0,0) --  plot[domain=1.5707963267948966:4.71238898038469,variable=\t]({1.*0.38666666666666644*cos(\t r)+0.*0.38666666666666644*sin(\t r)},{0.*0.38666666666666644*cos(\t r)+1.*0.38666666666666644*sin(\t r)}) -- cycle ;
\draw [shift={(9.726666666666668,0.)},color=zzttqq,fill=zzttqq,fill opacity=0.1]  (0,0) --  plot[domain=-1.5707963267948966:1.5707963267948966,variable=\t]({1.*0.37333333333333313*cos(\t r)+0.*0.37333333333333313*sin(\t r)},{0.*0.37333333333333313*cos(\t r)+1.*0.37333333333333313*sin(\t r)}) -- cycle ;
\draw [line width=2.pt,color=qqffqq] (6.833333333333335,0.)-- (6.846666666666668,0.4133333333333331);
\draw [line width=2.pt,color=qqffqq] (7.153333333333335,0.)-- (7.153333333333335,0.4);
\draw [line width=2.pt,color=qqffqq] (9.726666666666668,0.)-- (9.726666666666668,0.38666666666666644);
\draw [->] (1.,0.9995039532145752) -- (1.2391978767398628,0.9704599570588706);
\draw [->] (1.,-1.) -- (0.779534079620795,-0.9753946780413304);
\end{tikzpicture}
\end{ej}

\newpage

\begin{ej}[Otros modelos habituales]\

\definecolor{zzttqq}{rgb}{0.6,0.2,0.}
\begin{tikzpicture}[line cap=round,line join=round,>=triangle 45,x=1.0cm,y=1.0cm]
\clip(-0.892,-4.057333333333334) rectangle (14.228,3.156);
\fill [color=zzttqq,fill=zzttqq, fill opacity=0.1](0,0)--(0,2)--(3,2)--(3,0)--cycle;
\fill [color=zzttqq,fill=zzttqq, fill opacity=0.1](5,0)--(5,2)--(8,2)--(8,0)--cycle;
\fill [color=zzttqq,fill=zzttqq, fill opacity=0.1](10,0)--(10,2)--(13,2)--(13,0)--cycle;
\fill [color=zzttqq,fill=zzttqq, fill opacity=0.1](3.3066666666666666,-0.9866666666666666)-- (0.6666666666666667,-0.96)-- (0.6666666666666667,-2.96)--(3.3066666666666666,-2.9866666666666664)--cycle;
\fill [color=zzttqq,fill=zzttqq, fill opacity=0.1](5.,-1.)-- (8.,-1.)--(8,-3)--(5,-3)--cycle;
\draw [line width=1.2pt, fill=white] (6.521333333333334,-2.017333333333333) circle (0.5054195177166078cm);
\fill [color=zzttqq,fill=zzttqq, fill opacity=0.1](9.468,-2.9906666666666673)-- (9.441333333333333,-1.0173333333333336)--(12.613333333333332,-1.0133333333333332)--(12.613333333333332,-3.0133333333333328)--cycle;
\draw [line width=1.2pt, fill=white] (11.,-2.) circle (0.4681690103180918cm);
\draw [line width=1.2pt] (0.,0.)-- (0.,2.);
\draw [line width=1.2pt] (0.,0.)-- (3.,0.);
\draw [line width=1.2pt] (0.,2.)-- (3.,2.);
\draw [line width=1.2pt] (3.,2.)-- (3.,0.);
\draw [line width=1.2pt] (5.,0.)-- (5.,2.);
\draw [line width=1.2pt] (5.,0.)-- (8.,0.);
\draw [line width=1.2pt] (8.,0.)-- (8.,2.);
\draw [line width=1.2pt] (8.,2.)-- (5.,2.);
\draw [line width=1.2pt] (10.,0.)-- (13.,0.);
\draw [line width=1.2pt] (10.,0.)-- (10.,2.);
\draw [line width=1.2pt] (13.,0.)-- (13.,2.);
\draw [line width=1.2pt] (10.,2.)-- (13.,2.);
\draw [line width=1.2pt] (3.3066666666666666,-0.9866666666666666)-- (0.6666666666666667,-0.96);
\draw [line width=1.2pt] (5.,-1.)-- (8.,-1.);
\draw [line width=1.2pt] (0.6666666666666667,-0.96)-- (0.6666666666666667,-2.96);
\draw [line width=1.2pt] (3.3066666666666666,-0.9866666666666666)-- (3.3066666666666666,-2.9866666666666664);
\draw [line width=1.2pt] (3.3066666666666666,-2.9866666666666664)-- (0.6666666666666667,-2.96);
\draw [line width=1.2pt] (5.,-1.)-- (5.,-3.);
\draw [line width=1.2pt] (5.,-3.)-- (8.,-3.);
\draw [line width=1.2pt] (8.,-1.)-- (8.,-3.);
\draw [->] (0.,0.) -- (0.,1.);
\draw [->] (3.,0.) -- (3.,1.);
\draw [->] (5.,0.) -- (5.,1.);
\draw [->] (8.,2.) -- (8.,1.);
\draw [->] (10.,0.) -- (10.,1.);
\draw [->] (10.,2.) -- (11.494666666666665,2.);
\draw [->] (13.,0.) -- (13.,1.);
\draw [->] (10.,0.) -- (11.534666666666666,0.);
\draw [->] (0.6666666666666667,-2.96) -- (0.6666666666666667,-1.96);
\draw [->] (3.3066666666666666,-2.9866666666666664) -- (3.3066666666666666,-1.9866666666666664);
\draw [->] (0.6666666666666667,-0.96) -- (2.04064,-0.9738785185185185);
\draw [->] (3.3066666666666666,-2.9866666666666664) -- (1.9937066666666667,-2.9734044444444443);
\draw [->] (5.,-3.) -- (5.,-2.);
\draw [->] (8.,-3.) -- (8.,-2.);
\draw [->] (5.,-1.) -- (6.4946666666666655,-1.);
\draw [->] (5.,-3.) -- (6.534666666666666,-3.);
\draw [line width=1.2pt] (6.521333333333334,-2.017333333333333) circle (0.5054195177166078cm);
\draw [->] (7.026666666666666,-2.0266666666666664) -- (6.965038890401267,-1.775312120387188);
\draw (-0.3853333333333338,1.1826666666666668) node[anchor=north west] {$a$};
\draw (3.268,1.116) node[anchor=north west] {$a$};
\draw (4.574666666666666,1.156) node[anchor=north west] {$a$};
\draw (8,1.156) node[anchor=north west] {$a$};
\draw (9.4,1.0893333333333335) node[anchor=north west] {$a$};
\draw (13.281333333333333,1.156) node[anchor=north west] {$a$};
\draw (11.334666666666665,2.0093333333333336) node[anchor=north west] {$b$};
\draw (11.494666666666665,0.5293333333333333) node[anchor=north west] {$b$};
\draw (2.881333333333333,-1.7773333333333339) node[anchor=north west] {$a$};
\draw (0.7,-1.7373333333333338) node[anchor=north west] {$a$};
\draw (7.654666666666666,-1.7506666666666673) node[anchor=north west] {$a$};
\draw (5.148,-1.7506666666666673) node[anchor=north west] {$a$};
\draw (1.8546666666666662,-1.004) node[anchor=north west] {$b$};
\draw (1.9746666666666663,-2.484) node[anchor=north west] {$b$};
\draw (6.468,-2.524) node[anchor=north west] {$b$};
\draw (6.374666666666666,-0.9373333333333337) node[anchor=north west] {$b$};
\draw (6.5,-1.5773333333333337) node[anchor=north west] {$c$};
\draw (0.9613333333333329,-0.01733333333333349) node[anchor=north west] {$cilindro$};
\draw (5.2,-0.01733333333333349) node[anchor=north west] {$\textit{banda de Möbius}$};
\draw (11.281333333333333,-0.084) node[anchor=north west] {$toro$};
\draw (0.6,-3.084) node[anchor=north west] {$\textit{botella de Klein}$};
\draw (7.8,-3.0306666666666673) node[anchor=north west] {$\textit{toro doble}$};
\draw [line width=1.2pt] (9.468,-2.9906666666666673)-- (9.441333333333333,-1.0173333333333336);
\draw [line width=1.2pt] (9.441333333333333,-1.0173333333333336)-- (12.613333333333332,-1.0133333333333332);
\draw [line width=1.2pt] (12.613333333333332,-1.0133333333333332)-- (12.613333333333332,-3.0133333333333328);
\draw [line width=1.2pt] (9.468,-2.9906666666666673)-- (12.613333333333332,-3.0133333333333328);
\draw [line width=1.2pt] (11.,-2.) circle (0.4681690103180918cm);
\draw [->] (9.441333333333333,-1.0173333333333336) -- (11.,-1.);
\draw [->] (9.468,-2.9906666666666673) -- (11.188102842021598,-3.0030624904540204);
\draw [->] (11.201333333333332,-1.5773333333333337) -- (11.001275660142683,-1.5318327276352801);
\draw [->] (9.468,-2.9906666666666673) -- (9.452685046558335,-1.8573601119834466);
\draw [->] (12.613333333333332,-3.0133333333333328) -- (12.613333333333332,-1.964);
\draw (11.,-1.6) node[anchor=north west] {$c$};
\draw (10.868,-0.9773333333333337) node[anchor=north west] {$d$};
\draw (10.881333333333332,-2.5) node[anchor=north west] {$d$};
\draw (9.588,-1.764) node[anchor=north west] {$e$};
\draw (12.1,-1.7506666666666673) node[anchor=north west] {$e$};
\end{tikzpicture}

\end{ej}

\begin{ej} Distintos modelos pueden representar el mismo objeto.

\begin{tikzpicture}[line cap=round,line join=round,>=triangle 45,x=1.0cm,y=1.0cm]
\clip(-0.2,-1.2) rectangle (15.133333333333335,1.2);
\draw(2.,0.) circle (1.cm);
\draw(4.,0.) circle (1.cm);
\draw(7.,0.) circle (1.cm);
\draw(7.5,0.) circle (0.5cm);
\draw(10.,0.) circle (0.68cm);
\draw(11.7,0.) circle (0.6782329983125326cm);
\draw (3.133333333333334,0.22) node[anchor=north west] {$p$};
\draw (8.,0.15333333333333343) node[anchor=north west] {$p$};
\draw (10.253333333333334,0.22) node[anchor=north west] {$p$};
\draw (11.146666666666667,0.23333333333333348) node[anchor=north west] {$p$};
\draw (5.3,0.2466666666666668) node[anchor=north west] {$\cong$};
\draw (8.653333333333334,0.14) node[anchor=north west] {$\cong$};
\begin{scriptsize}
\draw [fill=black] (3.,0.) circle (1.5pt);
\draw [fill=black] (10.68,0.) circle (1.5pt);
\draw [fill=black] (11.021767001687467,0.) circle (1.5pt);
\draw [fill=black] (8.,0.) circle (1.5pt);
\end{scriptsize}
\end{tikzpicture}

\end{ej}

\end{document} 